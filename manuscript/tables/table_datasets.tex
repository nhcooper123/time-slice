% latex table generated in R 3.4.3 by xtable 1.8-2 package
% Wed Dec 20 13:12:18 2017
\begin{table}[!htbp]
\centering
\begin{tabular}{lcccc}
  \hline
 & \textbf{Beck2014} & \textbf{Brusatte2014} & \textbf{Bapst2016} & \textbf{Wright2017} \\ 
  \hline
Group & mammals & theropods & theropods & crinoids \\ 
  \# taxa & 106 & 152 &  89 &  42 \\ 
  \# characters & 421 & 853 & 374 &  87 \\ 
  Age range (MYA) & 171.8 - 0 & 168.5 - 0 & 207.2 - 0 & 485.4 - 372.2 \\ 
  Mass extinction (MYA) & 66 (K-Pg) & NA & NA & 443 (O-S) \\ 
  Reference & \cite{beckancient2014} & \cite{brusatte2014gradual} & \cite{bapst2016topology} & \cite{wright2017bayesian} \\ 
  Data reference &  \cite{beckancient2014} & \cite{dryad_84t75} & \cite{dryad_n2g80} &  \cite{dryad_6hb7j} \\ 
   \hline
\end{tabular}
\caption{Details of the datasets used in this study. Note that although Brusatte2014 and Bapst2016 cross the Cretaceous Palaeogene (K-Pg) mass extinction there are not enough taxa post K-Pg to investigate the effects of the mass extinction on disparity. Age ranges are root time to most recent tip taxon.} 
\label{table:datasets}  
\end{table}
