% Preamble
\documentclass[12pt,a4paper]{article}
\usepackage{enumerate} 	
\usepackage{setspace}						
\usepackage{authblk}	
\usepackage{graphicx} 	
%\usepackage[nomarkers, nolists]{endfloat} 
\usepackage{pdflscape}	
\usepackage{mathtools}	
\usepackage[osf]{mathpazo} 
\usepackage{lineno} 	
\usepackage{hyperref}
\usepackage[round]{natbib} 
\usepackage{setspace}
\usepackage{longtable}
\usepackage{fancyhdr}

\setcounter{secnumdepth}{0} 
%\raggedright 			
%\pagenumbering{arabic}	
\pagenumbering{gobble}

\pagestyle{fancy}
\rhead{\textit{Guillerme \& Cooper 2018}}
\lhead{\textit{Supporting Information}}
\renewcommand{\headrulewidth}{0pt}
\renewcommand{\footrulewidth}{0pt}
\setlength{\headsep}{0.3in}

\renewcommand{\thetable}{A\arabic{table}}
\renewcommand{\thefigure}{A\arabic{figure}}

% First order headings upper case bold
\usepackage{titlesec}
\titleformat*{\section}{\small\bfseries\uppercase}

% Second order headings normal case italics
\titleformat*{\subsection}{\small\itshape}

% Third order, italics, paragraph style
\titleformat*{\paragraph}{\small\itshape}

\begin{document}

\par{\centering{\Large \bf Supporting Information from ``Time for a rethink: time sub-sampling methods in disparity-through-time analyses''\par}}

\setlength{\parindent}{1cm}

\section{Appendix S1: Extra details of datasets}

\subsection{Beck2014 (Figure \ref{figure:beck})}
The following taxa were removed because they were in the phylogeny but not the character matrix or vice versa: \textit{Montanalestes}, \textit{Lainodon}, \textit{Kharmerungulatum}, \textit{Alymlestes}. 

\subsection{Brusatte2014 (Figure \ref{figure:brusatte})} 
We used one randomly selected time-scaled tree from Brusatte et al. (2014).
Zero-length branches were replaced with the minimum branch length in the phylogeny.
The following taxa were removed because they were in the phylogeny but not the character matrix or vice versa: \textit{Sinraptor dongi}, \textit{Hesperonychus elizabethae}, \textit{Pyroraptor olympius}, \textit{Limenavis patagonica}, \textit{Lithornis}, \textit{Crypturellus undulatus}, \textit{Gallus gallus}, \textit{Crax pauxi}, \textit{Anas}\\ \textit{platyrhynchus}, \textit{Chauna torquata}, \textit{Epidendrosaurus} and \textit{Kinnareemimus}. 
The following taxa were removed because they shared no characters in the morphological matrix: \textit{Shanag ashile}, \textit{Atrociraptor marshalli}, \textit{Proceratosaurus bradleyi}, \textit{Incisivosaurus gauthieri}, \textit{Enigmosaurus}, \textit{Nanshiungosaurus brevispinus}, \textit{Xixiasaurus}, \textit{Tsaagan mangas}, \textit{Mirischia}, \textit{Pedopenna}, \textit{Suzhousaurus}, \textit{Juratyrant}, \textit{Vorona}, \textit{Bonapartenykus}, \textit{Teratophoneus}, \textit{Gobipteryx}, \textit{Songlingornis}, \textit{Liaoningornis longidigitu} and \textit{Achillesaurus}.   

\subsection{Bapst2016 (Figure \ref{figure:bapst})} 
We used the maximum clade credibility tree from Bapst et al. (2016).
Zero-length branches were replaced with the minimum branch length in the phylogeny.
The following taxa were removed because they were in the phylogeny but not the character matrix or vice versa: \textit{Mei long} and \textit{Mei lon}. 
The following taxa were removed because they shared no characters in the morphological matrix: \textit{Hagryphus giganteus}, \textit{Atrociraptor marshalli}, \textit{IGM100 1015 UndesDromaeosaurid}, \textit{Dromaeosaurus albertensis}, \textit{Incisivosaurus gauthieri}, \textit{Deinocheirus mirificus}, \textit{Therizinosaurus cheloniformis},  \textit{Anserimimus planinychus} and \textit{Elmisaurus rarus}. 

\subsection{Wright2017 (Figure \ref{figure:wright})}
We used the maximum clade credibility tree from Wright (2017). 
To properly timescale the tree we followed the advice of Wright (2017) and divided the branch lengths by the corresponding clock rate (= 0.03517385)  and then set the root time to 485.4. 
Zero-length branches were replaced with the minimum branch length in the phylogeny.
No taxa were removed. 

\begin{figure}[!htbp]
    \centering
    \includegraphics[width=1\linewidth, height=1\textheight, keepaspectratio]{figures/fig-tree-Beck2014-appendix.pdf}
    \caption[Beck2014.]
    {Phylogeny from Beck \& Lee (2014).}
    \label{figure:beck}
  \end{figure} 

\begin{figure}[!htbp]
    \centering
    \includegraphics[width=1\linewidth, height=1\textheight, keepaspectratio]{figures/fig-tree-Brusatte2014-appendix.pdf}
    \caption[Brusatte2014.]
    {Phylogeny from Brusatte et al. (2014). This is one randomly selected tree from the time-scaled trees in the paper.}
    \label{figure:brusatte}
  \end{figure} 

\begin{figure}[!htbp]
    \centering
    \includegraphics[width=1\linewidth, height=1\textheight, keepaspectratio]{figures/fig-tree-Bapst2016-appendix.pdf}
    \caption[Bapst2016.]
    {Phylogeny from Bapst et al. (2016). This is the maximum clade credibility tree.}
    \label{figure:bapst}
  \end{figure} 

\begin{figure}[!htbp]
    \centering
    \includegraphics[width=1\linewidth, height=1\textheight, keepaspectratio]{figures/fig-tree-Wright2017-appendix.pdf}
    \caption[Wright2017.]
    {This is the maximum clade credibility tree from Wright (2017).}
    \label{figure:wright}
  \end{figure}  

\newpage
\section{Appendix S2: Additional tables and figures}

% Reset table and figure counters to 1
\setcounter{figure}{0}  
\setcounter{table}{0}  

% DTT figures
  \begin{figure}[!htbp]
    \centering
    \includegraphics[width=1\linewidth, height=1\textheight, keepaspectratio]{figures/fig-dtt-epoch-appendix.pdf}
    \caption[Relative disparity through time for four example datasets.]
    {Relative disparity-through-time for stratigraphic epochs. 
    Median bootstrapped disparities were calculated using time binning and time-slicing approaches. 
    Relative disparities (median bootstrapped disparity divided by the maximum median bootstrapped disparity for a dataset and analysis method) are presented so they can be compared across datasets/methods. 
    Stratigraphy uses unequal time bins or non-equidistant time-slices, where the width of the bin, or the interval between slices, is equivalent to stratigraphic epochs. 
    Duration uses equal time bins or equidistant time-slices, where the width of the bin, or the interval between slices, is the average duration of stratigraphic epochs in the time frame of the dataset. 
    Number uses equal time bins or equidistant time-slices, where the number of bins, or the number of slices, is the average number of stratigraphic epochs in the time frame of the dataset. 
    In all cases, time bin disparities are plotted at the midpoint of the bin, and error bars represent the 95\% confidence intervals around the bootstrapped median disparity.
    The four dataset names are on the first plot for each dataset (see Table 1 for details).
    Results for stratigraphic ages are shown in Figure A2.}
    \label{figure:dtt2}
  \end{figure}  

  \begin{figure}[!htbp]
    \centering
    \includegraphics[width=1\linewidth, height=1\textheight, keepaspectratio]{figures/fig-dtt-age-appendix.pdf}
    \caption[Relative disparity through time for four example datasets.]
    {Relative disparity-through-time for stratigraphic ages. 
    Median bootstrapped disparities were calculated using time binning and time-slicing approaches. 
    Relative disparities (median bootstrapped disparity divided by the maximum median bootstrapped disparity for a dataset and analysis method) are presented so they can be compared across datasets/methods. 
    Stratigraphy uses unequal time bins or non-equidistant time-slices, where the width of the bin, or the interval between slices, is equivalent to stratigraphic ages. 
    Duration uses equal time bins or equidistant time-slices, where the width of the bin, or the interval between slices, is the average duration of stratigraphic ages in the time frame of the dataset. 
    Number uses equal time bins or equidistant time-slices, where the number of bins, or the number of slices, is the average number of stratigraphic ages in the time frame of the dataset. 
    In all cases, time bin disparities are plotted at the midpoint of the bin, and error bars represent the 95\% confidence intervals around the bootstrapped median disparity.
    The four dataset names are on the first plot for each dataset (see Table 1 for details).
    Results for stratigraphic epochs are shown in Figure A1.}
    \label{figure:dtt3}
  \end{figure}  

% Wilcoxon results table
  \input{tables/table_wilcox_appendix.tex} 
  \label{table:wilcox2}  

% disparity peaks figures
\begin{figure}[!htbp]
    \centering
   % \includegraphics[width=1\linewidth, height=1\textheight, keepaspectratio]{figures/fig-peaks-epoch-appendix.pdf}
    \caption[Timing of peak disparity for four example datasets.]
    {Timing of peak disparity for stratigraphic epochs.
    Median bootstrapped disparities were calculated using time binning and time-slicing approaches. 
    Stratigraphy uses unequal time bins or non-equidistant time-slices, where the width of the bin, or the interval between slices, is equivalent to stratigraphic epochs. 
    Duration uses equal time bins or equidistant time-slices, where the width of the bin, or the interval between slices, is the average duration of stratigraphic epochs in the time frame of the dataset. 
    Number uses equal time bins or equidistant time-slices, where the number of bins, or the number of slices, is the average number of stratigraphic epochs in the time frame of the dataset. 
    For time bins there are two points indicating the maximum and minimum ages of the time bin within which peak disparities appeared.
    The four datasets are on separate rows, from top to bottom these are Beck2014, Brusatte2014, Bapst2016 and Wright2017 (see Table 1 for details).
    Results for stratigraphic ages are in Figure A4.}
    \label{figure:peak2}
  \end{figure}

\begin{figure}[!htbp]
    \centering
    %\includegraphics[width=1\linewidth, height=1\textheight, keepaspectratio]{figures/fig-peaks-age-appendix.pdf}
    \caption[Timing of peak disparity for four example datasets.]
    {Timing of peak disparity for stratigraphic ages.
    Median bootstrapped disparities were calculated using time binning and time-slicing approaches. 
    Stratigraphy uses unequal time bins or non-equidistant time-slices, where the width of the bin, or the interval between slices, is equivalent to stratigraphic ages. 
    Duration uses equal time bins or equidistant time-slices, where the width of the bin, or the interval between slices, is the average duration of stratigraphic ages in the time frame of the dataset. 
    Number uses equal time bins or equidistant time-slices, where the number of bins, or the number of slices, is the average number of stratigraphic ages in the time frame of the dataset. 
    For time bins there are two points indicating the maximum and minimum ages of the time bin within which peak disparities appeared.
    The four datasets are on separate rows, from top to bottom these are Beck2014, Brusatte2014, Bapst2016 and Wright2017 (see Table 1 for details).
    Results for stratigraphic epochs are in Figure A3.}
    \label{figure:peak3}
  \end{figure}

% Full results tables
% These are so huge (100 pages!!!) I'm going to leave them out for now...

%\begin{landscape}
%  % latex table generated in R 3.4.3 by xtable 1.8-2 package
% Thu Dec 21 15:01:41 2017
\begin{longtable}{llllrrrrrrr}
  \hline
stratigraphy & bin\_type & model & subsets & n & obs & bs.median & 2.5\% & 25\% & 75\% & 97.5\% \\ 
  \hline
Ages & stratigraphy & equalbins & 170.3 - 168.3 & 1.00 &  &  &  &  &  &  \\ 
   &  &  & 168.3 - 166.1 & 1.00 &  &  &  &  &  &  \\ 
   &  &  & 166.1 - 163.5 & 0.00 &  &  &  &  &  &  \\ 
   &  &  & 163.5 - 157.3 & 4.00 & 34.04 & 28.24 & 0.00 & 21.64 & 28.83 & 34.04 \\ 
   &  &  & 157.3 - 152.1 & 3.00 & 34.03 & 23.68 & 0.00 & 20.73 & 34.03 & 34.03 \\ 
   &  &  & 152.1 - 145 & 5.00 & 34.31 & 28.33 & 16.67 & 26.23 & 30.61 & 31.64 \\ 
   &  &  & 145 - 139.8 & 2.00 & 35.58 & 35.58 & 0.00 & 0.00 & 35.58 & 35.58 \\ 
   &  &  & 139.8 - 132.9 & 6.00 & 33.57 & 28.54 & 24.02 & 26.26 & 30.44 & 31.95 \\ 
   &  &  & 132.9 - 129.4 & 1.00 &  &  &  &  &  &  \\ 
   &  &  & 129.4 - 125 & 6.00 & 35.58 & 30.64 & 20.75 & 28.70 & 32.86 & 33.89 \\ 
   &  &  & 125 - 113 & 11.00 & 34.73 & 31.72 & 28.33 & 30.52 & 32.44 & 33.48 \\ 
   &  &  & 113 - 100.5 & 14.00 & 34.21 & 31.65 & 29.45 & 30.84 & 32.42 & 33.30 \\ 
   &  &  & 100.5 - 93.9 & 7.00 & 34.99 & 30.25 & 22.29 & 28.36 & 31.86 & 33.44 \\ 
   &  &  & 93.9 - 89.8 & 18.00 & 34.98 & 33.06 & 31.72 & 32.68 & 33.40 & 34.03 \\ 
   &  &  & 89.8 - 86.3 & 8.00 & 33.87 & 29.77 & 23.19 & 28.60 & 31.07 & 32.64 \\ 
   &  &  & 86.3 - 83.6 & 7.00 & 35.69 & 31.11 & 22.65 & 29.22 & 32.86 & 34.19 \\ 
   &  &  & 83.6 - 72.1 & 34.00 & 35.16 & 34.23 & 33.54 & 33.98 & 34.44 & 34.69 \\ 
   &  &  & 72.1 - 66 & 20.00 & 35.44 & 33.70 & 31.99 & 33.25 & 34.02 & 34.65 \\ 
   &  &  & 66 - 61.6 & 23.00 & 35.87 & 34.37 & 33.24 & 34.03 & 34.65 & 35.06 \\ 
   &  &  & 61.6 - 59.2 & 8.00 & 36.41 & 32.30 & 27.25 & 31.02 & 33.70 & 34.97 \\ 
   &  &  & 59.2 - 56 & 13.00 & 35.58 & 33.00 & 30.71 & 32.52 & 33.38 & 34.69 \\ 
   &  &  & 56 - 47.8 & 25.00 & 35.91 & 34.52 & 33.51 & 34.22 & 34.79 & 35.35 \\ 
   &  &  & 47.8 - 41.2 & 7.00 & 36.08 & 30.63 & 24.42 & 29.11 & 32.66 & 34.51 \\ 
   &  &  & 41.2 - 37.8 & 1.00 &  &  &  &  &  &  \\ 
   &  &  & 37.8 - 33.9 & 7.00 & 36.70 & 31.46 & 23.77 & 29.67 & 33.26 & 34.99 \\ 
   &  &  & 33.9 - 28.1 & 2.00 & 37.46 & 37.46 & 0.00 & 0.00 & 37.46 & 37.46 \\ 
   &  &  & 28.1 - 23.03 & 1.00 &  &  &  &  &  &  \\ 
   &  &  & 23.03 - 20.44 & 0.00 &  &  &  &  &  &  \\ 
   &  &  & 20.44 - 15.97 & 1.00 &  &  &  &  &  &  \\ 
   &  &  & 15.97 - 13.82 & 0.00 &  &  &  &  &  &  \\ 
   &  &  & 13.82 - 11.63 & 0.00 &  &  &  &  &  &  \\ 
   &  &  & 11.63 - 7.25 & 0.00 &  &  &  &  &  &  \\ 
   &  &  & 7.25 - 5.33 & 0.00 &  &  &  &  &  &  \\ 
   &  &  & 5.33 - 3.6 & 0.00 &  &  &  &  &  &  \\ 
   &  &  & 3.6 - 2.58 & 0.00 &  &  &  &  &  &  \\ 
   &  &  & 2.58 - 1.8 & 0.00 &  &  &  &  &  &  \\ 
   &  &  & 1.8 - 0.78 & 0.00 &  &  &  &  &  &  \\ 
   &  &  & 0.78 - 0.13 & 0.00 &  &  &  &  &  &  \\ 
   &  &  & 0.13 - 0.01 & 1.00 &  &  &  &  &  &  \\ 
   &  &  & 0.01 - 0 & 14.00 & 36.58 & 34.10 & 31.71 & 33.35 & 34.61 & 35.34 \\ 
   &  & acctran & 169.3 & 3.00 & 33.67 & 22.48 & 0.00 & 21.47 & 23.39 & 33.67 \\ 
   &  &  & 167.2 & 3.00 & 33.67 & 22.48 & 10.20 & 21.47 & 25.96 & 33.67 \\ 
   &  &  & 164.8 & 4.00 & 34.04 & 27.54 & 16.55 & 17.98 & 28.81 & 34.04 \\ 
   &  &  & 160.4 & 5.00 & 34.88 & 28.14 & 20.67 & 26.22 & 31.11 & 34.88 \\ 
   &  &  & 154.7 & 7.00 & 34.72 & 30.61 & 23.87 & 29.00 & 31.76 & 33.43 \\ 
   &  &  & 148.55 & 7.00 & 34.64 & 30.21 & 25.00 & 28.39 & 31.69 & 33.17 \\ 
   &  &  & 142.4 & 9.00 & 34.48 & 30.68 & 26.22 & 29.01 & 31.81 & 32.77 \\ 
   &  &  & 136.35 & 10.00 & 34.25 & 31.17 & 26.12 & 29.74 & 32.29 & 33.44 \\ 
   &  &  & 131.15 & 14.00 & 35.01 & 32.51 & 29.66 & 31.59 & 33.08 & 33.99 \\ 
   &  &  & 127.2 & 13.00 & 34.73 & 32.17 & 29.32 & 31.22 & 32.85 & 33.46 \\ 
   &  &  & 119 & 15.00 & 34.93 & 32.68 & 30.13 & 32.11 & 33.22 & 33.98 \\ 
   &  &  & 106.75 & 20.00 & 35.07 & 33.35 & 32.23 & 32.91 & 33.77 & 34.20 \\ 
   &  &  & 97.2 & 28.00 & 35.34 & 34.06 & 33.39 & 33.87 & 34.32 & 34.74 \\ 
   &  &  & 91.85 & 27.00 & 35.32 & 34.09 & 33.21 & 33.87 & 34.36 & 34.64 \\ 
   &  &  & 88.05 & 26.00 & 35.49 & 34.14 & 33.17 & 33.90 & 34.47 & 34.86 \\ 
   &  &  & 84.95 & 30.00 & 35.37 & 34.19 & 33.43 & 33.92 & 34.43 & 34.81 \\ 
   &  &  & 77.85 & 40.00 & 35.44 & 34.56 & 33.99 & 34.31 & 34.75 & 35.01 \\ 
   &  &  & 69.05 & 44.00 & 35.71 & 34.88 & 34.25 & 34.72 & 35.08 & 35.35 \\ 
   &  &  & 63.8 & 40.00 & 35.83 & 34.99 & 34.35 & 34.83 & 35.14 & 35.46 \\ 
   &  &  & 60.4 & 36.00 & 35.82 & 34.85 & 34.22 & 34.67 & 35.07 & 35.31 \\ 
   &  &  & 57.6 & 36.00 & 35.99 & 35.00 & 34.29 & 34.81 & 35.21 & 35.58 \\ 
   &  &  & 51.9 & 29.00 & 36.27 & 35.13 & 34.10 & 34.77 & 35.40 & 35.71 \\ 
   &  &  & 44.5 & 23.00 & 36.54 & 34.91 & 33.69 & 34.57 & 35.16 & 35.79 \\ 
   &  &  & 39.5 & 23.00 & 36.72 & 35.11 & 34.08 & 34.83 & 35.43 & 35.89 \\ 
   &  &  & 35.85 & 20.00 & 36.71 & 35.00 & 33.63 & 34.48 & 35.37 & 35.79 \\ 
   &  &  & 31 & 17.00 & 36.73 & 34.57 & 33.04 & 34.21 & 35.02 & 35.73 \\ 
   &  &  & 25.57 & 15.00 & 36.64 & 34.33 & 32.03 & 33.64 & 34.93 & 35.62 \\ 
   &  &  & 21.73 & 15.00 & 36.64 & 34.23 & 32.02 & 33.66 & 34.66 & 35.64 \\ 
   &  &  & 18.2 & 15.00 & 36.64 & 34.48 & 31.69 & 33.84 & 34.90 & 35.47 \\ 
   &  &  & 14.89 & 14.00 & 36.58 & 34.13 & 30.40 & 33.34 & 34.65 & 35.38 \\ 
   &  &  & 12.72 & 14.00 & 36.58 & 34.25 & 32.26 & 33.56 & 34.61 & 35.36 \\ 
   &  &  & 9.44 & 14.00 & 36.58 & 33.96 & 31.59 & 33.48 & 34.57 & 35.27 \\ 
   &  &  & 6.29 & 14.00 & 36.58 & 34.20 & 32.51 & 33.44 & 34.70 & 35.46 \\ 
   &  &  & 4.47 & 14.00 & 36.58 & 34.15 & 32.10 & 33.45 & 34.62 & 35.35 \\ 
   &  &  & 3.09 & 14.00 & 36.58 & 34.11 & 31.74 & 33.60 & 34.57 & 35.31 \\ 
   &  &  & 2.19 & 14.00 & 36.58 & 34.10 & 31.46 & 33.27 & 34.66 & 35.39 \\ 
   &  &  & 1.29 & 14.00 & 36.58 & 34.26 & 32.42 & 33.48 & 34.57 & 35.33 \\ 
   &  &  & 0.45 & 14.00 & 36.58 & 34.15 & 32.09 & 33.64 & 34.74 & 35.69 \\ 
   &  &  & 0.07 & 14.00 & 36.58 & 33.81 & 31.84 & 33.12 & 34.56 & 35.34 \\ 
   &  &  & 0.01 & 14.00 & 36.58 & 34.01 & 31.32 & 33.00 & 34.56 & 35.36 \\ 
   &  & deltran & 169.3 & 2.00 & 31.10 & 31.10 & 0.00 & 0.00 & 31.10 & 31.10 \\ 
   &  &  & 167.2 & 2.00 & 31.10 & 0.00 & 0.00 & 0.00 & 31.10 & 31.10 \\ 
   &  &  & 164.8 & 3.00 & 31.47 & 21.11 & 0.00 & 21.09 & 31.47 & 31.47 \\ 
   &  &  & 160.4 & 3.00 & 31.88 & 21.39 & 0.00 & 20.73 & 21.64 & 31.88 \\ 
   &  &  & 154.7 & 4.00 & 33.49 & 26.88 & 15.55 & 21.72 & 28.27 & 33.49 \\ 
   &  &  & 148.55 & 4.00 & 32.03 & 26.65 & 16.23 & 21.72 & 26.88 & 32.03 \\ 
   &  &  & 142.4 & 6.00 & 32.85 & 28.20 & 19.96 & 25.43 & 30.05 & 31.06 \\ 
   &  &  & 136.35 & 7.00 & 32.35 & 27.81 & 21.72 & 26.13 & 29.29 & 30.80 \\ 
   &  &  & 131.15 & 10.00 & 32.49 & 29.64 & 27.02 & 28.77 & 30.39 & 31.57 \\ 
   &  &  & 127.2 & 9.00 & 32.32 & 29.42 & 26.03 & 27.99 & 29.80 & 30.75 \\ 
   &  &  & 119 & 12.00 & 33.05 & 30.71 & 27.96 & 30.09 & 31.38 & 32.08 \\ 
   &  &  & 106.75 & 15.00 & 33.13 & 30.93 & 28.49 & 30.24 & 31.63 & 32.23 \\ 
   &  &  & 97.2 & 21.00 & 33.55 & 31.97 & 30.70 & 31.52 & 32.29 & 32.76 \\ 
   &  &  & 91.85 & 21.00 & 33.51 & 32.01 & 31.02 & 31.61 & 32.28 & 32.77 \\ 
   &  &  & 88.05 & 15.00 & 33.87 & 31.73 & 29.23 & 31.10 & 32.19 & 32.93 \\ 
   &  &  & 84.95 & 18.00 & 33.91 & 32.02 & 30.45 & 31.57 & 32.52 & 32.97 \\ 
   &  &  & 77.85 & 30.00 & 34.51 & 33.46 & 32.47 & 33.22 & 33.62 & 33.99 \\ 
   &  &  & 69.05 & 30.00 & 34.15 & 33.04 & 32.25 & 32.80 & 33.31 & 33.76 \\ 
   &  &  & 63.8 & 24.00 & 33.63 & 32.28 & 30.99 & 31.97 & 32.61 & 33.08 \\ 
   &  &  & 60.4 & 25.00 & 33.97 & 32.67 & 31.16 & 32.25 & 32.99 & 33.45 \\ 
   &  &  & 57.6 & 24.00 & 33.98 & 32.67 & 31.42 & 32.33 & 32.94 & 33.48 \\ 
   &  &  & 51.9 & 21.00 & 33.71 & 32.21 & 31.35 & 31.93 & 32.54 & 33.14 \\ 
   &  &  & 44.5 & 16.00 & 33.22 & 31.36 & 29.77 & 30.91 & 31.77 & 32.50 \\ 
   &  &  & 39.5 & 15.00 & 33.30 & 30.91 & 29.08 & 30.35 & 31.56 & 32.51 \\ 
   &  &  & 35.85 & 14.00 & 32.96 & 30.74 & 28.52 & 30.11 & 31.18 & 31.98 \\ 
   &  &  & 31 & 12.00 & 33.21 & 30.34 & 26.84 & 29.60 & 31.07 & 32.17 \\ 
   &  &  & 25.57 & 10.00 & 32.75 & 29.55 & 25.84 & 28.69 & 30.34 & 31.34 \\ 
   &  &  & 21.73 & 10.00 & 32.75 & 29.62 & 25.35 & 28.74 & 30.54 & 31.38 \\ 
   &  &  & 18.2 & 10.00 & 32.75 & 29.79 & 26.61 & 28.92 & 30.43 & 31.73 \\ 
   &  &  & 14.89 & 9.00 & 32.49 & 29.61 & 25.98 & 28.65 & 30.49 & 30.95 \\ 
   &  &  & 12.72 & 9.00 & 32.49 & 28.99 & 24.77 & 28.27 & 30.03 & 31.26 \\ 
   &  &  & 9.44 & 9.00 & 32.49 & 29.20 & 24.63 & 27.84 & 30.48 & 31.61 \\ 
   &  &  & 6.29 & 9.00 & 32.49 & 29.10 & 24.87 & 28.08 & 29.90 & 31.56 \\ 
   &  &  & 4.47 & 9.00 & 32.49 & 28.85 & 22.58 & 27.04 & 29.89 & 30.86 \\ 
   &  &  & 3.09 & 9.00 & 32.49 & 29.15 & 25.98 & 28.05 & 30.50 & 31.46 \\ 
   &  &  & 2.19 & 9.00 & 32.49 & 29.65 & 25.85 & 27.96 & 30.60 & 30.90 \\ 
   &  &  & 1.29 & 9.00 & 32.49 & 28.89 & 25.88 & 27.85 & 30.49 & 31.67 \\ 
   &  &  & 0.45 & 9.00 & 32.49 & 29.07 & 26.02 & 27.97 & 29.89 & 31.65 \\ 
   &  &  & 0.07 & 9.00 & 32.49 & 29.37 & 25.01 & 28.00 & 30.53 & 31.25 \\ 
   &  &  & 0.01 & 10.00 & 32.89 & 29.93 & 25.52 & 28.84 & 30.63 & 31.66 \\ 
   &  & random & 169.3 & 3.00 & 31.10 & 20.73 & 0.00 & 20.73 & 31.10 & 31.10 \\ 
   &  &  & 167.2 & 3.00 & 32.86 & 22.48 & 0.00 & 20.73 & 25.09 & 32.86 \\ 
   &  &  & 164.8 & 4.00 & 33.77 & 27.82 & 16.07 & 22.46 & 28.44 & 33.77 \\ 
   &  &  & 160.4 & 5.00 & 34.11 & 27.34 & 19.53 & 23.72 & 29.92 & 34.11 \\ 
   &  &  & 154.7 & 6.00 & 33.59 & 28.77 & 19.92 & 26.53 & 31.04 & 32.11 \\ 
   &  &  & 148.55 & 7.00 & 34.64 & 30.34 & 24.29 & 28.08 & 31.59 & 33.24 \\ 
   &  &  & 142.4 & 8.00 & 33.49 & 29.40 & 24.62 & 28.07 & 30.59 & 32.27 \\ 
   &  &  & 136.35 & 7.00 & 33.10 & 29.10 & 22.11 & 27.00 & 30.29 & 31.68 \\ 
   &  &  & 131.15 & 14.00 & 33.28 & 31.04 & 28.53 & 29.99 & 31.62 & 32.49 \\ 
   &  &  & 127.2 & 12.00 & 33.73 & 31.05 & 28.19 & 30.38 & 31.85 & 32.78 \\ 
   &  &  & 119 & 13.00 & 33.71 & 31.14 & 28.50 & 30.36 & 31.91 & 32.71 \\ 
   &  &  & 106.75 & 20.00 & 33.98 & 32.40 & 30.86 & 31.94 & 32.76 & 33.23 \\ 
   &  &  & 97.2 & 27.00 & 34.19 & 32.97 & 31.98 & 32.70 & 33.22 & 33.49 \\ 
   &  &  & 91.85 & 24.00 & 34.36 & 33.03 & 31.91 & 32.67 & 33.24 & 33.70 \\ 
   &  &  & 88.05 & 26.00 & 34.68 & 33.33 & 32.09 & 33.04 & 33.67 & 34.11 \\ 
   &  &  & 84.95 & 25.00 & 34.76 & 33.39 & 32.58 & 33.15 & 33.66 & 34.08 \\ 
   &  &  & 77.85 & 38.00 & 34.78 & 33.87 & 33.24 & 33.63 & 34.11 & 34.43 \\ 
   &  &  & 69.05 & 37.00 & 34.97 & 34.01 & 33.02 & 33.69 & 34.21 & 34.65 \\ 
   &  &  & 63.8 & 35.00 & 34.80 & 33.88 & 33.27 & 33.68 & 34.07 & 34.32 \\ 
   &  &  & 60.4 & 36.00 & 34.85 & 33.94 & 33.19 & 33.70 & 34.17 & 34.52 \\ 
   &  &  & 57.6 & 33.00 & 35.00 & 33.89 & 33.01 & 33.69 & 34.20 & 34.69 \\ 
   &  &  & 51.9 & 25.00 & 34.84 & 33.40 & 32.21 & 33.07 & 33.69 & 34.16 \\ 
   &  &  & 44.5 & 22.00 & 34.69 & 33.17 & 31.68 & 32.72 & 33.61 & 34.16 \\ 
   &  &  & 39.5 & 20.00 & 35.15 & 33.40 & 31.77 & 33.02 & 33.84 & 34.60 \\ 
   &  &  & 35.85 & 19.00 & 35.21 & 33.46 & 31.52 & 33.09 & 33.93 & 34.53 \\ 
   &  &  & 31 & 16.00 & 34.59 & 32.56 & 30.15 & 31.75 & 33.10 & 33.95 \\ 
   &  &  & 25.57 & 13.00 & 34.76 & 32.16 & 28.98 & 31.25 & 32.94 & 33.92 \\ 
   &  &  & 21.73 & 13.00 & 35.24 & 32.62 & 30.03 & 31.98 & 33.20 & 34.17 \\ 
   &  &  & 18.2 & 15.00 & 35.29 & 32.85 & 30.60 & 32.23 & 33.50 & 34.41 \\ 
   &  &  & 14.89 & 13.00 & 35.50 & 32.99 & 30.21 & 32.27 & 33.65 & 34.48 \\ 
   &  &  & 12.72 & 14.00 & 35.53 & 33.09 & 31.36 & 32.51 & 33.77 & 34.64 \\ 
   &  &  & 9.44 & 14.00 & 35.18 & 33.07 & 30.45 & 32.20 & 33.63 & 34.38 \\ 
   &  &  & 6.29 & 13.00 & 34.27 & 31.86 & 29.35 & 30.78 & 32.58 & 33.64 \\ 
   &  &  & 4.47 & 13.00 & 35.44 & 32.68 & 30.45 & 32.16 & 33.45 & 34.60 \\ 
   &  &  & 3.09 & 12.00 & 34.59 & 31.78 & 29.37 & 31.16 & 32.52 & 33.45 \\ 
   &  &  & 2.19 & 13.00 & 35.45 & 32.84 & 30.71 & 32.17 & 33.50 & 34.32 \\ 
   &  &  & 1.29 & 14.00 & 35.21 & 32.86 & 29.86 & 32.37 & 33.35 & 34.28 \\ 
   &  &  & 0.45 & 12.00 & 34.33 & 31.68 & 28.31 & 30.96 & 32.32 & 33.32 \\ 
   &  &  & 0.07 & 13.00 & 35.69 & 33.10 & 30.91 & 32.32 & 33.70 & 34.90 \\ 
   &  &  & 0.01 & 14.00 & 33.88 & 31.38 & 29.16 & 30.78 & 32.05 & 32.75 \\ 
   &  & proximity & 169.3 & 2.00 & 31.10 & 31.10 & 0.00 & 0.00 & 31.10 & 31.10 \\ 
   &  &  & 167.2 & 3.00 & 31.47 & 21.09 & 0.00 & 20.73 & 21.11 & 31.47 \\ 
   &  &  & 164.8 & 3.00 & 31.65 & 21.29 & 0.00 & 20.73 & 31.65 & 31.65 \\ 
   &  &  & 160.4 & 3.00 & 33.86 & 22.48 & 0.00 & 21.64 & 23.60 & 33.86 \\ 
   &  &  & 154.7 & 4.00 & 33.49 & 28.04 & 15.87 & 22.30 & 28.37 & 33.49 \\ 
   &  &  & 148.55 & 5.00 & 32.81 & 26.49 & 13.84 & 23.34 & 29.44 & 31.61 \\ 
   &  &  & 142.4 & 8.00 & 33.59 & 29.60 & 21.96 & 28.14 & 30.78 & 32.93 \\ 
   &  &  & 136.35 & 10.00 & 33.50 & 30.28 & 27.07 & 29.20 & 31.13 & 32.58 \\ 
   &  &  & 131.15 & 10.00 & 33.50 & 30.18 & 25.37 & 29.30 & 30.96 & 32.21 \\ 
   &  &  & 127.2 & 12.00 & 33.21 & 30.79 & 28.20 & 30.10 & 31.42 & 32.06 \\ 
   &  &  & 119 & 13.00 & 33.95 & 31.62 & 28.89 & 30.75 & 32.18 & 32.77 \\ 
   &  &  & 106.75 & 19.00 & 34.08 & 32.39 & 30.58 & 31.91 & 32.64 & 33.13 \\ 
   &  &  & 97.2 & 24.00 & 34.74 & 33.38 & 32.37 & 32.96 & 33.64 & 33.95 \\ 
   &  &  & 91.85 & 24.00 & 34.93 & 33.49 & 32.47 & 33.17 & 33.73 & 34.20 \\ 
   &  &  & 88.05 & 17.00 & 34.74 & 32.90 & 31.33 & 32.25 & 33.38 & 33.85 \\ 
   &  &  & 84.95 & 24.00 & 34.79 & 33.41 & 32.33 & 33.04 & 33.70 & 34.14 \\ 
   &  &  & 77.85 & 35.00 & 34.90 & 33.93 & 33.06 & 33.61 & 34.11 & 34.53 \\ 
   &  &  & 69.05 & 34.00 & 34.61 & 33.63 & 32.86 & 33.40 & 33.88 & 34.23 \\ 
   &  &  & 63.8 & 32.00 & 34.88 & 33.82 & 32.98 & 33.57 & 34.08 & 34.46 \\ 
   &  &  & 60.4 & 30.00 & 34.90 & 33.82 & 32.96 & 33.50 & 34.01 & 34.58 \\ 
   &  &  & 57.6 & 30.00 & 35.05 & 33.96 & 32.85 & 33.56 & 34.20 & 34.62 \\ 
   &  &  & 51.9 & 23.00 & 34.57 & 33.24 & 32.14 & 32.64 & 33.45 & 33.93 \\ 
   &  &  & 44.5 & 18.00 & 34.15 & 32.38 & 30.59 & 31.89 & 32.81 & 33.44 \\ 
   &  &  & 39.5 & 19.00 & 34.65 & 33.07 & 31.72 & 32.46 & 33.47 & 33.83 \\ 
   &  &  & 35.85 & 16.00 & 34.74 & 32.66 & 30.17 & 31.98 & 33.19 & 34.13 \\ 
   &  &  & 31 & 14.00 & 35.20 & 32.92 & 29.98 & 32.08 & 33.46 & 34.30 \\ 
   &  &  & 25.57 & 13.00 & 36.24 & 33.40 & 30.07 & 32.57 & 34.19 & 35.15 \\ 
   &  &  & 21.73 & 14.00 & 36.50 & 33.87 & 31.11 & 33.13 & 34.51 & 35.41 \\ 
   &  &  & 18.2 & 15.00 & 36.64 & 34.42 & 31.93 & 33.64 & 34.72 & 35.64 \\ 
   &  &  & 14.89 & 14.00 & 36.58 & 34.20 & 32.46 & 33.51 & 34.64 & 35.70 \\ 
   &  &  & 12.72 & 14.00 & 36.58 & 34.10 & 30.66 & 33.33 & 34.66 & 35.44 \\ 
   &  &  & 9.44 & 14.00 & 36.58 & 34.02 & 31.36 & 33.19 & 34.70 & 35.45 \\ 
   &  &  & 6.29 & 14.00 & 36.58 & 34.09 & 31.57 & 33.34 & 34.58 & 35.15 \\ 
   &  &  & 4.47 & 14.00 & 36.58 & 34.06 & 31.17 & 33.32 & 34.67 & 35.46 \\ 
   &  &  & 3.09 & 14.00 & 36.58 & 34.15 & 32.02 & 33.71 & 34.66 & 35.78 \\ 
   &  &  & 2.19 & 14.00 & 36.58 & 33.98 & 32.12 & 33.47 & 34.51 & 35.37 \\ 
   &  &  & 1.29 & 14.00 & 36.58 & 34.14 & 31.72 & 33.29 & 34.61 & 35.38 \\ 
   &  &  & 0.45 & 14.00 & 36.58 & 33.89 & 30.40 & 33.20 & 34.65 & 35.42 \\ 
   &  &  & 0.07 & 14.00 & 36.58 & 34.16 & 31.34 & 33.15 & 34.77 & 35.44 \\ 
   &  &  & 0.01 & 14.00 & 36.58 & 34.18 & 32.08 & 33.56 & 34.68 & 35.41 \\ 
   &  & equal.split & 169.3 & 3.00 & 31.47 & 21.11 & 0.00 & 20.73 & 22.50 & 33.58 \\ 
   &  &  & 167.2 & 3.00 & 31.65 & 21.09 & 0.00 & 20.73 & 22.50 & 33.49 \\ 
   &  &  & 164.8 & 4.00 & 34.04 & 26.38 & 15.68 & 17.66 & 27.69 & 33.48 \\ 
   &  &  & 160.4 & 5.00 & 34.11 & 26.49 & 12.99 & 23.27 & 29.45 & 31.25 \\ 
   &  &  & 154.7 & 8.00 & 31.12 & 28.49 & 19.24 & 27.03 & 30.08 & 32.60 \\ 
   &  &  & 148.55 & 7.00 & 32.00 & 29.07 & 21.29 & 26.94 & 30.06 & 31.69 \\ 
   &  &  & 142.4 & 10.00 & 32.26 & 29.95 & 24.66 & 28.51 & 31.02 & 32.87 \\ 
   &  &  & 136.35 & 10.00 & 33.11 & 29.40 & 24.42 & 28.13 & 30.45 & 32.61 \\ 
   &  &  & 131.15 & 14.00 & 33.07 & 31.33 & 28.46 & 30.44 & 31.89 & 33.25 \\ 
   &  &  & 127.2 & 13.00 & 32.49 & 30.74 & 27.48 & 29.87 & 31.60 & 32.47 \\ 
   &  &  & 119 & 16.00 & 33.25 & 31.76 & 29.54 & 31.11 & 32.43 & 33.22 \\ 
   &  &  & 106.75 & 20.00 & 33.92 & 32.37 & 30.22 & 31.76 & 32.88 & 33.41 \\ 
   &  &  & 97.2 & 29.00 & 34.13 & 33.23 & 31.96 & 32.81 & 33.56 & 33.97 \\ 
   &  &  & 91.85 & 27.00 & 34.60 & 33.22 & 32.28 & 32.80 & 33.41 & 33.88 \\ 
   &  &  & 88.05 & 26.00 & 34.72 & 33.21 & 31.49 & 32.76 & 33.49 & 34.08 \\ 
   &  &  & 84.95 & 30.00 & 34.45 & 33.36 & 32.28 & 32.95 & 33.65 & 34.25 \\ 
   &  &  & 77.85 & 43.00 & 34.98 & 34.07 & 33.33 & 33.80 & 34.29 & 34.54 \\ 
   &  &  & 69.05 & 48.00 & 34.75 & 34.06 & 33.43 & 33.85 & 34.26 & 34.67 \\ 
   &  &  & 63.8 & 40.00 & 34.72 & 33.87 & 32.76 & 33.41 & 34.17 & 34.56 \\ 
   &  &  & 60.4 & 38.00 & 34.35 & 33.99 & 32.92 & 33.68 & 34.18 & 34.61 \\ 
   &  &  & 57.6 & 37.00 & 35.10 & 33.92 & 32.87 & 33.50 & 34.14 & 34.77 \\ 
   &  &  & 51.9 & 30.00 & 34.89 & 33.66 & 32.12 & 33.18 & 33.99 & 34.60 \\ 
   &  &  & 44.5 & 23.00 & 34.05 & 33.35 & 31.10 & 32.53 & 33.90 & 34.62 \\ 
   &  &  & 39.5 & 23.00 & 34.96 & 33.28 & 31.59 & 32.64 & 33.72 & 34.62 \\ 
   &  &  & 35.85 & 20.00 & 34.42 & 33.09 & 30.52 & 32.30 & 33.75 & 34.69 \\ 
   &  &  & 31 & 17.00 & 36.37 & 32.88 & 30.84 & 32.16 & 33.49 & 34.63 \\ 
   &  &  & 25.57 & 15.00 & 34.33 & 32.56 & 30.55 & 31.82 & 33.48 & 34.28 \\ 
   &  &  & 21.73 & 15.00 & 33.63 & 32.40 & 29.78 & 31.61 & 33.24 & 34.18 \\ 
   &  &  & 18.2 & 15.00 & 35.70 & 32.40 & 29.47 & 31.73 & 33.04 & 34.39 \\ 
   &  &  & 14.89 & 14.00 & 35.63 & 32.07 & 28.42 & 31.28 & 32.70 & 33.75 \\ 
   &  &  & 12.72 & 14.00 & 34.19 & 32.06 & 29.58 & 31.27 & 32.73 & 34.10 \\ 
   &  &  & 9.44 & 14.00 & 34.10 & 31.89 & 28.88 & 31.34 & 32.85 & 34.13 \\ 
   &  &  & 6.29 & 14.00 & 32.70 & 31.98 & 27.95 & 30.91 & 32.92 & 34.05 \\ 
   &  &  & 4.47 & 14.00 & 35.61 & 31.91 & 28.15 & 31.11 & 32.66 & 33.83 \\ 
   &  &  & 3.09 & 14.00 & 32.75 & 31.79 & 28.51 & 31.04 & 32.78 & 34.35 \\ 
   &  &  & 2.19 & 14.00 & 33.65 & 32.10 & 27.87 & 31.10 & 32.94 & 34.33 \\ 
   &  &  & 1.29 & 14.00 & 34.98 & 32.20 & 28.49 & 31.17 & 32.79 & 33.77 \\ 
   &  &  & 0.45 & 14.00 & 34.65 & 32.00 & 28.36 & 31.03 & 32.79 & 34.58 \\ 
   &  &  & 0.07 & 14.00 & 34.86 & 32.06 & 28.79 & 30.98 & 32.76 & 33.93 \\ 
   &  &  & 0.01 & 15.00 & 33.94 & 31.84 & 29.29 & 30.99 & 32.83 & 34.11 \\ 
   &  & gradual.split & 169.3 & 3.00 & 20.73 & 20.73 & 0.00 & 0.00 & 20.73 & 31.47 \\ 
   &  &  & 167.2 & 3.00 & 33.67 & 21.09 & 0.00 & 20.73 & 22.50 & 31.65 \\ 
   &  &  & 164.8 & 4.00 & 26.47 & 24.79 & 0.00 & 17.54 & 27.27 & 33.29 \\ 
   &  &  & 160.4 & 5.00 & 26.81 & 23.73 & 6.10 & 19.48 & 26.81 & 30.58 \\ 
   &  &  & 154.7 & 8.00 & 28.70 & 26.90 & 19.45 & 24.66 & 28.33 & 31.22 \\ 
   &  &  & 148.55 & 7.00 & 31.57 & 26.58 & 17.45 & 23.23 & 29.39 & 31.72 \\ 
   &  &  & 142.4 & 10.00 & 32.65 & 29.32 & 25.47 & 27.79 & 30.59 & 32.17 \\ 
   &  &  & 136.35 & 10.00 & 33.02 & 29.84 & 25.10 & 28.36 & 30.93 & 32.08 \\ 
   &  &  & 131.15 & 14.00 & 32.76 & 30.67 & 27.60 & 29.59 & 31.30 & 32.51 \\ 
   &  &  & 127.2 & 13.00 & 33.47 & 30.62 & 28.25 & 30.08 & 31.26 & 32.56 \\ 
   &  &  & 119 & 16.00 & 33.55 & 31.56 & 29.13 & 30.66 & 32.07 & 32.95 \\ 
   &  &  & 106.75 & 20.00 & 33.50 & 32.30 & 30.69 & 31.82 & 32.73 & 33.29 \\ 
   &  &  & 97.2 & 29.00 & 34.18 & 33.25 & 32.15 & 32.97 & 33.48 & 33.83 \\ 
   &  &  & 91.85 & 27.00 & 34.36 & 33.35 & 31.94 & 33.03 & 33.74 & 34.19 \\ 
   &  &  & 88.05 & 26.00 & 33.72 & 32.62 & 30.59 & 32.16 & 32.96 & 33.45 \\ 
   &  &  & 84.95 & 30.00 & 34.17 & 33.17 & 32.09 & 32.85 & 33.46 & 33.95 \\ 
   &  &  & 77.85 & 43.00 & 34.47 & 33.73 & 33.08 & 33.50 & 33.94 & 34.29 \\ 
   &  &  & 69.05 & 48.00 & 34.35 & 33.86 & 33.31 & 33.67 & 34.06 & 34.48 \\ 
   &  &  & 63.8 & 40.00 & 34.42 & 33.67 & 32.90 & 33.40 & 33.91 & 34.30 \\ 
   &  &  & 60.4 & 38.00 & 34.67 & 33.77 & 32.89 & 33.40 & 33.99 & 34.48 \\ 
   &  &  & 57.6 & 37.00 & 34.93 & 33.51 & 32.40 & 33.12 & 33.83 & 34.21 \\ 
   &  &  & 51.9 & 30.00 & 34.02 & 33.06 & 31.50 & 32.66 & 33.47 & 33.99 \\ 
   &  &  & 44.5 & 23.00 & 33.19 & 32.53 & 30.22 & 31.92 & 33.07 & 33.73 \\ 
   &  &  & 39.5 & 23.00 & 34.09 & 33.25 & 31.64 & 32.58 & 33.83 & 34.61 \\ 
   &  &  & 35.85 & 20.00 & 34.91 & 32.85 & 30.78 & 31.98 & 33.43 & 34.41 \\ 
   &  &  & 31 & 17.00 & 34.85 & 32.54 & 30.04 & 31.86 & 33.18 & 34.75 \\ 
   &  &  & 25.57 & 15.00 & 35.27 & 32.32 & 29.01 & 31.30 & 33.14 & 34.58 \\ 
   &  &  & 21.73 & 15.00 & 35.00 & 32.77 & 29.90 & 31.69 & 33.58 & 34.55 \\ 
   &  &  & 18.2 & 15.00 & 36.44 & 32.90 & 29.59 & 31.85 & 33.68 & 34.71 \\ 
   &  &  & 14.89 & 14.00 & 34.77 & 33.05 & 30.09 & 32.10 & 33.70 & 34.53 \\ 
   &  &  & 12.72 & 14.00 & 35.49 & 33.15 & 30.80 & 32.47 & 33.86 & 34.66 \\ 
   &  &  & 9.44 & 14.00 & 36.28 & 33.53 & 29.91 & 32.69 & 34.22 & 35.10 \\ 
   &  &  & 6.29 & 14.00 & 36.30 & 33.64 & 31.39 & 33.06 & 34.34 & 35.06 \\ 
   &  &  & 4.47 & 14.00 & 36.08 & 33.78 & 31.36 & 32.95 & 34.55 & 35.22 \\ 
   &  &  & 3.09 & 14.00 & 36.58 & 33.92 & 32.07 & 33.28 & 34.69 & 35.32 \\ 
   &  &  & 2.19 & 14.00 & 36.58 & 33.92 & 32.04 & 33.36 & 34.64 & 35.45 \\ 
   &  &  & 1.29 & 14.00 & 36.58 & 34.19 & 31.57 & 33.29 & 34.65 & 35.37 \\ 
   &  &  & 0.45 & 14.00 & 36.58 & 33.94 & 31.89 & 33.37 & 34.55 & 35.40 \\ 
   &  &  & 0.07 & 14.00 & 36.58 & 34.00 & 31.99 & 33.25 & 34.61 & 35.33 \\ 
   &  &  & 0.01 & 15.00 & 36.24 & 34.03 & 31.64 & 33.31 & 34.63 & 35.38 \\ 
   & duration & equalbins & 170.2 - 166.5 & 2.00 & 31.64 & 31.64 & 0.00 & 0.00 & 31.64 & 31.64 \\ 
   &  &  & 166.5 - 162.8 & 0.00 &  &  &  &  &  &  \\ 
   &  &  & 162.8 - 159.1 & 3.00 & 33.06 & 22.06 & 0.00 & 21.64 & 22.41 & 33.06 \\ 
   &  &  & 159.1 - 155.4 & 3.00 & 34.03 & 23.66 & 0.00 & 20.73 & 23.68 & 34.03 \\ 
   &  &  & 155.4 - 151.7 & 1.00 &  &  &  &  &  &  \\ 
   &  &  & 151.7 - 148 & 2.00 & 31.63 & 31.63 & 0.00 & 0.00 & 31.63 & 31.63 \\ 
   &  &  & 148 - 144.3 & 2.00 & 35.32 & 35.32 & 0.00 & 0.00 & 35.32 & 35.32 \\ 
   &  &  & 144.3 - 140.6 & 1.00 &  &  &  &  &  &  \\ 
   &  &  & 140.6 - 136.9 & 3.00 & 34.13 & 23.81 & 0.00 & 20.73 & 23.81 & 34.13 \\ 
   &  &  & 136.9 - 133.2 & 4.00 & 32.93 & 27.33 & 7.65 & 21.93 & 27.55 & 32.93 \\ 
   &  &  & 133.2 - 129.5 & 0.00 &  &  &  &  &  &  \\ 
   &  &  & 129.5 - 125.8 & 3.00 & 35.14 & 24.25 & 0.00 & 21.76 & 24.26 & 35.14 \\ 
   &  &  & 125.8 - 122.1 & 5.00 & 35.22 & 28.75 & 14.31 & 25.04 & 31.87 & 35.22 \\ 
   &  &  & 122.1 - 118.4 & 3.00 & 33.36 & 22.19 & 0.00 & 20.73 & 33.36 & 33.36 \\ 
   &  &  & 118.4 - 114.7 & 2.00 & 31.10 & 0.00 & 0.00 & 0.00 & 31.10 & 31.10 \\ 
   &  &  & 114.7 - 111 & 8.00 & 34.66 & 31.07 & 24.88 & 29.39 & 32.26 & 33.62 \\ 
   &  &  & 111 - 107.3 & 2.00 & 35.61 & 0.00 & 0.00 & 0.00 & 35.61 & 35.61 \\ 
   &  &  & 107.3 - 103.6 & 3.00 & 34.87 & 23.74 & 0.00 & 22.63 & 34.87 & 34.87 \\ 
   &  &  & 103.6 - 99.9 & 5.00 & 34.35 & 27.71 & 13.70 & 24.04 & 31.01 & 34.35 \\ 
   &  &  & 99.9 - 96.2 & 4.00 & 35.76 & 29.59 & 17.54 & 23.84 & 30.06 & 35.76 \\ 
   &  &  & 96.2 - 92.5 & 7.00 & 35.17 & 30.30 & 23.56 & 28.62 & 31.89 & 33.64 \\ 
   &  &  & 92.5 - 88.8 & 15.00 & 34.88 & 32.92 & 29.37 & 32.32 & 33.43 & 33.98 \\ 
   &  &  & 88.8 - 85.1 & 8.00 & 33.42 & 29.97 & 25.34 & 28.78 & 30.79 & 32.27 \\ 
   &  &  & 85.1 - 81.4 & 10.00 & 35.10 & 31.92 & 27.84 & 31.02 & 32.73 & 34.20 \\ 
   &  &  & 81.4 - 77.7 & 14.00 & 35.44 & 33.25 & 30.71 & 32.51 & 33.66 & 34.44 \\ 
   &  &  & 77.7 - 74 & 10.00 & 35.39 & 32.11 & 29.00 & 30.99 & 32.89 & 34.07 \\ 
   &  &  & 74 - 70.3 & 14.00 & 34.94 & 32.72 & 30.61 & 31.98 & 33.31 & 33.92 \\ 
   &  &  & 70.3 - 66.6 & 12.00 & 35.46 & 32.90 & 29.91 & 31.92 & 33.42 & 34.18 \\ 
   &  &  & 66.6 - 62.9 & 17.00 & 35.74 & 33.82 & 32.19 & 33.27 & 34.22 & 34.89 \\ 
   &  &  & 62.9 - 59.2 & 11.00 & 36.15 & 32.94 & 29.75 & 31.90 & 33.87 & 34.93 \\ 
   &  &  & 59.2 - 55.5 & 12.00 & 35.33 & 32.49 & 29.69 & 31.48 & 33.12 & 34.01 \\ 
   &  &  & 55.5 - 51.8 & 14.00 & 35.98 & 33.48 & 31.37 & 32.65 & 34.11 & 34.77 \\ 
   &  &  & 51.8 - 48.1 & 8.00 & 35.59 & 30.94 & 26.45 & 29.59 & 32.49 & 34.52 \\ 
   &  &  & 48.1 - 44.4 & 5.00 & 36.12 & 29.29 & 17.61 & 25.68 & 32.37 & 32.86 \\ 
   &  &  & 44.4 - 40.7 & 2.00 & 36.14 & 0.00 & 0.00 & 0.00 & 36.14 & 36.14 \\ 
   &  &  & 40.7 - 37 & 3.00 & 35.57 & 23.70 & 0.00 & 22.46 & 24.98 & 35.57 \\ 
   &  &  & 37 - 33.3 & 6.00 & 36.66 & 31.83 & 22.12 & 29.34 & 33.85 & 34.51 \\ 
   &  &  & 33.3 - 29.6 & 2.00 & 37.46 & 37.46 & 0.00 & 0.00 & 37.46 & 37.46 \\ 
   &  &  & 29.6 - 25.9 & 1.00 &  &  &  &  &  &  \\ 
   &  &  & 25.9 - 22.2 & 0.00 &  &  &  &  &  &  \\ 
   &  &  & 22.2 - 18.5 & 0.00 &  &  &  &  &  &  \\ 
   &  &  & 18.5 - 14.8 & 1.00 &  &  &  &  &  &  \\ 
   &  &  & 14.8 - 11.1 & 0.00 &  &  &  &  &  &  \\ 
   &  &  & 11.1 - 7.4 & 0.00 &  &  &  &  &  &  \\ 
   &  &  & 7.4 - 3.7 & 0.00 &  &  &  &  &  &  \\ 
   &  &  & 3.7 - 0 & 14.00 & 36.58 & 34.10 & 31.70 & 33.37 & 34.68 & 35.55 \\ 
   &  & acctran & 168.35 & 3.00 & 33.67 & 23.39 & 0.00 & 21.47 & 33.67 & 33.67 \\ 
   &  &  & 164.65 & 4.00 & 34.04 & 27.96 & 16.23 & 22.32 & 28.85 & 34.04 \\ 
   &  &  & 160.95 & 6.00 & 34.89 & 29.37 & 21.17 & 27.51 & 32.23 & 33.06 \\ 
   &  &  & 157.25 & 5.00 & 34.88 & 28.01 & 16.49 & 24.80 & 31.12 & 33.51 \\ 
   &  &  & 153.55 & 7.00 & 34.72 & 30.36 & 24.02 & 28.75 & 31.42 & 33.51 \\ 
   &  &  & 149.85 & 6.00 & 34.31 & 29.61 & 21.13 & 27.02 & 30.82 & 32.43 \\ 
   &  &  & 146.15 & 8.00 & 34.58 & 30.40 & 25.17 & 29.09 & 31.76 & 33.53 \\ 
   &  &  & 142.45 & 9.00 & 34.48 & 30.94 & 26.74 & 29.90 & 31.80 & 33.11 \\ 
   &  &  & 138.75 & 10.00 & 34.25 & 31.06 & 26.71 & 29.63 & 32.00 & 33.28 \\ 
   &  &  & 135.05 & 12.00 & 34.85 & 31.93 & 29.38 & 31.21 & 32.66 & 33.61 \\ 
   &  &  & 131.35 & 14.00 & 35.01 & 32.86 & 30.09 & 32.18 & 33.33 & 33.98 \\ 
   &  &  & 127.65 & 13.00 & 34.73 & 32.26 & 29.55 & 31.62 & 32.86 & 33.57 \\ 
   &  &  & 123.95 & 15.00 & 34.89 & 32.64 & 30.54 & 32.12 & 33.21 & 33.87 \\ 
   &  &  & 120.25 & 15.00 & 34.76 & 32.74 & 30.84 & 32.11 & 33.28 & 33.95 \\ 
   &  &  & 116.55 & 15.00 & 34.93 & 32.87 & 30.43 & 32.31 & 33.28 & 34.13 \\ 
   &  &  & 112.85 & 15.00 & 34.75 & 32.37 & 30.49 & 31.79 & 32.95 & 33.68 \\ 
   &  &  & 109.15 & 19.00 & 35.04 & 33.27 & 31.53 & 32.80 & 33.64 & 34.15 \\ 
   &  &  & 105.45 & 22.00 & 35.04 & 33.53 & 32.34 & 33.00 & 33.94 & 34.34 \\ 
   &  &  & 101.75 & 25.00 & 35.20 & 33.85 & 32.75 & 33.55 & 34.15 & 34.49 \\ 
   &  &  & 98.05 & 26.00 & 35.19 & 33.87 & 33.07 & 33.59 & 34.15 & 34.51 \\ 
   &  &  & 94.35 & 26.00 & 35.35 & 34.09 & 33.04 & 33.79 & 34.32 & 34.61 \\ 
   &  &  & 90.65 & 30.00 & 35.44 & 34.35 & 33.47 & 34.10 & 34.55 & 34.81 \\ 
   &  &  & 86.95 & 28.00 & 35.48 & 34.13 & 32.83 & 33.80 & 34.40 & 34.87 \\ 
   &  &  & 83.25 & 30.00 & 35.12 & 33.99 & 33.12 & 33.68 & 34.22 & 34.54 \\ 
   &  &  & 79.55 & 36.00 & 35.28 & 34.30 & 33.50 & 34.10 & 34.56 & 34.87 \\ 
   &  &  & 75.85 & 34.00 & 35.17 & 34.19 & 33.37 & 33.96 & 34.40 & 34.77 \\ 
   &  &  & 72.15 & 40.00 & 35.54 & 34.67 & 34.00 & 34.47 & 34.87 & 35.20 \\ 
   &  &  & 68.45 & 44.00 & 35.71 & 34.88 & 34.33 & 34.71 & 35.05 & 35.37 \\ 
   &  &  & 64.75 & 40.00 & 35.74 & 34.88 & 34.22 & 34.65 & 35.06 & 35.34 \\ 
   &  &  & 61.05 & 36.00 & 35.82 & 34.81 & 34.19 & 34.65 & 35.06 & 35.42 \\ 
   &  &  & 57.35 & 36.00 & 35.99 & 35.01 & 33.88 & 34.79 & 35.24 & 35.63 \\ 
   &  &  & 53.65 & 32.00 & 36.26 & 35.16 & 34.49 & 34.91 & 35.29 & 35.72 \\ 
   &  &  & 49.95 & 27.00 & 36.38 & 35.09 & 34.07 & 34.78 & 35.31 & 35.78 \\ 
   &  &  & 46.25 & 23.00 & 36.54 & 34.89 & 33.94 & 34.65 & 35.31 & 35.61 \\ 
   &  &  & 42.55 & 23.00 & 36.72 & 35.24 & 34.13 & 34.98 & 35.48 & 35.89 \\ 
   &  &  & 38.85 & 24.00 & 36.82 & 35.33 & 34.35 & 35.05 & 35.62 & 35.98 \\ 
   &  &  & 35.15 & 19.00 & 36.75 & 34.98 & 33.09 & 34.40 & 35.33 & 35.81 \\ 
   &  &  & 31.45 & 17.00 & 36.73 & 34.72 & 32.97 & 34.25 & 35.13 & 35.76 \\ 
   &  &  & 27.75 & 16.00 & 36.71 & 34.57 & 32.62 & 34.03 & 34.98 & 35.68 \\ 
   &  &  & 24.05 & 15.00 & 36.64 & 34.25 & 32.22 & 33.86 & 34.78 & 35.68 \\ 
   &  &  & 20.35 & 15.00 & 36.64 & 34.43 & 32.00 & 33.65 & 35.01 & 35.48 \\ 
   &  &  & 16.65 & 15.00 & 36.64 & 34.28 & 32.46 & 33.61 & 34.71 & 35.49 \\ 
   &  &  & 12.95 & 14.00 & 36.58 & 34.26 & 30.95 & 33.52 & 34.69 & 35.31 \\ 
   &  &  & 9.25 & 14.00 & 36.58 & 34.15 & 32.45 & 33.45 & 34.63 & 35.48 \\ 
   &  &  & 5.55 & 14.00 & 36.58 & 34.07 & 31.81 & 33.36 & 34.70 & 35.57 \\ 
   &  &  & 1.85 & 14.00 & 36.58 & 34.11 & 31.86 & 33.26 & 34.54 & 35.33 \\ 
   &  & deltran & 168.35 & 2.00 & 31.10 & 31.10 & 0.00 & 0.00 & 31.10 & 31.10 \\ 
   &  &  & 164.65 & 3.00 & 31.47 & 21.09 & 0.00 & 20.73 & 21.11 & 31.47 \\ 
   &  &  & 160.95 & 4.00 & 31.89 & 26.47 & 0.00 & 21.39 & 26.70 & 31.89 \\ 
   &  &  & 157.25 & 3.00 & 33.86 & 23.60 & 0.00 & 21.64 & 33.86 & 33.86 \\ 
   &  &  & 153.55 & 4.00 & 33.49 & 27.98 & 15.87 & 21.41 & 28.48 & 33.49 \\ 
   &  &  & 149.85 & 3.00 & 32.06 & 21.64 & 0.00 & 20.73 & 32.06 & 32.06 \\ 
   &  &  & 146.15 & 4.00 & 32.16 & 26.87 & 15.81 & 21.59 & 26.91 & 32.16 \\ 
   &  &  & 142.45 & 6.00 & 32.85 & 28.34 & 19.63 & 26.00 & 30.32 & 32.85 \\ 
   &  &  & 138.75 & 7.00 & 32.35 & 27.92 & 22.25 & 26.30 & 29.32 & 31.71 \\ 
   &  &  & 135.05 & 9.00 & 32.40 & 29.01 & 24.03 & 28.12 & 30.01 & 31.48 \\ 
   &  &  & 131.35 & 10.00 & 32.49 & 29.40 & 26.18 & 28.62 & 30.28 & 31.24 \\ 
   &  &  & 127.65 & 9.00 & 32.32 & 29.02 & 25.75 & 27.71 & 29.79 & 31.41 \\ 
   &  &  & 123.95 & 11.00 & 33.40 & 30.27 & 26.18 & 29.37 & 31.24 & 32.38 \\ 
   &  &  & 120.25 & 11.00 & 33.05 & 30.10 & 26.59 & 29.29 & 30.90 & 32.13 \\ 
   &  &  & 116.55 & 12.00 & 33.05 & 30.52 & 27.61 & 29.46 & 31.37 & 32.12 \\ 
   &  &  & 112.85 & 12.00 & 33.04 & 30.55 & 27.99 & 29.89 & 31.27 & 32.09 \\ 
   &  &  & 109.15 & 15.00 & 33.08 & 30.93 & 28.87 & 30.44 & 31.20 & 31.92 \\ 
   &  &  & 105.45 & 17.00 & 33.18 & 31.34 & 29.32 & 30.80 & 31.73 & 32.31 \\ 
   &  &  & 101.75 & 19.00 & 33.21 & 31.48 & 29.91 & 31.01 & 31.93 & 32.45 \\ 
   &  &  & 98.05 & 20.00 & 33.44 & 31.87 & 30.43 & 31.27 & 32.23 & 32.62 \\ 
   &  &  & 94.35 & 20.00 & 33.59 & 32.03 & 30.67 & 31.62 & 32.41 & 33.03 \\ 
   &  &  & 90.65 & 22.00 & 33.57 & 32.17 & 31.24 & 31.85 & 32.39 & 32.85 \\ 
   &  &  & 86.95 & 17.00 & 33.91 & 32.01 & 30.45 & 31.52 & 32.34 & 33.02 \\ 
   &  &  & 83.25 & 20.00 & 34.15 & 32.46 & 31.12 & 32.02 & 32.87 & 33.31 \\ 
   &  &  & 79.55 & 26.00 & 34.24 & 32.85 & 31.88 & 32.57 & 33.18 & 33.63 \\ 
   &  &  & 75.85 & 25.00 & 34.10 & 32.77 & 31.52 & 32.31 & 33.08 & 33.56 \\ 
   &  &  & 72.15 & 27.00 & 34.03 & 32.78 & 31.73 & 32.46 & 33.07 & 33.48 \\ 
   &  &  & 68.45 & 30.00 & 34.18 & 33.10 & 32.17 & 32.85 & 33.30 & 33.77 \\ 
   &  &  & 64.75 & 25.00 & 33.74 & 32.33 & 31.30 & 32.05 & 32.69 & 33.12 \\ 
   &  &  & 61.05 & 25.00 & 33.97 & 32.75 & 31.71 & 32.50 & 32.99 & 33.50 \\ 
   &  &  & 57.35 & 24.00 & 33.98 & 32.61 & 31.68 & 32.27 & 32.91 & 33.37 \\ 
   &  &  & 53.65 & 24.00 & 33.65 & 32.17 & 30.90 & 31.81 & 32.48 & 32.93 \\ 
   &  &  & 49.95 & 20.00 & 33.82 & 32.21 & 30.44 & 31.63 & 32.59 & 33.05 \\ 
   &  &  & 46.25 & 16.00 & 33.22 & 31.27 & 29.41 & 30.83 & 31.77 & 32.28 \\ 
   &  &  & 42.55 & 15.00 & 33.30 & 31.18 & 28.58 & 30.69 & 31.81 & 32.40 \\ 
   &  &  & 38.85 & 16.00 & 33.30 & 31.44 & 29.50 & 30.87 & 31.83 & 32.53 \\ 
   &  &  & 35.15 & 13.00 & 33.02 & 30.74 & 27.65 & 30.04 & 31.34 & 32.14 \\ 
   &  &  & 31.45 & 11.00 & 32.86 & 30.00 & 28.00 & 29.21 & 30.97 & 31.56 \\ 
   &  &  & 27.75 & 11.00 & 32.86 & 29.99 & 27.16 & 28.98 & 30.76 & 31.61 \\ 
   &  &  & 24.05 & 10.00 & 32.75 & 29.61 & 26.38 & 28.58 & 30.45 & 31.40 \\ 
   &  &  & 20.35 & 10.00 & 32.75 & 29.58 & 25.87 & 28.77 & 30.23 & 31.39 \\ 
   &  &  & 16.65 & 10.00 & 32.75 & 29.59 & 26.03 & 28.41 & 30.53 & 32.07 \\ 
   &  &  & 12.95 & 9.00 & 32.49 & 29.06 & 24.39 & 27.88 & 29.87 & 30.78 \\ 
   &  &  & 9.25 & 9.00 & 32.49 & 28.97 & 25.44 & 27.88 & 30.23 & 30.84 \\ 
   &  &  & 5.55 & 9.00 & 32.49 & 29.03 & 23.69 & 27.91 & 29.84 & 31.56 \\ 
   &  &  & 1.85 & 9.00 & 32.49 & 28.88 & 25.11 & 27.87 & 29.90 & 31.25 \\ 
   &  & random & 168.35 & 3.00 & 32.86 & 22.48 & 0.00 & 20.73 & 22.50 & 32.86 \\ 
   &  &  & 164.65 & 4.00 & 33.51 & 27.86 & 7.51 & 22.50 & 28.53 & 33.51 \\ 
   &  &  & 160.95 & 5.00 & 35.31 & 28.34 & 14.45 & 24.82 & 31.82 & 32.34 \\ 
   &  &  & 157.25 & 4.00 & 33.47 & 27.96 & 15.55 & 21.72 & 28.34 & 33.47 \\ 
   &  &  & 153.55 & 5.00 & 33.84 & 27.61 & 14.19 & 23.94 & 30.62 & 33.84 \\ 
   &  &  & 149.85 & 6.00 & 33.25 & 28.74 & 16.80 & 26.55 & 29.99 & 31.57 \\ 
   &  &  & 146.15 & 7.00 & 34.50 & 30.20 & 22.68 & 27.81 & 31.41 & 33.16 \\ 
   &  &  & 142.45 & 8.00 & 33.99 & 30.07 & 21.66 & 28.33 & 31.28 & 32.70 \\ 
   &  &  & 138.75 & 10.00 & 33.39 & 30.39 & 26.96 & 29.32 & 31.24 & 31.99 \\ 
   &  &  & 135.05 & 11.00 & 33.72 & 30.91 & 26.72 & 29.69 & 31.71 & 32.81 \\ 
   &  &  & 131.35 & 13.00 & 34.41 & 31.87 & 29.28 & 31.29 & 32.66 & 33.54 \\ 
   &  &  & 127.65 & 12.00 & 33.62 & 31.02 & 28.26 & 30.31 & 31.59 & 32.54 \\ 
   &  &  & 123.95 & 15.00 & 34.17 & 32.01 & 30.35 & 31.49 & 32.46 & 33.19 \\ 
   &  &  & 120.25 & 13.00 & 33.88 & 31.21 & 28.62 & 30.46 & 31.96 & 32.93 \\ 
   &  &  & 116.55 & 15.00 & 34.27 & 32.28 & 29.67 & 31.55 & 32.72 & 33.30 \\ 
   &  &  & 112.85 & 15.00 & 33.81 & 31.77 & 29.99 & 31.01 & 32.23 & 33.04 \\ 
   &  &  & 109.15 & 18.00 & 33.97 & 32.20 & 30.67 & 31.77 & 32.54 & 33.20 \\ 
   &  &  & 105.45 & 19.00 & 34.06 & 32.42 & 30.89 & 32.09 & 32.79 & 33.24 \\ 
   &  &  & 101.75 & 23.00 & 34.02 & 32.62 & 31.47 & 32.21 & 32.87 & 33.38 \\ 
   &  &  & 98.05 & 21.00 & 34.29 & 32.78 & 31.45 & 32.48 & 33.11 & 33.72 \\ 
   &  &  & 94.35 & 24.00 & 34.58 & 33.10 & 31.90 & 32.68 & 33.55 & 34.05 \\ 
   &  &  & 90.65 & 27.00 & 34.42 & 33.15 & 32.41 & 32.91 & 33.37 & 33.73 \\ 
   &  &  & 86.95 & 26.00 & 34.81 & 33.55 & 32.42 & 33.28 & 33.86 & 34.14 \\ 
   &  &  & 83.25 & 28.00 & 34.73 & 33.55 & 32.41 & 33.20 & 33.75 & 34.12 \\ 
   &  &  & 79.55 & 33.00 & 34.70 & 33.61 & 32.69 & 33.37 & 33.78 & 34.15 \\ 
   &  &  & 75.85 & 34.00 & 34.67 & 33.66 & 32.87 & 33.43 & 33.93 & 34.24 \\ 
   &  &  & 72.15 & 37.00 & 35.08 & 34.10 & 33.25 & 33.96 & 34.29 & 34.56 \\ 
   &  &  & 68.45 & 40.00 & 35.10 & 34.27 & 33.69 & 34.10 & 34.42 & 34.82 \\ 
   &  &  & 64.75 & 38.00 & 35.00 & 34.00 & 33.20 & 33.81 & 34.24 & 34.55 \\ 
   &  &  & 61.05 & 33.00 & 34.91 & 33.84 & 32.92 & 33.56 & 34.13 & 34.47 \\ 
   &  &  & 57.35 & 34.00 & 34.91 & 33.95 & 33.13 & 33.70 & 34.18 & 34.63 \\ 
   &  &  & 53.65 & 30.00 & 34.84 & 33.72 & 32.64 & 33.45 & 34.02 & 34.48 \\ 
   &  &  & 49.95 & 25.00 & 35.10 & 33.81 & 32.55 & 33.36 & 34.07 & 34.47 \\ 
   &  &  & 46.25 & 22.00 & 35.39 & 33.88 & 32.69 & 33.45 & 34.27 & 34.68 \\ 
   &  &  & 42.55 & 21.00 & 35.53 & 33.81 & 32.15 & 33.40 & 34.34 & 34.76 \\ 
   &  &  & 38.85 & 23.00 & 35.70 & 34.17 & 32.93 & 33.86 & 34.56 & 35.09 \\ 
   &  &  & 35.15 & 17.00 & 35.53 & 33.63 & 31.19 & 32.87 & 34.11 & 34.86 \\ 
   &  &  & 31.45 & 16.00 & 35.16 & 33.15 & 30.38 & 32.60 & 33.61 & 34.46 \\ 
   &  &  & 27.75 & 15.00 & 35.47 & 33.20 & 30.72 & 32.62 & 33.73 & 34.51 \\ 
   &  &  & 24.05 & 15.00 & 34.33 & 32.14 & 29.82 & 31.63 & 32.69 & 33.25 \\ 
   &  &  & 20.35 & 14.00 & 34.76 & 32.60 & 30.29 & 31.87 & 33.17 & 34.17 \\ 
   &  &  & 16.65 & 15.00 & 35.69 & 33.54 & 31.33 & 32.66 & 34.13 & 34.80 \\ 
   &  &  & 12.95 & 13.00 & 34.84 & 32.51 & 30.29 & 31.75 & 33.09 & 34.08 \\ 
   &  &  & 9.25 & 12.00 & 34.77 & 32.11 & 28.90 & 31.51 & 32.66 & 33.84 \\ 
   &  &  & 5.55 & 13.00 & 35.22 & 32.93 & 29.85 & 31.80 & 33.49 & 34.25 \\ 
   &  &  & 1.85 & 13.00 & 34.76 & 32.53 & 30.08 & 31.56 & 33.20 & 34.05 \\ 
   &  & proximity & 168.35 & 2.00 & 31.10 & 31.10 & 0.00 & 0.00 & 31.10 & 31.10 \\ 
   &  &  & 164.65 & 3.00 & 31.65 & 21.11 & 0.00 & 20.73 & 31.65 & 31.65 \\ 
   &  &  & 160.95 & 4.00 & 34.04 & 27.64 & 16.23 & 17.98 & 28.81 & 34.04 \\ 
   &  &  & 157.25 & 4.00 & 33.49 & 28.12 & 15.55 & 25.89 & 28.48 & 33.49 \\ 
   &  &  & 153.55 & 4.00 & 33.49 & 27.98 & 15.55 & 22.30 & 28.48 & 33.49 \\ 
   &  &  & 149.85 & 5.00 & 32.81 & 26.50 & 12.65 & 23.23 & 29.61 & 32.81 \\ 
   &  &  & 146.15 & 6.00 & 32.85 & 27.92 & 19.15 & 26.04 & 28.86 & 30.91 \\ 
   &  &  & 142.45 & 8.00 & 33.59 & 29.56 & 24.43 & 28.23 & 30.87 & 32.49 \\ 
   &  &  & 138.75 & 9.00 & 33.41 & 29.89 & 26.35 & 28.46 & 31.12 & 32.53 \\ 
   &  &  & 135.05 & 10.00 & 33.50 & 30.52 & 26.28 & 29.26 & 31.25 & 32.19 \\ 
   &  &  & 131.35 & 10.00 & 33.50 & 30.48 & 27.34 & 29.33 & 31.51 & 32.22 \\ 
   &  &  & 127.65 & 12.00 & 33.21 & 30.77 & 28.33 & 30.21 & 31.35 & 32.25 \\ 
   &  &  & 123.95 & 13.00 & 33.78 & 31.72 & 27.98 & 30.90 & 32.14 & 32.64 \\ 
   &  &  & 120.25 & 13.00 & 33.87 & 31.42 & 29.31 & 30.50 & 32.16 & 33.08 \\ 
   &  &  & 116.55 & 15.00 & 33.97 & 31.67 & 29.56 & 31.16 & 32.13 & 32.98 \\ 
   &  &  & 112.85 & 14.00 & 33.77 & 31.60 & 29.38 & 30.81 & 32.13 & 32.95 \\ 
   &  &  & 109.15 & 15.00 & 34.26 & 31.95 & 29.18 & 31.23 & 32.58 & 33.39 \\ 
   &  &  & 105.45 & 19.00 & 34.08 & 32.42 & 31.18 & 31.86 & 32.76 & 33.31 \\ 
   &  &  & 101.75 & 21.00 & 34.59 & 32.91 & 31.99 & 32.68 & 33.19 & 33.69 \\ 
   &  &  & 98.05 & 23.00 & 34.62 & 33.16 & 32.03 & 32.74 & 33.42 & 33.95 \\ 
   &  &  & 94.35 & 25.00 & 34.76 & 33.45 & 32.25 & 33.06 & 33.67 & 34.02 \\ 
   &  &  & 90.65 & 25.00 & 34.98 & 33.59 & 32.61 & 33.35 & 33.84 & 34.31 \\ 
   &  &  & 86.95 & 19.00 & 34.86 & 33.06 & 31.83 & 32.59 & 33.46 & 34.16 \\ 
   &  &  & 83.25 & 26.00 & 34.74 & 33.48 & 32.28 & 33.18 & 33.67 & 34.04 \\ 
   &  &  & 79.55 & 32.00 & 34.78 & 33.79 & 32.71 & 33.52 & 34.01 & 34.28 \\ 
   &  &  & 75.85 & 30.00 & 34.60 & 33.41 & 32.61 & 33.14 & 33.60 & 33.99 \\ 
   &  &  & 72.15 & 33.00 & 34.58 & 33.57 & 32.62 & 33.25 & 33.87 & 34.15 \\ 
   &  &  & 68.45 & 36.00 & 34.74 & 33.85 & 33.09 & 33.60 & 34.04 & 34.35 \\ 
   &  &  & 64.75 & 32.00 & 34.93 & 33.87 & 33.06 & 33.72 & 34.15 & 34.61 \\ 
   &  &  & 61.05 & 30.00 & 34.64 & 33.49 & 32.40 & 33.17 & 33.82 & 34.27 \\ 
   &  &  & 57.35 & 30.00 & 35.05 & 33.94 & 32.82 & 33.61 & 34.13 & 34.52 \\ 
   &  &  & 53.65 & 26.00 & 34.75 & 33.47 & 32.38 & 33.10 & 33.84 & 34.16 \\ 
   &  &  & 49.95 & 21.00 & 34.39 & 32.88 & 31.67 & 32.36 & 33.20 & 33.75 \\ 
   &  &  & 46.25 & 17.00 & 33.97 & 32.05 & 30.16 & 31.51 & 32.58 & 33.42 \\ 
   &  &  & 42.55 & 17.00 & 34.36 & 32.50 & 30.85 & 32.01 & 32.91 & 33.95 \\ 
   &  &  & 38.85 & 19.00 & 34.84 & 33.05 & 30.93 & 32.49 & 33.50 & 34.23 \\ 
   &  &  & 35.15 & 15.00 & 34.86 & 32.61 & 30.43 & 31.80 & 33.32 & 34.04 \\ 
   &  &  & 31.45 & 14.00 & 35.20 & 32.54 & 31.01 & 31.94 & 33.28 & 33.98 \\ 
   &  &  & 27.75 & 14.00 & 35.86 & 33.53 & 30.82 & 32.96 & 34.08 & 35.02 \\ 
   &  &  & 24.05 & 14.00 & 36.50 & 34.01 & 31.90 & 33.32 & 34.46 & 35.48 \\ 
   &  &  & 20.35 & 14.00 & 36.50 & 34.08 & 31.93 & 33.62 & 34.65 & 35.49 \\ 
   &  &  & 16.65 & 15.00 & 36.64 & 34.19 & 32.17 & 33.46 & 34.68 & 35.59 \\ 
   &  &  & 12.95 & 14.00 & 36.58 & 34.04 & 31.52 & 33.47 & 34.60 & 35.38 \\ 
   &  &  & 9.25 & 14.00 & 36.58 & 34.09 & 31.21 & 33.31 & 34.58 & 35.42 \\ 
   &  &  & 5.55 & 14.00 & 36.58 & 34.18 & 31.71 & 33.58 & 34.73 & 35.36 \\ 
   &  &  & 1.85 & 14.00 & 36.58 & 34.14 & 32.10 & 33.38 & 34.60 & 35.41 \\ 
   &  & equal.split & 168.35 & 3.00 & 20.73 & 21.10 & 0.00 & 20.73 & 23.39 & 33.49 \\ 
   &  &  & 164.65 & 4.00 & 26.47 & 26.68 & 15.82 & 21.00 & 28.21 & 33.55 \\ 
   &  &  & 160.95 & 6.00 & 29.16 & 28.69 & 19.40 & 25.73 & 29.95 & 33.20 \\ 
   &  &  & 157.25 & 6.00 & 29.58 & 26.06 & 18.38 & 24.74 & 29.17 & 32.15 \\ 
   &  &  & 153.55 & 7.00 & 32.24 & 29.54 & 22.32 & 26.82 & 30.44 & 32.72 \\ 
   &  &  & 149.85 & 6.00 & 34.12 & 26.74 & 17.35 & 24.47 & 28.73 & 32.11 \\ 
   &  &  & 146.15 & 8.00 & 33.64 & 28.24 & 21.08 & 26.59 & 29.91 & 32.80 \\ 
   &  &  & 142.45 & 10.00 & 32.75 & 29.72 & 24.81 & 27.86 & 30.61 & 31.74 \\ 
   &  &  & 138.75 & 10.00 & 31.78 & 29.95 & 25.81 & 28.85 & 30.93 & 32.37 \\ 
   &  &  & 135.05 & 12.00 & 33.44 & 30.73 & 27.65 & 29.64 & 31.52 & 32.54 \\ 
   &  &  & 131.35 & 14.00 & 34.25 & 31.29 & 28.02 & 30.46 & 31.82 & 32.83 \\ 
   &  &  & 127.65 & 13.00 & 33.26 & 30.65 & 27.36 & 29.75 & 31.54 & 32.74 \\ 
   &  &  & 123.95 & 16.00 & 33.63 & 31.59 & 29.39 & 30.97 & 32.18 & 33.33 \\ 
   &  &  & 120.25 & 15.00 & 33.60 & 31.52 & 29.02 & 30.81 & 32.20 & 33.09 \\ 
   &  &  & 116.55 & 16.00 & 34.26 & 31.58 & 29.31 & 31.05 & 32.40 & 33.18 \\ 
   &  &  & 112.85 & 16.00 & 33.33 & 31.75 & 29.31 & 30.99 & 32.19 & 33.22 \\ 
   &  &  & 109.15 & 19.00 & 33.89 & 32.30 & 30.66 & 31.72 & 32.72 & 33.34 \\ 
   &  &  & 105.45 & 22.00 & 33.94 & 32.34 & 30.74 & 31.93 & 32.89 & 33.60 \\ 
   &  &  & 101.75 & 25.00 & 34.14 & 32.80 & 31.59 & 32.36 & 33.20 & 33.72 \\ 
   &  &  & 98.05 & 27.00 & 34.03 & 33.00 & 31.59 & 32.57 & 33.36 & 33.94 \\ 
   &  &  & 94.35 & 26.00 & 34.79 & 33.07 & 31.97 & 32.75 & 33.37 & 34.19 \\ 
   &  &  & 90.65 & 30.00 & 34.65 & 33.34 & 32.37 & 33.05 & 33.69 & 34.13 \\ 
   &  &  & 86.95 & 28.00 & 34.15 & 33.19 & 32.12 & 32.82 & 33.50 & 34.06 \\ 
   &  &  & 83.25 & 32.00 & 34.53 & 33.43 & 32.58 & 33.10 & 33.65 & 34.14 \\ 
   &  &  & 79.55 & 39.00 & 34.52 & 33.65 & 32.58 & 33.42 & 33.89 & 34.31 \\ 
   &  &  & 75.85 & 34.00 & 34.37 & 33.51 & 32.28 & 33.21 & 33.82 & 34.08 \\ 
   &  &  & 72.15 & 40.00 & 34.81 & 33.78 & 32.88 & 33.52 & 34.06 & 34.59 \\ 
   &  &  & 68.45 & 45.00 & 34.43 & 34.05 & 33.39 & 33.83 & 34.24 & 34.56 \\ 
   &  &  & 64.75 & 40.00 & 34.72 & 33.78 & 33.01 & 33.46 & 34.10 & 34.37 \\ 
   &  &  & 61.05 & 38.00 & 34.48 & 33.94 & 33.15 & 33.66 & 34.19 & 34.63 \\ 
   &  &  & 57.35 & 36.00 & 34.54 & 33.79 & 32.89 & 33.51 & 34.11 & 34.71 \\ 
   &  &  & 53.65 & 33.00 & 35.18 & 33.73 & 32.54 & 33.32 & 34.20 & 34.79 \\ 
   &  &  & 49.95 & 29.00 & 35.07 & 33.62 & 31.92 & 33.19 & 34.00 & 34.55 \\ 
   &  &  & 46.25 & 23.00 & 34.62 & 33.14 & 31.04 & 32.60 & 33.56 & 34.55 \\ 
   &  &  & 42.55 & 23.00 & 34.79 & 33.44 & 31.21 & 32.85 & 33.81 & 34.58 \\ 
   &  &  & 38.85 & 24.00 & 35.48 & 33.43 & 31.49 & 33.01 & 33.94 & 34.77 \\ 
   &  &  & 35.15 & 19.00 & 34.87 & 32.90 & 30.84 & 32.37 & 33.46 & 34.34 \\ 
   &  &  & 31.45 & 17.00 & 33.05 & 32.87 & 29.80 & 32.26 & 33.47 & 34.35 \\ 
   &  &  & 27.75 & 16.00 & 34.00 & 32.62 & 29.05 & 31.83 & 33.32 & 34.15 \\ 
   &  &  & 24.05 & 15.00 & 33.93 & 32.47 & 28.86 & 31.50 & 33.05 & 34.32 \\ 
   &  &  & 20.35 & 15.00 & 34.28 & 32.47 & 28.92 & 31.57 & 33.31 & 34.48 \\ 
   &  &  & 16.65 & 15.00 & 34.01 & 32.30 & 29.98 & 31.56 & 33.19 & 34.18 \\ 
   &  &  & 12.95 & 14.00 & 34.38 & 32.07 & 29.16 & 30.94 & 32.66 & 33.71 \\ 
   &  &  & 9.25 & 14.00 & 33.85 & 31.87 & 28.83 & 30.79 & 32.88 & 34.12 \\ 
   &  &  & 5.55 & 14.00 & 33.04 & 32.02 & 29.75 & 31.27 & 32.81 & 33.99 \\ 
   &  &  & 1.85 & 14.00 & 35.22 & 31.94 & 28.77 & 31.19 & 32.80 & 34.51 \\ 
   &  & gradual.split & 168.35 & 3.00 & 31.47 & 20.73 & 0.00 & 20.73 & 21.47 & 32.62 \\ 
   &  &  & 164.65 & 4.00 & 31.89 & 26.19 & 0.00 & 17.54 & 27.04 & 33.00 \\ 
   &  &  & 160.95 & 6.00 & 29.16 & 24.45 & 11.22 & 22.57 & 26.92 & 30.38 \\ 
   &  &  & 157.25 & 6.00 & 29.21 & 25.37 & 14.75 & 24.47 & 28.05 & 31.34 \\ 
   &  &  & 153.55 & 7.00 & 30.20 & 26.49 & 16.84 & 23.58 & 28.47 & 31.00 \\ 
   &  &  & 149.85 & 6.00 & 27.97 & 26.01 & 14.04 & 23.17 & 28.45 & 32.04 \\ 
   &  &  & 146.15 & 8.00 & 32.67 & 27.67 & 21.00 & 25.82 & 29.46 & 32.26 \\ 
   &  &  & 142.45 & 10.00 & 32.52 & 29.64 & 24.97 & 27.94 & 30.62 & 31.99 \\ 
   &  &  & 138.75 & 10.00 & 31.50 & 29.52 & 25.24 & 28.09 & 30.56 & 31.77 \\ 
   &  &  & 135.05 & 12.00 & 32.38 & 29.73 & 25.80 & 28.62 & 30.52 & 31.69 \\ 
   &  &  & 131.35 & 14.00 & 33.75 & 30.40 & 27.21 & 29.48 & 31.49 & 32.56 \\ 
   &  &  & 127.65 & 13.00 & 32.45 & 30.67 & 28.48 & 30.12 & 31.29 & 32.55 \\ 
   &  &  & 123.95 & 16.00 & 32.89 & 31.09 & 28.12 & 30.29 & 31.69 & 32.46 \\ 
   &  &  & 120.25 & 15.00 & 33.39 & 31.53 & 29.26 & 30.47 & 32.17 & 33.14 \\ 
   &  &  & 116.55 & 16.00 & 34.30 & 31.78 & 29.48 & 31.10 & 32.42 & 33.08 \\ 
   &  &  & 112.85 & 16.00 & 33.28 & 31.55 & 29.25 & 30.85 & 32.09 & 32.94 \\ 
   &  &  & 109.15 & 19.00 & 33.22 & 31.81 & 30.03 & 31.35 & 32.24 & 32.92 \\ 
   &  &  & 105.45 & 22.00 & 33.71 & 32.27 & 31.05 & 31.75 & 32.64 & 33.44 \\ 
   &  &  & 101.75 & 25.00 & 33.57 & 32.52 & 31.23 & 32.04 & 32.88 & 33.35 \\ 
   &  &  & 98.05 & 27.00 & 34.09 & 33.12 & 32.01 & 32.67 & 33.44 & 33.94 \\ 
   &  &  & 94.35 & 26.00 & 34.42 & 33.35 & 31.85 & 32.97 & 33.63 & 34.22 \\ 
   &  &  & 90.65 & 30.00 & 34.20 & 33.37 & 32.46 & 33.07 & 33.65 & 34.15 \\ 
   &  &  & 86.95 & 28.00 & 34.42 & 32.80 & 31.34 & 32.35 & 33.13 & 33.76 \\ 
   &  &  & 83.25 & 32.00 & 34.62 & 33.27 & 32.27 & 33.03 & 33.54 & 33.95 \\ 
   &  &  & 79.55 & 39.00 & 34.33 & 33.65 & 32.71 & 33.28 & 33.89 & 34.34 \\ 
   &  &  & 75.85 & 34.00 & 34.19 & 33.33 & 32.19 & 32.98 & 33.56 & 33.99 \\ 
   &  &  & 72.15 & 40.00 & 34.42 & 33.43 & 32.49 & 33.06 & 33.76 & 34.14 \\ 
   &  &  & 68.45 & 45.00 & 34.61 & 33.89 & 33.19 & 33.61 & 34.13 & 34.51 \\ 
   &  &  & 64.75 & 40.00 & 34.58 & 33.67 & 32.84 & 33.34 & 33.91 & 34.39 \\ 
   &  &  & 61.05 & 38.00 & 34.29 & 33.53 & 32.76 & 33.25 & 33.91 & 34.42 \\ 
   &  &  & 57.35 & 36.00 & 34.38 & 33.55 & 32.19 & 33.11 & 33.93 & 34.26 \\ 
   &  &  & 53.65 & 33.00 & 34.65 & 33.29 & 32.37 & 32.96 & 33.62 & 34.15 \\ 
   &  &  & 49.95 & 29.00 & 33.59 & 32.82 & 31.41 & 32.35 & 33.16 & 34.06 \\ 
   &  &  & 46.25 & 23.00 & 33.65 & 32.18 & 29.96 & 31.51 & 32.66 & 33.58 \\ 
   &  &  & 42.55 & 23.00 & 34.91 & 32.87 & 30.75 & 32.16 & 33.31 & 34.23 \\ 
   &  &  & 38.85 & 24.00 & 35.14 & 33.05 & 31.43 & 32.48 & 33.35 & 33.94 \\ 
   &  &  & 35.15 & 19.00 & 34.27 & 32.41 & 30.20 & 31.75 & 32.94 & 34.21 \\ 
   &  &  & 31.45 & 17.00 & 33.13 & 32.52 & 29.97 & 31.77 & 33.08 & 34.48 \\ 
   &  &  & 27.75 & 16.00 & 34.35 & 32.54 & 28.66 & 31.58 & 33.30 & 34.52 \\ 
   &  &  & 24.05 & 15.00 & 34.53 & 32.59 & 30.35 & 31.91 & 33.37 & 34.40 \\ 
   &  &  & 20.35 & 15.00 & 35.98 & 33.05 & 30.62 & 32.33 & 33.72 & 34.54 \\ 
   &  &  & 16.65 & 15.00 & 35.98 & 33.22 & 30.89 & 32.40 & 33.89 & 35.00 \\ 
   &  &  & 12.95 & 14.00 & 36.12 & 33.15 & 29.35 & 32.08 & 33.91 & 34.70 \\ 
   &  &  & 9.25 & 14.00 & 35.89 & 33.50 & 31.17 & 32.80 & 34.24 & 35.13 \\ 
   &  &  & 5.55 & 14.00 & 36.07 & 33.69 & 31.71 & 33.19 & 34.34 & 35.32 \\ 
   &  &  & 1.85 & 14.00 & 36.58 & 34.00 & 31.63 & 33.39 & 34.50 & 35.39 \\ 
   & number & equalbins & 168.35 - 164.14 & 1.00 &  &  &  &  &  &  \\ 
   &  &  & 164.14 - 159.93 & 3.00 & 33.06 & 22.06 & 0.00 & 21.64 & 22.41 & 33.06 \\ 
   &  &  & 159.93 - 155.72 & 3.00 & 34.03 & 23.66 & 0.00 & 20.73 & 34.03 & 34.03 \\ 
   &  &  & 155.72 - 151.52 & 1.00 &  &  &  &  &  &  \\ 
   &  &  & 151.52 - 147.31 & 2.00 & 31.63 & 0.00 & 0.00 & 0.00 & 31.63 & 31.63 \\ 
   &  &  & 147.31 - 143.1 & 2.00 & 35.32 & 0.00 & 0.00 & 0.00 & 35.32 & 35.32 \\ 
   &  &  & 143.1 - 138.89 & 3.00 & 34.13 & 23.72 & 0.00 & 20.73 & 23.81 & 34.13 \\ 
   &  &  & 138.89 - 134.68 & 2.00 & 32.22 & 32.22 & 0.00 & 0.00 & 32.22 & 32.22 \\ 
   &  &  & 134.68 - 130.47 & 2.00 & 32.26 & 32.26 & 0.00 & 0.00 & 32.26 & 32.26 \\ 
   &  &  & 130.47 - 126.26 & 3.00 & 35.14 & 24.25 & 0.00 & 21.76 & 24.26 & 35.14 \\ 
   &  &  & 126.26 - 122.05 & 5.00 & 35.22 & 28.64 & 20.61 & 27.04 & 31.75 & 35.22 \\ 
   &  &  & 122.05 - 117.85 & 3.00 & 33.36 & 22.19 & 0.00 & 20.73 & 23.81 & 33.36 \\ 
   &  &  & 117.85 - 113.64 & 3.00 & 33.74 & 22.59 & 0.00 & 20.73 & 24.15 & 33.74 \\ 
   &  &  & 113.64 - 109.43 & 7.00 & 34.82 & 29.89 & 23.81 & 27.97 & 31.36 & 33.39 \\ 
   &  &  & 109.43 - 105.22 & 4.00 & 34.21 & 28.17 & 7.39 & 22.63 & 29.03 & 34.21 \\ 
   &  &  & 105.22 - 101.01 & 5.00 & 34.35 & 27.97 & 16.66 & 24.60 & 31.13 & 34.35 \\ 
   &  &  & 101.01 - 96.8 & 4.00 & 35.76 & 29.63 & 17.61 & 23.65 & 30.14 & 35.76 \\ 
   &  &  & 96.8 - 92.59 & 7.00 & 35.17 & 30.53 & 23.13 & 28.80 & 31.97 & 33.68 \\ 
   &  &  & 92.59 - 88.38 & 19.00 & 34.73 & 32.97 & 31.40 & 32.53 & 33.27 & 33.88 \\ 
   &  &  & 88.38 - 84.18 & 6.00 & 32.97 & 27.65 & 19.36 & 25.38 & 29.52 & 31.60 \\ 
   &  &  & 84.18 - 79.97 & 11.00 & 35.45 & 32.53 & 28.53 & 31.74 & 33.12 & 34.49 \\ 
   &  &  & 79.97 - 75.76 & 15.00 & 35.44 & 33.27 & 31.18 & 32.64 & 33.79 & 34.54 \\ 
   &  &  & 75.76 - 71.55 & 15.00 & 35.12 & 32.79 & 30.71 & 32.15 & 33.27 & 34.23 \\ 
   &  &  & 71.55 - 67.34 & 13.00 & 35.16 & 32.70 & 29.61 & 32.02 & 33.22 & 34.43 \\ 
   &  &  & 67.34 - 63.13 & 17.00 & 35.34 & 33.28 & 31.51 & 32.71 & 33.75 & 34.38 \\ 
   &  &  & 63.13 - 58.92 & 15.00 & 36.19 & 34.04 & 31.40 & 33.26 & 34.52 & 35.12 \\ 
   &  &  & 58.92 - 54.71 & 13.00 & 35.42 & 32.97 & 30.63 & 32.14 & 33.58 & 34.60 \\ 
   &  &  & 54.71 - 50.51 & 13.00 & 35.85 & 33.23 & 31.34 & 32.72 & 33.89 & 34.74 \\ 
   &  &  & 50.51 - 46.3 & 8.00 & 35.90 & 31.60 & 27.06 & 30.38 & 32.99 & 34.73 \\ 
   &  &  & 46.3 - 42.09 & 2.00 & 36.14 & 36.14 & 0.00 & 0.00 & 36.14 & 36.14 \\ 
   &  &  & 42.09 - 37.88 & 1.00 &  &  &  &  &  &  \\ 
   &  &  & 37.88 - 33.67 & 7.00 & 36.70 & 31.82 & 24.74 & 29.85 & 33.22 & 34.96 \\ 
   &  &  & 33.67 - 29.46 & 2.00 & 37.46 & 0.00 & 0.00 & 0.00 & 37.46 & 37.46 \\ 
   &  &  & 29.46 - 25.25 & 1.00 &  &  &  &  &  &  \\ 
   &  &  & 25.25 - 21.04 & 0.00 &  &  &  &  &  &  \\ 
   &  &  & 21.04 - 16.84 & 0.00 &  &  &  &  &  &  \\ 
   &  &  & 16.84 - 12.63 & 1.00 &  &  &  &  &  &  \\ 
   &  &  & 12.63 - 8.42 & 0.00 &  &  &  &  &  &  \\ 
   &  &  & 8.42 - 4.21 & 0.00 &  &  &  &  &  &  \\ 
   &  &  & 4.21 - 0 & 14.00 & 36.58 & 34.15 & 32.13 & 33.49 & 34.59 & 35.29 \\ 
   &  & acctran & 168.35 & 3.00 & 33.67 & 23.39 & 0.00 & 21.47 & 33.67 & 33.67 \\ 
   &  &  & 164.03 & 4.00 & 34.04 & 28.24 & 16.23 & 23.88 & 28.82 & 34.04 \\ 
   &  &  & 159.72 & 5.00 & 34.88 & 28.54 & 18.66 & 26.26 & 31.12 & 34.88 \\ 
   &  &  & 155.4 & 7.00 & 34.72 & 30.32 & 23.50 & 28.56 & 31.96 & 33.52 \\ 
   &  &  & 151.08 & 6.00 & 34.31 & 29.58 & 20.54 & 27.53 & 31.47 & 32.51 \\ 
   &  &  & 146.77 & 8.00 & 34.58 & 30.26 & 25.42 & 29.04 & 31.47 & 32.68 \\ 
   &  &  & 142.45 & 9.00 & 34.48 & 31.11 & 25.71 & 29.84 & 32.24 & 33.14 \\ 
   &  &  & 138.13 & 10.00 & 34.25 & 31.10 & 27.53 & 30.06 & 32.12 & 33.39 \\ 
   &  &  & 133.82 & 13.00 & 35.04 & 32.62 & 30.03 & 31.97 & 33.23 & 33.81 \\ 
   &  &  & 129.5 & 14.00 & 35.01 & 32.61 & 30.59 & 32.04 & 33.11 & 33.94 \\ 
   &  &  & 125.18 & 15.00 & 34.91 & 32.69 & 31.13 & 32.03 & 33.19 & 33.87 \\ 
   &  &  & 120.87 & 15.00 & 34.76 & 32.65 & 30.62 & 32.05 & 33.18 & 33.77 \\ 
   &  &  & 116.55 & 15.00 & 34.93 & 32.97 & 30.73 & 32.20 & 33.43 & 34.11 \\ 
   &  &  & 112.23 & 17.00 & 34.89 & 32.85 & 31.12 & 32.39 & 33.31 & 34.00 \\ 
   &  &  & 107.92 & 20.00 & 35.07 & 33.32 & 32.00 & 32.98 & 33.68 & 34.39 \\ 
   &  &  & 103.6 & 22.00 & 35.04 & 33.51 & 32.19 & 33.00 & 33.79 & 34.35 \\ 
   &  &  & 99.28 & 26.00 & 35.19 & 33.93 & 32.69 & 33.55 & 34.14 & 34.52 \\ 
   &  &  & 94.97 & 26.00 & 35.35 & 34.06 & 33.31 & 33.76 & 34.28 & 34.68 \\ 
   &  &  & 90.65 & 30.00 & 35.44 & 34.34 & 33.42 & 34.09 & 34.56 & 34.87 \\ 
   &  &  & 86.33 & 29.00 & 35.40 & 34.16 & 33.35 & 33.74 & 34.41 & 34.75 \\ 
   &  &  & 82.02 & 32.00 & 35.07 & 34.00 & 33.20 & 33.75 & 34.18 & 34.55 \\ 
   &  &  & 77.7 & 35.00 & 35.30 & 34.26 & 33.48 & 34.04 & 34.56 & 35.03 \\ 
   &  &  & 73.38 & 39.00 & 35.51 & 34.62 & 33.84 & 34.44 & 34.85 & 35.08 \\ 
   &  &  & 69.07 & 44.00 & 35.71 & 34.91 & 34.24 & 34.71 & 35.04 & 35.38 \\ 
   &  &  & 64.75 & 40.00 & 35.74 & 34.90 & 34.20 & 34.60 & 35.07 & 35.44 \\ 
   &  &  & 60.43 & 36.00 & 35.82 & 34.90 & 34.17 & 34.65 & 35.07 & 35.36 \\ 
   &  &  & 56.12 & 38.00 & 36.26 & 35.33 & 34.72 & 35.15 & 35.52 & 35.83 \\ 
   &  &  & 51.8 & 28.00 & 36.24 & 34.98 & 33.99 & 34.69 & 35.22 & 35.68 \\ 
   &  &  & 47.48 & 26.00 & 36.51 & 35.12 & 34.29 & 34.84 & 35.38 & 35.81 \\ 
   &  &  & 43.17 & 23.00 & 36.72 & 35.19 & 34.14 & 34.87 & 35.50 & 35.91 \\ 
   &  &  & 38.85 & 24.00 & 36.82 & 35.39 & 33.87 & 35.04 & 35.59 & 35.85 \\ 
   &  &  & 34.53 & 19.00 & 36.75 & 34.82 & 33.25 & 34.44 & 35.24 & 35.72 \\ 
   &  &  & 30.22 & 16.00 & 36.71 & 34.49 & 32.79 & 33.89 & 35.07 & 35.73 \\ 
   &  &  & 25.9 & 15.00 & 36.64 & 34.30 & 32.16 & 33.83 & 34.82 & 35.37 \\ 
   &  &  & 21.58 & 15.00 & 36.64 & 34.44 & 32.33 & 33.74 & 34.90 & 35.68 \\ 
   &  &  & 17.27 & 15.00 & 36.64 & 34.24 & 32.52 & 33.75 & 34.82 & 35.26 \\ 
   &  &  & 12.95 & 14.00 & 36.58 & 33.85 & 32.01 & 33.33 & 34.57 & 35.50 \\ 
   &  &  & 8.63 & 14.00 & 36.58 & 34.19 & 31.51 & 33.64 & 34.62 & 35.47 \\ 
   &  &  & 4.32 & 14.00 & 36.58 & 34.13 & 31.90 & 33.44 & 34.65 & 35.42 \\ 
   &  &  & 0 & 14.00 & 36.58 & 33.95 & 31.15 & 33.23 & 34.61 & 35.47 \\ 
   &  & deltran & 168.35 & 2.00 & 31.10 & 15.55 & 0.00 & 0.00 & 31.10 & 31.10 \\ 
   &  &  & 164.03 & 3.00 & 31.47 & 21.09 & 0.00 & 20.73 & 21.11 & 31.47 \\ 
   &  &  & 159.72 & 3.00 & 31.88 & 21.39 & 0.00 & 20.73 & 21.64 & 31.88 \\ 
   &  &  & 155.4 & 4.00 & 33.49 & 26.88 & 15.55 & 21.64 & 28.34 & 33.49 \\ 
   &  &  & 151.08 & 3.00 & 32.06 & 21.64 & 0.00 & 20.73 & 21.75 & 32.06 \\ 
   &  &  & 146.77 & 4.00 & 32.16 & 26.36 & 15.81 & 21.09 & 26.91 & 32.16 \\ 
   &  &  & 142.45 & 6.00 & 32.85 & 28.34 & 19.60 & 25.74 & 29.04 & 30.96 \\ 
   &  &  & 138.13 & 7.00 & 32.35 & 28.05 & 21.72 & 26.61 & 29.41 & 30.99 \\ 
   &  &  & 133.82 & 9.00 & 32.43 & 28.78 & 24.96 & 27.87 & 29.70 & 30.76 \\ 
   &  &  & 129.5 & 10.00 & 32.49 & 29.60 & 25.23 & 28.61 & 30.43 & 31.33 \\ 
   &  &  & 125.18 & 11.00 & 32.47 & 29.51 & 26.53 & 28.74 & 30.31 & 31.30 \\ 
   &  &  & 120.87 & 11.00 & 33.05 & 30.28 & 27.19 & 29.18 & 30.87 & 32.03 \\ 
   &  &  & 116.55 & 12.00 & 33.05 & 30.24 & 28.12 & 29.37 & 31.15 & 32.16 \\ 
   &  &  & 112.23 & 13.00 & 33.09 & 30.95 & 29.38 & 30.29 & 31.42 & 32.30 \\ 
   &  &  & 107.92 & 15.00 & 33.13 & 31.05 & 29.35 & 30.60 & 31.45 & 32.10 \\ 
   &  &  & 103.6 & 17.00 & 33.18 & 31.31 & 29.60 & 30.83 & 31.67 & 32.23 \\ 
   &  &  & 99.28 & 20.00 & 33.26 & 31.67 & 30.55 & 31.34 & 31.98 & 32.37 \\ 
   &  &  & 94.97 & 20.00 & 33.59 & 32.00 & 30.52 & 31.54 & 32.46 & 33.08 \\ 
   &  &  & 90.65 & 22.00 & 33.57 & 32.18 & 31.20 & 31.89 & 32.37 & 32.85 \\ 
   &  &  & 86.33 & 17.00 & 33.89 & 32.02 & 30.15 & 31.49 & 32.49 & 33.15 \\ 
   &  &  & 82.02 & 22.00 & 34.18 & 32.70 & 31.50 & 32.38 & 33.04 & 33.44 \\ 
   &  &  & 77.7 & 25.00 & 34.18 & 32.73 & 31.79 & 32.37 & 33.17 & 33.60 \\ 
   &  &  & 73.38 & 27.00 & 34.00 & 32.78 & 31.88 & 32.45 & 33.18 & 33.50 \\ 
   &  &  & 69.07 & 30.00 & 34.15 & 33.08 & 32.40 & 32.76 & 33.37 & 33.77 \\ 
   &  &  & 64.75 & 25.00 & 33.56 & 32.31 & 30.95 & 31.89 & 32.61 & 33.11 \\ 
   &  &  & 60.43 & 25.00 & 33.97 & 32.72 & 31.24 & 32.40 & 32.94 & 33.33 \\ 
   &  &  & 56.12 & 25.00 & 33.91 & 32.56 & 31.52 & 32.29 & 32.76 & 33.16 \\ 
   &  &  & 51.8 & 20.00 & 33.51 & 31.93 & 30.60 & 31.55 & 32.31 & 32.83 \\ 
   &  &  & 47.48 & 19.00 & 33.65 & 31.99 & 30.12 & 31.58 & 32.32 & 32.79 \\ 
   &  &  & 43.17 & 15.00 & 33.30 & 31.26 & 29.83 & 30.69 & 31.66 & 32.44 \\ 
   &  &  & 38.85 & 16.00 & 33.30 & 31.31 & 29.94 & 30.80 & 31.68 & 32.37 \\ 
   &  &  & 34.53 & 13.00 & 33.02 & 30.54 & 27.98 & 29.71 & 31.05 & 31.94 \\ 
   &  &  & 30.22 & 11.00 & 32.86 & 30.32 & 26.76 & 29.44 & 30.75 & 31.62 \\ 
   &  &  & 25.9 & 10.00 & 32.75 & 29.81 & 26.27 & 28.87 & 30.49 & 31.51 \\ 
   &  &  & 21.58 & 10.00 & 32.75 & 29.66 & 26.18 & 28.89 & 30.55 & 31.50 \\ 
   &  &  & 17.27 & 10.00 & 32.75 & 29.80 & 25.78 & 28.83 & 30.51 & 31.35 \\ 
   &  &  & 12.95 & 9.00 & 32.49 & 28.94 & 24.75 & 28.34 & 29.93 & 30.95 \\ 
   &  &  & 8.63 & 9.00 & 32.49 & 29.64 & 24.34 & 28.64 & 30.60 & 31.63 \\ 
   &  &  & 4.32 & 9.00 & 32.49 & 28.95 & 24.20 & 27.93 & 30.25 & 31.59 \\ 
   &  &  & 0 & 14.00 & 36.58 & 34.32 & 31.15 & 33.34 & 34.72 & 35.52 \\ 
   &  & random & 168.35 & 2.00 & 33.74 & 0.00 & 0.00 & 0.00 & 33.74 & 33.74 \\ 
   &  &  & 164.03 & 4.00 & 33.80 & 27.83 & 16.29 & 22.06 & 28.66 & 33.80 \\ 
   &  &  & 159.72 & 4.00 & 33.49 & 27.98 & 0.00 & 19.99 & 28.48 & 33.49 \\ 
   &  &  & 155.4 & 6.00 & 34.64 & 29.69 & 19.61 & 27.42 & 31.62 & 32.94 \\ 
   &  &  & 151.08 & 6.00 & 34.12 & 29.15 & 21.19 & 26.77 & 30.64 & 32.59 \\ 
   &  &  & 146.77 & 7.00 & 33.08 & 28.92 & 21.41 & 27.07 & 30.03 & 32.45 \\ 
   &  &  & 142.45 & 9.00 & 33.43 & 30.17 & 25.94 & 28.98 & 31.13 & 32.54 \\ 
   &  &  & 138.13 & 7.00 & 32.82 & 28.54 & 17.75 & 27.03 & 29.97 & 31.54 \\ 
   &  &  & 133.82 & 11.00 & 33.84 & 30.86 & 28.43 & 30.06 & 31.92 & 32.61 \\ 
   &  &  & 129.5 & 13.00 & 33.87 & 31.11 & 28.22 & 30.56 & 31.91 & 32.76 \\ 
   &  &  & 125.18 & 15.00 & 34.23 & 32.12 & 29.46 & 31.63 & 32.57 & 33.35 \\ 
   &  &  & 120.87 & 13.00 & 34.21 & 31.90 & 29.17 & 31.17 & 32.46 & 33.21 \\ 
   &  &  & 116.55 & 15.00 & 33.90 & 31.86 & 29.66 & 31.12 & 32.36 & 32.86 \\ 
   &  &  & 112.23 & 16.00 & 33.46 & 31.45 & 29.68 & 31.02 & 31.90 & 32.55 \\ 
   &  &  & 107.92 & 19.00 & 33.94 & 32.22 & 30.34 & 31.85 & 32.68 & 33.26 \\ 
   &  &  & 103.6 & 21.00 & 34.25 & 32.65 & 31.46 & 32.25 & 33.01 & 33.58 \\ 
   &  &  & 99.28 & 24.00 & 34.54 & 33.16 & 32.00 & 32.89 & 33.45 & 33.99 \\ 
   &  &  & 94.97 & 26.00 & 34.72 & 33.42 & 32.17 & 33.06 & 33.70 & 34.11 \\ 
   &  &  & 90.65 & 28.00 & 35.11 & 33.90 & 33.04 & 33.58 & 34.12 & 34.54 \\ 
   &  &  & 86.33 & 24.00 & 34.39 & 32.99 & 31.78 & 32.75 & 33.39 & 33.84 \\ 
   &  &  & 82.02 & 31.00 & 34.61 & 33.47 & 32.74 & 33.26 & 33.72 & 34.13 \\ 
   &  &  & 77.7 & 32.00 & 34.85 & 33.78 & 33.01 & 33.51 & 34.02 & 34.36 \\ 
   &  &  & 73.38 & 36.00 & 34.97 & 34.04 & 33.08 & 33.76 & 34.27 & 34.64 \\ 
   &  &  & 69.07 & 41.00 & 34.85 & 34.02 & 33.45 & 33.80 & 34.26 & 34.55 \\ 
   &  &  & 64.75 & 36.00 & 35.01 & 34.07 & 33.40 & 33.86 & 34.29 & 34.65 \\ 
   &  &  & 60.43 & 35.00 & 35.14 & 34.13 & 33.22 & 33.85 & 34.35 & 34.66 \\ 
   &  &  & 56.12 & 33.00 & 35.22 & 34.11 & 33.24 & 33.82 & 34.44 & 34.76 \\ 
   &  &  & 51.8 & 26.00 & 34.74 & 33.55 & 32.01 & 33.13 & 33.85 & 34.35 \\ 
   &  &  & 47.48 & 23.00 & 35.04 & 33.48 & 32.44 & 33.15 & 33.87 & 34.37 \\ 
   &  &  & 43.17 & 22.00 & 35.28 & 33.78 & 32.57 & 33.43 & 34.16 & 34.67 \\ 
   &  &  & 38.85 & 21.00 & 34.85 & 33.23 & 32.02 & 32.71 & 33.64 & 34.25 \\ 
   &  &  & 34.53 & 19.00 & 35.25 & 33.46 & 31.77 & 33.02 & 33.90 & 34.50 \\ 
   &  &  & 30.22 & 14.00 & 35.74 & 33.18 & 30.86 & 32.47 & 34.05 & 34.71 \\ 
   &  &  & 25.9 & 11.00 & 33.96 & 31.25 & 27.54 & 30.18 & 32.12 & 33.06 \\ 
   &  &  & 21.58 & 13.00 & 34.67 & 32.09 & 29.99 & 31.42 & 32.68 & 33.65 \\ 
   &  &  & 17.27 & 13.00 & 34.12 & 31.68 & 29.40 & 31.02 & 32.27 & 33.29 \\ 
   &  &  & 12.95 & 14.00 & 35.08 & 32.90 & 29.83 & 32.03 & 33.32 & 34.08 \\ 
   &  &  & 8.63 & 14.00 & 34.78 & 32.32 & 29.60 & 31.40 & 32.94 & 34.07 \\ 
   &  &  & 4.32 & 12.00 & 35.41 & 32.85 & 30.79 & 32.02 & 33.71 & 34.45 \\ 
   &  &  & 0 & 14.00 & 36.58 & 34.16 & 31.71 & 33.38 & 34.69 & 35.44 \\ 
   &  & proximity & 168.35 & 2.00 & 31.10 & 0.00 & 0.00 & 0.00 & 31.10 & 31.10 \\ 
   &  &  & 164.03 & 4.00 & 31.89 & 26.38 & 0.00 & 16.10 & 26.59 & 31.89 \\ 
   &  &  & 159.72 & 3.00 & 33.86 & 22.48 & 0.00 & 21.64 & 23.60 & 33.86 \\ 
   &  &  & 155.4 & 4.00 & 33.49 & 26.88 & 7.39 & 21.64 & 28.27 & 33.49 \\ 
   &  &  & 151.08 & 5.00 & 32.81 & 25.96 & 19.27 & 23.40 & 29.60 & 32.81 \\ 
   &  &  & 146.77 & 6.00 & 32.85 & 28.41 & 23.44 & 26.46 & 30.30 & 31.30 \\ 
   &  &  & 142.45 & 8.00 & 33.59 & 29.60 & 25.23 & 28.32 & 30.76 & 31.79 \\ 
   &  &  & 138.13 & 9.00 & 33.41 & 30.32 & 26.10 & 29.10 & 31.49 & 32.40 \\ 
   &  &  & 133.82 & 10.00 & 33.50 & 30.70 & 26.40 & 29.49 & 31.26 & 32.33 \\ 
   &  &  & 129.5 & 13.00 & 33.63 & 31.23 & 29.05 & 30.55 & 31.82 & 32.78 \\ 
   &  &  & 125.18 & 12.00 & 33.25 & 30.77 & 27.88 & 29.95 & 31.50 & 32.38 \\ 
   &  &  & 120.87 & 13.00 & 33.87 & 31.40 & 28.93 & 30.76 & 32.25 & 33.08 \\ 
   &  &  & 116.55 & 15.00 & 33.97 & 31.72 & 29.66 & 30.99 & 32.26 & 32.93 \\ 
   &  &  & 112.23 & 14.00 & 34.08 & 31.92 & 29.71 & 31.17 & 32.34 & 33.15 \\ 
   &  &  & 107.92 & 17.00 & 34.19 & 32.31 & 30.42 & 31.80 & 32.68 & 33.26 \\ 
   &  &  & 103.6 & 21.00 & 34.39 & 32.89 & 31.68 & 32.51 & 33.23 & 33.68 \\ 
   &  &  & 99.28 & 23.00 & 34.61 & 33.14 & 31.83 & 32.74 & 33.46 & 33.93 \\ 
   &  &  & 94.97 & 25.00 & 34.76 & 33.30 & 32.29 & 33.07 & 33.69 & 34.29 \\ 
   &  &  & 90.65 & 25.00 & 34.98 & 33.62 & 32.58 & 33.31 & 33.85 & 34.20 \\ 
   &  &  & 86.33 & 20.00 & 34.77 & 33.04 & 31.74 & 32.56 & 33.40 & 33.96 \\ 
   &  &  & 82.02 & 28.00 & 34.77 & 33.58 & 32.58 & 33.35 & 33.86 & 34.19 \\ 
   &  &  & 77.7 & 30.00 & 34.64 & 33.52 & 32.64 & 33.15 & 33.80 & 34.18 \\ 
   &  &  & 73.38 & 30.00 & 34.60 & 33.53 & 32.55 & 33.21 & 33.84 & 34.14 \\ 
   &  &  & 69.07 & 34.00 & 34.61 & 33.61 & 32.91 & 33.39 & 33.85 & 34.16 \\ 
   &  &  & 64.75 & 32.00 & 34.93 & 33.82 & 33.18 & 33.57 & 34.15 & 34.53 \\ 
   &  &  & 60.43 & 30.00 & 34.90 & 33.76 & 32.94 & 33.53 & 34.01 & 34.38 \\ 
   &  &  & 56.12 & 32.00 & 35.12 & 34.10 & 33.02 & 33.71 & 34.35 & 34.79 \\ 
   &  &  & 51.8 & 22.00 & 34.43 & 33.03 & 32.00 & 32.73 & 33.32 & 33.71 \\ 
   &  &  & 47.48 & 19.00 & 34.33 & 32.69 & 31.48 & 32.32 & 33.13 & 33.63 \\ 
   &  &  & 43.17 & 17.00 & 34.36 & 32.64 & 30.51 & 32.03 & 33.00 & 33.73 \\ 
   &  &  & 38.85 & 19.00 & 34.84 & 33.16 & 31.76 & 32.54 & 33.58 & 34.41 \\ 
   &  &  & 34.53 & 15.00 & 34.86 & 32.61 & 30.78 & 32.18 & 33.25 & 33.94 \\ 
   &  &  & 30.22 & 13.00 & 35.09 & 32.51 & 28.32 & 31.54 & 33.09 & 34.17 \\ 
   &  &  & 25.9 & 13.00 & 35.97 & 33.53 & 29.91 & 32.67 & 34.11 & 34.90 \\ 
   &  &  & 21.58 & 14.00 & 36.50 & 34.08 & 31.75 & 33.36 & 34.50 & 35.28 \\ 
   &  &  & 17.27 & 15.00 & 36.64 & 34.30 & 32.48 & 33.69 & 34.87 & 35.51 \\ 
   &  &  & 12.95 & 14.00 & 36.58 & 34.30 & 30.99 & 33.30 & 34.74 & 35.42 \\ 
   &  &  & 8.63 & 14.00 & 36.58 & 34.09 & 31.95 & 33.65 & 34.70 & 35.45 \\ 
   &  &  & 4.32 & 14.00 & 36.58 & 34.19 & 31.90 & 33.55 & 34.63 & 35.18 \\ 
   &  &  & 0 & 14.00 & 36.58 & 34.10 & 31.93 & 33.34 & 34.75 & 35.30 \\ 
   &  & equal.split & 168.35 & 3.00 & 22.50 & 21.11 & 0.00 & 20.73 & 22.50 & 33.67 \\ 
   &  &  & 164.03 & 4.00 & 33.80 & 26.47 & 7.39 & 21.00 & 27.57 & 33.00 \\ 
   &  &  & 159.72 & 5.00 & 32.44 & 26.29 & 12.99 & 23.05 & 28.96 & 32.70 \\ 
   &  &  & 155.4 & 8.00 & 31.47 & 28.40 & 16.20 & 25.75 & 30.07 & 32.38 \\ 
   &  &  & 151.08 & 6.00 & 27.97 & 26.60 & 17.35 & 23.64 & 28.71 & 31.50 \\ 
   &  &  & 146.77 & 8.00 & 34.58 & 28.75 & 22.10 & 26.71 & 30.43 & 32.30 \\ 
   &  &  & 142.45 & 10.00 & 32.43 & 29.59 & 25.92 & 28.39 & 30.56 & 32.31 \\ 
   &  &  & 138.13 & 10.00 & 31.76 & 29.98 & 25.00 & 28.92 & 30.87 & 32.89 \\ 
   &  &  & 133.82 & 13.00 & 32.29 & 31.15 & 27.36 & 30.38 & 31.80 & 32.77 \\ 
   &  &  & 129.5 & 14.00 & 33.26 & 31.16 & 27.75 & 30.32 & 32.03 & 33.10 \\ 
   &  &  & 125.18 & 15.00 & 33.27 & 31.33 & 28.46 & 30.58 & 32.09 & 33.00 \\ 
   &  &  & 120.87 & 15.00 & 33.91 & 31.44 & 28.55 & 30.75 & 32.18 & 33.22 \\ 
   &  &  & 116.55 & 16.00 & 33.91 & 31.55 & 29.49 & 30.88 & 32.18 & 33.16 \\ 
   &  &  & 112.23 & 18.00 & 34.03 & 31.83 & 29.44 & 31.08 & 32.38 & 33.08 \\ 
   &  &  & 107.92 & 20.00 & 33.78 & 32.25 & 30.49 & 31.73 & 32.87 & 33.47 \\ 
   &  &  & 103.6 & 22.00 & 33.95 & 32.64 & 31.35 & 32.15 & 33.04 & 33.59 \\ 
   &  &  & 99.28 & 26.00 & 34.32 & 32.89 & 31.45 & 32.45 & 33.25 & 33.70 \\ 
   &  &  & 94.97 & 26.00 & 34.35 & 33.03 & 31.69 & 32.73 & 33.39 & 33.80 \\ 
   &  &  & 90.65 & 30.00 & 34.41 & 33.36 & 32.32 & 33.06 & 33.66 & 34.18 \\ 
   &  &  & 86.33 & 29.00 & 34.70 & 33.17 & 32.27 & 32.84 & 33.61 & 34.18 \\ 
   &  &  & 82.02 & 34.00 & 34.42 & 33.47 & 32.67 & 33.18 & 33.70 & 34.15 \\ 
   &  &  & 77.7 & 36.00 & 34.41 & 33.60 & 32.87 & 33.28 & 33.93 & 34.40 \\ 
   &  &  & 73.38 & 39.00 & 34.83 & 33.81 & 32.88 & 33.45 & 34.05 & 34.50 \\ 
   &  &  & 69.07 & 48.00 & 34.66 & 34.14 & 33.42 & 33.86 & 34.32 & 34.67 \\ 
   &  &  & 64.75 & 40.00 & 34.80 & 33.76 & 33.20 & 33.54 & 33.99 & 34.42 \\ 
   &  &  & 60.43 & 38.00 & 35.25 & 33.89 & 32.93 & 33.56 & 34.13 & 34.62 \\ 
   &  &  & 56.12 & 38.00 & 35.51 & 33.98 & 33.11 & 33.61 & 34.34 & 34.93 \\ 
   &  &  & 51.8 & 28.00 & 34.57 & 33.50 & 31.99 & 32.99 & 33.89 & 34.63 \\ 
   &  &  & 47.48 & 26.00 & 34.64 & 33.52 & 32.02 & 33.09 & 34.04 & 34.58 \\ 
   &  &  & 43.17 & 23.00 & 34.73 & 33.14 & 31.57 & 32.63 & 33.62 & 34.76 \\ 
   &  &  & 38.85 & 24.00 & 33.96 & 33.46 & 32.04 & 33.00 & 34.00 & 34.96 \\ 
   &  &  & 34.53 & 19.00 & 35.14 & 33.10 & 30.37 & 32.50 & 33.85 & 34.74 \\ 
   &  &  & 30.22 & 16.00 & 34.97 & 32.55 & 29.38 & 31.86 & 33.16 & 34.37 \\ 
   &  &  & 25.9 & 15.00 & 35.27 & 32.17 & 29.28 & 31.50 & 33.07 & 34.16 \\ 
   &  &  & 21.58 & 15.00 & 34.19 & 32.20 & 29.95 & 31.37 & 32.92 & 33.84 \\ 
   &  &  & 17.27 & 15.00 & 33.28 & 32.37 & 29.40 & 31.64 & 33.13 & 34.45 \\ 
   &  &  & 12.95 & 14.00 & 33.32 & 31.76 & 28.02 & 30.91 & 32.59 & 34.23 \\ 
   &  &  & 8.63 & 14.00 & 33.71 & 32.14 & 29.17 & 31.14 & 33.01 & 34.37 \\ 
   &  &  & 4.32 & 14.00 & 34.80 & 31.69 & 29.24 & 31.05 & 32.77 & 34.06 \\ 
   &  &  & 0 & 14.00 & 36.58 & 33.97 & 32.31 & 33.55 & 34.59 & 35.37 \\ 
   &  & gradual.split & 168.35 & 3.00 & 31.47 & 20.73 & 0.00 & 20.73 & 21.11 & 31.47 \\ 
   &  &  & 164.03 & 4.00 & 26.47 & 26.47 & 7.39 & 21.10 & 28.05 & 33.55 \\ 
   &  &  & 159.72 & 5.00 & 34.86 & 25.55 & 12.70 & 20.99 & 27.12 & 30.51 \\ 
   &  &  & 155.4 & 8.00 & 28.70 & 26.81 & 18.48 & 24.68 & 28.32 & 30.59 \\ 
   &  &  & 151.08 & 6.00 & 30.46 & 26.23 & 17.08 & 23.90 & 28.20 & 31.53 \\ 
   &  &  & 146.77 & 8.00 & 31.50 & 28.16 & 18.61 & 25.39 & 29.23 & 31.01 \\ 
   &  &  & 142.45 & 10.00 & 32.68 & 29.24 & 25.77 & 28.14 & 30.16 & 31.51 \\ 
   &  &  & 138.13 & 10.00 & 32.42 & 29.31 & 24.85 & 28.28 & 30.56 & 31.89 \\ 
   &  &  & 133.82 & 13.00 & 31.94 & 29.85 & 26.41 & 29.06 & 30.50 & 32.31 \\ 
   &  &  & 129.5 & 14.00 & 33.15 & 30.62 & 27.93 & 29.63 & 31.51 & 33.05 \\ 
   &  &  & 125.18 & 15.00 & 32.39 & 31.18 & 29.12 & 30.56 & 31.72 & 32.84 \\ 
   &  &  & 120.87 & 15.00 & 32.58 & 31.30 & 28.87 & 30.54 & 31.88 & 32.79 \\ 
   &  &  & 116.55 & 16.00 & 34.11 & 31.65 & 29.44 & 31.13 & 32.22 & 33.19 \\ 
   &  &  & 112.23 & 18.00 & 33.23 & 31.33 & 28.84 & 30.49 & 32.00 & 32.71 \\ 
   &  &  & 107.92 & 20.00 & 33.62 & 31.87 & 30.25 & 31.51 & 32.39 & 33.07 \\ 
   &  &  & 103.6 & 22.00 & 33.99 & 32.58 & 31.04 & 32.23 & 32.88 & 33.40 \\ 
   &  &  & 99.28 & 26.00 & 33.73 & 32.98 & 31.13 & 32.46 & 33.38 & 33.90 \\ 
   &  &  & 94.97 & 26.00 & 34.77 & 33.27 & 31.98 & 32.89 & 33.63 & 34.03 \\ 
   &  &  & 90.65 & 30.00 & 34.43 & 33.25 & 32.27 & 32.92 & 33.66 & 34.14 \\ 
   &  &  & 86.33 & 29.00 & 34.06 & 32.90 & 31.63 & 32.48 & 33.26 & 33.82 \\ 
   &  &  & 82.02 & 34.00 & 34.25 & 33.40 & 32.64 & 33.16 & 33.67 & 34.07 \\ 
   &  &  & 77.7 & 36.00 & 34.18 & 33.24 & 32.39 & 32.93 & 33.52 & 33.84 \\ 
   &  &  & 73.38 & 39.00 & 34.18 & 33.33 & 32.40 & 33.04 & 33.61 & 33.99 \\ 
   &  &  & 69.07 & 48.00 & 34.73 & 33.81 & 33.24 & 33.64 & 34.06 & 34.43 \\ 
   &  &  & 64.75 & 40.00 & 34.50 & 33.65 & 32.88 & 33.39 & 33.96 & 34.36 \\ 
   &  &  & 60.43 & 38.00 & 34.53 & 33.67 & 32.76 & 33.37 & 33.94 & 34.37 \\ 
   &  &  & 56.12 & 38.00 & 34.58 & 33.63 & 32.30 & 33.24 & 33.98 & 34.51 \\ 
   &  &  & 51.8 & 28.00 & 34.23 & 32.86 & 31.61 & 32.46 & 33.29 & 34.07 \\ 
   &  &  & 47.48 & 26.00 & 33.81 & 32.60 & 31.09 & 32.06 & 33.04 & 33.86 \\ 
   &  &  & 43.17 & 23.00 & 33.43 & 32.50 & 30.93 & 31.90 & 33.10 & 33.90 \\ 
   &  &  & 38.85 & 24.00 & 34.95 & 33.22 & 31.50 & 32.58 & 33.72 & 34.39 \\ 
   &  &  & 34.53 & 19.00 & 34.85 & 32.56 & 29.87 & 31.79 & 33.32 & 34.21 \\ 
   &  &  & 30.22 & 16.00 & 34.91 & 32.34 & 29.77 & 31.48 & 33.20 & 34.09 \\ 
   &  &  & 25.9 & 15.00 & 34.44 & 32.48 & 29.89 & 31.83 & 33.27 & 34.18 \\ 
   &  &  & 21.58 & 15.00 & 34.71 & 32.71 & 29.72 & 32.04 & 33.69 & 34.55 \\ 
   &  &  & 17.27 & 15.00 & 35.15 & 33.42 & 30.38 & 32.42 & 34.09 & 34.71 \\ 
   &  &  & 12.95 & 14.00 & 35.11 & 33.16 & 29.63 & 32.30 & 33.92 & 34.81 \\ 
   &  &  & 8.63 & 14.00 & 36.11 & 33.72 & 30.00 & 32.85 & 34.22 & 35.07 \\ 
   &  &  & 4.32 & 14.00 & 36.04 & 33.91 & 31.29 & 32.93 & 34.36 & 35.29 \\ 
   &  &  & 0 & 14.00 & 36.58 & 34.12 & 31.35 & 33.38 & 34.76 & 35.66 \\ 
  Epoch & stratigraphy & equalbins & 163.5 - 145 & 10.00 & 33.32 & 30.22 & 26.86 & 29.26 & 31.18 & 32.21 \\ 
   &  &  & 145 - 100.5 & 32.00 & 34.06 & 33.00 & 32.22 & 32.66 & 33.31 & 33.69 \\ 
   &  &  & 100.5 - 66 & 82.00 & 35.04 & 34.66 & 34.30 & 34.53 & 34.76 & 34.91 \\ 
   &  &  & 66 - 56 & 35.00 & 35.75 & 34.79 & 33.93 & 34.46 & 34.98 & 35.30 \\ 
   &  &  & 56 - 33.9 & 35.00 & 35.99 & 35.02 & 34.34 & 34.82 & 35.22 & 35.45 \\ 
   &  &  & 33.9 - 23.03 & 3.00 & 37.41 & 25.16 & 0.00 & 24.48 & 37.41 & 37.41 \\ 
   &  &  & 23.03 - 5.33 & 1.00 &  &  &  &  &  &  \\ 
   &  &  & 5.33 - 2.58 & 0.00 &  &  &  &  &  &  \\ 
   &  &  & 2.58 - 0 & 14.00 & 36.58 & 34.15 & 31.42 & 33.61 & 34.65 & 35.38 \\ 
   &  & acctran & 154.25 & 7.00 & 34.72 & 30.37 & 23.05 & 28.43 & 32.10 & 33.53 \\ 
   &  &  & 122.75 & 15.00 & 34.89 & 32.65 & 30.42 & 32.09 & 33.27 & 33.94 \\ 
   &  &  & 83.25 & 30.00 & 35.12 & 33.95 & 33.15 & 33.67 & 34.24 & 34.60 \\ 
   &  &  & 61 & 36.00 & 35.82 & 34.84 & 34.10 & 34.63 & 35.07 & 35.41 \\ 
   &  &  & 44.95 & 23.00 & 36.54 & 34.96 & 34.18 & 34.64 & 35.34 & 35.73 \\ 
   &  &  & 28.46 & 16.00 & 36.71 & 34.37 & 32.51 & 33.82 & 34.87 & 35.33 \\ 
   &  &  & 14.18 & 14.00 & 36.58 & 34.19 & 31.01 & 33.58 & 34.64 & 35.40 \\ 
   &  &  & 3.96 & 14.00 & 36.58 & 34.22 & 31.65 & 33.70 & 34.66 & 35.34 \\ 
   &  &  & 1.29 & 14.00 & 36.58 & 34.15 & 31.20 & 33.56 & 34.61 & 35.44 \\ 
   &  & deltran & 154.25 & 4.00 & 33.49 & 26.65 & 7.39 & 19.99 & 28.16 & 33.49 \\ 
   &  &  & 122.75 & 11.00 & 33.40 & 30.72 & 27.39 & 30.08 & 31.57 & 32.58 \\ 
   &  &  & 83.25 & 20.00 & 34.15 & 32.57 & 30.85 & 32.07 & 32.94 & 33.33 \\ 
   &  &  & 61 & 25.00 & 33.97 & 32.62 & 31.33 & 32.20 & 32.95 & 33.49 \\ 
   &  &  & 44.95 & 16.00 & 33.22 & 31.40 & 29.64 & 30.75 & 31.66 & 32.15 \\ 
   &  &  & 28.46 & 11.00 & 32.86 & 29.92 & 26.44 & 29.20 & 30.73 & 31.78 \\ 
   &  &  & 14.18 & 9.00 & 32.49 & 28.99 & 23.85 & 27.83 & 29.98 & 30.86 \\ 
   &  &  & 3.96 & 9.00 & 32.49 & 28.99 & 25.19 & 28.02 & 30.13 & 31.20 \\ 
   &  &  & 1.29 & 9.00 & 32.49 & 29.51 & 23.72 & 28.16 & 29.79 & 30.81 \\ 
   &  & random & 154.25 & 5.00 & 33.59 & 27.97 & 16.32 & 24.46 & 30.28 & 32.45 \\ 
   &  &  & 122.75 & 13.00 & 34.76 & 32.64 & 30.32 & 31.61 & 33.10 & 33.92 \\ 
   &  &  & 83.25 & 27.00 & 34.39 & 33.19 & 32.25 & 32.87 & 33.42 & 33.87 \\ 
   &  &  & 61 & 34.00 & 35.16 & 34.23 & 33.22 & 33.94 & 34.42 & 34.87 \\ 
   &  &  & 44.95 & 22.00 & 34.90 & 33.40 & 32.17 & 33.06 & 33.81 & 34.21 \\ 
   &  &  & 28.46 & 15.00 & 35.64 & 33.46 & 30.67 & 32.59 & 34.00 & 34.80 \\ 
   &  &  & 14.18 & 10.00 & 33.73 & 30.41 & 24.67 & 28.87 & 31.25 & 32.68 \\ 
   &  &  & 3.96 & 14.00 & 35.83 & 33.32 & 31.02 & 32.64 & 34.03 & 34.93 \\ 
   &  &  & 1.29 & 13.00 & 35.10 & 32.78 & 30.46 & 31.97 & 33.32 & 34.16 \\ 
   &  & proximity & 154.25 & 4.00 & 33.49 & 26.88 & 15.55 & 21.72 & 28.37 & 33.49 \\ 
   &  &  & 122.75 & 14.00 & 33.77 & 31.63 & 29.72 & 30.96 & 32.15 & 32.93 \\ 
   &  &  & 83.25 & 26.00 & 34.74 & 33.46 & 32.25 & 33.17 & 33.76 & 34.21 \\ 
   &  &  & 61 & 30.00 & 34.64 & 33.51 & 32.20 & 33.17 & 33.75 & 34.13 \\ 
   &  &  & 44.95 & 18.00 & 34.15 & 32.29 & 30.93 & 31.81 & 32.73 & 33.63 \\ 
   &  &  & 28.46 & 13.00 & 35.50 & 32.63 & 29.57 & 31.89 & 33.63 & 34.60 \\ 
   &  &  & 14.18 & 14.00 & 36.58 & 34.02 & 31.76 & 33.21 & 34.40 & 35.73 \\ 
   &  &  & 3.96 & 14.00 & 36.58 & 33.93 & 31.98 & 33.25 & 34.49 & 35.21 \\ 
   &  &  & 1.29 & 14.00 & 36.58 & 34.12 & 31.82 & 33.48 & 34.57 & 35.35 \\ 
   &  & equal.split & 154.25 & 7.00 & 33.78 & 28.33 & 18.79 & 26.22 & 30.53 & 32.75 \\ 
   &  &  & 122.75 & 15.00 & 32.98 & 31.55 & 28.88 & 30.88 & 32.43 & 33.25 \\ 
   &  &  & 83.25 & 32.00 & 34.79 & 33.38 & 32.44 & 33.04 & 33.60 & 34.01 \\ 
   &  &  & 61 & 38.00 & 34.78 & 33.93 & 33.10 & 33.65 & 34.21 & 34.53 \\ 
   &  &  & 44.95 & 23.00 & 34.27 & 33.30 & 30.96 & 32.79 & 33.80 & 34.64 \\ 
   &  &  & 28.46 & 16.00 & 34.73 & 32.51 & 29.56 & 31.74 & 33.33 & 34.38 \\ 
   &  &  & 14.18 & 14.00 & 33.70 & 32.16 & 28.20 & 31.10 & 32.75 & 33.73 \\ 
   &  &  & 3.96 & 14.00 & 33.75 & 32.12 & 28.24 & 31.30 & 33.12 & 33.98 \\ 
   &  &  & 1.29 & 14.00 & 34.16 & 31.84 & 28.25 & 30.89 & 32.90 & 34.20 \\ 
   &  & gradual.split & 154.25 & 7.00 & 28.30 & 26.89 & 21.68 & 23.73 & 29.06 & 31.26 \\ 
   &  &  & 122.75 & 15.00 & 34.18 & 31.26 & 28.51 & 30.63 & 31.94 & 32.86 \\ 
   &  &  & 83.25 & 32.00 & 33.93 & 33.23 & 32.37 & 32.94 & 33.53 & 33.95 \\ 
   &  &  & 61 & 38.00 & 34.74 & 33.72 & 32.85 & 33.36 & 33.91 & 34.24 \\ 
   &  &  & 44.95 & 23.00 & 35.07 & 32.59 & 30.71 & 31.92 & 32.94 & 34.00 \\ 
   &  &  & 28.46 & 16.00 & 34.30 & 32.70 & 29.35 & 31.74 & 33.47 & 34.35 \\ 
   &  &  & 14.18 & 14.00 & 36.08 & 33.25 & 29.94 & 32.16 & 34.02 & 34.74 \\ 
   &  &  & 3.96 & 14.00 & 36.39 & 33.73 & 31.60 & 33.33 & 34.35 & 35.10 \\ 
   &  &  & 1.29 & 14.00 & 36.36 & 34.08 & 31.76 & 33.26 & 34.61 & 35.31 \\ 
   & duration & equalbins & 159.28 - 141.58 & 7.00 & 33.56 & 29.61 & 21.77 & 27.13 & 30.86 & 32.05 \\ 
   &  &  & 141.58 - 123.89 & 15.00 & 34.26 & 32.08 & 29.97 & 31.43 & 32.79 & 33.48 \\ 
   &  &  & 123.89 - 106.19 & 14.00 & 34.23 & 31.89 & 28.71 & 31.12 & 32.45 & 33.17 \\ 
   &  &  & 106.19 - 88.49 & 33.00 & 34.66 & 33.61 & 33.08 & 33.45 & 33.83 & 34.19 \\ 
   &  &  & 88.49 - 70.79 & 45.00 & 34.89 & 34.15 & 33.43 & 33.92 & 34.35 & 34.68 \\ 
   &  &  & 70.79 - 53.09 & 52.00 & 35.61 & 34.94 & 34.41 & 34.79 & 35.08 & 35.38 \\ 
   &  &  & 53.09 - 35.4 & 23.00 & 35.99 & 34.56 & 33.64 & 34.28 & 34.87 & 35.28 \\ 
   &  &  & 35.4 - 17.7 & 4.00 & 37.73 & 31.29 & 8.83 & 25.36 & 31.52 & 37.73 \\ 
   &  &  & 17.7 - 0 & 15.00 & 36.64 & 34.54 & 32.28 & 33.75 & 34.90 & 35.67 \\ 
   &  & acctran & 150.43 & 6.00 & 34.31 & 29.08 & 20.70 & 26.66 & 30.62 & 32.54 \\ 
   &  &  & 132.74 & 14.00 & 35.01 & 32.84 & 30.54 & 32.14 & 33.26 & 34.10 \\ 
   &  &  & 115.04 & 15.00 & 34.93 & 32.60 & 31.07 & 32.04 & 33.12 & 33.92 \\ 
   &  &  & 97.34 & 28.00 & 35.34 & 34.18 & 33.29 & 33.91 & 34.43 & 34.67 \\ 
   &  &  & 79.64 & 36.00 & 35.28 & 34.37 & 33.56 & 34.09 & 34.56 & 34.90 \\ 
   &  &  & 61.94 & 37.00 & 35.83 & 34.91 & 34.24 & 34.71 & 35.10 & 35.42 \\ 
   &  &  & 44.24 & 23.00 & 36.54 & 35.07 & 33.85 & 34.69 & 35.33 & 35.68 \\ 
   &  &  & 26.55 & 15.00 & 36.64 & 34.34 & 32.19 & 33.55 & 34.81 & 35.58 \\ 
   &  &  & 8.85 & 14.00 & 36.58 & 34.00 & 31.94 & 33.29 & 34.61 & 35.40 \\ 
   &  & deltran & 150.43 & 3.00 & 32.06 & 21.64 & 0.00 & 20.73 & 32.06 & 32.06 \\ 
   &  &  & 132.74 & 10.00 & 32.49 & 29.59 & 26.09 & 28.52 & 30.35 & 31.17 \\ 
   &  &  & 115.04 & 12.00 & 33.05 & 30.25 & 27.74 & 29.43 & 31.05 & 32.15 \\ 
   &  &  & 97.34 & 21.00 & 33.55 & 31.91 & 30.74 & 31.57 & 32.34 & 32.89 \\ 
   &  &  & 79.64 & 26.00 & 34.24 & 33.03 & 32.00 & 32.68 & 33.22 & 33.69 \\ 
   &  &  & 61.94 & 23.00 & 33.79 & 32.41 & 31.15 & 31.90 & 32.70 & 33.31 \\ 
   &  &  & 44.24 & 16.00 & 33.22 & 31.34 & 29.61 & 30.75 & 31.70 & 32.20 \\ 
   &  &  & 26.55 & 10.00 & 32.75 & 29.82 & 26.62 & 28.78 & 30.53 & 31.45 \\ 
   &  &  & 8.85 & 9.00 & 32.49 & 29.07 & 24.15 & 27.98 & 29.99 & 31.18 \\ 
   &  & random & 150.43 & 6.00 & 32.63 & 28.12 & 19.50 & 25.94 & 29.03 & 30.86 \\ 
   &  &  & 132.74 & 11.00 & 33.23 & 30.31 & 26.13 & 29.48 & 31.14 & 32.33 \\ 
   &  &  & 115.04 & 14.00 & 34.00 & 31.71 & 29.10 & 30.98 & 32.24 & 32.91 \\ 
   &  &  & 97.34 & 25.00 & 34.78 & 33.47 & 32.63 & 33.17 & 33.76 & 34.32 \\ 
   &  &  & 79.64 & 34.00 & 35.14 & 34.03 & 33.43 & 33.76 & 34.30 & 34.89 \\ 
   &  &  & 61.94 & 33.00 & 34.76 & 33.77 & 32.95 & 33.51 & 33.99 & 34.30 \\ 
   &  &  & 44.24 & 21.00 & 35.27 & 33.74 & 32.03 & 33.24 & 34.07 & 34.61 \\ 
   &  &  & 26.55 & 14.00 & 35.69 & 33.34 & 30.83 & 32.66 & 33.92 & 34.72 \\ 
   &  &  & 8.85 & 11.00 & 35.02 & 32.07 & 29.09 & 31.20 & 32.64 & 33.69 \\ 
   &  & proximity & 150.43 & 5.00 & 32.81 & 28.04 & 16.10 & 25.31 & 29.69 & 30.07 \\ 
   &  &  & 132.74 & 10.00 & 33.50 & 30.23 & 25.63 & 28.98 & 31.13 & 32.19 \\ 
   &  &  & 115.04 & 15.00 & 34.04 & 31.89 & 29.73 & 31.43 & 32.42 & 33.30 \\ 
   &  &  & 97.34 & 24.00 & 34.74 & 33.42 & 32.56 & 33.13 & 33.60 & 34.13 \\ 
   &  &  & 79.64 & 31.00 & 34.79 & 33.72 & 32.87 & 33.45 & 33.92 & 34.44 \\ 
   &  &  & 61.94 & 31.00 & 34.80 & 33.69 & 32.95 & 33.42 & 34.04 & 34.41 \\ 
   &  &  & 44.24 & 18.00 & 34.15 & 32.20 & 30.45 & 31.79 & 32.72 & 33.33 \\ 
   &  &  & 26.55 & 13.00 & 35.97 & 33.53 & 31.02 & 33.07 & 34.00 & 34.78 \\ 
   &  &  & 8.85 & 14.00 & 36.58 & 34.11 & 32.10 & 33.46 & 34.64 & 35.31 \\ 
   &  & equal.split & 150.43 & 6.00 & 33.48 & 27.31 & 13.73 & 25.16 & 29.04 & 32.54 \\ 
   &  &  & 132.74 & 14.00 & 33.74 & 31.43 & 28.40 & 30.54 & 32.06 & 32.99 \\ 
   &  &  & 115.04 & 16.00 & 33.48 & 31.55 & 29.01 & 30.73 & 32.23 & 33.34 \\ 
   &  &  & 97.34 & 29.00 & 34.44 & 33.11 & 31.97 & 32.75 & 33.44 & 34.13 \\ 
   &  &  & 79.64 & 39.00 & 34.64 & 33.71 & 33.05 & 33.47 & 33.93 & 34.31 \\ 
   &  &  & 61.94 & 37.00 & 34.65 & 33.85 & 33.20 & 33.64 & 34.13 & 34.50 \\ 
   &  &  & 44.24 & 23.00 & 34.51 & 33.34 & 31.79 & 32.80 & 33.78 & 34.50 \\ 
   &  &  & 26.55 & 15.00 & 33.11 & 32.34 & 29.02 & 31.61 & 33.05 & 34.17 \\ 
   &  &  & 8.85 & 14.00 & 32.96 & 31.86 & 27.28 & 30.61 & 32.83 & 34.23 \\ 
   &  & gradual.split & 150.43 & 6.00 & 27.97 & 26.89 & 14.05 & 23.80 & 28.83 & 31.54 \\ 
   &  &  & 132.74 & 14.00 & 31.92 & 30.16 & 26.31 & 29.20 & 30.95 & 32.12 \\ 
   &  &  & 115.04 & 16.00 & 34.30 & 32.07 & 29.97 & 31.16 & 32.50 & 33.15 \\ 
   &  &  & 97.34 & 29.00 & 34.34 & 33.17 & 31.84 & 32.93 & 33.44 & 33.86 \\ 
   &  &  & 79.64 & 39.00 & 34.38 & 33.64 & 33.00 & 33.43 & 33.86 & 34.22 \\ 
   &  &  & 61.94 & 37.00 & 34.42 & 33.47 & 32.67 & 33.11 & 33.78 & 34.48 \\ 
   &  &  & 44.24 & 23.00 & 33.52 & 32.66 & 30.08 & 32.10 & 33.19 & 34.13 \\ 
   &  &  & 26.55 & 15.00 & 34.60 & 32.54 & 29.25 & 31.47 & 33.26 & 34.29 \\ 
   &  &  & 8.85 & 14.00 & 36.00 & 33.65 & 31.33 & 32.97 & 34.15 & 34.80 \\ 
   & number & equalbins & 168.35 - 149.64 & 7.00 & 33.10 & 29.03 & 23.76 & 27.30 & 30.44 & 32.17 \\ 
   &  &  & 149.64 - 130.94 & 10.00 & 32.83 & 29.80 & 26.49 & 28.58 & 30.58 & 31.59 \\ 
   &  &  & 130.94 - 112.23 & 16.00 & 34.65 & 32.65 & 30.35 & 32.09 & 33.20 & 33.94 \\ 
   &  &  & 112.23 - 93.53 & 16.00 & 34.19 & 32.04 & 30.31 & 31.59 & 32.69 & 33.41 \\ 
   &  &  & 93.53 - 74.82 & 53.00 & 34.86 & 34.24 & 33.59 & 33.98 & 34.40 & 34.58 \\ 
   &  &  & 74.82 - 56.12 & 54.00 & 35.44 & 34.81 & 34.38 & 34.62 & 34.97 & 35.22 \\ 
   &  &  & 56.12 - 37.41 & 28.00 & 35.81 & 34.63 & 33.44 & 34.33 & 34.84 & 35.19 \\ 
   &  &  & 37.41 - 18.71 & 9.00 & 37.03 & 33.60 & 29.71 & 32.63 & 34.21 & 35.90 \\ 
   &  &  & 18.71 - 0 & 15.00 & 36.64 & 34.18 & 31.95 & 33.55 & 34.76 & 35.41 \\ 
   &  & acctran & 168.35 & 3.00 & 33.67 & 22.48 & 0.00 & 21.47 & 23.39 & 33.67 \\ 
   &  &  & 147.31 & 8.00 & 34.58 & 31.00 & 26.14 & 29.53 & 31.86 & 33.52 \\ 
   &  &  & 126.26 & 15.00 & 34.91 & 32.60 & 30.65 & 31.96 & 33.04 & 33.97 \\ 
   &  &  & 105.22 & 22.00 & 35.04 & 33.58 & 32.22 & 33.17 & 33.86 & 34.24 \\ 
   &  &  & 84.18 & 32.00 & 35.26 & 34.15 & 33.48 & 33.91 & 34.39 & 34.79 \\ 
   &  &  & 63.13 & 40.00 & 35.87 & 35.02 & 34.31 & 34.79 & 35.20 & 35.60 \\ 
   &  &  & 42.09 & 23.00 & 36.72 & 35.30 & 34.12 & 34.88 & 35.48 & 35.85 \\ 
   &  &  & 21.04 & 15.00 & 36.64 & 34.28 & 32.53 & 33.78 & 34.78 & 35.59 \\ 
   &  &  & 0 & 14.00 & 36.58 & 34.18 & 31.68 & 33.60 & 34.75 & 35.39 \\ 
   &  & deltran & 168.35 & 2.00 & 31.10 & 31.10 & 0.00 & 0.00 & 31.10 & 31.10 \\ 
   &  &  & 147.31 & 4.00 & 32.16 & 26.87 & 15.81 & 21.10 & 26.91 & 32.16 \\ 
   &  &  & 126.26 & 11.00 & 32.47 & 29.95 & 27.46 & 28.95 & 30.42 & 31.43 \\ 
   &  &  & 105.22 & 17.00 & 33.18 & 31.43 & 29.84 & 31.09 & 31.83 & 32.24 \\ 
   &  &  & 84.18 & 20.00 & 33.86 & 32.22 & 30.60 & 31.77 & 32.54 & 32.96 \\ 
   &  &  & 63.13 & 24.00 & 33.70 & 32.36 & 31.26 & 31.95 & 32.63 & 32.99 \\ 
   &  &  & 42.09 & 15.00 & 33.30 & 31.24 & 28.99 & 30.68 & 31.63 & 32.27 \\ 
   &  &  & 21.04 & 10.00 & 32.75 & 29.49 & 26.74 & 28.71 & 30.48 & 31.39 \\ 
   &  &  & 0 & 14.00 & 36.58 & 34.10 & 31.77 & 33.43 & 34.76 & 35.48 \\ 
   &  & random & 168.35 & 3.00 & 31.65 & 21.11 & 0.00 & 20.73 & 24.01 & 31.65 \\ 
   &  &  & 147.31 & 6.00 & 33.96 & 28.59 & 18.69 & 25.90 & 30.09 & 32.18 \\ 
   &  &  & 126.26 & 14.00 & 34.29 & 31.84 & 30.01 & 31.31 & 32.44 & 33.29 \\ 
   &  &  & 105.22 & 20.00 & 33.76 & 32.01 & 30.88 & 31.55 & 32.40 & 32.98 \\ 
   &  &  & 84.18 & 30.00 & 34.58 & 33.47 & 32.75 & 33.22 & 33.73 & 34.15 \\ 
   &  &  & 63.13 & 35.00 & 35.11 & 34.12 & 33.23 & 33.96 & 34.39 & 34.62 \\ 
   &  &  & 42.09 & 20.00 & 35.35 & 33.67 & 32.26 & 33.18 & 34.02 & 34.68 \\ 
   &  &  & 21.04 & 12.00 & 34.95 & 32.40 & 28.77 & 31.49 & 33.00 & 33.85 \\ 
   &  &  & 0 & 14.00 & 36.58 & 34.09 & 30.97 & 33.50 & 34.65 & 35.41 \\ 
   &  & proximity & 168.35 & 2.00 & 31.10 & 31.10 & 0.00 & 0.00 & 31.10 & 31.10 \\ 
   &  &  & 147.31 & 5.00 & 32.81 & 26.66 & 15.40 & 23.36 & 29.61 & 32.81 \\ 
   &  &  & 126.26 & 12.00 & 33.25 & 30.61 & 27.76 & 29.80 & 31.16 & 32.32 \\ 
   &  &  & 105.22 & 19.00 & 34.08 & 32.31 & 30.86 & 31.89 & 32.72 & 33.22 \\ 
   &  &  & 84.18 & 25.00 & 34.81 & 33.49 & 32.72 & 33.13 & 33.82 & 34.13 \\ 
   &  &  & 63.13 & 33.00 & 34.95 & 33.81 & 32.84 & 33.61 & 34.12 & 34.45 \\ 
   &  &  & 42.09 & 17.00 & 34.36 & 32.50 & 30.78 & 31.60 & 32.92 & 33.60 \\ 
   &  &  & 21.04 & 14.00 & 36.50 & 34.12 & 31.47 & 33.34 & 34.64 & 35.37 \\ 
   &  &  & 0 & 14.00 & 36.58 & 34.20 & 31.96 & 33.42 & 34.70 & 35.42 \\ 
   &  & equal.split & 168.35 & 3.00 & 22.50 & 21.29 & 0.00 & 20.73 & 22.72 & 33.58 \\ 
   &  &  & 147.31 & 8.00 & 32.80 & 28.89 & 22.12 & 26.92 & 30.55 & 31.83 \\ 
   &  &  & 126.26 & 15.00 & 33.29 & 31.41 & 29.19 & 30.83 & 32.10 & 32.96 \\ 
   &  &  & 105.22 & 22.00 & 33.81 & 32.40 & 30.97 & 31.94 & 32.87 & 33.71 \\ 
   &  &  & 84.18 & 32.00 & 34.04 & 33.40 & 32.44 & 33.11 & 33.71 & 34.06 \\ 
   &  &  & 63.13 & 40.00 & 34.78 & 33.90 & 33.08 & 33.60 & 34.21 & 34.60 \\ 
   &  &  & 42.09 & 23.00 & 34.55 & 33.41 & 31.81 & 32.99 & 33.82 & 34.85 \\ 
   &  &  & 21.04 & 15.00 & 34.28 & 32.34 & 29.80 & 31.48 & 33.20 & 34.35 \\ 
   &  &  & 0 & 14.00 & 36.58 & 34.18 & 31.27 & 33.30 & 34.73 & 35.37 \\ 
   &  & gradual.split & 168.35 & 3.00 & 31.47 & 20.73 & 0.00 & 20.73 & 21.47 & 31.65 \\ 
   &  &  & 147.31 & 8.00 & 31.13 & 27.62 & 16.95 & 25.09 & 29.15 & 31.65 \\ 
   &  &  & 126.26 & 15.00 & 33.05 & 30.62 & 28.43 & 29.89 & 31.12 & 32.39 \\ 
   &  &  & 105.22 & 22.00 & 33.80 & 32.20 & 30.74 & 31.81 & 32.65 & 33.32 \\ 
   &  &  & 84.18 & 32.00 & 34.29 & 33.28 & 31.98 & 32.69 & 33.59 & 33.98 \\ 
   &  &  & 63.13 & 40.00 & 34.38 & 33.75 & 32.78 & 33.43 & 34.09 & 34.46 \\ 
   &  &  & 42.09 & 23.00 & 33.78 & 32.71 & 30.86 & 32.07 & 33.30 & 34.24 \\ 
   &  &  & 21.04 & 15.00 & 34.11 & 32.88 & 29.59 & 32.14 & 33.49 & 34.35 \\ 
   &  &  & 0 & 14.00 & 36.58 & 34.16 & 31.64 & 33.36 & 34.65 & 35.55 \\ 
   \hline
\hline
\caption{Disparity measurement (sum of variance) for Beck2014 dataset} 
\end{longtable}

 
%\end{landscape}

%\newpage
%\begin{landscape}
%  % latex table generated in R 3.4.3 by xtable 1.8-2 package
% Thu Dec 21 15:01:41 2017
\begin{longtable}{llllrrrrrrr}
  \hline
stratigraphy & bin\_type & model & subsets & n & obs & bs.median & 2.5\% & 25\% & 75\% & 97.5\% \\ 
  \hline
Ages & stratigraphy & equalbins & 168.3 - 166.1 & 16.00 & 7.40 & 6.98 & 6.53 & 6.85 & 7.08 & 7.25 \\ 
   &  &  & 166.1 - 163.5 & 17.00 & 7.40 & 7.02 & 6.59 & 6.92 & 7.13 & 7.32 \\ 
   &  &  & 163.5 - 157.3 & 17.00 & 7.53 & 7.10 & 6.74 & 6.95 & 7.21 & 7.31 \\ 
   &  &  & 157.3 - 152.1 & 16.00 & 7.66 & 7.19 & 6.84 & 7.04 & 7.29 & 7.45 \\ 
   &  &  & 152.1 - 145 & 16.00 & 7.53 & 7.14 & 6.47 & 7.01 & 7.21 & 7.32 \\ 
   &  &  & 145 - 139.8 & 12.00 & 7.63 & 6.98 & 6.13 & 6.73 & 7.14 & 7.37 \\ 
   &  &  & 139.8 - 132.9 & 13.00 & 7.62 & 7.10 & 6.50 & 6.87 & 7.18 & 7.41 \\ 
   &  &  & 132.9 - 129.4 & 27.00 & 7.88 & 7.59 & 7.45 & 7.54 & 7.67 & 7.74 \\ 
   &  &  & 129.4 - 125 & 47.00 & 7.99 & 7.82 & 7.71 & 7.79 & 7.85 & 7.92 \\ 
   &  &  & 125 - 113 & 46.00 & 7.93 & 7.75 & 7.64 & 7.71 & 7.79 & 7.86 \\ 
   &  &  & 113 - 100.5 & 19.00 & 7.98 & 7.58 & 7.22 & 7.46 & 7.68 & 7.82 \\ 
   &  &  & 100.5 - 93.9 & 27.00 & 8.08 & 7.78 & 7.58 & 7.71 & 7.84 & 7.91 \\ 
   &  &  & 93.9 - 89.8 & 19.00 & 8.32 & 7.88 & 7.38 & 7.79 & 7.99 & 8.13 \\ 
   &  &  & 89.8 - 86.3 & 21.00 & 8.28 & 7.90 & 7.59 & 7.81 & 7.98 & 8.07 \\ 
   &  &  & 86.3 - 83.6 & 37.00 & 8.29 & 8.08 & 7.88 & 8.03 & 8.13 & 8.22 \\ 
   &  &  & 83.6 - 72.1 & 66.00 & 8.31 & 8.19 & 8.08 & 8.15 & 8.23 & 8.29 \\ 
   &  &  & 72.1 - 66 & 46.00 & 8.30 & 8.13 & 7.98 & 8.08 & 8.17 & 8.25 \\ 
   &  & acctran & 167.2 & 14.00 & 7.65 & 7.10 & 6.63 & 6.98 & 7.25 & 7.43 \\ 
   &  &  & 164.8 & 21.00 & 7.54 & 7.21 & 6.89 & 7.11 & 7.27 & 7.37 \\ 
   &  &  & 160.4 & 26.00 & 7.63 & 7.35 & 7.09 & 7.28 & 7.41 & 7.51 \\ 
   &  &  & 154.7 & 23.00 & 7.60 & 7.30 & 7.05 & 7.19 & 7.36 & 7.47 \\ 
   &  &  & 148.55 & 26.00 & 7.60 & 7.31 & 7.15 & 7.26 & 7.37 & 7.46 \\ 
   &  &  & 142.4 & 27.00 & 7.73 & 7.44 & 7.21 & 7.37 & 7.51 & 7.59 \\ 
   &  &  & 136.35 & 36.00 & 7.84 & 7.64 & 7.51 & 7.58 & 7.67 & 7.73 \\ 
   &  &  & 131.15 & 39.00 & 7.86 & 7.66 & 7.55 & 7.62 & 7.70 & 7.77 \\ 
   &  &  & 127.2 & 50.00 & 7.98 & 7.82 & 7.72 & 7.78 & 7.86 & 7.91 \\ 
   &  &  & 119 & 32.00 & 8.00 & 7.77 & 7.56 & 7.70 & 7.83 & 7.93 \\ 
   &  &  & 106.75 & 41.00 & 8.12 & 7.93 & 7.76 & 7.86 & 7.98 & 8.09 \\ 
   &  &  & 97.2 & 43.00 & 8.25 & 8.07 & 7.92 & 8.01 & 8.12 & 8.19 \\ 
   &  &  & 91.85 & 47.00 & 8.32 & 8.14 & 8.00 & 8.09 & 8.18 & 8.27 \\ 
   &  &  & 88.05 & 49.00 & 8.33 & 8.15 & 8.02 & 8.10 & 8.20 & 8.26 \\ 
   &  &  & 84.95 & 51.00 & 8.33 & 8.17 & 8.04 & 8.12 & 8.20 & 8.26 \\ 
   &  &  & 77.85 & 46.00 & 8.29 & 8.11 & 7.97 & 8.05 & 8.15 & 8.23 \\ 
   &  &  & 69.05 & 25.00 & 8.26 & 7.96 & 7.72 & 7.89 & 8.03 & 8.13 \\ 
   &  & deltran & 167.2 & 13.00 & 7.48 & 6.97 & 6.33 & 6.71 & 7.13 & 7.33 \\ 
   &  &  & 164.8 & 21.00 & 7.29 & 6.97 & 6.64 & 6.81 & 7.03 & 7.17 \\ 
   &  &  & 160.4 & 24.00 & 7.37 & 7.09 & 6.86 & 6.99 & 7.16 & 7.25 \\ 
   &  &  & 154.7 & 17.00 & 7.28 & 6.83 & 6.52 & 6.72 & 6.96 & 7.08 \\ 
   &  &  & 148.55 & 22.00 & 7.33 & 7.01 & 6.68 & 6.93 & 7.07 & 7.15 \\ 
   &  &  & 142.4 & 23.00 & 7.33 & 7.02 & 6.67 & 6.94 & 7.07 & 7.16 \\ 
   &  &  & 136.35 & 29.00 & 7.46 & 7.19 & 7.01 & 7.13 & 7.26 & 7.31 \\ 
   &  &  & 131.15 & 30.00 & 7.46 & 7.23 & 7.04 & 7.17 & 7.27 & 7.34 \\ 
   &  &  & 127.2 & 47.00 & 7.79 & 7.62 & 7.49 & 7.58 & 7.66 & 7.73 \\ 
   &  &  & 119 & 28.00 & 7.63 & 7.37 & 7.10 & 7.31 & 7.43 & 7.52 \\ 
   &  &  & 106.75 & 35.00 & 7.66 & 7.44 & 7.25 & 7.38 & 7.48 & 7.57 \\ 
   &  &  & 97.2 & 36.00 & 7.78 & 7.58 & 7.37 & 7.52 & 7.62 & 7.70 \\ 
   &  &  & 91.85 & 40.00 & 7.86 & 7.67 & 7.48 & 7.59 & 7.72 & 7.81 \\ 
   &  &  & 88.05 & 44.00 & 7.93 & 7.75 & 7.59 & 7.69 & 7.82 & 7.89 \\ 
   &  &  & 84.95 & 43.00 & 7.99 & 7.79 & 7.64 & 7.74 & 7.86 & 7.97 \\ 
   &  &  & 77.85 & 44.00 & 8.05 & 7.87 & 7.72 & 7.81 & 7.91 & 8.01 \\ 
   &  &  & 69.05 & 25.00 & 8.26 & 7.96 & 7.69 & 7.86 & 8.04 & 8.12 \\ 
   &  & random & 167.2 & 13.00 & 7.59 & 7.03 & 6.37 & 6.83 & 7.20 & 7.33 \\ 
   &  &  & 164.8 & 21.00 & 7.45 & 7.10 & 6.78 & 7.00 & 7.20 & 7.30 \\ 
   &  &  & 160.4 & 26.00 & 7.47 & 7.21 & 6.91 & 7.12 & 7.25 & 7.36 \\ 
   &  &  & 154.7 & 21.00 & 7.49 & 7.14 & 6.82 & 7.06 & 7.21 & 7.34 \\ 
   &  &  & 148.55 & 25.00 & 7.54 & 7.25 & 6.95 & 7.18 & 7.32 & 7.42 \\ 
   &  &  & 142.4 & 26.00 & 7.52 & 7.24 & 7.06 & 7.17 & 7.30 & 7.38 \\ 
   &  &  & 136.35 & 33.00 & 7.69 & 7.47 & 7.31 & 7.41 & 7.52 & 7.59 \\ 
   &  &  & 131.15 & 35.00 & 7.70 & 7.50 & 7.32 & 7.46 & 7.53 & 7.61 \\ 
   &  &  & 127.2 & 50.00 & 7.91 & 7.74 & 7.61 & 7.70 & 7.79 & 7.85 \\ 
   &  &  & 119 & 31.00 & 7.81 & 7.55 & 7.35 & 7.49 & 7.62 & 7.70 \\ 
   &  &  & 106.75 & 38.00 & 7.93 & 7.71 & 7.50 & 7.65 & 7.78 & 7.86 \\ 
   &  &  & 97.2 & 42.00 & 8.05 & 7.87 & 7.68 & 7.83 & 7.91 & 7.99 \\ 
   &  &  & 91.85 & 45.00 & 8.08 & 7.91 & 7.78 & 7.87 & 7.96 & 8.03 \\ 
   &  &  & 88.05 & 49.00 & 8.17 & 8.02 & 7.84 & 7.95 & 8.07 & 8.15 \\ 
   &  &  & 84.95 & 48.00 & 8.16 & 8.01 & 7.85 & 7.96 & 8.05 & 8.12 \\ 
   &  &  & 77.85 & 45.00 & 8.20 & 8.03 & 7.88 & 7.97 & 8.07 & 8.15 \\ 
   &  &  & 69.05 & 25.00 & 8.26 & 7.97 & 7.72 & 7.90 & 8.03 & 8.14 \\ 
   &  & proximity & 167.2 & 13.00 & 7.48 & 6.89 & 6.37 & 6.70 & 7.07 & 7.24 \\ 
   &  &  & 164.8 & 21.00 & 7.29 & 6.95 & 6.64 & 6.82 & 7.03 & 7.14 \\ 
   &  &  & 160.4 & 24.00 & 7.44 & 7.14 & 6.79 & 7.05 & 7.22 & 7.30 \\ 
   &  &  & 154.7 & 20.00 & 7.32 & 6.98 & 6.67 & 6.86 & 7.03 & 7.15 \\ 
   &  &  & 148.55 & 25.00 & 7.42 & 7.13 & 6.91 & 7.06 & 7.20 & 7.27 \\ 
   &  &  & 142.4 & 24.00 & 7.48 & 7.19 & 6.97 & 7.12 & 7.26 & 7.37 \\ 
   &  &  & 136.35 & 32.00 & 7.57 & 7.34 & 7.17 & 7.28 & 7.39 & 7.46 \\ 
   &  &  & 131.15 & 38.00 & 7.73 & 7.53 & 7.36 & 7.49 & 7.57 & 7.65 \\ 
   &  &  & 127.2 & 48.00 & 7.84 & 7.68 & 7.58 & 7.65 & 7.72 & 7.77 \\ 
   &  &  & 119 & 29.00 & 7.65 & 7.42 & 7.24 & 7.35 & 7.47 & 7.56 \\ 
   &  &  & 106.75 & 39.00 & 7.86 & 7.68 & 7.54 & 7.62 & 7.71 & 7.76 \\ 
   &  &  & 97.2 & 40.00 & 8.05 & 7.86 & 7.68 & 7.79 & 7.92 & 8.00 \\ 
   &  &  & 91.85 & 45.00 & 8.20 & 8.02 & 7.83 & 7.97 & 8.07 & 8.16 \\ 
   &  &  & 88.05 & 44.00 & 8.26 & 8.07 & 7.91 & 8.01 & 8.13 & 8.22 \\ 
   &  &  & 84.95 & 47.00 & 8.26 & 8.09 & 7.94 & 8.04 & 8.13 & 8.19 \\ 
   &  &  & 77.85 & 46.00 & 8.27 & 8.09 & 7.95 & 8.04 & 8.14 & 8.19 \\ 
   &  &  & 69.05 & 25.00 & 8.26 & 7.96 & 7.60 & 7.88 & 8.01 & 8.14 \\ 
   &  & equal.split & 167.2 & 14.00 & 7.55 & 6.98 & 6.32 & 6.79 & 7.16 & 7.36 \\ 
   &  &  & 164.8 & 24.00 & 7.40 & 7.10 & 6.86 & 7.02 & 7.18 & 7.27 \\ 
   &  &  & 160.4 & 32.00 & 7.51 & 7.24 & 7.09 & 7.20 & 7.31 & 7.37 \\ 
   &  &  & 154.7 & 27.00 & 7.44 & 7.09 & 6.81 & 6.99 & 7.16 & 7.29 \\ 
   &  &  & 148.55 & 27.00 & 7.37 & 7.16 & 6.93 & 7.10 & 7.23 & 7.36 \\ 
   &  &  & 142.4 & 28.00 & 7.48 & 7.25 & 7.02 & 7.16 & 7.33 & 7.45 \\ 
   &  &  & 136.35 & 37.00 & 7.62 & 7.43 & 7.24 & 7.37 & 7.49 & 7.56 \\ 
   &  &  & 131.15 & 40.00 & 7.68 & 7.45 & 7.29 & 7.40 & 7.51 & 7.60 \\ 
   &  &  & 127.2 & 62.00 & 7.90 & 7.70 & 7.62 & 7.67 & 7.74 & 7.81 \\ 
   &  &  & 119 & 35.00 & 7.77 & 7.58 & 7.34 & 7.47 & 7.65 & 7.74 \\ 
   &  &  & 106.75 & 43.00 & 7.86 & 7.70 & 7.53 & 7.63 & 7.78 & 7.88 \\ 
   &  &  & 97.2 & 48.00 & 8.01 & 7.85 & 7.69 & 7.79 & 7.89 & 8.00 \\ 
   &  &  & 91.85 & 52.00 & 7.99 & 7.92 & 7.79 & 7.86 & 7.97 & 8.04 \\ 
   &  &  & 88.05 & 55.00 & 8.14 & 7.98 & 7.80 & 7.93 & 8.03 & 8.12 \\ 
   &  &  & 84.95 & 55.00 & 8.18 & 8.00 & 7.84 & 7.95 & 8.05 & 8.17 \\ 
   &  &  & 77.85 & 56.00 & 8.13 & 7.97 & 7.86 & 7.91 & 8.02 & 8.09 \\ 
   &  &  & 69.05 & 36.00 & 8.17 & 7.94 & 7.75 & 7.89 & 8.00 & 8.08 \\ 
   &  & gradual.split & 167.2 & 14.00 & 7.36 & 6.84 & 6.35 & 6.66 & 7.02 & 7.26 \\ 
   &  &  & 164.8 & 24.00 & 7.40 & 7.01 & 6.83 & 6.94 & 7.10 & 7.18 \\ 
   &  &  & 160.4 & 32.00 & 7.42 & 7.19 & 6.98 & 7.11 & 7.24 & 7.33 \\ 
   &  &  & 154.7 & 27.00 & 7.31 & 7.03 & 6.80 & 6.93 & 7.10 & 7.20 \\ 
   &  &  & 148.55 & 27.00 & 7.35 & 7.11 & 6.85 & 7.02 & 7.18 & 7.28 \\ 
   &  &  & 142.4 & 28.00 & 7.46 & 7.17 & 6.84 & 7.08 & 7.23 & 7.34 \\ 
   &  &  & 136.35 & 37.00 & 7.47 & 7.37 & 7.19 & 7.32 & 7.42 & 7.51 \\ 
   &  &  & 131.15 & 40.00 & 7.65 & 7.48 & 7.32 & 7.42 & 7.53 & 7.62 \\ 
   &  &  & 127.2 & 62.00 & 7.79 & 7.69 & 7.58 & 7.63 & 7.72 & 7.77 \\ 
   &  &  & 119 & 35.00 & 7.71 & 7.50 & 7.31 & 7.45 & 7.56 & 7.65 \\ 
   &  &  & 106.75 & 43.00 & 7.90 & 7.67 & 7.49 & 7.61 & 7.72 & 7.81 \\ 
   &  &  & 97.2 & 48.00 & 7.89 & 7.83 & 7.69 & 7.77 & 7.87 & 8.00 \\ 
   &  &  & 91.85 & 52.00 & 8.18 & 7.92 & 7.72 & 7.87 & 7.98 & 8.07 \\ 
   &  &  & 88.05 & 55.00 & 8.17 & 7.99 & 7.83 & 7.94 & 8.05 & 8.16 \\ 
   &  &  & 84.95 & 55.00 & 8.21 & 8.03 & 7.85 & 7.97 & 8.09 & 8.15 \\ 
   &  &  & 77.85 & 56.00 & 8.19 & 8.01 & 7.85 & 7.97 & 8.06 & 8.13 \\ 
   &  &  & 69.05 & 36.00 & 8.17 & 7.94 & 7.75 & 7.88 & 7.98 & 8.07 \\ 
   & duration & equalbins & 164.8 - 159.6 & 15.00 & 7.55 & 7.07 & 6.67 & 6.94 & 7.16 & 7.30 \\ 
   &  &  & 159.6 - 154.4 & 17.00 & 7.64 & 7.21 & 6.94 & 7.12 & 7.30 & 7.50 \\ 
   &  &  & 154.4 - 149.2 & 10.00 & 7.44 & 6.81 & 5.93 & 6.63 & 6.98 & 7.17 \\ 
   &  &  & 149.2 - 144 & 13.00 & 7.59 & 7.05 & 6.63 & 6.91 & 7.18 & 7.33 \\ 
   &  &  & 144 - 138.8 & 11.00 & 7.60 & 6.93 & 6.11 & 6.75 & 7.09 & 7.34 \\ 
   &  &  & 138.8 - 133.6 & 12.00 & 7.66 & 6.98 & 6.36 & 6.78 & 7.19 & 7.38 \\ 
   &  &  & 133.6 - 128.4 & 33.00 & 7.90 & 7.67 & 7.50 & 7.60 & 7.70 & 7.78 \\ 
   &  &  & 128.4 - 123.2 & 45.00 & 8.01 & 7.83 & 7.68 & 7.79 & 7.86 & 7.91 \\ 
   &  &  & 123.2 - 118 & 24.00 & 7.99 & 7.68 & 7.41 & 7.57 & 7.75 & 7.84 \\ 
   &  &  & 118 - 112.8 & 13.00 & 7.90 & 7.23 & 6.44 & 7.06 & 7.44 & 7.63 \\ 
   &  &  & 112.8 - 107.6 & 12.00 & 8.01 & 7.40 & 6.80 & 7.18 & 7.58 & 7.77 \\ 
   &  &  & 107.6 - 102.4 & 13.00 & 8.00 & 7.46 & 6.66 & 7.26 & 7.56 & 7.74 \\ 
   &  &  & 102.4 - 97.2 & 21.00 & 7.99 & 7.61 & 7.40 & 7.56 & 7.69 & 7.86 \\ 
   &  &  & 97.2 - 92 & 18.00 & 8.07 & 7.65 & 7.30 & 7.56 & 7.75 & 7.86 \\ 
   &  &  & 92 - 86.8 & 19.00 & 8.33 & 7.93 & 7.57 & 7.80 & 7.99 & 8.16 \\ 
   &  &  & 86.8 - 81.6 & 37.00 & 8.29 & 8.05 & 7.90 & 8.01 & 8.10 & 8.20 \\ 
   &  &  & 81.6 - 76.4 & 34.00 & 8.23 & 8.00 & 7.77 & 7.94 & 8.05 & 8.15 \\ 
   &  &  & 76.4 - 71.2 & 53.00 & 8.30 & 8.15 & 8.05 & 8.11 & 8.18 & 8.24 \\ 
   &  &  & 71.2 - 66 & 26.00 & 8.25 & 7.96 & 7.69 & 7.88 & 8.01 & 8.11 \\ 
   &  & acctran & 162.2 & 23.00 & 7.57 & 7.25 & 7.01 & 7.19 & 7.31 & 7.43 \\ 
   &  &  & 157 & 29.00 & 7.64 & 7.37 & 7.22 & 7.33 & 7.42 & 7.49 \\ 
   &  &  & 151.8 & 25.00 & 7.60 & 7.31 & 7.07 & 7.25 & 7.36 & 7.44 \\ 
   &  &  & 146.6 & 26.00 & 7.61 & 7.34 & 7.10 & 7.29 & 7.38 & 7.48 \\ 
   &  &  & 141.4 & 29.00 & 7.76 & 7.50 & 7.31 & 7.43 & 7.55 & 7.66 \\ 
   &  &  & 136.2 & 36.00 & 7.84 & 7.62 & 7.50 & 7.57 & 7.66 & 7.71 \\ 
   &  &  & 131 & 39.00 & 7.86 & 7.66 & 7.55 & 7.63 & 7.70 & 7.76 \\ 
   &  &  & 125.8 & 51.00 & 7.99 & 7.85 & 7.74 & 7.82 & 7.87 & 7.92 \\ 
   &  &  & 120.6 & 39.00 & 8.00 & 7.82 & 7.66 & 7.77 & 7.85 & 7.90 \\ 
   &  &  & 115.4 & 36.00 & 8.04 & 7.84 & 7.61 & 7.77 & 7.88 & 7.99 \\ 
   &  &  & 110.2 & 39.00 & 8.06 & 7.86 & 7.66 & 7.81 & 7.91 & 8.00 \\ 
   &  &  & 105 & 43.00 & 8.12 & 7.93 & 7.76 & 7.86 & 7.97 & 8.06 \\ 
   &  &  & 99.8 & 41.00 & 8.24 & 8.05 & 7.89 & 7.99 & 8.10 & 8.18 \\ 
   &  &  & 94.6 & 45.00 & 8.30 & 8.12 & 7.94 & 8.06 & 8.16 & 8.26 \\ 
   &  &  & 89.4 & 47.00 & 8.32 & 8.14 & 7.98 & 8.08 & 8.20 & 8.27 \\ 
   &  &  & 84.2 & 52.00 & 8.33 & 8.17 & 8.04 & 8.12 & 8.21 & 8.27 \\ 
   &  &  & 79 & 46.00 & 8.29 & 8.10 & 7.95 & 8.07 & 8.16 & 8.21 \\ 
   &  &  & 73.8 & 45.00 & 8.31 & 8.14 & 8.04 & 8.09 & 8.18 & 8.24 \\ 
   &  &  & 68.6 & 25.00 & 8.26 & 7.94 & 7.65 & 7.86 & 8.01 & 8.12 \\ 
   &  & deltran & 162.2 & 23.00 & 7.34 & 7.05 & 6.74 & 6.96 & 7.13 & 7.25 \\ 
   &  &  & 157 & 23.00 & 7.39 & 7.07 & 6.79 & 7.01 & 7.15 & 7.24 \\ 
   &  &  & 151.8 & 21.00 & 7.33 & 7.00 & 6.61 & 6.90 & 7.09 & 7.19 \\ 
   &  &  & 146.6 & 21.00 & 7.31 & 6.96 & 6.73 & 6.89 & 7.04 & 7.17 \\ 
   &  &  & 141.4 & 23.00 & 7.36 & 7.05 & 6.75 & 6.96 & 7.12 & 7.20 \\ 
   &  &  & 136.2 & 29.00 & 7.46 & 7.20 & 6.99 & 7.14 & 7.25 & 7.33 \\ 
   &  &  & 131 & 30.00 & 7.46 & 7.22 & 7.02 & 7.17 & 7.26 & 7.33 \\ 
   &  &  & 125.8 & 49.00 & 7.86 & 7.72 & 7.59 & 7.67 & 7.75 & 7.82 \\ 
   &  &  & 120.6 & 36.00 & 7.79 & 7.58 & 7.41 & 7.52 & 7.64 & 7.71 \\ 
   &  &  & 115.4 & 29.00 & 7.63 & 7.38 & 7.11 & 7.31 & 7.43 & 7.53 \\ 
   &  &  & 110.2 & 33.00 & 7.64 & 7.41 & 7.20 & 7.35 & 7.48 & 7.58 \\ 
   &  &  & 105 & 36.00 & 7.67 & 7.45 & 7.25 & 7.37 & 7.52 & 7.61 \\ 
   &  &  & 99.8 & 34.00 & 7.76 & 7.53 & 7.39 & 7.49 & 7.60 & 7.68 \\ 
   &  &  & 94.6 & 37.00 & 7.81 & 7.60 & 7.43 & 7.55 & 7.65 & 7.74 \\ 
   &  &  & 89.4 & 42.00 & 7.95 & 7.76 & 7.58 & 7.70 & 7.82 & 7.94 \\ 
   &  &  & 84.2 & 44.00 & 7.99 & 7.81 & 7.61 & 7.76 & 7.86 & 7.95 \\ 
   &  &  & 79 & 44.00 & 8.04 & 7.86 & 7.71 & 7.81 & 7.90 & 7.97 \\ 
   &  &  & 73.8 & 44.00 & 8.15 & 7.97 & 7.81 & 7.93 & 8.02 & 8.09 \\ 
   &  &  & 68.6 & 25.00 & 8.26 & 7.95 & 7.63 & 7.85 & 8.01 & 8.13 \\ 
   &  & random & 162.2 & 23.00 & 7.48 & 7.19 & 6.83 & 7.12 & 7.24 & 7.31 \\ 
   &  &  & 157 & 27.00 & 7.59 & 7.32 & 7.11 & 7.24 & 7.38 & 7.46 \\ 
   &  &  & 151.8 & 23.00 & 7.52 & 7.19 & 7.02 & 7.11 & 7.26 & 7.39 \\ 
   &  &  & 146.6 & 24.00 & 7.51 & 7.22 & 7.01 & 7.13 & 7.28 & 7.39 \\ 
   &  &  & 141.4 & 29.00 & 7.60 & 7.36 & 7.11 & 7.30 & 7.42 & 7.51 \\ 
   &  &  & 136.2 & 34.00 & 7.68 & 7.46 & 7.29 & 7.40 & 7.48 & 7.55 \\ 
   &  &  & 131 & 36.00 & 7.72 & 7.51 & 7.37 & 7.46 & 7.54 & 7.64 \\ 
   &  &  & 125.8 & 51.00 & 7.91 & 7.76 & 7.61 & 7.72 & 7.80 & 7.87 \\ 
   &  &  & 120.6 & 38.00 & 7.90 & 7.70 & 7.51 & 7.63 & 7.75 & 7.82 \\ 
   &  &  & 115.4 & 34.00 & 7.92 & 7.68 & 7.48 & 7.59 & 7.75 & 7.90 \\ 
   &  &  & 110.2 & 37.00 & 7.85 & 7.64 & 7.50 & 7.58 & 7.71 & 7.79 \\ 
   &  &  & 105 & 41.00 & 7.91 & 7.73 & 7.52 & 7.65 & 7.77 & 7.85 \\ 
   &  &  & 99.8 & 38.00 & 8.02 & 7.80 & 7.63 & 7.74 & 7.86 & 7.99 \\ 
   &  &  & 94.6 & 43.00 & 8.11 & 7.93 & 7.77 & 7.88 & 7.99 & 8.08 \\ 
   &  &  & 89.4 & 46.00 & 8.10 & 7.94 & 7.78 & 7.89 & 7.98 & 8.06 \\ 
   &  &  & 84.2 & 51.00 & 8.19 & 8.04 & 7.88 & 7.98 & 8.08 & 8.16 \\ 
   &  &  & 79 & 44.00 & 8.21 & 8.01 & 7.87 & 7.96 & 8.07 & 8.14 \\ 
   &  &  & 73.8 & 45.00 & 8.23 & 8.06 & 7.91 & 8.00 & 8.08 & 8.16 \\ 
   &  &  & 68.6 & 25.00 & 8.26 & 7.94 & 7.68 & 7.87 & 8.01 & 8.12 \\ 
   &  & proximity & 162.2 & 23.00 & 7.39 & 7.06 & 6.84 & 6.99 & 7.15 & 7.24 \\ 
   &  &  & 157 & 25.00 & 7.44 & 7.18 & 6.95 & 7.11 & 7.24 & 7.33 \\ 
   &  &  & 151.8 & 21.00 & 7.38 & 7.04 & 6.83 & 6.97 & 7.11 & 7.20 \\ 
   &  &  & 146.6 & 24.00 & 7.43 & 7.14 & 6.88 & 7.06 & 7.20 & 7.30 \\ 
   &  &  & 141.4 & 24.00 & 7.48 & 7.18 & 6.90 & 7.12 & 7.25 & 7.31 \\ 
   &  &  & 136.2 & 32.00 & 7.58 & 7.35 & 7.17 & 7.28 & 7.40 & 7.47 \\ 
   &  &  & 131 & 38.00 & 7.73 & 7.53 & 7.37 & 7.48 & 7.57 & 7.64 \\ 
   &  &  & 125.8 & 50.00 & 7.86 & 7.70 & 7.57 & 7.66 & 7.74 & 7.80 \\ 
   &  &  & 120.6 & 38.00 & 7.79 & 7.58 & 7.41 & 7.51 & 7.64 & 7.72 \\ 
   &  &  & 115.4 & 31.00 & 7.68 & 7.44 & 7.22 & 7.37 & 7.50 & 7.60 \\ 
   &  &  & 110.2 & 36.00 & 7.81 & 7.61 & 7.47 & 7.56 & 7.64 & 7.73 \\ 
   &  &  & 105 & 40.00 & 7.89 & 7.70 & 7.54 & 7.64 & 7.75 & 7.80 \\ 
   &  &  & 99.8 & 36.00 & 7.95 & 7.74 & 7.58 & 7.68 & 7.77 & 7.86 \\ 
   &  &  & 94.6 & 41.00 & 8.08 & 7.90 & 7.72 & 7.83 & 7.95 & 8.04 \\ 
   &  &  & 89.4 & 44.00 & 8.21 & 8.02 & 7.84 & 7.96 & 8.08 & 8.17 \\ 
   &  &  & 84.2 & 48.00 & 8.26 & 8.09 & 7.93 & 8.05 & 8.15 & 8.23 \\ 
   &  &  & 79 & 46.00 & 8.27 & 8.10 & 7.96 & 8.06 & 8.14 & 8.20 \\ 
   &  &  & 73.8 & 44.00 & 8.31 & 8.13 & 7.97 & 8.08 & 8.19 & 8.25 \\ 
   &  &  & 68.6 & 25.00 & 8.26 & 7.95 & 7.70 & 7.89 & 8.01 & 8.12 \\ 
   &  & equal.split & 162.2 & 29.00 & 7.45 & 7.20 & 6.89 & 7.14 & 7.25 & 7.35 \\ 
   &  &  & 157 & 30.00 & 7.40 & 7.26 & 7.03 & 7.20 & 7.32 & 7.38 \\ 
   &  &  & 151.8 & 28.00 & 7.34 & 7.15 & 6.87 & 7.07 & 7.20 & 7.32 \\ 
   &  &  & 146.6 & 26.00 & 7.46 & 7.17 & 6.82 & 7.08 & 7.25 & 7.36 \\ 
   &  &  & 141.4 & 30.00 & 7.60 & 7.30 & 7.09 & 7.22 & 7.38 & 7.46 \\ 
   &  &  & 136.2 & 37.00 & 7.56 & 7.44 & 7.25 & 7.38 & 7.49 & 7.55 \\ 
   &  &  & 131 & 40.00 & 7.76 & 7.45 & 7.29 & 7.39 & 7.50 & 7.62 \\ 
   &  &  & 125.8 & 67.00 & 7.90 & 7.77 & 7.68 & 7.74 & 7.80 & 7.86 \\ 
   &  &  & 120.6 & 42.00 & 7.86 & 7.69 & 7.49 & 7.63 & 7.75 & 7.83 \\ 
   &  &  & 115.4 & 38.00 & 7.78 & 7.61 & 7.37 & 7.54 & 7.67 & 7.81 \\ 
   &  &  & 110.2 & 43.00 & 7.78 & 7.66 & 7.46 & 7.57 & 7.71 & 7.83 \\ 
   &  &  & 105 & 44.00 & 7.83 & 7.68 & 7.49 & 7.61 & 7.75 & 7.84 \\ 
   &  &  & 99.8 & 46.00 & 8.02 & 7.80 & 7.60 & 7.76 & 7.86 & 8.02 \\ 
   &  &  & 94.6 & 48.00 & 8.02 & 7.89 & 7.73 & 7.82 & 7.94 & 8.08 \\ 
   &  &  & 89.4 & 56.00 & 8.19 & 8.01 & 7.86 & 7.96 & 8.06 & 8.14 \\ 
   &  &  & 84.2 & 55.00 & 8.14 & 8.00 & 7.85 & 7.95 & 8.04 & 8.14 \\ 
   &  &  & 79 & 57.00 & 8.10 & 7.96 & 7.82 & 7.92 & 8.01 & 8.09 \\ 
   &  &  & 73.8 & 55.00 & 8.16 & 8.00 & 7.87 & 7.95 & 8.08 & 8.15 \\ 
   &  &  & 68.6 & 35.00 & 8.18 & 7.94 & 7.73 & 7.88 & 8.02 & 8.10 \\ 
   &  & gradual.split & 162.2 & 29.00 & 7.40 & 7.17 & 6.92 & 7.05 & 7.22 & 7.33 \\ 
   &  &  & 157 & 30.00 & 7.38 & 7.16 & 6.85 & 7.10 & 7.22 & 7.32 \\ 
   &  &  & 151.8 & 28.00 & 7.34 & 7.06 & 6.82 & 6.99 & 7.15 & 7.29 \\ 
   &  &  & 146.6 & 26.00 & 7.35 & 7.13 & 6.92 & 7.05 & 7.20 & 7.35 \\ 
   &  &  & 141.4 & 30.00 & 7.44 & 7.18 & 6.92 & 7.11 & 7.24 & 7.36 \\ 
   &  &  & 136.2 & 37.00 & 7.63 & 7.36 & 7.19 & 7.31 & 7.42 & 7.52 \\ 
   &  &  & 131 & 40.00 & 7.62 & 7.49 & 7.27 & 7.44 & 7.52 & 7.60 \\ 
   &  &  & 125.8 & 67.00 & 7.87 & 7.74 & 7.65 & 7.71 & 7.78 & 7.82 \\ 
   &  &  & 120.6 & 42.00 & 7.84 & 7.64 & 7.46 & 7.59 & 7.69 & 7.76 \\ 
   &  &  & 115.4 & 38.00 & 7.65 & 7.50 & 7.33 & 7.46 & 7.56 & 7.65 \\ 
   &  &  & 110.2 & 43.00 & 7.83 & 7.60 & 7.42 & 7.55 & 7.65 & 7.79 \\ 
   &  &  & 105 & 44.00 & 7.82 & 7.68 & 7.49 & 7.63 & 7.73 & 7.83 \\ 
   &  &  & 99.8 & 46.00 & 7.96 & 7.76 & 7.56 & 7.69 & 7.81 & 7.93 \\ 
   &  &  & 94.6 & 48.00 & 8.02 & 7.88 & 7.69 & 7.79 & 7.93 & 8.04 \\ 
   &  &  & 89.4 & 56.00 & 8.16 & 8.02 & 7.88 & 7.97 & 8.07 & 8.17 \\ 
   &  &  & 84.2 & 55.00 & 8.20 & 8.05 & 7.89 & 8.00 & 8.12 & 8.19 \\ 
   &  &  & 79 & 57.00 & 8.15 & 8.02 & 7.91 & 7.98 & 8.06 & 8.15 \\ 
   &  &  & 73.8 & 55.00 & 8.19 & 8.06 & 7.93 & 8.01 & 8.12 & 8.17 \\ 
   &  &  & 68.6 & 35.00 & 8.18 & 7.96 & 7.79 & 7.91 & 8.01 & 8.07 \\ 
   & number & equalbins & 166.85 - 160.92 & 27.00 & 7.37 & 7.10 & 6.89 & 7.04 & 7.16 & 7.23 \\ 
   &  &  & 160.92 - 154.98 & 16.00 & 7.65 & 7.20 & 6.68 & 7.08 & 7.31 & 7.46 \\ 
   &  &  & 154.98 - 149.05 & 12.00 & 7.43 & 6.82 & 6.09 & 6.67 & 6.97 & 7.20 \\ 
   &  &  & 149.05 - 143.12 & 14.00 & 7.57 & 7.07 & 6.60 & 6.92 & 7.19 & 7.34 \\ 
   &  &  & 143.12 - 137.19 & 13.00 & 7.63 & 7.05 & 6.52 & 6.89 & 7.20 & 7.45 \\ 
   &  &  & 137.19 - 131.26 & 9.00 & 7.63 & 6.79 & 5.55 & 6.58 & 7.03 & 7.31 \\ 
   &  &  & 131.26 - 125.32 & 50.00 & 7.95 & 7.80 & 7.67 & 7.76 & 7.84 & 7.90 \\ 
   &  &  & 125.32 - 119.39 & 40.00 & 7.98 & 7.78 & 7.62 & 7.74 & 7.81 & 7.87 \\ 
   &  &  & 119.39 - 113.46 & 15.00 & 7.85 & 7.35 & 6.87 & 7.23 & 7.48 & 7.63 \\ 
   &  &  & 113.46 - 107.53 & 12.00 & 8.01 & 7.43 & 6.79 & 7.27 & 7.58 & 7.77 \\ 
   &  &  & 107.53 - 101.59 & 13.00 & 8.00 & 7.40 & 6.62 & 7.20 & 7.52 & 7.69 \\ 
   &  &  & 101.59 - 95.66 & 26.00 & 7.99 & 7.69 & 7.42 & 7.60 & 7.76 & 7.85 \\ 
   &  &  & 95.66 - 89.73 & 19.00 & 8.32 & 7.90 & 7.45 & 7.79 & 8.01 & 8.15 \\ 
   &  &  & 89.73 - 83.8 & 23.00 & 8.24 & 7.91 & 7.54 & 7.82 & 8.01 & 8.13 \\ 
   &  &  & 83.8 - 77.86 & 37.00 & 8.29 & 8.07 & 7.89 & 8.01 & 8.13 & 8.20 \\ 
   &  &  & 77.86 - 71.93 & 53.00 & 8.30 & 8.14 & 8.04 & 8.11 & 8.17 & 8.24 \\ 
   &  &  & 71.93 - 66 & 27.00 & 8.25 & 7.96 & 7.75 & 7.89 & 8.02 & 8.12 \\ 
   &  & acctran & 166.85 & 14.00 & 7.66 & 7.17 & 6.72 & 7.01 & 7.28 & 7.41 \\ 
   &  &  & 160.55 & 26.00 & 7.63 & 7.35 & 7.13 & 7.29 & 7.42 & 7.50 \\ 
   &  &  & 154.24 & 24.00 & 7.59 & 7.30 & 7.11 & 7.23 & 7.36 & 7.45 \\ 
   &  &  & 147.94 & 25.00 & 7.59 & 7.32 & 7.09 & 7.24 & 7.38 & 7.47 \\ 
   &  &  & 141.64 & 27.00 & 7.73 & 7.45 & 7.25 & 7.37 & 7.50 & 7.58 \\ 
   &  &  & 135.33 & 38.00 & 7.86 & 7.67 & 7.51 & 7.62 & 7.70 & 7.75 \\ 
   &  &  & 129.03 & 46.00 & 7.90 & 7.73 & 7.59 & 7.70 & 7.77 & 7.84 \\ 
   &  &  & 122.73 & 38.00 & 7.99 & 7.78 & 7.58 & 7.75 & 7.83 & 7.90 \\ 
   &  &  & 116.42 & 36.00 & 8.04 & 7.80 & 7.66 & 7.76 & 7.87 & 7.97 \\ 
   &  &  & 110.12 & 39.00 & 8.06 & 7.86 & 7.69 & 7.80 & 7.91 & 7.99 \\ 
   &  &  & 103.82 & 44.00 & 8.15 & 7.97 & 7.79 & 7.92 & 8.01 & 8.12 \\ 
   &  &  & 97.52 & 42.00 & 8.25 & 8.07 & 7.92 & 8.01 & 8.11 & 8.20 \\ 
   &  &  & 91.21 & 48.00 & 8.31 & 8.15 & 8.02 & 8.11 & 8.18 & 8.25 \\ 
   &  &  & 84.91 & 51.00 & 8.33 & 8.17 & 8.03 & 8.14 & 8.21 & 8.28 \\ 
   &  &  & 78.61 & 46.00 & 8.29 & 8.11 & 7.98 & 8.06 & 8.16 & 8.24 \\ 
   &  &  & 72.3 & 44.00 & 8.29 & 8.11 & 7.98 & 8.05 & 8.15 & 8.24 \\ 
   &  &  & 66 & 1.00 &  &  &  &  &  &  \\ 
   &  & deltran & 166.85 & 14.00 & 7.44 & 6.94 & 6.36 & 6.82 & 7.07 & 7.30 \\ 
   &  &  & 160.55 & 24.00 & 7.37 & 7.09 & 6.84 & 6.98 & 7.14 & 7.27 \\ 
   &  &  & 154.24 & 18.00 & 7.29 & 6.91 & 6.51 & 6.83 & 6.98 & 7.09 \\ 
   &  &  & 147.94 & 21.00 & 7.30 & 6.96 & 6.70 & 6.87 & 7.03 & 7.10 \\ 
   &  &  & 141.64 & 23.00 & 7.33 & 7.00 & 6.77 & 6.94 & 7.08 & 7.16 \\ 
   &  &  & 135.33 & 29.00 & 7.46 & 7.20 & 7.00 & 7.15 & 7.26 & 7.32 \\ 
   &  &  & 129.03 & 41.00 & 7.72 & 7.54 & 7.39 & 7.49 & 7.59 & 7.64 \\ 
   &  &  & 122.73 & 35.00 & 7.79 & 7.60 & 7.37 & 7.53 & 7.65 & 7.74 \\ 
   &  &  & 116.42 & 29.00 & 7.63 & 7.37 & 7.12 & 7.28 & 7.45 & 7.52 \\ 
   &  &  & 110.12 & 33.00 & 7.64 & 7.42 & 7.19 & 7.36 & 7.46 & 7.54 \\ 
   &  &  & 103.82 & 37.00 & 7.71 & 7.52 & 7.34 & 7.47 & 7.56 & 7.67 \\ 
   &  &  & 97.52 & 35.00 & 7.76 & 7.54 & 7.34 & 7.47 & 7.61 & 7.69 \\ 
   &  &  & 91.21 & 41.00 & 7.85 & 7.68 & 7.52 & 7.62 & 7.72 & 7.80 \\ 
   &  &  & 84.91 & 43.00 & 7.99 & 7.82 & 7.63 & 7.76 & 7.87 & 7.94 \\ 
   &  &  & 78.61 & 44.00 & 8.04 & 7.86 & 7.73 & 7.83 & 7.91 & 7.98 \\ 
   &  &  & 72.3 & 43.00 & 8.12 & 7.94 & 7.77 & 7.89 & 7.99 & 8.03 \\ 
   &  &  & 66 & 1.00 &  &  &  &  &  &  \\ 
   &  & random & 166.85 & 14.00 & 7.51 & 7.01 & 6.37 & 6.81 & 7.14 & 7.33 \\ 
   &  &  & 160.55 & 26.00 & 7.49 & 7.19 & 6.89 & 7.12 & 7.27 & 7.39 \\ 
   &  &  & 154.24 & 23.00 & 7.47 & 7.17 & 6.85 & 7.10 & 7.22 & 7.30 \\ 
   &  &  & 147.94 & 24.00 & 7.41 & 7.07 & 6.88 & 7.01 & 7.15 & 7.25 \\ 
   &  &  & 141.64 & 26.00 & 7.57 & 7.29 & 7.10 & 7.23 & 7.35 & 7.44 \\ 
   &  &  & 135.33 & 35.00 & 7.57 & 7.37 & 7.17 & 7.31 & 7.41 & 7.46 \\ 
   &  &  & 129.03 & 44.00 & 7.80 & 7.63 & 7.47 & 7.59 & 7.67 & 7.73 \\ 
   &  &  & 122.73 & 38.00 & 7.87 & 7.70 & 7.49 & 7.64 & 7.74 & 7.80 \\ 
   &  &  & 116.42 & 35.00 & 7.88 & 7.66 & 7.48 & 7.60 & 7.74 & 7.84 \\ 
   &  &  & 110.12 & 38.00 & 7.91 & 7.70 & 7.49 & 7.63 & 7.77 & 7.85 \\ 
   &  &  & 103.82 & 43.00 & 7.96 & 7.78 & 7.59 & 7.71 & 7.84 & 7.95 \\ 
   &  &  & 97.52 & 40.00 & 7.99 & 7.79 & 7.62 & 7.75 & 7.83 & 7.90 \\ 
   &  &  & 91.21 & 46.00 & 8.11 & 7.94 & 7.80 & 7.90 & 7.99 & 8.08 \\ 
   &  &  & 84.91 & 50.00 & 8.16 & 8.01 & 7.85 & 7.97 & 8.06 & 8.13 \\ 
   &  &  & 78.61 & 46.00 & 8.18 & 8.00 & 7.84 & 7.93 & 8.04 & 8.11 \\ 
   &  &  & 72.3 & 44.00 & 8.18 & 8.01 & 7.80 & 7.94 & 8.05 & 8.12 \\ 
   &  &  & 66 & 1.00 &  &  &  &  &  &  \\ 
   &  & proximity & 166.85 & 14.00 & 7.44 & 6.99 & 6.33 & 6.82 & 7.10 & 7.28 \\ 
   &  &  & 160.55 & 24.00 & 7.44 & 7.14 & 6.95 & 7.07 & 7.25 & 7.35 \\ 
   &  &  & 154.24 & 20.00 & 7.32 & 6.95 & 6.60 & 6.85 & 7.04 & 7.14 \\ 
   &  &  & 147.94 & 24.00 & 7.42 & 7.13 & 6.89 & 7.06 & 7.20 & 7.28 \\ 
   &  &  & 141.64 & 24.00 & 7.48 & 7.20 & 6.95 & 7.13 & 7.27 & 7.34 \\ 
   &  &  & 135.33 & 34.00 & 7.58 & 7.34 & 7.16 & 7.29 & 7.38 & 7.46 \\ 
   &  &  & 129.03 & 43.00 & 7.80 & 7.62 & 7.49 & 7.58 & 7.66 & 7.73 \\ 
   &  &  & 122.73 & 38.00 & 7.79 & 7.59 & 7.41 & 7.53 & 7.64 & 7.75 \\ 
   &  &  & 116.42 & 30.00 & 7.68 & 7.44 & 7.20 & 7.38 & 7.50 & 7.59 \\ 
   &  &  & 110.12 & 36.00 & 7.81 & 7.60 & 7.44 & 7.54 & 7.64 & 7.70 \\ 
   &  &  & 103.82 & 41.00 & 7.89 & 7.71 & 7.58 & 7.67 & 7.75 & 7.82 \\ 
   &  &  & 97.52 & 39.00 & 8.07 & 7.86 & 7.67 & 7.80 & 7.91 & 7.98 \\ 
   &  &  & 91.21 & 45.00 & 8.21 & 8.03 & 7.87 & 7.97 & 8.08 & 8.15 \\ 
   &  &  & 84.91 & 47.00 & 8.26 & 8.08 & 7.96 & 8.04 & 8.15 & 8.20 \\ 
   &  &  & 78.61 & 46.00 & 8.27 & 8.08 & 7.96 & 8.04 & 8.13 & 8.22 \\ 
   &  &  & 72.3 & 44.00 & 8.29 & 8.12 & 7.97 & 8.05 & 8.17 & 8.22 \\ 
   &  &  & 66 & 1.00 &  &  &  &  &  &  \\ 
   &  & equal.split & 166.85 & 15.00 & 7.57 & 7.02 & 6.20 & 6.83 & 7.14 & 7.36 \\ 
   &  &  & 160.55 & 32.00 & 7.46 & 7.22 & 7.01 & 7.17 & 7.30 & 7.41 \\ 
   &  &  & 154.24 & 27.00 & 7.33 & 7.11 & 6.87 & 7.02 & 7.20 & 7.28 \\ 
   &  &  & 147.94 & 26.00 & 7.42 & 7.12 & 6.87 & 7.06 & 7.20 & 7.30 \\ 
   &  &  & 141.64 & 28.00 & 7.59 & 7.26 & 6.99 & 7.18 & 7.33 & 7.45 \\ 
   &  &  & 135.33 & 39.00 & 7.66 & 7.44 & 7.24 & 7.37 & 7.48 & 7.57 \\ 
   &  &  & 129.03 & 61.00 & 7.78 & 7.62 & 7.54 & 7.59 & 7.65 & 7.72 \\ 
   &  &  & 122.73 & 43.00 & 7.86 & 7.67 & 7.46 & 7.60 & 7.72 & 7.79 \\ 
   &  &  & 116.42 & 38.00 & 7.77 & 7.60 & 7.37 & 7.53 & 7.67 & 7.82 \\ 
   &  &  & 110.12 & 43.00 & 7.75 & 7.65 & 7.44 & 7.59 & 7.70 & 7.79 \\ 
   &  &  & 103.82 & 46.00 & 7.81 & 7.76 & 7.58 & 7.70 & 7.84 & 7.94 \\ 
   &  &  & 97.52 & 47.00 & 7.98 & 7.80 & 7.62 & 7.73 & 7.88 & 7.97 \\ 
   &  &  & 91.21 & 53.00 & 8.06 & 7.91 & 7.72 & 7.85 & 7.98 & 8.05 \\ 
   &  &  & 84.91 & 55.00 & 8.18 & 8.01 & 7.86 & 7.94 & 8.06 & 8.15 \\ 
   &  &  & 78.61 & 57.00 & 8.12 & 7.99 & 7.84 & 7.93 & 8.04 & 8.10 \\ 
   &  &  & 72.3 & 48.00 & 8.20 & 8.00 & 7.82 & 7.96 & 8.05 & 8.15 \\ 
   &  &  & 66 & 1.00 &  &  &  &  &  &  \\ 
   &  & gradual.split & 166.85 & 15.00 & 7.40 & 6.97 & 6.47 & 6.81 & 7.12 & 7.29 \\ 
   &  &  & 160.55 & 32.00 & 7.35 & 7.17 & 6.96 & 7.12 & 7.22 & 7.31 \\ 
   &  &  & 154.24 & 27.00 & 7.27 & 7.01 & 6.77 & 6.92 & 7.08 & 7.18 \\ 
   &  &  & 147.94 & 26.00 & 7.36 & 7.11 & 6.84 & 7.03 & 7.18 & 7.29 \\ 
   &  &  & 141.64 & 28.00 & 7.50 & 7.20 & 6.97 & 7.13 & 7.26 & 7.37 \\ 
   &  &  & 135.33 & 39.00 & 7.61 & 7.35 & 7.21 & 7.30 & 7.42 & 7.51 \\ 
   &  &  & 129.03 & 61.00 & 7.73 & 7.62 & 7.53 & 7.58 & 7.67 & 7.71 \\ 
   &  &  & 122.73 & 43.00 & 7.86 & 7.62 & 7.46 & 7.56 & 7.67 & 7.75 \\ 
   &  &  & 116.42 & 38.00 & 7.68 & 7.49 & 7.28 & 7.43 & 7.54 & 7.66 \\ 
   &  &  & 110.12 & 43.00 & 7.78 & 7.60 & 7.44 & 7.54 & 7.65 & 7.72 \\ 
   &  &  & 103.82 & 46.00 & 7.80 & 7.69 & 7.53 & 7.63 & 7.76 & 7.84 \\ 
   &  &  & 97.52 & 47.00 & 8.01 & 7.82 & 7.63 & 7.76 & 7.87 & 7.96 \\ 
   &  &  & 91.21 & 53.00 & 8.10 & 7.94 & 7.76 & 7.89 & 7.98 & 8.06 \\ 
   &  &  & 84.91 & 55.00 & 8.18 & 8.04 & 7.88 & 7.99 & 8.08 & 8.17 \\ 
   &  &  & 78.61 & 57.00 & 8.14 & 8.01 & 7.87 & 7.96 & 8.06 & 8.15 \\ 
   &  &  & 72.3 & 48.00 & 8.26 & 8.06 & 7.90 & 8.03 & 8.10 & 8.19 \\ 
   &  &  & 66 & 1.00 &  &  &  &  &  &  \\ 
  Epoch & stratigraphy & equalbins & 163.5 - 145 & 36.00 & 7.52 & 7.33 & 7.18 & 7.28 & 7.36 & 7.42 \\ 
   &  &  & 145 - 100.5 & 95.00 & 7.84 & 7.76 & 7.68 & 7.73 & 7.79 & 7.83 \\ 
   &  &  & 100.5 - 66 & 103.00 & 8.20 & 8.13 & 8.04 & 8.09 & 8.15 & 8.18 \\ 
   &  & acctran & 154.25 & 24.00 & 7.59 & 7.28 & 7.02 & 7.22 & 7.34 & 7.44 \\ 
   &  &  & 122.75 & 38.00 & 7.99 & 7.77 & 7.63 & 7.71 & 7.82 & 7.89 \\ 
   &  &  & 83.25 & 45.00 & 8.30 & 8.10 & 7.97 & 8.06 & 8.16 & 8.22 \\ 
   &  & deltran & 154.25 & 18.00 & 7.29 & 6.90 & 6.50 & 6.79 & 6.97 & 7.10 \\ 
   &  &  & 122.75 & 35.00 & 7.79 & 7.58 & 7.38 & 7.51 & 7.65 & 7.72 \\ 
   &  &  & 83.25 & 43.00 & 8.03 & 7.85 & 7.66 & 7.80 & 7.90 & 7.99 \\ 
   &  & random & 154.25 & 23.00 & 7.51 & 7.20 & 6.95 & 7.11 & 7.26 & 7.39 \\ 
   &  &  & 122.75 & 37.00 & 7.90 & 7.69 & 7.53 & 7.63 & 7.74 & 7.82 \\ 
   &  &  & 83.25 & 45.00 & 8.19 & 8.02 & 7.84 & 7.96 & 8.06 & 8.14 \\ 
   &  & proximity & 154.25 & 20.00 & 7.32 & 6.96 & 6.63 & 6.88 & 7.03 & 7.14 \\ 
   &  &  & 122.75 & 38.00 & 7.79 & 7.58 & 7.43 & 7.53 & 7.65 & 7.74 \\ 
   &  &  & 83.25 & 43.00 & 8.21 & 8.01 & 7.87 & 7.97 & 8.06 & 8.15 \\ 
   &  & equal.split & 154.25 & 27.00 & 7.33 & 7.12 & 6.78 & 7.04 & 7.19 & 7.27 \\ 
   &  &  & 122.75 & 43.00 & 7.81 & 7.66 & 7.47 & 7.61 & 7.70 & 7.79 \\ 
   &  &  & 83.25 & 61.00 & 8.14 & 7.99 & 7.83 & 7.93 & 8.02 & 8.09 \\ 
   &  & gradual.split & 154.25 & 27.00 & 7.22 & 7.02 & 6.75 & 6.92 & 7.06 & 7.17 \\ 
   &  &  & 122.75 & 43.00 & 7.81 & 7.62 & 7.44 & 7.56 & 7.66 & 7.75 \\ 
   &  &  & 83.25 & 61.00 & 8.08 & 7.99 & 7.85 & 7.95 & 8.03 & 8.09 \\ 
   & duration & equalbins & 135 - 100.5 & 78.00 & 7.89 & 7.80 & 7.71 & 7.77 & 7.82 & 7.87 \\ 
   &  &  & 100.5 - 66 & 103.00 & 8.20 & 8.12 & 8.03 & 8.09 & 8.15 & 8.19 \\ 
   &  & acctran & 117.75 & 34.00 & 8.01 & 7.76 & 7.60 & 7.71 & 7.83 & 7.92 \\ 
   &  &  & 83.25 & 45.00 & 8.30 & 8.13 & 7.97 & 8.09 & 8.18 & 8.26 \\ 
   &  & deltran & 117.75 & 28.00 & 7.64 & 7.36 & 7.16 & 7.28 & 7.43 & 7.54 \\ 
   &  &  & 83.25 & 43.00 & 8.03 & 7.85 & 7.66 & 7.79 & 7.92 & 7.98 \\ 
   &  & random & 117.75 & 33.00 & 7.91 & 7.67 & 7.46 & 7.62 & 7.73 & 7.82 \\ 
   &  &  & 83.25 & 45.00 & 8.16 & 7.99 & 7.86 & 7.94 & 8.03 & 8.11 \\ 
   &  & proximity & 117.75 & 29.00 & 7.67 & 7.41 & 7.18 & 7.34 & 7.49 & 7.58 \\ 
   &  &  & 83.25 & 43.00 & 8.21 & 8.02 & 7.85 & 7.97 & 8.08 & 8.15 \\ 
   &  & equal.split & 117.75 & 36.00 & 7.77 & 7.55 & 7.36 & 7.47 & 7.63 & 7.76 \\ 
   &  &  & 83.25 & 61.00 & 8.12 & 7.98 & 7.84 & 7.92 & 8.02 & 8.11 \\ 
   &  & gradual.split & 117.75 & 36.00 & 7.71 & 7.46 & 7.28 & 7.41 & 7.53 & 7.58 \\ 
   &  &  & 83.25 & 61.00 & 8.12 & 7.98 & 7.86 & 7.94 & 8.02 & 8.11 \\ 
   & number & equalbins & 166.85 - 133.23 & 67.00 & 7.46 & 7.35 & 7.26 & 7.32 & 7.37 & 7.43 \\ 
   &  &  & 133.23 - 99.62 & 83.00 & 7.91 & 7.82 & 7.72 & 7.79 & 7.85 & 7.89 \\ 
   &  &  & 99.62 - 66 & 94.00 & 8.22 & 8.13 & 8.03 & 8.11 & 8.16 & 8.20 \\ 
   &  & acctran & 166.85 & 14.00 & 7.66 & 7.15 & 6.71 & 7.02 & 7.27 & 7.43 \\ 
   &  &  & 116.42 & 36.00 & 8.04 & 7.83 & 7.66 & 7.77 & 7.90 & 8.00 \\ 
   &  &  & 66 & 1.00 &  &  &  &  &  &  \\ 
   &  & deltran & 166.85 & 14.00 & 7.44 & 6.91 & 6.49 & 6.77 & 7.08 & 7.27 \\ 
   &  &  & 116.42 & 29.00 & 7.63 & 7.39 & 7.13 & 7.32 & 7.45 & 7.53 \\ 
   &  &  & 66 & 1.00 &  &  &  &  &  &  \\ 
   &  & random & 166.85 & 14.00 & 7.53 & 7.03 & 6.47 & 6.88 & 7.12 & 7.27 \\ 
   &  &  & 116.42 & 35.00 & 7.86 & 7.64 & 7.46 & 7.61 & 7.69 & 7.77 \\ 
   &  &  & 66 & 1.00 &  &  &  &  &  &  \\ 
   &  & proximity & 166.85 & 14.00 & 7.44 & 6.92 & 6.44 & 6.78 & 7.08 & 7.27 \\ 
   &  &  & 116.42 & 30.00 & 7.68 & 7.42 & 7.21 & 7.35 & 7.51 & 7.62 \\ 
   &  &  & 66 & 1.00 &  &  &  &  &  &  \\ 
   &  & equal.split & 166.85 & 15.00 & 7.52 & 7.04 & 6.35 & 6.85 & 7.19 & 7.38 \\ 
   &  &  & 116.42 & 38.00 & 7.84 & 7.59 & 7.42 & 7.52 & 7.68 & 7.81 \\ 
   &  &  & 66 & 1.00 &  &  &  &  &  &  \\ 
   &  & gradual.split & 166.85 & 15.00 & 7.40 & 6.95 & 6.42 & 6.78 & 7.09 & 7.28 \\ 
   &  &  & 116.42 & 38.00 & 7.61 & 7.50 & 7.32 & 7.42 & 7.55 & 7.67 \\ 
   &  &  & 66 & 1.00 &  &  &  &  &  &  \\ 
   \hline
\hline
\caption{Disparity measurement (sum of variance) for Brusatte2014 dataset} 
\end{longtable}

 
%\end{landscape}

%\newpage
%\begin{landscape}
%  % latex table generated in R 3.4.3 by xtable 1.8-2 package
% Thu Dec 21 15:01:42 2017
\begin{longtable}{llllrrrrrrr}
  \hline
stratigraphy & bin\_type & model & subsets & n & obs & bs.median & 2.5\% & 25\% & 75\% & 97.5\% \\ 
  \hline
Ages & stratigraphy & equalbins & 201.3 - 199.3 & 0.00 &  &  &  &  &  &  \\ 
   &  &  & 199.3 - 190.8 & 2.00 & 6.02 & 0.00 & 0.00 & 0.00 & 6.02 & 6.02 \\ 
   &  &  & 190.8 - 182.7 & 2.00 & 5.81 & 0.00 & 0.00 & 0.00 & 5.81 & 5.81 \\ 
   &  &  & 182.7 - 174.1 & 3.00 & 6.31 & 4.35 & 0.00 & 3.87 & 4.40 & 6.31 \\ 
   &  &  & 174.1 - 170.3 & 3.00 & 6.22 & 4.26 & 0.00 & 3.87 & 6.22 & 6.22 \\ 
   &  &  & 170.3 - 168.3 & 2.00 & 7.53 & 7.53 & 0.00 & 0.00 & 7.53 & 7.53 \\ 
   &  &  & 168.3 - 166.1 & 2.00 & 7.21 & 3.60 & 0.00 & 0.00 & 7.21 & 7.21 \\ 
   &  &  & 166.1 - 163.5 & 8.00 & 7.81 & 6.92 & 5.54 & 6.48 & 7.19 & 7.52 \\ 
   &  &  & 163.5 - 157.3 & 8.00 & 7.92 & 6.99 & 5.73 & 6.73 & 7.30 & 7.65 \\ 
   &  &  & 157.3 - 152.1 & 14.00 & 7.81 & 7.24 & 6.78 & 7.14 & 7.40 & 7.54 \\ 
   &  &  & 152.1 - 145 & 14.00 & 7.64 & 7.14 & 6.63 & 6.95 & 7.28 & 7.42 \\ 
   &  &  & 145 - 139.8 & 11.00 & 7.66 & 7.05 & 6.31 & 6.73 & 7.15 & 7.34 \\ 
   &  &  & 139.8 - 132.9 & 8.00 & 7.67 & 6.72 & 5.30 & 6.42 & 6.99 & 7.36 \\ 
   &  &  & 132.9 - 129.4 & 22.00 & 7.93 & 7.56 & 7.29 & 7.49 & 7.63 & 7.78 \\ 
   &  &  & 129.4 - 125 & 27.00 & 8.04 & 7.76 & 7.49 & 7.67 & 7.83 & 7.91 \\ 
   &  &  & 125 - 113 & 18.00 & 7.86 & 7.40 & 7.06 & 7.27 & 7.52 & 7.63 \\ 
   &  &  & 113 - 100.5 & 20.00 & 7.77 & 7.41 & 6.95 & 7.24 & 7.51 & 7.64 \\ 
   &  &  & 100.5 - 93.9 & 22.00 & 7.98 & 7.62 & 7.22 & 7.51 & 7.73 & 7.84 \\ 
   &  &  & 93.9 - 89.8 & 17.00 & 8.06 & 7.60 & 7.11 & 7.45 & 7.73 & 7.91 \\ 
   &  &  & 89.8 - 86.3 & 15.00 & 8.12 & 7.64 & 6.99 & 7.47 & 7.79 & 8.01 \\ 
   &  &  & 86.3 - 83.6 & 34.00 & 8.21 & 7.99 & 7.82 & 7.92 & 8.03 & 8.10 \\ 
   &  &  & 83.6 - 72.1 & 44.00 & 8.12 & 7.93 & 7.82 & 7.89 & 7.99 & 8.06 \\ 
   &  &  & 72.1 - 66 & 27.00 & 8.20 & 7.90 & 7.59 & 7.83 & 7.97 & 8.06 \\ 
   &  & acctran & 200.3 & 3.00 & 6.54 & 4.32 & 0.00 & 4.31 & 4.97 & 6.54 \\ 
   &  &  & 195.05 & 4.00 & 6.76 & 5.58 & 1.50 & 4.65 & 5.68 & 6.76 \\ 
   &  &  & 186.75 & 6.00 & 6.79 & 5.84 & 4.31 & 5.44 & 6.06 & 6.42 \\ 
   &  &  & 178.4 & 8.00 & 7.01 & 6.25 & 4.60 & 5.97 & 6.49 & 6.75 \\ 
   &  &  & 172.2 & 12.00 & 7.27 & 6.68 & 6.09 & 6.52 & 6.83 & 7.10 \\ 
   &  &  & 169.3 & 14.00 & 7.31 & 6.79 & 6.27 & 6.66 & 6.92 & 7.11 \\ 
   &  &  & 167.2 & 16.00 & 7.50 & 7.08 & 6.70 & 6.95 & 7.20 & 7.41 \\ 
   &  &  & 164.8 & 17.00 & 7.62 & 7.16 & 6.74 & 7.04 & 7.26 & 7.48 \\ 
   &  &  & 160.4 & 18.00 & 7.64 & 7.23 & 6.88 & 7.09 & 7.36 & 7.52 \\ 
   &  &  & 154.7 & 18.00 & 7.54 & 7.17 & 6.77 & 7.05 & 7.24 & 7.37 \\ 
   &  &  & 148.55 & 21.00 & 7.58 & 7.24 & 6.99 & 7.14 & 7.32 & 7.41 \\ 
   &  &  & 142.4 & 17.00 & 7.58 & 7.16 & 6.83 & 7.07 & 7.25 & 7.38 \\ 
   &  &  & 136.35 & 23.00 & 7.72 & 7.40 & 7.11 & 7.33 & 7.48 & 7.56 \\ 
   &  &  & 131.15 & 28.00 & 7.80 & 7.52 & 7.36 & 7.46 & 7.60 & 7.68 \\ 
   &  &  & 127.2 & 33.00 & 7.94 & 7.71 & 7.48 & 7.64 & 7.76 & 7.85 \\ 
   &  &  & 119 & 23.00 & 7.91 & 7.56 & 7.21 & 7.45 & 7.67 & 7.79 \\ 
   &  &  & 106.75 & 16.00 & 7.81 & 7.35 & 6.88 & 7.20 & 7.46 & 7.63 \\ 
   &  &  & 97.2 & 20.00 & 8.11 & 7.69 & 7.46 & 7.62 & 7.78 & 7.95 \\ 
   &  &  & 91.85 & 23.00 & 8.14 & 7.78 & 7.53 & 7.70 & 7.86 & 7.98 \\ 
   &  &  & 88.05 & 25.00 & 8.17 & 7.85 & 7.65 & 7.79 & 7.93 & 8.00 \\ 
   &  &  & 84.95 & 24.00 & 8.18 & 7.86 & 7.58 & 7.75 & 7.92 & 8.05 \\ 
   &  &  & 77.85 & 26.00 & 8.22 & 7.93 & 7.67 & 7.86 & 7.98 & 8.09 \\ 
   &  &  & 69.05 & 20.00 & 8.28 & 7.90 & 7.47 & 7.80 & 7.99 & 8.10 \\ 
   &  & deltran & 200.3 & 2.00 & 5.81 & 5.81 & 0.00 & 0.00 & 5.81 & 5.81 \\ 
   &  &  & 195.05 & 3.00 & 5.81 & 3.87 & 0.00 & 3.87 & 3.87 & 5.81 \\ 
   &  &  & 186.75 & 5.00 & 6.06 & 4.87 & 3.48 & 4.28 & 5.45 & 5.55 \\ 
   &  &  & 178.4 & 7.00 & 6.20 & 5.44 & 4.22 & 5.07 & 5.70 & 5.97 \\ 
   &  &  & 172.2 & 9.00 & 6.45 & 5.81 & 4.98 & 5.57 & 6.03 & 6.22 \\ 
   &  &  & 169.3 & 10.00 & 6.54 & 5.94 & 5.44 & 5.72 & 6.09 & 6.31 \\ 
   &  &  & 167.2 & 13.00 & 6.90 & 6.41 & 5.86 & 6.26 & 6.58 & 6.77 \\ 
   &  &  & 164.8 & 15.00 & 7.12 & 6.73 & 6.09 & 6.57 & 6.87 & 7.07 \\ 
   &  &  & 160.4 & 16.00 & 7.26 & 6.81 & 6.42 & 6.67 & 6.95 & 7.20 \\ 
   &  &  & 154.7 & 15.00 & 7.27 & 6.79 & 6.37 & 6.65 & 6.90 & 7.09 \\ 
   &  &  & 148.55 & 18.00 & 7.28 & 6.91 & 6.50 & 6.79 & 7.01 & 7.11 \\ 
   &  &  & 142.4 & 12.00 & 7.06 & 6.51 & 5.88 & 6.29 & 6.66 & 6.84 \\ 
   &  &  & 136.35 & 17.00 & 7.27 & 6.86 & 6.44 & 6.73 & 6.97 & 7.11 \\ 
   &  &  & 131.15 & 22.00 & 7.38 & 7.08 & 6.67 & 6.98 & 7.14 & 7.26 \\ 
   &  &  & 127.2 & 31.00 & 7.70 & 7.48 & 7.28 & 7.39 & 7.52 & 7.65 \\ 
   &  &  & 119 & 23.00 & 7.70 & 7.40 & 7.08 & 7.33 & 7.49 & 7.64 \\ 
   &  &  & 106.75 & 15.00 & 7.48 & 7.02 & 6.48 & 6.85 & 7.18 & 7.38 \\ 
   &  &  & 97.2 & 16.00 & 7.79 & 7.35 & 6.80 & 7.17 & 7.47 & 7.64 \\ 
   &  &  & 91.85 & 21.00 & 7.92 & 7.56 & 7.07 & 7.43 & 7.67 & 7.79 \\ 
   &  &  & 88.05 & 24.00 & 7.88 & 7.58 & 7.25 & 7.50 & 7.64 & 7.76 \\ 
   &  &  & 84.95 & 23.00 & 7.93 & 7.59 & 7.26 & 7.52 & 7.66 & 7.80 \\ 
   &  &  & 77.85 & 26.00 & 8.12 & 7.80 & 7.49 & 7.73 & 7.88 & 7.96 \\ 
   &  &  & 69.05 & 20.00 & 8.28 & 7.89 & 7.56 & 7.81 & 7.98 & 8.14 \\ 
   &  & random & 200.3 & 3.00 & 6.20 & 4.07 & 0.00 & 3.87 & 4.45 & 6.20 \\ 
   &  &  & 195.05 & 4.00 & 6.29 & 5.08 & 2.90 & 3.45 & 5.36 & 6.29 \\ 
   &  &  & 186.75 & 6.00 & 6.63 & 5.53 & 3.91 & 5.14 & 5.92 & 6.37 \\ 
   &  &  & 178.4 & 7.00 & 6.41 & 5.63 & 4.46 & 5.21 & 5.88 & 6.21 \\ 
   &  &  & 172.2 & 12.00 & 6.84 & 6.27 & 5.66 & 6.06 & 6.45 & 6.70 \\ 
   &  &  & 169.3 & 14.00 & 7.04 & 6.55 & 6.04 & 6.40 & 6.67 & 6.86 \\ 
   &  &  & 167.2 & 16.00 & 7.26 & 6.84 & 6.27 & 6.67 & 6.94 & 7.12 \\ 
   &  &  & 164.8 & 17.00 & 7.30 & 6.85 & 6.39 & 6.70 & 6.98 & 7.20 \\ 
   &  &  & 160.4 & 18.00 & 7.40 & 7.04 & 6.54 & 6.85 & 7.15 & 7.29 \\ 
   &  &  & 154.7 & 18.00 & 7.45 & 7.07 & 6.63 & 6.90 & 7.16 & 7.36 \\ 
   &  &  & 148.55 & 20.00 & 7.40 & 7.04 & 6.72 & 6.92 & 7.12 & 7.30 \\ 
   &  &  & 142.4 & 16.00 & 7.35 & 6.91 & 6.46 & 6.79 & 7.02 & 7.12 \\ 
   &  &  & 136.35 & 21.00 & 7.50 & 7.17 & 6.80 & 7.08 & 7.23 & 7.37 \\ 
   &  &  & 131.15 & 26.00 & 7.73 & 7.42 & 7.22 & 7.34 & 7.50 & 7.58 \\ 
   &  &  & 127.2 & 32.00 & 7.89 & 7.66 & 7.46 & 7.58 & 7.72 & 7.79 \\ 
   &  &  & 119 & 23.00 & 7.79 & 7.43 & 7.08 & 7.35 & 7.51 & 7.69 \\ 
   &  &  & 106.75 & 16.00 & 7.76 & 7.31 & 6.93 & 7.20 & 7.45 & 7.57 \\ 
   &  &  & 97.2 & 18.00 & 7.84 & 7.44 & 6.97 & 7.30 & 7.52 & 7.67 \\ 
   &  &  & 91.85 & 22.00 & 8.07 & 7.70 & 7.41 & 7.60 & 7.79 & 7.90 \\ 
   &  &  & 88.05 & 25.00 & 8.08 & 7.79 & 7.60 & 7.73 & 7.85 & 7.97 \\ 
   &  &  & 84.95 & 23.00 & 8.08 & 7.76 & 7.43 & 7.68 & 7.83 & 7.94 \\ 
   &  &  & 77.85 & 26.00 & 8.15 & 7.87 & 7.61 & 7.78 & 7.92 & 8.02 \\ 
   &  &  & 69.05 & 20.00 & 8.28 & 7.87 & 7.48 & 7.78 & 7.97 & 8.14 \\ 
   &  & proximity & 200.3 & 2.00 & 5.81 & 5.81 & 0.00 & 0.00 & 5.81 & 5.81 \\ 
   &  &  & 195.05 & 3.00 & 5.81 & 3.87 & 0.00 & 3.87 & 4.35 & 5.81 \\ 
   &  &  & 186.75 & 5.00 & 6.06 & 5.37 & 2.53 & 4.58 & 5.44 & 6.06 \\ 
   &  &  & 178.4 & 8.00 & 6.73 & 5.98 & 4.66 & 5.75 & 6.20 & 6.55 \\ 
   &  &  & 172.2 & 9.00 & 6.82 & 6.13 & 5.23 & 5.85 & 6.31 & 6.65 \\ 
   &  &  & 169.3 & 11.00 & 6.85 & 6.35 & 5.46 & 6.14 & 6.46 & 6.66 \\ 
   &  &  & 167.2 & 15.00 & 7.32 & 6.84 & 6.31 & 6.69 & 6.97 & 7.21 \\ 
   &  &  & 164.8 & 16.00 & 7.49 & 7.06 & 6.48 & 6.88 & 7.20 & 7.35 \\ 
   &  &  & 160.4 & 17.00 & 7.53 & 7.16 & 6.51 & 7.01 & 7.29 & 7.47 \\ 
   &  &  & 154.7 & 15.00 & 7.41 & 6.96 & 6.56 & 6.81 & 7.09 & 7.29 \\ 
   &  &  & 148.55 & 18.00 & 7.34 & 6.97 & 6.60 & 6.86 & 7.04 & 7.16 \\ 
   &  &  & 142.4 & 14.00 & 7.23 & 6.81 & 6.27 & 6.59 & 6.92 & 7.07 \\ 
   &  &  & 136.35 & 19.00 & 7.50 & 7.10 & 6.85 & 7.02 & 7.21 & 7.35 \\ 
   &  &  & 131.15 & 25.00 & 7.62 & 7.32 & 7.05 & 7.23 & 7.40 & 7.50 \\ 
   &  &  & 127.2 & 31.00 & 7.78 & 7.56 & 7.37 & 7.50 & 7.62 & 7.69 \\ 
   &  &  & 119 & 23.00 & 7.77 & 7.46 & 7.11 & 7.35 & 7.56 & 7.69 \\ 
   &  &  & 106.75 & 16.00 & 7.70 & 7.23 & 6.71 & 7.08 & 7.34 & 7.52 \\ 
   &  &  & 97.2 & 17.00 & 7.86 & 7.44 & 6.97 & 7.29 & 7.55 & 7.70 \\ 
   &  &  & 91.85 & 23.00 & 8.08 & 7.73 & 7.45 & 7.65 & 7.83 & 7.96 \\ 
   &  &  & 88.05 & 25.00 & 8.04 & 7.74 & 7.46 & 7.66 & 7.82 & 7.91 \\ 
   &  &  & 84.95 & 23.00 & 8.11 & 7.78 & 7.47 & 7.69 & 7.84 & 7.94 \\ 
   &  &  & 77.85 & 26.00 & 8.15 & 7.87 & 7.58 & 7.78 & 7.92 & 8.00 \\ 
   &  &  & 69.05 & 20.00 & 8.28 & 7.89 & 7.43 & 7.74 & 7.98 & 8.11 \\ 
   &  & equal.split & 200.3 & 3.00 & 6.13 & 3.87 & 0.00 & 3.87 & 4.31 & 6.20 \\ 
   &  &  & 195.05 & 4.00 & 6.29 & 5.04 & 2.90 & 4.06 & 5.39 & 6.08 \\ 
   &  &  & 186.75 & 6.00 & 6.42 & 5.38 & 3.86 & 4.96 & 5.89 & 6.33 \\ 
   &  &  & 178.4 & 8.00 & 6.46 & 5.89 & 4.63 & 5.62 & 6.13 & 6.48 \\ 
   &  &  & 172.2 & 12.00 & 6.96 & 6.28 & 5.56 & 6.05 & 6.44 & 6.76 \\ 
   &  &  & 169.3 & 14.00 & 6.58 & 6.43 & 5.81 & 6.26 & 6.61 & 6.83 \\ 
   &  &  & 167.2 & 17.00 & 7.24 & 6.77 & 6.29 & 6.60 & 6.89 & 7.15 \\ 
   &  &  & 164.8 & 19.00 & 7.29 & 6.97 & 6.37 & 6.84 & 7.11 & 7.28 \\ 
   &  &  & 160.4 & 23.00 & 7.37 & 7.12 & 6.82 & 7.02 & 7.24 & 7.38 \\ 
   &  &  & 154.7 & 21.00 & 7.23 & 6.92 & 6.49 & 6.79 & 7.00 & 7.19 \\ 
   &  &  & 148.55 & 23.00 & 7.32 & 7.05 & 6.83 & 6.99 & 7.15 & 7.28 \\ 
   &  &  & 142.4 & 17.00 & 7.27 & 6.91 & 6.32 & 6.77 & 7.01 & 7.21 \\ 
   &  &  & 136.35 & 23.00 & 7.55 & 7.21 & 6.90 & 7.10 & 7.30 & 7.45 \\ 
   &  &  & 131.15 & 29.00 & 7.60 & 7.27 & 7.00 & 7.20 & 7.36 & 7.49 \\ 
   &  &  & 127.2 & 35.00 & 7.79 & 7.58 & 7.36 & 7.52 & 7.63 & 7.74 \\ 
   &  &  & 119 & 25.00 & 7.74 & 7.50 & 7.15 & 7.36 & 7.59 & 7.71 \\ 
   &  &  & 106.75 & 18.00 & 7.41 & 7.12 & 6.53 & 6.93 & 7.22 & 7.38 \\ 
   &  &  & 97.2 & 23.00 & 7.73 & 7.53 & 7.21 & 7.42 & 7.61 & 7.78 \\ 
   &  &  & 91.85 & 30.00 & 7.91 & 7.65 & 7.36 & 7.56 & 7.73 & 7.84 \\ 
   &  &  & 88.05 & 28.00 & 8.06 & 7.70 & 7.43 & 7.62 & 7.77 & 7.89 \\ 
   &  &  & 84.95 & 26.00 & 8.02 & 7.71 & 7.41 & 7.61 & 7.81 & 7.89 \\ 
   &  &  & 77.85 & 38.00 & 8.04 & 7.82 & 7.59 & 7.78 & 7.87 & 7.98 \\ 
   &  &  & 69.05 & 22.00 & 8.20 & 7.83 & 7.44 & 7.71 & 7.93 & 8.06 \\ 
   &  & gradual.split & 200.3 & 3.00 & 5.81 & 3.87 & 0.00 & 3.87 & 3.87 & 6.12 \\ 
   &  &  & 195.05 & 4.00 & 4.84 & 3.87 & 2.90 & 2.90 & 5.00 & 6.13 \\ 
   &  &  & 186.75 & 6.00 & 5.78 & 5.26 & 3.72 & 4.93 & 5.61 & 5.97 \\ 
   &  &  & 178.4 & 8.00 & 6.68 & 5.89 & 4.89 & 5.55 & 6.13 & 6.43 \\ 
   &  &  & 172.2 & 12.00 & 6.68 & 6.14 & 5.23 & 5.89 & 6.29 & 6.54 \\ 
   &  &  & 169.3 & 14.00 & 6.75 & 6.37 & 5.67 & 6.14 & 6.53 & 6.74 \\ 
   &  &  & 167.2 & 17.00 & 7.02 & 6.72 & 5.96 & 6.54 & 6.89 & 7.05 \\ 
   &  &  & 164.8 & 19.00 & 7.34 & 7.01 & 6.44 & 6.83 & 7.15 & 7.32 \\ 
   &  &  & 160.4 & 23.00 & 7.52 & 7.13 & 6.76 & 7.01 & 7.24 & 7.42 \\ 
   &  &  & 154.7 & 21.00 & 7.20 & 6.91 & 6.52 & 6.78 & 7.01 & 7.19 \\ 
   &  &  & 148.55 & 23.00 & 7.26 & 6.99 & 6.68 & 6.90 & 7.06 & 7.17 \\ 
   &  &  & 142.4 & 17.00 & 7.16 & 6.83 & 6.31 & 6.66 & 6.96 & 7.12 \\ 
   &  &  & 136.35 & 23.00 & 7.43 & 7.11 & 6.74 & 7.00 & 7.19 & 7.30 \\ 
   &  &  & 131.15 & 29.00 & 7.53 & 7.29 & 7.07 & 7.22 & 7.37 & 7.51 \\ 
   &  &  & 127.2 & 35.00 & 7.77 & 7.57 & 7.34 & 7.51 & 7.62 & 7.74 \\ 
   &  &  & 119 & 25.00 & 7.73 & 7.49 & 7.18 & 7.37 & 7.57 & 7.71 \\ 
   &  &  & 106.75 & 18.00 & 7.43 & 7.15 & 6.55 & 6.98 & 7.26 & 7.46 \\ 
   &  &  & 97.2 & 23.00 & 7.69 & 7.42 & 7.06 & 7.31 & 7.50 & 7.71 \\ 
   &  &  & 91.85 & 30.00 & 7.91 & 7.68 & 7.39 & 7.60 & 7.72 & 7.86 \\ 
   &  &  & 88.05 & 28.00 & 8.03 & 7.74 & 7.52 & 7.67 & 7.81 & 7.90 \\ 
   &  &  & 84.95 & 26.00 & 8.03 & 7.74 & 7.50 & 7.63 & 7.82 & 7.92 \\ 
   &  &  & 77.85 & 38.00 & 7.97 & 7.82 & 7.61 & 7.75 & 7.86 & 7.93 \\ 
   &  &  & 69.05 & 22.00 & 8.20 & 7.85 & 7.60 & 7.78 & 7.92 & 8.08 \\ 
   & duration & equalbins & 201.2 - 196 & 1.00 &  &  &  &  &  &  \\ 
   &  &  & 196 - 190.8 & 1.00 &  &  &  &  &  &  \\ 
   &  &  & 190.8 - 185.6 & 1.00 &  &  &  &  &  &  \\ 
   &  &  & 185.6 - 180.4 & 2.00 & 6.17 & 6.17 & 0.00 & 0.00 & 6.17 & 6.17 \\ 
   &  &  & 180.4 - 175.2 & 2.00 & 6.59 & 0.00 & 0.00 & 0.00 & 6.59 & 6.59 \\ 
   &  &  & 175.2 - 170 & 3.00 & 6.22 & 4.26 & 0.00 & 3.87 & 4.32 & 6.22 \\ 
   &  &  & 170 - 164.8 & 6.00 & 7.75 & 6.63 & 4.78 & 6.01 & 7.07 & 7.40 \\ 
   &  &  & 164.8 - 159.6 & 9.00 & 7.71 & 6.91 & 5.91 & 6.59 & 7.19 & 7.59 \\ 
   &  &  & 159.6 - 154.4 & 14.00 & 7.78 & 7.20 & 6.67 & 6.99 & 7.38 & 7.58 \\ 
   &  &  & 154.4 - 149.2 & 12.00 & 7.70 & 7.10 & 6.37 & 6.94 & 7.26 & 7.52 \\ 
   &  &  & 149.2 - 144 & 10.00 & 7.54 & 6.86 & 6.06 & 6.62 & 7.03 & 7.29 \\ 
   &  &  & 144 - 138.8 & 5.00 & 7.48 & 6.04 & 3.54 & 5.28 & 6.65 & 6.83 \\ 
   &  &  & 138.8 - 133.6 & 6.00 & 7.75 & 6.70 & 5.01 & 6.19 & 7.17 & 7.40 \\ 
   &  &  & 133.6 - 128.4 & 24.00 & 7.88 & 7.56 & 7.33 & 7.46 & 7.66 & 7.76 \\ 
   &  &  & 128.4 - 123.2 & 28.00 & 8.02 & 7.75 & 7.55 & 7.66 & 7.81 & 7.88 \\ 
   &  &  & 123.2 - 118 & 14.00 & 7.90 & 7.36 & 6.78 & 7.24 & 7.51 & 7.69 \\ 
   &  &  & 118 - 112.8 & 16.00 & 7.92 & 7.48 & 7.05 & 7.34 & 7.58 & 7.74 \\ 
   &  &  & 112.8 - 107.6 & 7.00 & 7.23 & 6.17 & 4.88 & 5.80 & 6.63 & 7.17 \\ 
   &  &  & 107.6 - 102.4 & 8.00 & 7.59 & 6.75 & 5.36 & 6.43 & 6.99 & 7.32 \\ 
   &  &  & 102.4 - 97.2 & 14.00 & 7.96 & 7.42 & 6.86 & 7.25 & 7.55 & 7.81 \\ 
   &  &  & 97.2 - 92 & 15.00 & 7.95 & 7.47 & 6.89 & 7.22 & 7.64 & 7.88 \\ 
   &  &  & 92 - 86.8 & 16.00 & 7.94 & 7.51 & 6.85 & 7.36 & 7.62 & 7.82 \\ 
   &  &  & 86.8 - 81.6 & 36.00 & 8.19 & 7.95 & 7.76 & 7.89 & 8.02 & 8.10 \\ 
   &  &  & 81.6 - 76.4 & 27.00 & 8.09 & 7.80 & 7.55 & 7.72 & 7.86 & 7.98 \\ 
   &  &  & 76.4 - 71.2 & 28.00 & 8.17 & 7.90 & 7.61 & 7.82 & 7.97 & 8.05 \\ 
   &  &  & 71.2 - 66 & 20.00 & 8.28 & 7.90 & 7.59 & 7.78 & 7.98 & 8.07 \\ 
   &  & acctran & 198.6 & 3.00 & 6.54 & 4.32 & 0.00 & 4.31 & 4.97 & 6.54 \\ 
   &  &  & 193.4 & 4.00 & 6.76 & 5.57 & 3.15 & 3.54 & 5.68 & 6.76 \\ 
   &  &  & 188.2 & 5.00 & 6.74 & 5.57 & 2.87 & 4.92 & 6.11 & 6.74 \\ 
   &  &  & 183 & 7.00 & 6.96 & 6.06 & 4.48 & 5.74 & 6.36 & 6.76 \\ 
   &  &  & 177.8 & 9.00 & 7.09 & 6.37 & 5.35 & 6.12 & 6.61 & 6.88 \\ 
   &  &  & 172.6 & 11.00 & 7.24 & 6.63 & 5.97 & 6.41 & 6.75 & 7.00 \\ 
   &  &  & 167.4 & 16.00 & 7.50 & 7.06 & 6.55 & 6.96 & 7.17 & 7.32 \\ 
   &  &  & 162.2 & 18.00 & 7.64 & 7.24 & 6.93 & 7.11 & 7.37 & 7.46 \\ 
   &  &  & 157 & 19.00 & 7.65 & 7.30 & 6.96 & 7.17 & 7.39 & 7.52 \\ 
   &  &  & 151.8 & 18.00 & 7.48 & 7.07 & 6.69 & 6.97 & 7.17 & 7.31 \\ 
   &  &  & 146.6 & 21.00 & 7.58 & 7.25 & 6.97 & 7.15 & 7.33 & 7.44 \\ 
   &  &  & 141.4 & 17.00 & 7.58 & 7.14 & 6.70 & 7.03 & 7.26 & 7.39 \\ 
   &  &  & 136.2 & 23.00 & 7.72 & 7.41 & 7.11 & 7.31 & 7.47 & 7.54 \\ 
   &  &  & 131 & 28.00 & 7.80 & 7.54 & 7.28 & 7.46 & 7.59 & 7.67 \\ 
   &  &  & 125.8 & 23.00 & 7.90 & 7.56 & 7.27 & 7.47 & 7.68 & 7.83 \\ 
   &  &  & 120.6 & 23.00 & 7.90 & 7.56 & 7.23 & 7.49 & 7.65 & 7.80 \\ 
   &  &  & 115.4 & 24.00 & 7.91 & 7.61 & 7.27 & 7.53 & 7.68 & 7.80 \\ 
   &  &  & 110.2 & 16.00 & 7.81 & 7.35 & 6.84 & 7.18 & 7.46 & 7.63 \\ 
   &  &  & 105 & 18.00 & 7.86 & 7.45 & 7.08 & 7.34 & 7.53 & 7.68 \\ 
   &  &  & 99.8 & 19.00 & 8.10 & 7.70 & 7.28 & 7.54 & 7.82 & 7.96 \\ 
   &  &  & 94.6 & 23.00 & 8.09 & 7.74 & 7.46 & 7.65 & 7.82 & 7.96 \\ 
   &  &  & 89.4 & 25.00 & 8.17 & 7.85 & 7.57 & 7.76 & 7.92 & 8.05 \\ 
   &  &  & 84.2 & 25.00 & 8.20 & 7.86 & 7.66 & 7.79 & 7.96 & 8.04 \\ 
   &  &  & 79 & 26.00 & 8.19 & 7.91 & 7.60 & 7.82 & 7.96 & 8.07 \\ 
   &  &  & 73.8 & 26.00 & 8.22 & 7.90 & 7.62 & 7.84 & 7.97 & 8.07 \\ 
   &  &  & 68.6 & 20.00 & 8.28 & 7.92 & 7.52 & 7.80 & 7.99 & 8.11 \\ 
   &  & deltran & 198.6 & 2.00 & 5.81 & 0.00 & 0.00 & 0.00 & 5.81 & 5.81 \\ 
   &  &  & 193.4 & 3.00 & 5.81 & 3.87 & 0.00 & 3.87 & 3.87 & 5.81 \\ 
   &  &  & 188.2 & 4.00 & 6.01 & 4.84 & 2.90 & 3.15 & 5.08 & 6.01 \\ 
   &  &  & 183 & 6.00 & 6.08 & 5.23 & 3.51 & 4.83 & 5.46 & 5.76 \\ 
   &  &  & 177.8 & 8.00 & 6.27 & 5.59 & 4.66 & 5.35 & 5.75 & 5.97 \\ 
   &  &  & 172.6 & 8.00 & 6.36 & 5.64 & 4.73 & 5.33 & 5.85 & 6.13 \\ 
   &  &  & 167.4 & 13.00 & 6.90 & 6.44 & 5.87 & 6.25 & 6.61 & 6.84 \\ 
   &  &  & 162.2 & 16.00 & 7.26 & 6.87 & 6.24 & 6.64 & 7.02 & 7.22 \\ 
   &  &  & 157 & 17.00 & 7.25 & 6.81 & 6.27 & 6.65 & 6.98 & 7.27 \\ 
   &  &  & 151.8 & 15.00 & 7.22 & 6.79 & 6.22 & 6.58 & 6.93 & 7.10 \\ 
   &  &  & 146.6 & 18.00 & 7.28 & 6.91 & 6.52 & 6.77 & 7.00 & 7.15 \\ 
   &  &  & 141.4 & 12.00 & 7.06 & 6.45 & 5.91 & 6.28 & 6.66 & 6.90 \\ 
   &  &  & 136.2 & 17.00 & 7.27 & 6.89 & 6.55 & 6.73 & 6.99 & 7.13 \\ 
   &  &  & 131 & 22.00 & 7.38 & 7.05 & 6.63 & 6.95 & 7.11 & 7.24 \\ 
   &  &  & 125.8 & 23.00 & 7.69 & 7.39 & 6.99 & 7.24 & 7.48 & 7.60 \\ 
   &  &  & 120.6 & 23.00 & 7.69 & 7.35 & 7.11 & 7.27 & 7.46 & 7.56 \\ 
   &  &  & 115.4 & 24.00 & 7.70 & 7.40 & 7.04 & 7.28 & 7.49 & 7.59 \\ 
   &  &  & 110.2 & 15.00 & 7.48 & 6.99 & 6.44 & 6.84 & 7.13 & 7.29 \\ 
   &  &  & 105 & 15.00 & 7.49 & 7.00 & 6.45 & 6.81 & 7.12 & 7.32 \\ 
   &  &  & 99.8 & 15.00 & 7.79 & 7.30 & 6.74 & 7.13 & 7.39 & 7.56 \\ 
   &  &  & 94.6 & 17.00 & 7.78 & 7.35 & 6.94 & 7.23 & 7.41 & 7.59 \\ 
   &  &  & 89.4 & 24.00 & 7.88 & 7.58 & 7.29 & 7.49 & 7.66 & 7.78 \\ 
   &  &  & 84.2 & 23.00 & 7.96 & 7.62 & 7.24 & 7.50 & 7.70 & 7.83 \\ 
   &  &  & 79 & 26.00 & 8.12 & 7.84 & 7.53 & 7.77 & 7.87 & 8.02 \\ 
   &  &  & 73.8 & 26.00 & 8.12 & 7.82 & 7.60 & 7.75 & 7.89 & 7.99 \\ 
   &  &  & 68.6 & 20.00 & 8.28 & 7.89 & 7.51 & 7.80 & 7.99 & 8.10 \\ 
   &  & random & 198.6 & 3.00 & 6.54 & 4.32 & 0.00 & 4.31 & 6.54 & 6.54 \\ 
   &  &  & 193.4 & 4.00 & 6.19 & 5.06 & 2.90 & 4.26 & 5.38 & 6.19 \\ 
   &  &  & 188.2 & 5.00 & 6.40 & 5.30 & 2.42 & 4.56 & 5.77 & 6.40 \\ 
   &  &  & 183 & 6.00 & 6.54 & 5.61 & 3.79 & 5.04 & 5.97 & 6.45 \\ 
   &  &  & 177.8 & 9.00 & 6.76 & 5.97 & 5.20 & 5.74 & 6.29 & 6.56 \\ 
   &  &  & 172.6 & 11.00 & 6.92 & 6.37 & 5.69 & 6.15 & 6.52 & 6.74 \\ 
   &  &  & 167.4 & 16.00 & 7.15 & 6.72 & 6.20 & 6.53 & 6.84 & 7.03 \\ 
   &  &  & 162.2 & 18.00 & 7.46 & 7.06 & 6.58 & 6.89 & 7.21 & 7.36 \\ 
   &  &  & 157 & 19.00 & 7.50 & 7.14 & 6.67 & 6.99 & 7.24 & 7.47 \\ 
   &  &  & 151.8 & 18.00 & 7.29 & 6.90 & 6.63 & 6.80 & 7.00 & 7.14 \\ 
   &  &  & 146.6 & 20.00 & 7.44 & 7.09 & 6.73 & 6.97 & 7.20 & 7.32 \\ 
   &  &  & 141.4 & 17.00 & 7.41 & 6.96 & 6.55 & 6.84 & 7.07 & 7.21 \\ 
   &  &  & 136.2 & 21.00 & 7.49 & 7.16 & 6.83 & 7.05 & 7.23 & 7.31 \\ 
   &  &  & 131 & 28.00 & 7.61 & 7.36 & 7.10 & 7.26 & 7.41 & 7.51 \\ 
   &  &  & 125.8 & 23.00 & 7.76 & 7.44 & 6.93 & 7.32 & 7.54 & 7.66 \\ 
   &  &  & 120.6 & 23.00 & 7.74 & 7.41 & 7.00 & 7.30 & 7.51 & 7.60 \\ 
   &  &  & 115.4 & 24.00 & 7.80 & 7.49 & 7.21 & 7.41 & 7.56 & 7.71 \\ 
   &  &  & 110.2 & 16.00 & 7.48 & 7.04 & 6.52 & 6.90 & 7.16 & 7.37 \\ 
   &  &  & 105 & 15.00 & 7.67 & 7.19 & 6.75 & 7.05 & 7.32 & 7.49 \\ 
   &  &  & 99.8 & 19.00 & 8.01 & 7.61 & 7.14 & 7.46 & 7.70 & 7.85 \\ 
   &  &  & 94.6 & 22.00 & 7.97 & 7.64 & 7.21 & 7.52 & 7.72 & 7.84 \\ 
   &  &  & 89.4 & 25.00 & 8.11 & 7.81 & 7.54 & 7.73 & 7.88 & 8.00 \\ 
   &  &  & 84.2 & 25.00 & 8.06 & 7.75 & 7.47 & 7.67 & 7.82 & 7.92 \\ 
   &  &  & 79 & 26.00 & 8.12 & 7.84 & 7.50 & 7.75 & 7.89 & 7.99 \\ 
   &  &  & 73.8 & 26.00 & 8.13 & 7.83 & 7.53 & 7.74 & 7.88 & 7.97 \\ 
   &  &  & 68.6 & 20.00 & 8.28 & 7.87 & 7.57 & 7.77 & 7.98 & 8.08 \\ 
   &  & proximity & 198.6 & 3.00 & 5.81 & 3.87 & 0.00 & 3.87 & 4.35 & 5.81 \\ 
   &  &  & 193.4 & 4.00 & 6.01 & 4.99 & 2.90 & 3.87 & 5.09 & 6.01 \\ 
   &  &  & 188.2 & 5.00 & 6.06 & 4.92 & 2.41 & 4.58 & 5.42 & 5.81 \\ 
   &  &  & 183 & 6.00 & 6.23 & 5.35 & 3.72 & 4.65 & 5.55 & 6.05 \\ 
   &  &  & 177.8 & 8.00 & 6.73 & 6.00 & 4.77 & 5.62 & 6.28 & 6.55 \\ 
   &  &  & 172.6 & 9.00 & 6.82 & 6.16 & 5.42 & 5.82 & 6.33 & 6.58 \\ 
   &  &  & 167.4 & 15.00 & 7.32 & 6.84 & 6.27 & 6.66 & 7.04 & 7.19 \\ 
   &  &  & 162.2 & 17.00 & 7.51 & 7.07 & 6.53 & 6.91 & 7.21 & 7.35 \\ 
   &  &  & 157 & 18.00 & 7.53 & 7.14 & 6.83 & 7.04 & 7.27 & 7.46 \\ 
   &  &  & 151.8 & 17.00 & 7.33 & 6.90 & 6.53 & 6.76 & 7.02 & 7.16 \\ 
   &  &  & 146.6 & 18.00 & 7.41 & 7.02 & 6.65 & 6.89 & 7.12 & 7.23 \\ 
   &  &  & 141.4 & 14.00 & 7.25 & 6.77 & 6.21 & 6.62 & 6.90 & 7.08 \\ 
   &  &  & 136.2 & 19.00 & 7.50 & 7.11 & 6.81 & 7.02 & 7.19 & 7.34 \\ 
   &  &  & 131 & 25.00 & 7.62 & 7.33 & 7.07 & 7.26 & 7.41 & 7.50 \\ 
   &  &  & 125.8 & 23.00 & 7.77 & 7.45 & 7.16 & 7.33 & 7.53 & 7.66 \\ 
   &  &  & 120.6 & 23.00 & 7.77 & 7.44 & 7.08 & 7.37 & 7.53 & 7.68 \\ 
   &  &  & 115.4 & 24.00 & 7.79 & 7.48 & 7.12 & 7.38 & 7.55 & 7.66 \\ 
   &  &  & 110.2 & 15.00 & 7.73 & 7.24 & 6.76 & 7.06 & 7.38 & 7.53 \\ 
   &  &  & 105 & 16.00 & 7.75 & 7.25 & 6.83 & 7.08 & 7.41 & 7.58 \\ 
   &  &  & 99.8 & 17.00 & 7.86 & 7.41 & 6.99 & 7.24 & 7.53 & 7.71 \\ 
   &  &  & 94.6 & 19.00 & 7.94 & 7.54 & 7.16 & 7.42 & 7.65 & 7.81 \\ 
   &  &  & 89.4 & 24.00 & 8.05 & 7.70 & 7.39 & 7.65 & 7.79 & 7.91 \\ 
   &  &  & 84.2 & 23.00 & 8.16 & 7.83 & 7.47 & 7.72 & 7.91 & 8.00 \\ 
   &  &  & 79 & 26.00 & 8.15 & 7.87 & 7.59 & 7.79 & 7.94 & 8.02 \\ 
   &  &  & 73.8 & 26.00 & 8.19 & 7.88 & 7.66 & 7.79 & 7.96 & 8.06 \\ 
   &  &  & 68.6 & 20.00 & 8.28 & 7.88 & 7.48 & 7.79 & 7.98 & 8.07 \\ 
   &  & equal.split & 198.6 & 3.00 & 5.81 & 3.87 & 0.00 & 3.87 & 4.32 & 6.38 \\ 
   &  &  & 193.4 & 4.00 & 5.06 & 4.92 & 0.00 & 3.24 & 5.36 & 6.48 \\ 
   &  &  & 188.2 & 5.00 & 6.45 & 5.11 & 2.48 & 4.46 & 5.63 & 6.14 \\ 
   &  &  & 183 & 7.00 & 6.15 & 5.66 & 4.29 & 5.34 & 6.02 & 6.41 \\ 
   &  &  & 177.8 & 9.00 & 6.42 & 6.04 & 5.02 & 5.77 & 6.26 & 6.69 \\ 
   &  &  & 172.6 & 11.00 & 6.62 & 6.17 & 5.29 & 5.90 & 6.34 & 6.81 \\ 
   &  &  & 167.4 & 17.00 & 7.14 & 6.77 & 6.13 & 6.59 & 6.88 & 7.09 \\ 
   &  &  & 162.2 & 24.00 & 7.36 & 7.14 & 6.84 & 7.06 & 7.23 & 7.36 \\ 
   &  &  & 157 & 20.00 & 7.42 & 7.09 & 6.71 & 6.96 & 7.20 & 7.36 \\ 
   &  &  & 151.8 & 20.00 & 7.28 & 6.98 & 6.53 & 6.86 & 7.06 & 7.25 \\ 
   &  &  & 146.6 & 21.00 & 7.36 & 7.06 & 6.70 & 6.97 & 7.15 & 7.32 \\ 
   &  &  & 141.4 & 17.00 & 7.23 & 6.91 & 6.47 & 6.79 & 7.03 & 7.21 \\ 
   &  &  & 136.2 & 23.00 & 7.58 & 7.19 & 6.85 & 7.07 & 7.25 & 7.39 \\ 
   &  &  & 131 & 29.00 & 7.63 & 7.29 & 7.02 & 7.21 & 7.38 & 7.52 \\ 
   &  &  & 125.8 & 31.00 & 7.70 & 7.47 & 7.15 & 7.38 & 7.55 & 7.64 \\ 
   &  &  & 120.6 & 29.00 & 7.76 & 7.50 & 7.19 & 7.41 & 7.58 & 7.68 \\ 
   &  &  & 115.4 & 26.00 & 7.71 & 7.50 & 7.20 & 7.39 & 7.56 & 7.67 \\ 
   &  &  & 110.2 & 19.00 & 7.43 & 7.16 & 6.80 & 7.03 & 7.28 & 7.41 \\ 
   &  &  & 105 & 20.00 & 7.66 & 7.22 & 6.89 & 7.10 & 7.32 & 7.50 \\ 
   &  &  & 99.8 & 21.00 & 7.71 & 7.46 & 7.17 & 7.35 & 7.58 & 7.73 \\ 
   &  &  & 94.6 & 25.00 & 7.84 & 7.52 & 7.15 & 7.43 & 7.60 & 7.76 \\ 
   &  &  & 89.4 & 30.00 & 7.95 & 7.71 & 7.38 & 7.63 & 7.77 & 7.89 \\ 
   &  &  & 84.2 & 27.00 & 8.06 & 7.76 & 7.53 & 7.68 & 7.84 & 7.94 \\ 
   &  &  & 79 & 39.00 & 8.03 & 7.84 & 7.58 & 7.78 & 7.90 & 7.98 \\ 
   &  &  & 73.8 & 29.00 & 8.15 & 7.86 & 7.61 & 7.79 & 7.93 & 8.03 \\ 
   &  &  & 68.6 & 22.00 & 8.20 & 7.85 & 7.45 & 7.72 & 7.94 & 8.08 \\ 
   &  & gradual.split & 198.6 & 3.00 & 5.81 & 3.87 & 0.00 & 3.87 & 4.45 & 6.12 \\ 
   &  &  & 193.4 & 4.00 & 5.06 & 4.49 & 2.90 & 3.31 & 5.06 & 6.20 \\ 
   &  &  & 188.2 & 5.00 & 6.27 & 5.23 & 3.91 & 4.39 & 5.59 & 6.26 \\ 
   &  &  & 183 & 7.00 & 5.97 & 5.52 & 3.78 & 5.08 & 5.79 & 6.42 \\ 
   &  &  & 177.8 & 9.00 & 6.37 & 5.83 & 4.77 & 5.51 & 6.02 & 6.45 \\ 
   &  &  & 172.6 & 11.00 & 6.58 & 6.09 & 5.00 & 5.78 & 6.28 & 6.77 \\ 
   &  &  & 167.4 & 17.00 & 6.99 & 6.68 & 6.06 & 6.53 & 6.90 & 7.11 \\ 
   &  &  & 162.2 & 24.00 & 7.47 & 7.17 & 6.83 & 7.07 & 7.27 & 7.41 \\ 
   &  &  & 157 & 20.00 & 7.58 & 7.14 & 6.78 & 7.03 & 7.27 & 7.42 \\ 
   &  &  & 151.8 & 20.00 & 7.24 & 6.92 & 6.56 & 6.82 & 7.02 & 7.13 \\ 
   &  &  & 146.6 & 21.00 & 7.47 & 7.01 & 6.64 & 6.85 & 7.12 & 7.24 \\ 
   &  &  & 141.4 & 17.00 & 7.34 & 6.84 & 6.46 & 6.72 & 6.96 & 7.19 \\ 
   &  &  & 136.2 & 23.00 & 7.44 & 7.12 & 6.81 & 7.05 & 7.21 & 7.33 \\ 
   &  &  & 131 & 29.00 & 7.55 & 7.32 & 7.09 & 7.25 & 7.38 & 7.49 \\ 
   &  &  & 125.8 & 31.00 & 7.74 & 7.47 & 7.25 & 7.41 & 7.52 & 7.63 \\ 
   &  &  & 120.6 & 29.00 & 7.73 & 7.48 & 7.16 & 7.37 & 7.56 & 7.68 \\ 
   &  &  & 115.4 & 26.00 & 7.84 & 7.45 & 7.22 & 7.38 & 7.55 & 7.67 \\ 
   &  &  & 110.2 & 19.00 & 7.49 & 7.12 & 6.75 & 6.98 & 7.24 & 7.43 \\ 
   &  &  & 105 & 20.00 & 7.55 & 7.18 & 6.85 & 7.04 & 7.28 & 7.46 \\ 
   &  &  & 99.8 & 21.00 & 7.67 & 7.37 & 6.84 & 7.24 & 7.50 & 7.65 \\ 
   &  &  & 94.6 & 25.00 & 7.90 & 7.46 & 7.14 & 7.35 & 7.54 & 7.75 \\ 
   &  &  & 89.4 & 30.00 & 7.98 & 7.69 & 7.47 & 7.63 & 7.77 & 7.89 \\ 
   &  &  & 84.2 & 27.00 & 8.10 & 7.77 & 7.49 & 7.69 & 7.83 & 7.91 \\ 
   &  &  & 79 & 39.00 & 8.05 & 7.83 & 7.60 & 7.77 & 7.89 & 7.96 \\ 
   &  &  & 73.8 & 29.00 & 8.18 & 7.91 & 7.62 & 7.83 & 7.96 & 8.04 \\ 
   &  &  & 68.6 & 22.00 & 8.20 & 7.85 & 7.55 & 7.78 & 7.96 & 8.07 \\ 
   & number & equalbins & 203.07 - 197.11 & 0.00 &  &  &  &  &  &  \\ 
   &  &  & 197.11 - 191.15 & 1.00 &  &  &  &  &  &  \\ 
   &  &  & 191.15 - 185.19 & 2.00 & 5.81 & 5.81 & 0.00 & 0.00 & 5.81 & 5.81 \\ 
   &  &  & 185.19 - 179.23 & 2.00 & 6.17 & 0.00 & 0.00 & 0.00 & 6.17 & 6.17 \\ 
   &  &  & 179.23 - 173.27 & 2.00 & 6.59 & 6.59 & 0.00 & 0.00 & 6.59 & 6.59 \\ 
   &  &  & 173.27 - 167.31 & 7.00 & 7.06 & 6.19 & 4.87 & 5.79 & 6.45 & 6.78 \\ 
   &  &  & 167.31 - 161.35 & 9.00 & 7.71 & 6.98 & 6.08 & 6.67 & 7.19 & 7.55 \\ 
   &  &  & 161.35 - 155.39 & 14.00 & 7.78 & 7.30 & 6.69 & 7.07 & 7.41 & 7.65 \\ 
   &  &  & 155.39 - 149.43 & 12.00 & 7.70 & 7.10 & 6.34 & 6.87 & 7.25 & 7.47 \\ 
   &  &  & 149.43 - 143.48 & 10.00 & 7.54 & 6.93 & 6.10 & 6.65 & 7.09 & 7.29 \\ 
   &  &  & 143.48 - 137.52 & 5.00 & 7.48 & 6.01 & 3.07 & 5.23 & 6.66 & 7.17 \\ 
   &  &  & 137.52 - 131.56 & 10.00 & 7.68 & 6.99 & 6.34 & 6.77 & 7.16 & 7.41 \\ 
   &  &  & 131.56 - 125.6 & 30.00 & 7.97 & 7.71 & 7.56 & 7.66 & 7.78 & 7.87 \\ 
   &  &  & 125.6 - 119.64 & 16.00 & 7.84 & 7.33 & 6.84 & 7.20 & 7.46 & 7.60 \\ 
   &  &  & 119.64 - 113.68 & 14.00 & 7.93 & 7.35 & 6.81 & 7.19 & 7.52 & 7.69 \\ 
   &  &  & 113.68 - 107.72 & 16.00 & 7.87 & 7.34 & 6.92 & 7.22 & 7.51 & 7.68 \\ 
   &  &  & 107.72 - 101.76 & 9.00 & 7.48 & 6.73 & 5.89 & 6.50 & 7.02 & 7.32 \\ 
   &  &  & 101.76 - 95.8 & 15.00 & 7.93 & 7.43 & 6.93 & 7.26 & 7.60 & 7.75 \\ 
   &  &  & 95.8 - 89.84 & 18.00 & 7.90 & 7.49 & 7.05 & 7.36 & 7.66 & 7.80 \\ 
   &  &  & 89.84 - 83.88 & 17.00 & 8.16 & 7.71 & 7.26 & 7.55 & 7.84 & 8.05 \\ 
   &  &  & 83.88 - 77.92 & 38.00 & 8.10 & 7.90 & 7.74 & 7.84 & 7.96 & 8.01 \\ 
   &  &  & 77.92 - 71.96 & 28.00 & 8.17 & 7.91 & 7.58 & 7.83 & 7.96 & 8.09 \\ 
   &  &  & 71.96 - 66 & 20.00 & 8.28 & 7.86 & 7.49 & 7.77 & 7.98 & 8.12 \\ 
   &  & acctran & 203.07 & 3.00 & 6.54 & 4.32 & 0.00 & 4.31 & 4.45 & 6.54 \\ 
   &  &  & 196.84 & 3.00 & 6.54 & 4.32 & 0.00 & 4.31 & 4.45 & 6.54 \\ 
   &  &  & 190.61 & 5.00 & 6.74 & 5.49 & 2.68 & 4.78 & 6.08 & 6.74 \\ 
   &  &  & 184.38 & 7.00 & 6.96 & 6.04 & 4.69 & 5.72 & 6.32 & 6.72 \\ 
   &  &  & 178.15 & 8.00 & 7.01 & 6.19 & 5.11 & 5.86 & 6.41 & 6.72 \\ 
   &  &  & 171.92 & 12.00 & 7.27 & 6.71 & 6.25 & 6.55 & 6.81 & 7.02 \\ 
   &  &  & 165.69 & 17.00 & 7.62 & 7.23 & 6.84 & 7.10 & 7.34 & 7.49 \\ 
   &  &  & 159.46 & 18.00 & 7.64 & 7.22 & 6.82 & 7.08 & 7.35 & 7.49 \\ 
   &  &  & 153.23 & 18.00 & 7.54 & 7.12 & 6.80 & 7.01 & 7.23 & 7.41 \\ 
   &  &  & 147 & 21.00 & 7.58 & 7.22 & 6.88 & 7.12 & 7.31 & 7.44 \\ 
   &  &  & 140.77 & 17.00 & 7.58 & 7.14 & 6.75 & 7.03 & 7.24 & 7.40 \\ 
   &  &  & 134.54 & 24.00 & 7.76 & 7.45 & 7.16 & 7.37 & 7.51 & 7.60 \\ 
   &  &  & 128.31 & 33.00 & 7.94 & 7.69 & 7.47 & 7.62 & 7.75 & 7.87 \\ 
   &  &  & 122.07 & 23.00 & 7.90 & 7.58 & 7.12 & 7.48 & 7.64 & 7.74 \\ 
   &  &  & 115.84 & 24.00 & 7.91 & 7.60 & 7.16 & 7.49 & 7.68 & 7.79 \\ 
   &  &  & 109.61 & 16.00 & 7.81 & 7.35 & 6.99 & 7.18 & 7.47 & 7.68 \\ 
   &  &  & 103.38 & 19.00 & 7.84 & 7.46 & 7.05 & 7.32 & 7.54 & 7.72 \\ 
   &  &  & 97.15 & 20.00 & 8.11 & 7.73 & 7.35 & 7.62 & 7.83 & 8.03 \\ 
   &  &  & 90.92 & 25.00 & 8.16 & 7.85 & 7.63 & 7.77 & 7.93 & 8.05 \\ 
   &  &  & 84.69 & 25.00 & 8.20 & 7.90 & 7.61 & 7.80 & 7.96 & 8.04 \\ 
   &  &  & 78.46 & 26.00 & 8.19 & 7.91 & 7.63 & 7.82 & 7.98 & 8.07 \\ 
   &  &  & 72.23 & 27.00 & 8.20 & 7.92 & 7.60 & 7.83 & 7.97 & 8.07 \\ 
   &  &  & 66 & 1.00 &  &  &  &  &  &  \\ 
   &  & deltran & 203.07 & 2.00 & 5.81 & 2.90 & 0.00 & 0.00 & 5.81 & 5.81 \\ 
   &  &  & 196.84 & 2.00 & 5.81 & 0.00 & 0.00 & 0.00 & 5.81 & 5.81 \\ 
   &  &  & 190.61 & 4.00 & 6.01 & 4.84 & 2.90 & 3.87 & 5.07 & 6.01 \\ 
   &  &  & 184.38 & 6.00 & 6.08 & 5.24 & 3.71 & 4.93 & 5.64 & 5.76 \\ 
   &  &  & 178.15 & 7.00 & 6.20 & 5.41 & 3.70 & 5.09 & 5.65 & 5.94 \\ 
   &  &  & 171.92 & 9.00 & 6.45 & 5.76 & 5.01 & 5.58 & 5.98 & 6.27 \\ 
   &  &  & 165.69 & 15.00 & 7.12 & 6.63 & 6.05 & 6.43 & 6.82 & 7.08 \\ 
   &  &  & 159.46 & 16.00 & 7.26 & 6.78 & 6.23 & 6.64 & 6.98 & 7.19 \\ 
   &  &  & 153.23 & 15.00 & 7.27 & 6.83 & 6.26 & 6.67 & 6.98 & 7.17 \\ 
   &  &  & 147 & 18.00 & 7.28 & 6.93 & 6.52 & 6.79 & 7.02 & 7.18 \\ 
   &  &  & 140.77 & 12.00 & 7.06 & 6.51 & 5.92 & 6.33 & 6.67 & 6.83 \\ 
   &  &  & 134.54 & 18.00 & 7.30 & 6.91 & 6.51 & 6.76 & 7.01 & 7.17 \\ 
   &  &  & 128.31 & 31.00 & 7.70 & 7.47 & 7.18 & 7.39 & 7.52 & 7.65 \\ 
   &  &  & 122.07 & 23.00 & 7.69 & 7.36 & 6.95 & 7.23 & 7.49 & 7.64 \\ 
   &  &  & 115.84 & 24.00 & 7.70 & 7.43 & 7.04 & 7.29 & 7.48 & 7.62 \\ 
   &  &  & 109.61 & 15.00 & 7.48 & 7.03 & 6.48 & 6.88 & 7.15 & 7.31 \\ 
   &  &  & 103.38 & 16.00 & 7.46 & 7.01 & 6.60 & 6.88 & 7.15 & 7.31 \\ 
   &  &  & 97.15 & 16.00 & 7.79 & 7.36 & 6.95 & 7.23 & 7.47 & 7.65 \\ 
   &  &  & 90.92 & 23.00 & 7.89 & 7.56 & 7.26 & 7.46 & 7.63 & 7.79 \\ 
   &  &  & 84.69 & 23.00 & 7.96 & 7.60 & 7.34 & 7.52 & 7.68 & 7.82 \\ 
   &  &  & 78.46 & 26.00 & 8.12 & 7.82 & 7.63 & 7.74 & 7.88 & 8.00 \\ 
   &  &  & 72.23 & 27.00 & 8.10 & 7.85 & 7.50 & 7.73 & 7.90 & 7.98 \\ 
   &  &  & 66 & 1.00 &  &  &  &  &  &  \\ 
   &  & random & 203.07 & 2.00 & 6.11 & 6.11 & 0.00 & 0.00 & 6.11 & 6.11 \\ 
   &  &  & 196.84 & 3.00 & 5.81 & 3.87 & 0.00 & 3.87 & 5.81 & 5.81 \\ 
   &  &  & 190.61 & 5.00 & 6.42 & 5.16 & 3.73 & 4.69 & 5.71 & 6.04 \\ 
   &  &  & 184.38 & 7.00 & 6.58 & 5.74 & 4.90 & 5.44 & 5.94 & 6.28 \\ 
   &  &  & 178.15 & 8.00 & 6.92 & 6.14 & 5.02 & 5.88 & 6.35 & 6.64 \\ 
   &  &  & 171.92 & 12.00 & 7.16 & 6.63 & 6.18 & 6.46 & 6.78 & 7.02 \\ 
   &  &  & 165.69 & 17.00 & 7.44 & 7.01 & 6.41 & 6.87 & 7.15 & 7.35 \\ 
   &  &  & 159.46 & 17.00 & 7.39 & 6.96 & 6.56 & 6.80 & 7.12 & 7.31 \\ 
   &  &  & 153.23 & 18.00 & 7.34 & 6.92 & 6.43 & 6.84 & 7.03 & 7.26 \\ 
   &  &  & 147 & 21.00 & 7.44 & 7.12 & 6.79 & 7.00 & 7.21 & 7.34 \\ 
   &  &  & 140.77 & 16.00 & 7.37 & 6.93 & 6.55 & 6.83 & 7.06 & 7.20 \\ 
   &  &  & 134.54 & 23.00 & 7.60 & 7.29 & 7.04 & 7.20 & 7.36 & 7.51 \\ 
   &  &  & 128.31 & 33.00 & 7.81 & 7.60 & 7.41 & 7.53 & 7.67 & 7.77 \\ 
   &  &  & 122.07 & 23.00 & 7.88 & 7.54 & 7.17 & 7.44 & 7.62 & 7.75 \\ 
   &  &  & 115.84 & 24.00 & 7.85 & 7.52 & 7.24 & 7.44 & 7.61 & 7.74 \\ 
   &  &  & 109.61 & 16.00 & 7.68 & 7.24 & 6.77 & 7.09 & 7.34 & 7.52 \\ 
   &  &  & 103.38 & 19.00 & 7.69 & 7.31 & 6.88 & 7.19 & 7.43 & 7.55 \\ 
   &  &  & 97.15 & 20.00 & 7.98 & 7.59 & 7.26 & 7.49 & 7.71 & 7.86 \\ 
   &  &  & 90.92 & 25.00 & 8.15 & 7.83 & 7.56 & 7.74 & 7.92 & 8.04 \\ 
   &  &  & 84.69 & 25.00 & 8.03 & 7.72 & 7.40 & 7.62 & 7.80 & 7.93 \\ 
   &  &  & 78.46 & 26.00 & 8.19 & 7.90 & 7.70 & 7.83 & 7.95 & 8.04 \\ 
   &  &  & 72.23 & 27.00 & 8.17 & 7.90 & 7.59 & 7.81 & 7.96 & 8.06 \\ 
   &  &  & 66 & 1.00 &  &  &  &  &  &  \\ 
   &  & proximity & 203.07 & 2.00 & 5.81 & 5.81 & 0.00 & 0.00 & 5.81 & 5.81 \\ 
   &  &  & 196.84 & 3.00 & 5.81 & 3.87 & 0.00 & 3.87 & 5.81 & 5.81 \\ 
   &  &  & 190.61 & 4.00 & 6.01 & 4.84 & 2.90 & 3.87 & 5.08 & 6.01 \\ 
   &  &  & 184.38 & 6.00 & 6.08 & 5.22 & 3.48 & 4.59 & 5.34 & 5.76 \\ 
   &  &  & 178.15 & 8.00 & 6.73 & 6.05 & 4.90 & 5.69 & 6.27 & 6.57 \\ 
   &  &  & 171.92 & 10.00 & 6.82 & 6.24 & 5.23 & 6.01 & 6.37 & 6.63 \\ 
   &  &  & 165.69 & 16.00 & 7.45 & 6.98 & 6.46 & 6.81 & 7.15 & 7.33 \\ 
   &  &  & 159.46 & 18.00 & 7.50 & 7.12 & 6.63 & 6.94 & 7.25 & 7.37 \\ 
   &  &  & 153.23 & 16.00 & 7.42 & 6.99 & 6.57 & 6.83 & 7.09 & 7.22 \\ 
   &  &  & 147 & 18.00 & 7.41 & 6.99 & 6.64 & 6.88 & 7.10 & 7.26 \\ 
   &  &  & 140.77 & 14.00 & 7.25 & 6.76 & 6.12 & 6.62 & 6.89 & 7.08 \\ 
   &  &  & 134.54 & 21.00 & 7.52 & 7.16 & 6.87 & 7.07 & 7.27 & 7.39 \\ 
   &  &  & 128.31 & 31.00 & 7.77 & 7.54 & 7.31 & 7.48 & 7.62 & 7.70 \\ 
   &  &  & 122.07 & 23.00 & 7.77 & 7.42 & 7.08 & 7.32 & 7.54 & 7.66 \\ 
   &  &  & 115.84 & 24.00 & 7.79 & 7.46 & 7.20 & 7.39 & 7.55 & 7.65 \\ 
   &  &  & 109.61 & 15.00 & 7.73 & 7.25 & 6.77 & 7.12 & 7.36 & 7.55 \\ 
   &  &  & 103.38 & 17.00 & 7.70 & 7.28 & 6.64 & 7.13 & 7.37 & 7.53 \\ 
   &  &  & 97.15 & 17.00 & 7.86 & 7.43 & 6.98 & 7.34 & 7.57 & 7.71 \\ 
   &  &  & 90.92 & 24.00 & 8.05 & 7.73 & 7.42 & 7.66 & 7.82 & 7.92 \\ 
   &  &  & 84.69 & 23.00 & 8.16 & 7.83 & 7.63 & 7.77 & 7.88 & 7.97 \\ 
   &  &  & 78.46 & 26.00 & 8.15 & 7.85 & 7.59 & 7.76 & 7.91 & 8.02 \\ 
   &  &  & 72.23 & 27.00 & 8.19 & 7.92 & 7.62 & 7.81 & 7.99 & 8.07 \\ 
   &  &  & 66 & 1.00 &  &  &  &  &  &  \\ 
   &  & equal.split & 203.07 & 3.00 & 5.81 & 3.87 & 0.00 & 3.87 & 4.31 & 6.13 \\ 
   &  &  & 196.84 & 3.00 & 4.07 & 3.87 & 0.00 & 3.87 & 4.31 & 6.54 \\ 
   &  &  & 190.61 & 5.00 & 5.98 & 5.15 & 2.55 & 4.60 & 5.66 & 6.35 \\ 
   &  &  & 184.38 & 7.00 & 6.29 & 5.56 & 4.20 & 5.20 & 5.84 & 6.36 \\ 
   &  &  & 178.15 & 8.00 & 6.81 & 5.82 & 4.45 & 5.55 & 6.04 & 6.30 \\ 
   &  &  & 171.92 & 12.00 & 6.78 & 6.33 & 5.54 & 6.14 & 6.52 & 6.72 \\ 
   &  &  & 165.69 & 19.00 & 7.19 & 6.89 & 6.38 & 6.75 & 7.07 & 7.26 \\ 
   &  &  & 159.46 & 21.00 & 7.47 & 7.08 & 6.82 & 6.99 & 7.21 & 7.42 \\ 
   &  &  & 153.23 & 19.00 & 7.34 & 6.96 & 6.56 & 6.85 & 7.09 & 7.27 \\ 
   &  &  & 147 & 21.00 & 7.48 & 7.07 & 6.71 & 6.92 & 7.15 & 7.28 \\ 
   &  &  & 140.77 & 17.00 & 7.30 & 6.94 & 6.46 & 6.81 & 7.06 & 7.23 \\ 
   &  &  & 134.54 & 24.00 & 7.50 & 7.20 & 6.94 & 7.12 & 7.31 & 7.44 \\ 
   &  &  & 128.31 & 39.00 & 7.80 & 7.60 & 7.47 & 7.56 & 7.66 & 7.75 \\ 
   &  &  & 122.07 & 32.00 & 7.71 & 7.49 & 7.28 & 7.43 & 7.55 & 7.67 \\ 
   &  &  & 115.84 & 26.00 & 7.80 & 7.49 & 7.18 & 7.41 & 7.58 & 7.67 \\ 
   &  &  & 109.61 & 19.00 & 7.58 & 7.16 & 6.52 & 7.01 & 7.29 & 7.49 \\ 
   &  &  & 103.38 & 19.00 & 7.62 & 7.24 & 6.88 & 7.11 & 7.36 & 7.52 \\ 
   &  &  & 97.15 & 23.00 & 7.89 & 7.49 & 7.13 & 7.38 & 7.63 & 7.79 \\ 
   &  &  & 90.92 & 32.00 & 8.00 & 7.69 & 7.46 & 7.63 & 7.77 & 7.88 \\ 
   &  &  & 84.69 & 27.00 & 8.03 & 7.77 & 7.49 & 7.66 & 7.84 & 7.94 \\ 
   &  &  & 78.46 & 38.00 & 8.04 & 7.82 & 7.59 & 7.79 & 7.88 & 7.97 \\ 
   &  &  & 72.23 & 28.00 & 8.09 & 7.84 & 7.43 & 7.74 & 7.92 & 8.02 \\ 
   &  &  & 66 & 1.00 &  &  &  &  &  &  \\ 
   &  & gradual.split & 203.07 & 3.00 & 5.81 & 3.87 & 0.00 & 0.00 & 3.87 & 4.27 \\ 
   &  &  & 196.84 & 3.00 & 5.81 & 3.87 & 0.00 & 3.87 & 4.07 & 6.03 \\ 
   &  &  & 190.61 & 5.00 & 5.67 & 4.83 & 2.36 & 4.17 & 5.40 & 5.95 \\ 
   &  &  & 184.38 & 7.00 & 6.29 & 5.38 & 3.44 & 4.98 & 5.67 & 6.26 \\ 
   &  &  & 178.15 & 8.00 & 6.63 & 5.81 & 4.97 & 5.50 & 6.12 & 6.44 \\ 
   &  &  & 171.92 & 12.00 & 6.72 & 6.21 & 5.54 & 5.97 & 6.37 & 6.69 \\ 
   &  &  & 165.69 & 19.00 & 7.45 & 6.95 & 6.37 & 6.75 & 7.06 & 7.30 \\ 
   &  &  & 159.46 & 21.00 & 7.52 & 7.14 & 6.78 & 7.00 & 7.24 & 7.44 \\ 
   &  &  & 153.23 & 19.00 & 7.29 & 6.90 & 6.42 & 6.75 & 7.07 & 7.19 \\ 
   &  &  & 147 & 21.00 & 7.37 & 7.03 & 6.56 & 6.89 & 7.10 & 7.24 \\ 
   &  &  & 140.77 & 17.00 & 7.28 & 6.89 & 6.17 & 6.73 & 6.97 & 7.14 \\ 
   &  &  & 134.54 & 24.00 & 7.39 & 7.19 & 6.92 & 7.09 & 7.27 & 7.39 \\ 
   &  &  & 128.31 & 39.00 & 7.75 & 7.59 & 7.40 & 7.54 & 7.63 & 7.72 \\ 
   &  &  & 122.07 & 32.00 & 7.70 & 7.50 & 7.27 & 7.43 & 7.55 & 7.66 \\ 
   &  &  & 115.84 & 26.00 & 7.74 & 7.48 & 7.22 & 7.39 & 7.56 & 7.71 \\ 
   &  &  & 109.61 & 19.00 & 7.52 & 7.16 & 6.73 & 7.02 & 7.27 & 7.45 \\ 
   &  &  & 103.38 & 19.00 & 7.63 & 7.26 & 6.78 & 7.15 & 7.38 & 7.53 \\ 
   &  &  & 97.15 & 23.00 & 7.83 & 7.47 & 7.12 & 7.36 & 7.55 & 7.66 \\ 
   &  &  & 90.92 & 32.00 & 7.94 & 7.71 & 7.38 & 7.61 & 7.76 & 7.86 \\ 
   &  &  & 84.69 & 27.00 & 8.03 & 7.78 & 7.41 & 7.69 & 7.85 & 7.94 \\ 
   &  &  & 78.46 & 38.00 & 8.02 & 7.82 & 7.58 & 7.75 & 7.87 & 7.95 \\ 
   &  &  & 72.23 & 28.00 & 8.15 & 7.86 & 7.52 & 7.78 & 7.92 & 8.03 \\ 
   &  &  & 66 & 1.00 &  &  &  &  &  &  \\ 
  Epoch & stratigraphy & equalbins & 201.3 - 174.1 & 7.00 & 6.18 & 5.29 & 4.34 & 5.03 & 5.60 & 5.90 \\ 
   &  &  & 174.1 - 163.5 & 13.00 & 7.42 & 6.86 & 6.23 & 6.68 & 7.03 & 7.29 \\ 
   &  &  & 163.5 - 145 & 24.00 & 7.63 & 7.34 & 7.04 & 7.26 & 7.44 & 7.53 \\ 
   &  &  & 145 - 100.5 & 61.00 & 7.79 & 7.67 & 7.56 & 7.64 & 7.70 & 7.77 \\ 
   &  &  & 100.5 - 66 & 68.00 & 8.05 & 7.94 & 7.82 & 7.89 & 7.97 & 8.02 \\ 
   &  & acctran & 187.7 & 6.00 & 6.79 & 5.90 & 3.80 & 5.52 & 6.17 & 6.46 \\ 
   &  &  & 168.8 & 15.00 & 7.39 & 6.90 & 6.40 & 6.78 & 7.02 & 7.17 \\ 
   &  &  & 154.25 & 18.00 & 7.54 & 7.14 & 6.67 & 7.04 & 7.25 & 7.42 \\ 
   &  &  & 122.75 & 23.00 & 7.90 & 7.56 & 7.22 & 7.46 & 7.67 & 7.82 \\ 
   &  &  & 83.25 & 26.00 & 8.17 & 7.89 & 7.55 & 7.80 & 7.94 & 8.06 \\ 
   &  & deltran & 187.7 & 5.00 & 6.06 & 4.87 & 2.41 & 4.35 & 5.44 & 6.06 \\ 
   &  &  & 168.8 & 11.00 & 6.60 & 6.05 & 5.52 & 5.88 & 6.21 & 6.41 \\ 
   &  &  & 154.25 & 15.00 & 7.27 & 6.83 & 6.29 & 6.63 & 6.99 & 7.17 \\ 
   &  &  & 122.75 & 23.00 & 7.69 & 7.42 & 7.03 & 7.30 & 7.51 & 7.62 \\ 
   &  &  & 83.25 & 26.00 & 8.09 & 7.79 & 7.53 & 7.69 & 7.87 & 7.97 \\ 
   &  & random & 187.7 & 6.00 & 6.40 & 5.49 & 3.70 & 5.09 & 5.82 & 6.16 \\ 
   &  &  & 168.8 & 15.00 & 7.16 & 6.63 & 6.08 & 6.49 & 6.80 & 7.03 \\ 
   &  &  & 154.25 & 18.00 & 7.42 & 7.06 & 6.67 & 6.93 & 7.16 & 7.30 \\ 
   &  &  & 122.75 & 23.00 & 7.75 & 7.43 & 7.12 & 7.31 & 7.49 & 7.63 \\ 
   &  &  & 83.25 & 26.00 & 8.09 & 7.78 & 7.42 & 7.67 & 7.87 & 7.96 \\ 
   &  & proximity & 187.7 & 5.00 & 6.06 & 4.94 & 2.98 & 4.36 & 5.45 & 6.06 \\ 
   &  &  & 168.8 & 11.00 & 6.96 & 6.34 & 5.49 & 6.08 & 6.56 & 6.80 \\ 
   &  &  & 154.25 & 15.00 & 7.41 & 6.90 & 6.43 & 6.76 & 7.07 & 7.22 \\ 
   &  &  & 122.75 & 23.00 & 7.77 & 7.46 & 7.13 & 7.36 & 7.56 & 7.64 \\ 
   &  &  & 83.25 & 26.00 & 8.14 & 7.84 & 7.57 & 7.75 & 7.91 & 8.03 \\ 
   &  & equal.split & 187.7 & 6.00 & 6.31 & 5.43 & 3.93 & 4.85 & 5.83 & 6.32 \\ 
   &  &  & 168.8 & 15.00 & 7.00 & 6.57 & 5.85 & 6.34 & 6.74 & 6.96 \\ 
   &  &  & 154.25 & 19.00 & 7.28 & 6.97 & 6.47 & 6.80 & 7.10 & 7.29 \\ 
   &  &  & 122.75 & 33.00 & 7.72 & 7.49 & 7.25 & 7.41 & 7.54 & 7.69 \\ 
   &  &  & 83.25 & 38.00 & 7.99 & 7.82 & 7.55 & 7.76 & 7.85 & 7.93 \\ 
   &  & gradual.split & 187.7 & 6.00 & 5.83 & 4.94 & 3.64 & 4.55 & 5.37 & 6.07 \\ 
   &  &  & 168.8 & 15.00 & 6.65 & 6.36 & 5.63 & 6.13 & 6.53 & 6.79 \\ 
   &  &  & 154.25 & 19.00 & 7.20 & 6.89 & 6.45 & 6.73 & 7.06 & 7.26 \\ 
   &  &  & 122.75 & 33.00 & 7.73 & 7.50 & 7.28 & 7.41 & 7.55 & 7.66 \\ 
   &  &  & 83.25 & 38.00 & 7.99 & 7.81 & 7.60 & 7.74 & 7.86 & 7.96 \\ 
   & duration & equalbins & 174.8 - 147.6 & 30.00 & 7.45 & 7.22 & 6.98 & 7.15 & 7.29 & 7.44 \\ 
   &  &  & 147.6 - 120.4 & 52.00 & 7.83 & 7.68 & 7.54 & 7.63 & 7.71 & 7.80 \\ 
   &  &  & 120.4 - 93.2 & 40.00 & 7.89 & 7.70 & 7.47 & 7.64 & 7.75 & 7.82 \\ 
   &  &  & 93.2 - 66 & 56.00 & 8.10 & 7.96 & 7.83 & 7.92 & 7.99 & 8.05 \\ 
   &  & acctran & 161.2 & 18.00 & 7.64 & 7.23 & 6.94 & 7.11 & 7.35 & 7.52 \\ 
   &  &  & 134 & 24.00 & 7.76 & 7.46 & 7.23 & 7.38 & 7.52 & 7.62 \\ 
   &  &  & 106.8 & 16.00 & 7.81 & 7.37 & 6.95 & 7.26 & 7.48 & 7.62 \\ 
   &  &  & 79.6 & 26.00 & 8.19 & 7.89 & 7.55 & 7.82 & 7.95 & 8.04 \\ 
   &  & deltran & 161.2 & 16.00 & 7.26 & 6.82 & 6.39 & 6.69 & 7.01 & 7.24 \\ 
   &  &  & 134 & 18.00 & 7.30 & 6.91 & 6.46 & 6.74 & 7.01 & 7.13 \\ 
   &  &  & 106.8 & 15.00 & 7.48 & 6.97 & 6.36 & 6.79 & 7.10 & 7.36 \\ 
   &  &  & 79.6 & 26.00 & 8.12 & 7.80 & 7.51 & 7.72 & 7.88 & 7.99 \\ 
   &  & random & 161.2 & 18.00 & 7.45 & 7.03 & 6.61 & 6.93 & 7.15 & 7.37 \\ 
   &  &  & 134 & 23.00 & 7.66 & 7.32 & 6.98 & 7.23 & 7.37 & 7.50 \\ 
   &  &  & 106.8 & 16.00 & 7.72 & 7.22 & 6.80 & 7.12 & 7.33 & 7.51 \\ 
   &  &  & 79.6 & 26.00 & 8.15 & 7.88 & 7.52 & 7.78 & 7.93 & 7.99 \\ 
   &  & proximity & 161.2 & 17.00 & 7.53 & 7.13 & 6.73 & 7.02 & 7.21 & 7.42 \\ 
   &  &  & 134 & 22.00 & 7.54 & 7.24 & 6.96 & 7.16 & 7.31 & 7.41 \\ 
   &  &  & 106.8 & 16.00 & 7.70 & 7.24 & 6.84 & 7.10 & 7.35 & 7.52 \\ 
   &  &  & 79.6 & 26.00 & 8.15 & 7.86 & 7.51 & 7.75 & 7.92 & 8.01 \\ 
   &  & equal.split & 161.2 & 24.00 & 7.52 & 7.20 & 6.81 & 7.07 & 7.29 & 7.36 \\ 
   &  &  & 134 & 24.00 & 7.60 & 7.22 & 6.94 & 7.10 & 7.30 & 7.43 \\ 
   &  &  & 106.8 & 18.00 & 7.61 & 7.15 & 6.57 & 6.98 & 7.31 & 7.54 \\ 
   &  &  & 79.6 & 42.00 & 8.03 & 7.84 & 7.65 & 7.77 & 7.89 & 7.97 \\ 
   &  & gradual.split & 161.2 & 24.00 & 7.54 & 7.16 & 6.85 & 7.06 & 7.25 & 7.41 \\ 
   &  &  & 134 & 24.00 & 7.54 & 7.23 & 6.94 & 7.11 & 7.30 & 7.43 \\ 
   &  &  & 106.8 & 18.00 & 7.62 & 7.13 & 6.33 & 6.99 & 7.23 & 7.41 \\ 
   &  &  & 79.6 & 42.00 & 8.04 & 7.85 & 7.63 & 7.78 & 7.91 & 7.95 \\ 
   & number & equalbins & 203.07 - 175.66 & 6.00 & 6.16 & 5.31 & 4.11 & 4.96 & 5.47 & 5.82 \\ 
   &  &  & 175.66 - 148.24 & 31.00 & 7.43 & 7.19 & 6.96 & 7.12 & 7.27 & 7.39 \\ 
   &  &  & 148.24 - 120.83 & 52.00 & 7.83 & 7.68 & 7.56 & 7.64 & 7.73 & 7.80 \\ 
   &  &  & 120.83 - 93.41 & 40.00 & 7.89 & 7.70 & 7.54 & 7.64 & 7.74 & 7.83 \\ 
   &  &  & 93.41 - 66 & 56.00 & 8.10 & 7.96 & 7.82 & 7.91 & 8.00 & 8.05 \\ 
   &  & acctran & 203.07 & 3.00 & 6.54 & 4.32 & 0.00 & 4.31 & 4.45 & 6.54 \\ 
   &  &  & 168.8 & 15.00 & 7.39 & 6.91 & 6.53 & 6.76 & 7.05 & 7.20 \\ 
   &  &  & 134.54 & 24.00 & 7.76 & 7.47 & 7.19 & 7.37 & 7.54 & 7.64 \\ 
   &  &  & 100.27 & 20.00 & 7.91 & 7.54 & 7.26 & 7.48 & 7.63 & 7.74 \\ 
   &  &  & 66 & 1.00 &  &  &  &  &  &  \\ 
   &  & deltran & 203.07 & 2.00 & 5.81 & 0.00 & 0.00 & 0.00 & 5.81 & 5.81 \\ 
   &  &  & 168.8 & 11.00 & 6.60 & 6.10 & 5.36 & 5.89 & 6.19 & 6.44 \\ 
   &  &  & 134.54 & 18.00 & 7.30 & 6.97 & 6.62 & 6.86 & 7.08 & 7.19 \\ 
   &  &  & 100.27 & 16.00 & 7.50 & 7.06 & 6.55 & 6.91 & 7.18 & 7.35 \\ 
   &  &  & 66 & 1.00 &  &  &  &  &  &  \\ 
   &  & random & 203.07 & 3.00 & 6.03 & 3.87 & 0.00 & 3.87 & 6.03 & 6.03 \\ 
   &  &  & 168.8 & 15.00 & 7.08 & 6.56 & 5.99 & 6.43 & 6.76 & 6.96 \\ 
   &  &  & 134.54 & 22.00 & 7.49 & 7.19 & 6.88 & 7.08 & 7.25 & 7.33 \\ 
   &  &  & 100.27 & 20.00 & 7.83 & 7.44 & 7.08 & 7.38 & 7.56 & 7.72 \\ 
   &  &  & 66 & 1.00 &  &  &  &  &  &  \\ 
   &  & proximity & 203.07 & 2.00 & 5.81 & 0.00 & 0.00 & 0.00 & 5.81 & 5.81 \\ 
   &  &  & 168.8 & 11.00 & 6.96 & 6.42 & 5.66 & 6.14 & 6.58 & 6.76 \\ 
   &  &  & 134.54 & 21.00 & 7.52 & 7.19 & 6.87 & 7.09 & 7.26 & 7.37 \\ 
   &  &  & 100.27 & 19.00 & 7.73 & 7.35 & 6.97 & 7.26 & 7.45 & 7.61 \\ 
   &  &  & 66 & 1.00 &  &  &  &  &  &  \\ 
   &  & equal.split & 203.07 & 3.00 & 6.03 & 4.07 & 0.00 & 3.87 & 4.32 & 6.38 \\ 
   &  &  & 168.8 & 15.00 & 7.10 & 6.55 & 5.56 & 6.35 & 6.68 & 6.90 \\ 
   &  &  & 134.54 & 24.00 & 7.58 & 7.21 & 6.79 & 7.10 & 7.32 & 7.42 \\ 
   &  &  & 100.27 & 20.00 & 7.72 & 7.31 & 6.79 & 7.14 & 7.43 & 7.59 \\ 
   &  &  & 66 & 1.00 &  &  &  &  &  &  \\ 
   &  & gradual.split & 203.07 & 3.00 & 3.87 & 3.87 & 0.00 & 0.00 & 3.87 & 4.20 \\ 
   &  &  & 168.8 & 15.00 & 6.62 & 6.38 & 5.82 & 6.16 & 6.53 & 6.82 \\ 
   &  &  & 134.54 & 24.00 & 7.50 & 7.20 & 6.96 & 7.13 & 7.28 & 7.40 \\ 
   &  &  & 100.27 & 20.00 & 7.63 & 7.26 & 6.86 & 7.16 & 7.36 & 7.54 \\ 
   &  &  & 66 & 1.00 &  &  &  &  &  &  \\ 
   \hline
\hline
\caption{Disparity measurement (sum of variance) for Bapst2016 dataset} 
\end{longtable}

 
%\end{landscape}

%\newpage
%\begin{landscape}
 % % latex table generated in R 3.4.3 by xtable 1.8-2 package
% Mon Dec 18 15:17:01 2017
\begin{longtable}{llllrrrrrrr}
  \hline
stratigraphy & bin\_type & model & subsets & n & obs & bs.median & 2.5\% & 25\% & 75\% & 97.5\% \\ 
  \hline
Ages & stratigraphy & equalbins & 477.7 - 470 & 14.00 & 2.27 & 2.13 & 1.96 & 2.08 & 2.16 & 2.22 \\ 
   &  &  & 470 - 467.3 & 14.00 & 2.13 & 1.98 & 1.74 & 1.93 & 2.05 & 2.13 \\ 
   &  &  & 467.3 - 458.4 & 15.00 & 2.19 & 2.07 & 1.86 & 2.02 & 2.10 & 2.16 \\ 
   &  &  & 458.4 - 453 & 18.00 & 2.17 & 2.05 & 1.87 & 2.00 & 2.11 & 2.18 \\ 
   &  &  & 453 - 445.2 & 18.00 & 2.14 & 2.03 & 1.87 & 1.99 & 2.08 & 2.13 \\ 
   &  &  & 445.2 - 443.8 & 22.00 & 2.20 & 2.10 & 1.97 & 2.05 & 2.13 & 2.19 \\ 
   &  &  & 443.8 - 440.8 & 14.00 & 2.06 & 1.91 & 1.76 & 1.88 & 1.97 & 2.06 \\ 
   &  &  & 440.8 - 438.5 & 11.00 & 2.10 & 1.92 & 1.71 & 1.86 & 1.98 & 2.06 \\ 
   &  &  & 438.5 - 433.4 & 11.00 & 2.11 & 1.91 & 1.65 & 1.82 & 1.96 & 2.08 \\ 
   &  &  & 433.4 - 430.5 & 11.00 & 2.09 & 1.92 & 1.68 & 1.84 & 1.96 & 2.06 \\ 
   &  &  & 430.5 - 427.4 & 14.00 & 2.10 & 1.97 & 1.78 & 1.91 & 2.01 & 2.08 \\ 
   &  &  & 427.4 - 425.6 & 14.00 & 2.11 & 1.98 & 1.79 & 1.91 & 2.02 & 2.08 \\ 
   &  &  & 425.6 - 423 & 16.00 & 2.12 & 2.00 & 1.83 & 1.96 & 2.04 & 2.09 \\ 
   &  &  & 423 - 419.2 & 19.00 & 2.16 & 2.05 & 1.97 & 2.02 & 2.09 & 2.14 \\ 
   &  &  & 419.2 - 410.8 & 16.00 & 2.13 & 2.00 & 1.82 & 1.96 & 2.05 & 2.11 \\ 
   &  &  & 410.8 - 407.6 & 7.00 & 2.05 & 1.81 & 1.31 & 1.68 & 1.88 & 1.99 \\ 
   &  &  & 407.6 - 393.3 & 8.00 & 1.97 & 1.75 & 1.36 & 1.64 & 1.84 & 1.93 \\ 
   &  &  & 393.3 - 387.7 & 8.00 & 2.11 & 1.86 & 1.51 & 1.72 & 1.96 & 2.08 \\ 
   &  &  & 387.7 - 382.7 & 7.00 & 2.16 & 1.89 & 1.39 & 1.74 & 1.97 & 2.07 \\ 
   &  &  & 382.7 - 372.2 & 8.00 & 1.93 & 1.73 & 1.33 & 1.59 & 1.85 & 1.97 \\ 
   &  & acctran & 473.85 & 3.00 & 1.73 & 1.27 & 0.00 & 0.92 & 1.46 & 1.73 \\ 
   &  &  & 468.65 & 12.00 & 2.24 & 2.07 & 1.81 & 1.98 & 2.12 & 2.20 \\ 
   &  &  & 462.85 & 12.00 & 2.20 & 2.03 & 1.74 & 1.94 & 2.09 & 2.14 \\ 
   &  &  & 455.7 & 10.00 & 2.24 & 2.05 & 1.80 & 1.95 & 2.12 & 2.22 \\ 
   &  &  & 449.1 & 13.00 & 2.18 & 2.04 & 1.82 & 1.98 & 2.08 & 2.13 \\ 
   &  &  & 444.5 & 6.00 & 1.95 & 1.70 & 1.15 & 1.53 & 1.78 & 1.89 \\ 
   &  &  & 442.3 & 12.00 & 2.09 & 1.92 & 1.65 & 1.85 & 1.98 & 2.05 \\ 
   &  &  & 439.65 & 12.00 & 2.05 & 1.89 & 1.60 & 1.78 & 1.97 & 2.05 \\ 
   &  &  & 435.95 & 12.00 & 2.05 & 1.87 & 1.62 & 1.80 & 1.93 & 2.04 \\ 
   &  &  & 431.95 & 12.00 & 2.04 & 1.88 & 1.59 & 1.77 & 1.95 & 2.06 \\ 
   &  &  & 428.95 & 12.00 & 2.16 & 2.01 & 1.78 & 1.95 & 2.04 & 2.12 \\ 
   &  &  & 426.5 & 15.00 & 2.14 & 1.99 & 1.83 & 1.95 & 2.04 & 2.11 \\ 
   &  &  & 424.3 & 15.00 & 2.15 & 2.00 & 1.89 & 1.95 & 2.05 & 2.12 \\ 
   &  &  & 421.1 & 11.00 & 2.13 & 1.94 & 1.76 & 1.87 & 2.01 & 2.09 \\ 
   &  &  & 415 & 8.00 & 2.00 & 1.77 & 1.46 & 1.69 & 1.86 & 1.96 \\ 
   &  &  & 409.2 & 7.00 & 2.02 & 1.78 & 1.33 & 1.66 & 1.90 & 1.99 \\ 
   &  &  & 400.45 & 7.00 & 2.11 & 1.86 & 1.41 & 1.75 & 1.97 & 2.08 \\ 
   &  &  & 390.5 & 7.00 & 2.16 & 1.89 & 1.42 & 1.81 & 1.96 & 2.07 \\ 
   &  &  & 385.2 & 4.00 & 2.07 & 1.54 & 0.57 & 1.46 & 1.86 & 2.07 \\ 
   &  &  & 377.45 & 2.00 & 1.14 & 0.57 & 0.00 & 0.00 & 1.14 & 1.14 \\ 
   &  & deltran & 473.85 & 3.00 & 1.61 & 1.22 & 0.00 & 0.76 & 1.34 & 1.61 \\ 
   &  &  & 468.65 & 12.00 & 2.17 & 2.00 & 1.76 & 1.93 & 2.06 & 2.19 \\ 
   &  &  & 462.85 & 12.00 & 2.24 & 2.07 & 1.87 & 2.01 & 2.13 & 2.21 \\ 
   &  &  & 455.7 & 10.00 & 2.27 & 2.07 & 1.70 & 1.99 & 2.12 & 2.20 \\ 
   &  &  & 449.1 & 13.00 & 2.17 & 2.02 & 1.76 & 1.96 & 2.08 & 2.16 \\ 
   &  &  & 444.5 & 5.00 & 1.81 & 1.53 & 0.62 & 1.29 & 1.64 & 1.81 \\ 
   &  &  & 442.3 & 12.00 & 2.09 & 1.93 & 1.71 & 1.86 & 1.98 & 2.06 \\ 
   &  &  & 439.65 & 12.00 & 2.08 & 1.92 & 1.69 & 1.85 & 1.97 & 2.06 \\ 
   &  &  & 435.95 & 12.00 & 2.08 & 1.92 & 1.72 & 1.86 & 1.98 & 2.06 \\ 
   &  &  & 431.95 & 12.00 & 2.09 & 1.93 & 1.67 & 1.85 & 1.99 & 2.05 \\ 
   &  &  & 428.95 & 12.00 & 2.04 & 1.88 & 1.68 & 1.81 & 1.94 & 2.04 \\ 
   &  &  & 426.5 & 15.00 & 2.04 & 1.90 & 1.71 & 1.83 & 1.95 & 2.03 \\ 
   &  &  & 424.3 & 15.00 & 2.09 & 1.95 & 1.82 & 1.89 & 1.99 & 2.08 \\ 
   &  &  & 421.1 & 11.00 & 2.13 & 1.93 & 1.65 & 1.84 & 1.98 & 2.08 \\ 
   &  &  & 415 & 8.00 & 1.98 & 1.76 & 1.19 & 1.65 & 1.85 & 1.95 \\ 
   &  &  & 409.2 & 7.00 & 2.02 & 1.75 & 1.35 & 1.60 & 1.86 & 1.99 \\ 
   &  &  & 400.45 & 7.00 & 2.02 & 1.73 & 1.32 & 1.62 & 1.84 & 1.95 \\ 
   &  &  & 390.5 & 7.00 & 2.17 & 1.89 & 1.42 & 1.79 & 2.00 & 2.11 \\ 
   &  &  & 385.2 & 4.00 & 2.17 & 1.77 & 0.45 & 1.49 & 1.87 & 2.17 \\ 
   &  &  & 377.45 & 1.00 &  &  &  &  &  &  \\ 
   &  & random & 473.85 & 3.00 & 1.61 & 1.24 & 0.00 & 0.76 & 1.24 & 1.61 \\ 
   &  &  & 468.65 & 12.00 & 2.24 & 2.06 & 1.81 & 1.98 & 2.12 & 2.18 \\ 
   &  &  & 462.85 & 12.00 & 2.24 & 2.06 & 1.80 & 2.00 & 2.11 & 2.21 \\ 
   &  &  & 455.7 & 10.00 & 2.28 & 2.07 & 1.84 & 2.01 & 2.13 & 2.21 \\ 
   &  &  & 449.1 & 13.00 & 2.18 & 2.03 & 1.82 & 1.97 & 2.07 & 2.17 \\ 
   &  &  & 444.5 & 6.00 & 1.91 & 1.68 & 1.11 & 1.56 & 1.74 & 1.83 \\ 
   &  &  & 442.3 & 12.00 & 2.09 & 1.94 & 1.77 & 1.87 & 2.00 & 2.07 \\ 
   &  &  & 439.65 & 12.00 & 2.05 & 1.88 & 1.62 & 1.81 & 1.93 & 2.02 \\ 
   &  &  & 435.95 & 12.00 & 2.09 & 1.93 & 1.70 & 1.85 & 1.99 & 2.05 \\ 
   &  &  & 431.95 & 12.00 & 2.09 & 1.92 & 1.73 & 1.86 & 1.99 & 2.06 \\ 
   &  &  & 428.95 & 12.00 & 2.09 & 1.92 & 1.68 & 1.85 & 1.98 & 2.06 \\ 
   &  &  & 426.5 & 15.00 & 2.14 & 2.00 & 1.83 & 1.96 & 2.05 & 2.08 \\ 
   &  &  & 424.3 & 15.00 & 2.15 & 2.00 & 1.86 & 1.96 & 2.05 & 2.11 \\ 
   &  &  & 421.1 & 11.00 & 2.12 & 1.93 & 1.74 & 1.86 & 2.00 & 2.07 \\ 
   &  &  & 415 & 8.00 & 2.00 & 1.79 & 1.44 & 1.70 & 1.86 & 1.96 \\ 
   &  &  & 409.2 & 7.00 & 2.02 & 1.76 & 1.34 & 1.65 & 1.86 & 1.99 \\ 
   &  &  & 400.45 & 7.00 & 2.02 & 1.74 & 1.31 & 1.60 & 1.86 & 1.99 \\ 
   &  &  & 390.5 & 7.00 & 2.17 & 1.90 & 1.46 & 1.80 & 1.99 & 2.11 \\ 
   &  &  & 385.2 & 4.00 & 2.17 & 1.73 & 0.94 & 1.43 & 1.87 & 2.17 \\ 
   &  &  & 377.45 & 2.00 & 1.14 & 0.00 & 0.00 & 0.00 & 1.14 & 1.14 \\ 
   &  & proximity & 473.85 & 3.00 & 1.61 & 1.24 & 0.00 & 0.76 & 1.61 & 1.61 \\ 
   &  &  & 468.65 & 12.00 & 2.17 & 2.00 & 1.71 & 1.92 & 2.06 & 2.15 \\ 
   &  &  & 462.85 & 12.00 & 2.24 & 2.06 & 1.81 & 2.01 & 2.12 & 2.19 \\ 
   &  &  & 455.7 & 10.00 & 2.23 & 2.01 & 1.73 & 1.92 & 2.09 & 2.20 \\ 
   &  &  & 449.1 & 13.00 & 2.18 & 2.03 & 1.83 & 1.96 & 2.08 & 2.15 \\ 
   &  &  & 444.5 & 5.00 & 1.81 & 1.54 & 0.93 & 1.29 & 1.64 & 1.70 \\ 
   &  &  & 442.3 & 12.00 & 2.08 & 1.91 & 1.70 & 1.87 & 1.97 & 2.08 \\ 
   &  &  & 439.65 & 12.00 & 2.08 & 1.95 & 1.71 & 1.87 & 2.00 & 2.07 \\ 
   &  &  & 435.95 & 12.00 & 2.05 & 1.92 & 1.72 & 1.84 & 1.97 & 2.05 \\ 
   &  &  & 431.95 & 12.00 & 2.04 & 1.88 & 1.58 & 1.80 & 1.96 & 2.04 \\ 
   &  &  & 428.95 & 12.00 & 2.04 & 1.90 & 1.61 & 1.80 & 1.94 & 2.03 \\ 
   &  &  & 426.5 & 15.00 & 2.09 & 1.95 & 1.78 & 1.89 & 1.99 & 2.06 \\ 
   &  &  & 424.3 & 15.00 & 2.14 & 2.01 & 1.83 & 1.96 & 2.04 & 2.10 \\ 
   &  &  & 421.1 & 11.00 & 2.13 & 1.94 & 1.76 & 1.89 & 2.01 & 2.09 \\ 
   &  &  & 415 & 8.00 & 1.98 & 1.75 & 1.28 & 1.66 & 1.82 & 1.94 \\ 
   &  &  & 409.2 & 7.00 & 2.03 & 1.79 & 1.37 & 1.68 & 1.88 & 1.99 \\ 
   &  &  & 400.45 & 7.00 & 2.02 & 1.78 & 1.36 & 1.67 & 1.86 & 1.98 \\ 
   &  &  & 390.5 & 7.00 & 2.16 & 1.82 & 1.44 & 1.76 & 1.97 & 2.09 \\ 
   &  &  & 385.2 & 4.00 & 2.07 & 1.54 & 0.27 & 1.17 & 1.86 & 2.07 \\ 
   &  &  & 377.45 & 2.00 & 1.14 & 1.14 & 0.00 & 0.00 & 1.14 & 1.14 \\ 
   &  & equal.split & 473.85 & 3.00 & 1.73 & 1.24 & 0.00 & 0.92 & 1.36 & 1.73 \\ 
   &  &  & 468.65 & 14.00 & 2.10 & 1.99 & 1.76 & 1.92 & 2.04 & 2.14 \\ 
   &  &  & 462.85 & 15.00 & 2.19 & 2.05 & 1.88 & 2.00 & 2.11 & 2.15 \\ 
   &  &  & 455.7 & 11.00 & 2.20 & 2.01 & 1.74 & 1.96 & 2.09 & 2.17 \\ 
   &  &  & 449.1 & 13.00 & 2.18 & 2.03 & 1.77 & 1.96 & 2.08 & 2.14 \\ 
   &  &  & 444.5 & 6.00 & 1.89 & 1.59 & 1.01 & 1.41 & 1.72 & 1.85 \\ 
   &  &  & 442.3 & 14.00 & 2.04 & 1.91 & 1.70 & 1.86 & 1.95 & 2.04 \\ 
   &  &  & 439.65 & 14.00 & 2.05 & 1.90 & 1.67 & 1.81 & 1.95 & 2.03 \\ 
   &  &  & 435.95 & 13.00 & 2.04 & 1.91 & 1.72 & 1.84 & 1.98 & 2.06 \\ 
   &  &  & 431.95 & 12.00 & 2.07 & 1.91 & 1.67 & 1.81 & 1.96 & 2.06 \\ 
   &  &  & 428.95 & 14.00 & 2.05 & 1.96 & 1.69 & 1.89 & 2.00 & 2.08 \\ 
   &  &  & 426.5 & 17.00 & 2.09 & 1.99 & 1.84 & 1.94 & 2.03 & 2.09 \\ 
   &  &  & 424.3 & 17.00 & 2.09 & 1.99 & 1.85 & 1.94 & 2.03 & 2.08 \\ 
   &  &  & 421.1 & 12.00 & 2.13 & 1.97 & 1.81 & 1.92 & 2.03 & 2.09 \\ 
   &  &  & 415 & 8.00 & 1.99 & 1.75 & 1.33 & 1.70 & 1.84 & 1.95 \\ 
   &  &  & 409.2 & 7.00 & 2.01 & 1.78 & 1.39 & 1.67 & 1.89 & 1.98 \\ 
   &  &  & 400.45 & 7.00 & 2.11 & 1.79 & 1.42 & 1.68 & 1.92 & 2.04 \\ 
   &  &  & 390.5 & 8.00 & 2.09 & 1.89 & 1.51 & 1.76 & 1.94 & 2.04 \\ 
   &  &  & 385.2 & 4.00 & 2.07 & 1.73 & 0.75 & 1.44 & 1.87 & 2.17 \\ 
   &  &  & 377.45 & 2.00 & 1.14 & 0.00 & 0.00 & 0.00 & 1.14 & 1.14 \\ 
   &  & gradual.split & 473.85 & 3.00 & 1.59 & 1.24 & 0.00 & 0.92 & 1.59 & 1.73 \\ 
   &  &  & 468.65 & 14.00 & 2.05 & 1.94 & 1.65 & 1.85 & 2.02 & 2.10 \\ 
   &  &  & 462.85 & 15.00 & 2.21 & 2.07 & 1.90 & 2.03 & 2.11 & 2.18 \\ 
   &  &  & 455.7 & 11.00 & 2.22 & 2.03 & 1.78 & 1.95 & 2.09 & 2.17 \\ 
   &  &  & 449.1 & 13.00 & 2.18 & 2.02 & 1.75 & 1.93 & 2.07 & 2.14 \\ 
   &  &  & 444.5 & 6.00 & 1.84 & 1.50 & 0.83 & 1.39 & 1.64 & 1.80 \\ 
   &  &  & 442.3 & 14.00 & 2.04 & 1.90 & 1.61 & 1.83 & 1.95 & 2.02 \\ 
   &  &  & 439.65 & 14.00 & 2.05 & 1.89 & 1.68 & 1.82 & 1.96 & 2.04 \\ 
   &  &  & 435.95 & 13.00 & 2.07 & 1.90 & 1.69 & 1.83 & 1.96 & 2.03 \\ 
   &  &  & 431.95 & 12.00 & 2.04 & 1.91 & 1.71 & 1.85 & 1.97 & 2.04 \\ 
   &  &  & 428.95 & 14.00 & 2.05 & 1.92 & 1.72 & 1.85 & 1.97 & 2.03 \\ 
   &  &  & 426.5 & 17.00 & 2.10 & 1.98 & 1.80 & 1.90 & 2.03 & 2.09 \\ 
   &  &  & 424.3 & 17.00 & 2.14 & 2.01 & 1.88 & 1.97 & 2.05 & 2.10 \\ 
   &  &  & 421.1 & 12.00 & 2.14 & 1.96 & 1.76 & 1.91 & 2.02 & 2.08 \\ 
   &  &  & 415 & 8.00 & 1.98 & 1.75 & 1.33 & 1.60 & 1.85 & 1.94 \\ 
   &  &  & 409.2 & 7.00 & 2.03 & 1.79 & 1.12 & 1.61 & 1.86 & 1.96 \\ 
   &  &  & 400.45 & 7.00 & 2.11 & 1.78 & 1.37 & 1.65 & 1.87 & 2.03 \\ 
   &  &  & 390.5 & 8.00 & 2.09 & 1.86 & 1.36 & 1.75 & 1.95 & 2.05 \\ 
   &  &  & 385.2 & 4.00 & 2.07 & 1.54 & 1.08 & 1.44 & 1.89 & 2.07 \\ 
   &  &  & 377.45 & 2.00 & 1.14 & 0.00 & 0.00 & 0.00 & 1.14 & 1.14 \\ 
   & duration & equalbins & 473.4 - 469 & 12.00 & 2.21 & 2.04 & 1.82 & 1.98 & 2.10 & 2.17 \\ 
   &  &  & 469 - 464.6 & 14.00 & 2.14 & 2.00 & 1.68 & 1.93 & 2.06 & 2.15 \\ 
   &  &  & 464.6 - 460.2 & 12.00 & 2.25 & 2.06 & 1.88 & 1.98 & 2.10 & 2.18 \\ 
   &  &  & 460.2 - 455.8 & 12.00 & 2.24 & 2.06 & 1.85 & 2.00 & 2.12 & 2.18 \\ 
   &  &  & 455.8 - 451.4 & 16.00 & 2.17 & 2.02 & 1.79 & 1.97 & 2.08 & 2.13 \\ 
   &  &  & 451.4 - 447 & 12.00 & 2.23 & 2.05 & 1.86 & 1.99 & 2.11 & 2.17 \\ 
   &  &  & 447 - 442.6 & 28.00 & 2.11 & 2.03 & 1.93 & 1.99 & 2.08 & 2.14 \\ 
   &  &  & 442.6 - 438.2 & 12.00 & 2.09 & 1.92 & 1.68 & 1.84 & 1.98 & 2.08 \\ 
   &  &  & 438.2 - 433.8 & 11.00 & 2.11 & 1.92 & 1.69 & 1.85 & 1.99 & 2.07 \\ 
   &  &  & 433.8 - 429.4 & 12.00 & 2.04 & 1.88 & 1.67 & 1.81 & 1.95 & 2.04 \\ 
   &  &  & 429.4 - 425 & 14.00 & 2.11 & 1.97 & 1.83 & 1.94 & 2.03 & 2.07 \\ 
   &  &  & 425 - 420.6 & 16.00 & 2.12 & 2.00 & 1.87 & 1.96 & 2.03 & 2.11 \\ 
   &  &  & 420.6 - 416.2 & 15.00 & 2.15 & 2.00 & 1.87 & 1.98 & 2.05 & 2.11 \\ 
   &  &  & 416.2 - 411.8 & 7.00 & 2.00 & 1.73 & 1.26 & 1.60 & 1.84 & 1.99 \\ 
   &  &  & 411.8 - 407.4 & 7.00 & 2.05 & 1.79 & 1.45 & 1.68 & 1.90 & 2.00 \\ 
   &  &  & 407.4 - 403 & 5.00 & 2.15 & 1.71 & 1.06 & 1.54 & 1.94 & 2.00 \\ 
   &  &  & 403 - 398.6 & 6.00 & 2.04 & 1.78 & 1.17 & 1.59 & 1.86 & 1.98 \\ 
   &  &  & 398.6 - 394.2 & 6.00 & 2.03 & 1.72 & 1.16 & 1.58 & 1.87 & 2.02 \\ 
   &  &  & 394.2 - 389.8 & 8.00 & 2.11 & 1.91 & 1.47 & 1.77 & 1.99 & 2.10 \\ 
   &  &  & 389.8 - 385.4 & 6.00 & 2.17 & 1.85 & 1.30 & 1.71 & 1.99 & 2.17 \\ 
   &  &  & 385.4 - 381 & 4.00 & 2.07 & 1.65 & 0.27 & 1.48 & 1.88 & 2.07 \\ 
   &  &  & 381 - 376.6 & 2.00 & 1.14 & 1.14 & 0.00 & 0.00 & 1.14 & 1.14 \\ 
   &  &  & 376.6 - 372.2 & 3.00 & 2.10 & 1.41 & 0.00 & 1.34 & 1.45 & 2.10 \\ 
   &  & acctran & 471.2 & 3.00 & 1.73 & 1.18 & 0.00 & 0.92 & 1.73 & 1.73 \\ 
   &  &  & 466.8 & 12.00 & 2.24 & 2.08 & 1.82 & 2.02 & 2.12 & 2.20 \\ 
   &  &  & 462.4 & 12.00 & 2.20 & 2.03 & 1.82 & 1.96 & 2.10 & 2.17 \\ 
   &  &  & 458 & 10.00 & 2.24 & 2.04 & 1.69 & 1.91 & 2.12 & 2.21 \\ 
   &  &  & 453.6 & 10.00 & 2.22 & 2.05 & 1.75 & 1.94 & 2.10 & 2.18 \\ 
   &  &  & 449.2 & 13.00 & 2.18 & 2.00 & 1.81 & 1.94 & 2.07 & 2.14 \\ 
   &  &  & 444.8 & 6.00 & 1.95 & 1.67 & 1.17 & 1.50 & 1.79 & 1.90 \\ 
   &  &  & 440.4 & 12.00 & 2.05 & 1.91 & 1.72 & 1.84 & 1.97 & 2.05 \\ 
   &  &  & 436 & 12.00 & 2.05 & 1.89 & 1.69 & 1.82 & 1.94 & 2.03 \\ 
   &  &  & 431.6 & 12.00 & 2.04 & 1.88 & 1.63 & 1.83 & 1.96 & 2.04 \\ 
   &  &  & 427.2 & 15.00 & 2.14 & 2.01 & 1.85 & 1.95 & 2.05 & 2.12 \\ 
   &  &  & 422.8 & 11.00 & 2.13 & 1.94 & 1.73 & 1.86 & 2.01 & 2.09 \\ 
   &  &  & 418.4 & 8.00 & 1.98 & 1.74 & 1.37 & 1.64 & 1.82 & 1.95 \\ 
   &  &  & 414 & 8.00 & 2.00 & 1.79 & 1.47 & 1.71 & 1.86 & 1.95 \\ 
   &  &  & 409.6 & 7.00 & 2.02 & 1.74 & 1.20 & 1.61 & 1.88 & 1.94 \\ 
   &  &  & 405.2 & 7.00 & 2.11 & 1.88 & 1.30 & 1.70 & 1.98 & 2.08 \\ 
   &  &  & 400.8 & 7.00 & 2.11 & 1.83 & 1.27 & 1.68 & 1.96 & 2.08 \\ 
   &  &  & 396.4 & 7.00 & 2.13 & 1.84 & 1.22 & 1.71 & 1.95 & 2.06 \\ 
   &  &  & 392 & 7.00 & 2.16 & 1.94 & 1.48 & 1.78 & 2.01 & 2.07 \\ 
   &  &  & 387.6 & 4.00 & 2.07 & 1.56 & 0.57 & 1.47 & 1.86 & 2.07 \\ 
   &  &  & 383.2 & 4.00 & 2.00 & 1.65 & 0.69 & 1.42 & 1.89 & 2.00 \\ 
   &  &  & 378.8 & 1.00 &  &  &  &  &  &  \\ 
   &  &  & 374.4 & 3.00 & 2.10 & 1.41 & 0.00 & 1.34 & 1.45 & 2.10 \\ 
   &  & deltran & 471.2 & 3.00 & 1.61 & 1.22 & 0.00 & 0.76 & 1.24 & 1.61 \\ 
   &  &  & 466.8 & 12.00 & 2.18 & 2.01 & 1.79 & 1.94 & 2.07 & 2.15 \\ 
   &  &  & 462.4 & 12.00 & 2.24 & 2.08 & 1.89 & 2.01 & 2.13 & 2.19 \\ 
   &  &  & 458 & 10.00 & 2.27 & 2.07 & 1.77 & 1.98 & 2.13 & 2.18 \\ 
   &  &  & 453.6 & 9.00 & 2.21 & 2.01 & 1.69 & 1.92 & 2.07 & 2.16 \\ 
   &  &  & 449.2 & 13.00 & 2.17 & 2.01 & 1.80 & 1.94 & 2.07 & 2.15 \\ 
   &  &  & 444.8 & 5.00 & 1.81 & 1.55 & 0.90 & 1.41 & 1.64 & 1.81 \\ 
   &  &  & 440.4 & 12.00 & 2.08 & 1.94 & 1.63 & 1.83 & 2.00 & 2.07 \\ 
   &  &  & 436 & 12.00 & 2.08 & 1.91 & 1.69 & 1.84 & 1.98 & 2.05 \\ 
   &  &  & 431.6 & 12.00 & 2.09 & 1.91 & 1.73 & 1.86 & 1.98 & 2.05 \\ 
   &  &  & 427.2 & 15.00 & 2.04 & 1.93 & 1.73 & 1.85 & 1.97 & 2.05 \\ 
   &  &  & 422.8 & 11.00 & 2.13 & 1.93 & 1.68 & 1.83 & 2.01 & 2.10 \\ 
   &  &  & 418.4 & 8.00 & 2.05 & 1.80 & 1.49 & 1.70 & 1.86 & 1.98 \\ 
   &  &  & 414 & 8.00 & 1.98 & 1.77 & 1.43 & 1.65 & 1.85 & 1.93 \\ 
   &  &  & 409.6 & 7.00 & 2.02 & 1.72 & 1.32 & 1.62 & 1.82 & 1.96 \\ 
   &  &  & 405.2 & 7.00 & 2.03 & 1.77 & 1.41 & 1.69 & 1.86 & 1.98 \\ 
   &  &  & 400.8 & 7.00 & 2.02 & 1.75 & 1.32 & 1.67 & 1.84 & 1.95 \\ 
   &  &  & 396.4 & 7.00 & 2.01 & 1.76 & 1.28 & 1.65 & 1.86 & 1.96 \\ 
   &  &  & 392 & 7.00 & 2.17 & 1.86 & 1.48 & 1.75 & 1.97 & 2.11 \\ 
   &  &  & 387.6 & 4.00 & 2.17 & 1.77 & 1.07 & 1.46 & 1.88 & 2.17 \\ 
   &  &  & 383.2 & 4.00 & 2.07 & 1.56 & 0.57 & 1.46 & 1.87 & 2.07 \\ 
   &  &  & 378.8 & 1.00 &  &  &  &  &  &  \\ 
   &  &  & 374.4 & 2.00 & 1.14 & 0.00 & 0.00 & 0.00 & 1.14 & 1.14 \\ 
   &  & random & 471.2 & 3.00 & 1.73 & 1.34 & 0.00 & 0.76 & 1.36 & 1.73 \\ 
   &  &  & 466.8 & 12.00 & 2.24 & 2.07 & 1.79 & 1.98 & 2.13 & 2.20 \\ 
   &  &  & 462.4 & 12.00 & 2.20 & 2.04 & 1.79 & 1.97 & 2.11 & 2.17 \\ 
   &  &  & 458 & 10.00 & 2.28 & 2.08 & 1.77 & 2.01 & 2.12 & 2.22 \\ 
   &  &  & 453.6 & 9.00 & 2.21 & 1.99 & 1.70 & 1.91 & 2.07 & 2.14 \\ 
   &  &  & 449.2 & 13.00 & 2.18 & 2.05 & 1.83 & 1.99 & 2.09 & 2.16 \\ 
   &  &  & 444.8 & 6.00 & 1.88 & 1.62 & 1.03 & 1.49 & 1.72 & 1.83 \\ 
   &  &  & 440.4 & 12.00 & 2.05 & 1.89 & 1.62 & 1.83 & 1.95 & 2.05 \\ 
   &  &  & 436 & 12.00 & 2.08 & 1.91 & 1.66 & 1.83 & 1.97 & 2.06 \\ 
   &  &  & 431.6 & 12.00 & 2.05 & 1.89 & 1.62 & 1.81 & 1.97 & 2.04 \\ 
   &  &  & 427.2 & 15.00 & 2.04 & 1.91 & 1.72 & 1.84 & 1.97 & 2.07 \\ 
   &  &  & 422.8 & 11.00 & 2.12 & 1.95 & 1.69 & 1.85 & 2.01 & 2.07 \\ 
   &  &  & 418.4 & 8.00 & 2.00 & 1.80 & 1.34 & 1.70 & 1.86 & 1.97 \\ 
   &  &  & 414 & 8.00 & 2.00 & 1.81 & 1.44 & 1.67 & 1.87 & 1.96 \\ 
   &  &  & 409.6 & 7.00 & 2.01 & 1.75 & 1.38 & 1.63 & 1.85 & 1.97 \\ 
   &  &  & 405.2 & 7.00 & 2.03 & 1.73 & 1.48 & 1.64 & 1.84 & 1.96 \\ 
   &  &  & 400.8 & 7.00 & 2.11 & 1.84 & 1.45 & 1.73 & 1.92 & 2.06 \\ 
   &  &  & 396.4 & 7.00 & 2.01 & 1.76 & 1.20 & 1.60 & 1.86 & 1.98 \\ 
   &  &  & 392 & 7.00 & 2.17 & 1.94 & 1.47 & 1.81 & 2.01 & 2.09 \\ 
   &  &  & 387.6 & 4.00 & 2.17 & 1.74 & 0.51 & 1.38 & 1.88 & 2.17 \\ 
   &  &  & 383.2 & 4.00 & 2.00 & 1.56 & 0.33 & 1.37 & 1.69 & 2.00 \\ 
   &  &  & 378.8 & 1.00 &  &  &  &  &  &  \\ 
   &  &  & 374.4 & 3.00 & 1.47 & 1.07 & 0.00 & 0.99 & 1.11 & 1.47 \\ 
   &  & proximity & 471.2 & 3.00 & 1.73 & 1.18 & 0.00 & 0.92 & 1.36 & 1.73 \\ 
   &  &  & 466.8 & 12.00 & 2.18 & 2.00 & 1.74 & 1.95 & 2.08 & 2.15 \\ 
   &  &  & 462.4 & 12.00 & 2.24 & 2.07 & 1.86 & 2.01 & 2.12 & 2.18 \\ 
   &  &  & 458 & 10.00 & 2.23 & 2.01 & 1.77 & 1.92 & 2.07 & 2.14 \\ 
   &  &  & 453.6 & 10.00 & 2.17 & 1.97 & 1.65 & 1.90 & 2.04 & 2.14 \\ 
   &  &  & 449.2 & 13.00 & 2.18 & 2.03 & 1.78 & 1.95 & 2.08 & 2.14 \\ 
   &  &  & 444.8 & 5.00 & 1.81 & 1.55 & 0.87 & 1.39 & 1.64 & 1.81 \\ 
   &  &  & 440.4 & 12.00 & 2.08 & 1.93 & 1.69 & 1.87 & 1.98 & 2.04 \\ 
   &  &  & 436 & 12.00 & 2.05 & 1.92 & 1.69 & 1.84 & 1.98 & 2.05 \\ 
   &  &  & 431.6 & 12.00 & 2.04 & 1.88 & 1.65 & 1.82 & 1.93 & 2.03 \\ 
   &  &  & 427.2 & 15.00 & 2.09 & 1.97 & 1.79 & 1.93 & 2.02 & 2.07 \\ 
   &  &  & 422.8 & 11.00 & 2.13 & 1.94 & 1.72 & 1.87 & 2.00 & 2.07 \\ 
   &  &  & 418.4 & 8.00 & 1.98 & 1.79 & 1.40 & 1.65 & 1.85 & 1.97 \\ 
   &  &  & 414 & 8.00 & 1.98 & 1.75 & 1.43 & 1.63 & 1.84 & 1.94 \\ 
   &  &  & 409.6 & 7.00 & 2.03 & 1.74 & 1.35 & 1.62 & 1.85 & 1.96 \\ 
   &  &  & 405.2 & 7.00 & 2.02 & 1.78 & 1.33 & 1.64 & 1.85 & 1.97 \\ 
   &  &  & 400.8 & 7.00 & 2.02 & 1.77 & 1.41 & 1.68 & 1.89 & 1.97 \\ 
   &  &  & 396.4 & 7.00 & 2.01 & 1.73 & 1.33 & 1.62 & 1.83 & 1.98 \\ 
   &  &  & 392 & 7.00 & 2.16 & 1.89 & 1.51 & 1.76 & 1.96 & 2.07 \\ 
   &  &  & 387.6 & 4.00 & 2.07 & 1.54 & 0.27 & 1.15 & 1.87 & 2.07 \\ 
   &  &  & 383.2 & 4.00 & 2.07 & 1.56 & 0.81 & 1.17 & 1.86 & 2.07 \\ 
   &  &  & 378.8 & 1.00 &  &  &  &  &  &  \\ 
   &  &  & 374.4 & 3.00 & 2.10 & 1.41 & 0.00 & 1.34 & 1.45 & 2.10 \\ 
   &  & equal.split & 471.2 & 3.00 & 1.61 & 1.24 & 0.00 & 1.18 & 1.59 & 1.73 \\ 
   &  &  & 466.8 & 14.00 & 2.19 & 1.98 & 1.76 & 1.91 & 2.06 & 2.13 \\ 
   &  &  & 462.4 & 14.00 & 2.23 & 2.06 & 1.82 & 1.98 & 2.11 & 2.17 \\ 
   &  &  & 458 & 12.00 & 2.24 & 2.06 & 1.79 & 2.00 & 2.11 & 2.19 \\ 
   &  &  & 453.6 & 11.00 & 2.19 & 1.97 & 1.72 & 1.90 & 2.05 & 2.14 \\ 
   &  &  & 449.2 & 13.00 & 2.17 & 2.03 & 1.81 & 1.97 & 2.09 & 2.15 \\ 
   &  &  & 444.8 & 7.00 & 1.67 & 1.58 & 1.01 & 1.41 & 1.68 & 1.78 \\ 
   &  &  & 440.4 & 13.00 & 2.05 & 1.90 & 1.72 & 1.83 & 1.97 & 2.02 \\ 
   &  &  & 436 & 13.00 & 2.07 & 1.92 & 1.66 & 1.85 & 1.97 & 2.04 \\ 
   &  &  & 431.6 & 12.00 & 2.05 & 1.90 & 1.60 & 1.83 & 1.97 & 2.05 \\ 
   &  &  & 427.2 & 17.00 & 2.06 & 1.96 & 1.76 & 1.91 & 2.01 & 2.08 \\ 
   &  &  & 422.8 & 12.00 & 2.14 & 1.97 & 1.75 & 1.90 & 2.02 & 2.08 \\ 
   &  &  & 418.4 & 8.00 & 2.05 & 1.78 & 1.16 & 1.70 & 1.87 & 1.98 \\ 
   &  &  & 414 & 8.00 & 2.00 & 1.78 & 1.27 & 1.70 & 1.84 & 1.95 \\ 
   &  &  & 409.6 & 7.00 & 2.02 & 1.78 & 1.41 & 1.67 & 1.86 & 1.96 \\ 
   &  &  & 405.2 & 7.00 & 2.13 & 1.79 & 1.30 & 1.64 & 1.92 & 2.10 \\ 
   &  &  & 400.8 & 7.00 & 2.02 & 1.77 & 1.23 & 1.62 & 1.87 & 2.04 \\ 
   &  &  & 396.4 & 8.00 & 2.00 & 1.79 & 1.32 & 1.66 & 1.89 & 2.01 \\ 
   &  &  & 392 & 8.00 & 2.09 & 1.87 & 1.53 & 1.77 & 1.96 & 2.04 \\ 
   &  &  & 387.6 & 4.00 & 2.07 & 1.56 & 0.45 & 1.44 & 1.87 & 2.12 \\ 
   &  &  & 383.2 & 4.00 & 2.00 & 1.54 & 0.27 & 1.17 & 1.85 & 2.03 \\ 
   &  &  & 378.8 & 1.00 &  &  &  &  &  &  \\ 
   &  &  & 374.4 & 3.00 & 1.59 & 1.07 & 0.00 & 0.76 & 1.41 & 1.73 \\ 
   &  & gradual.split & 471.2 & 3.00 & 1.73 & 1.24 & 0.00 & 0.92 & 1.36 & 1.73 \\ 
   &  &  & 466.8 & 14.00 & 2.13 & 2.01 & 1.84 & 1.96 & 2.07 & 2.13 \\ 
   &  &  & 462.4 & 14.00 & 2.22 & 2.06 & 1.85 & 2.01 & 2.11 & 2.19 \\ 
   &  &  & 458 & 12.00 & 2.18 & 2.03 & 1.82 & 1.96 & 2.09 & 2.14 \\ 
   &  &  & 453.6 & 11.00 & 2.15 & 1.95 & 1.73 & 1.89 & 2.04 & 2.14 \\ 
   &  &  & 449.2 & 13.00 & 2.18 & 2.02 & 1.82 & 1.97 & 2.09 & 2.16 \\ 
   &  &  & 444.8 & 7.00 & 1.61 & 1.50 & 0.92 & 1.29 & 1.59 & 1.72 \\ 
   &  &  & 440.4 & 13.00 & 2.06 & 1.92 & 1.70 & 1.85 & 1.97 & 2.03 \\ 
   &  &  & 436 & 13.00 & 2.04 & 1.89 & 1.60 & 1.83 & 1.96 & 2.06 \\ 
   &  &  & 431.6 & 12.00 & 2.04 & 1.88 & 1.62 & 1.80 & 1.93 & 2.04 \\ 
   &  &  & 427.2 & 17.00 & 2.06 & 1.96 & 1.74 & 1.92 & 2.01 & 2.06 \\ 
   &  &  & 422.8 & 12.00 & 2.13 & 1.99 & 1.81 & 1.92 & 2.04 & 2.09 \\ 
   &  &  & 418.4 & 8.00 & 1.98 & 1.75 & 1.39 & 1.64 & 1.82 & 1.93 \\ 
   &  &  & 414 & 8.00 & 1.98 & 1.77 & 1.51 & 1.67 & 1.86 & 1.95 \\ 
   &  &  & 409.6 & 7.00 & 2.01 & 1.75 & 1.41 & 1.66 & 1.81 & 1.96 \\ 
   &  &  & 405.2 & 7.00 & 2.02 & 1.77 & 1.35 & 1.60 & 1.87 & 1.94 \\ 
   &  &  & 400.8 & 7.00 & 2.02 & 1.80 & 1.39 & 1.68 & 1.89 & 2.03 \\ 
   &  &  & 396.4 & 8.00 & 1.93 & 1.76 & 1.36 & 1.63 & 1.83 & 2.02 \\ 
   &  &  & 392 & 8.00 & 2.09 & 1.87 & 1.42 & 1.72 & 1.95 & 2.03 \\ 
   &  &  & 387.6 & 4.00 & 2.17 & 1.54 & 0.66 & 1.39 & 1.86 & 2.07 \\ 
   &  &  & 383.2 & 4.00 & 2.07 & 1.54 & 0.57 & 1.17 & 1.86 & 2.07 \\ 
   &  &  & 378.8 & 1.00 &  &  &  &  &  &  \\ 
   &  &  & 374.4 & 3.00 & 1.62 & 1.07 & 0.00 & 0.76 & 1.36 & 1.84 \\ 
   & number & equalbins & 465.98 - 461.29 & 14.00 & 2.21 & 2.06 & 1.88 & 1.99 & 2.10 & 2.18 \\ 
   &  &  & 461.29 - 456.61 & 11.00 & 2.26 & 2.06 & 1.78 & 1.97 & 2.13 & 2.18 \\ 
   &  &  & 456.61 - 451.92 & 17.00 & 2.16 & 2.05 & 1.87 & 2.01 & 2.09 & 2.18 \\ 
   &  &  & 451.92 - 447.23 & 12.00 & 2.23 & 2.06 & 1.87 & 2.01 & 2.11 & 2.18 \\ 
   &  &  & 447.23 - 442.54 & 28.00 & 2.11 & 2.03 & 1.90 & 1.99 & 2.07 & 2.13 \\ 
   &  &  & 442.54 - 437.85 & 12.00 & 2.09 & 1.93 & 1.72 & 1.86 & 2.01 & 2.05 \\ 
   &  &  & 437.85 - 433.16 & 11.00 & 2.11 & 1.94 & 1.66 & 1.87 & 2.00 & 2.08 \\ 
   &  &  & 433.16 - 428.47 & 12.00 & 2.04 & 1.90 & 1.64 & 1.81 & 1.94 & 2.02 \\ 
   &  &  & 428.47 - 423.78 & 16.00 & 2.07 & 1.95 & 1.81 & 1.90 & 1.99 & 2.07 \\ 
   &  &  & 423.78 - 419.09 & 19.00 & 2.16 & 2.05 & 1.93 & 2.02 & 2.08 & 2.12 \\ 
   &  &  & 419.09 - 414.4 & 8.00 & 1.98 & 1.75 & 1.45 & 1.66 & 1.83 & 1.94 \\ 
   &  &  & 414.4 - 409.71 & 6.00 & 2.09 & 1.81 & 1.18 & 1.68 & 1.95 & 2.06 \\ 
   &  &  & 409.71 - 405.02 & 6.00 & 2.08 & 1.74 & 1.27 & 1.61 & 1.90 & 2.01 \\ 
   &  &  & 405.02 - 400.34 & 6.00 & 2.04 & 1.74 & 1.12 & 1.59 & 1.85 & 2.01 \\ 
   &  &  & 400.34 - 395.65 & 6.00 & 2.03 & 1.74 & 1.11 & 1.58 & 1.88 & 2.03 \\ 
   &  &  & 395.65 - 390.96 & 7.00 & 2.19 & 1.90 & 1.43 & 1.78 & 2.01 & 2.11 \\ 
   &  &  & 390.96 - 386.27 & 7.00 & 2.06 & 1.86 & 1.33 & 1.70 & 1.95 & 2.03 \\ 
   &  &  & 386.27 - 381.58 & 4.00 & 2.07 & 1.56 & 0.57 & 1.37 & 1.87 & 2.07 \\ 
   &  &  & 381.58 - 376.89 & 2.00 & 1.14 & 0.00 & 0.00 & 0.00 & 1.14 & 1.14 \\ 
   &  &  & 376.89 - 372.2 & 3.00 & 2.10 & 1.41 & 0.00 & 1.34 & 1.45 & 2.10 \\ 
   &  & acctran & 465.98 & 12.00 & 2.24 & 2.07 & 1.82 & 2.01 & 2.12 & 2.18 \\ 
   &  &  & 461.05 & 12.00 & 2.20 & 2.03 & 1.83 & 1.96 & 2.10 & 2.16 \\ 
   &  &  & 456.11 & 10.00 & 2.24 & 2.07 & 1.77 & 1.99 & 2.11 & 2.22 \\ 
   &  &  & 451.18 & 12.00 & 2.20 & 2.04 & 1.84 & 1.97 & 2.10 & 2.18 \\ 
   &  &  & 446.24 & 14.00 & 2.17 & 2.02 & 1.82 & 1.96 & 2.08 & 2.15 \\ 
   &  &  & 441.3 & 12.00 & 2.05 & 1.89 & 1.70 & 1.80 & 1.97 & 2.05 \\ 
   &  &  & 436.37 & 12.00 & 2.05 & 1.90 & 1.65 & 1.82 & 1.96 & 2.03 \\ 
   &  &  & 431.43 & 12.00 & 2.04 & 1.90 & 1.61 & 1.82 & 1.96 & 2.06 \\ 
   &  &  & 426.5 & 15.00 & 2.14 & 2.01 & 1.84 & 1.93 & 2.04 & 2.12 \\ 
   &  &  & 421.56 & 11.00 & 2.13 & 1.94 & 1.75 & 1.88 & 2.00 & 2.09 \\ 
   &  &  & 416.62 & 8.00 & 1.99 & 1.74 & 1.39 & 1.65 & 1.83 & 1.92 \\ 
   &  &  & 411.69 & 8.00 & 2.00 & 1.79 & 1.47 & 1.70 & 1.87 & 1.95 \\ 
   &  &  & 406.75 & 7.00 & 2.11 & 1.86 & 1.51 & 1.73 & 1.95 & 2.10 \\ 
   &  &  & 401.82 & 7.00 & 2.11 & 1.88 & 1.39 & 1.73 & 1.95 & 2.05 \\ 
   &  &  & 396.88 & 7.00 & 2.11 & 1.82 & 1.46 & 1.71 & 1.96 & 2.08 \\ 
   &  &  & 391.94 & 7.00 & 2.16 & 1.87 & 1.48 & 1.79 & 1.98 & 2.10 \\ 
   &  &  & 387.01 & 4.00 & 2.07 & 1.54 & 0.00 & 1.44 & 1.87 & 2.07 \\ 
   &  &  & 382.07 & 1.00 &  &  &  &  &  &  \\ 
   &  &  & 377.14 & 3.00 & 2.10 & 1.41 & 0.00 & 1.34 & 2.10 & 2.10 \\ 
   &  &  & 372.2 & 3.00 & 2.10 & 1.41 & 0.00 & 1.34 & 1.45 & 2.10 \\ 
   &  & deltran & 465.98 & 12.00 & 2.18 & 2.01 & 1.72 & 1.91 & 2.08 & 2.17 \\ 
   &  &  & 461.05 & 12.00 & 2.24 & 2.08 & 1.89 & 2.03 & 2.12 & 2.18 \\ 
   &  &  & 456.11 & 10.00 & 2.27 & 2.07 & 1.83 & 2.01 & 2.13 & 2.18 \\ 
   &  &  & 451.18 & 12.00 & 2.19 & 2.01 & 1.76 & 1.94 & 2.08 & 2.18 \\ 
   &  &  & 446.24 & 13.00 & 2.18 & 2.02 & 1.80 & 1.95 & 2.07 & 2.16 \\ 
   &  &  & 441.3 & 12.00 & 2.08 & 1.93 & 1.73 & 1.86 & 2.00 & 2.07 \\ 
   &  &  & 436.37 & 12.00 & 2.08 & 1.91 & 1.66 & 1.82 & 1.97 & 2.05 \\ 
   &  &  & 431.43 & 12.00 & 2.09 & 1.93 & 1.67 & 1.88 & 1.98 & 2.06 \\ 
   &  &  & 426.5 & 15.00 & 2.04 & 1.92 & 1.73 & 1.86 & 1.99 & 2.04 \\ 
   &  &  & 421.56 & 11.00 & 2.13 & 1.94 & 1.63 & 1.88 & 2.01 & 2.09 \\ 
   &  &  & 416.62 & 8.00 & 2.03 & 1.79 & 1.55 & 1.70 & 1.86 & 1.97 \\ 
   &  &  & 411.69 & 8.00 & 1.98 & 1.75 & 1.35 & 1.69 & 1.84 & 1.93 \\ 
   &  &  & 406.75 & 7.00 & 2.03 & 1.76 & 1.36 & 1.59 & 1.87 & 1.97 \\ 
   &  &  & 401.82 & 7.00 & 2.03 & 1.79 & 1.42 & 1.65 & 1.86 & 1.96 \\ 
   &  &  & 396.88 & 7.00 & 2.02 & 1.74 & 1.22 & 1.57 & 1.87 & 1.98 \\ 
   &  &  & 391.94 & 7.00 & 2.17 & 1.86 & 1.33 & 1.72 & 1.97 & 2.09 \\ 
   &  &  & 387.01 & 4.00 & 2.17 & 1.73 & 1.07 & 1.43 & 1.86 & 2.17 \\ 
   &  &  & 382.07 & 1.00 &  &  &  &  &  &  \\ 
   &  &  & 377.14 & 2.00 & 1.14 & 0.00 & 0.00 & 0.00 & 1.14 & 1.14 \\ 
   &  &  & 372.2 & 3.00 & 2.10 & 1.45 & 0.00 & 1.34 & 1.45 & 2.10 \\ 
   &  & random & 465.98 & 12.00 & 2.18 & 2.01 & 1.78 & 1.91 & 2.08 & 2.16 \\ 
   &  &  & 461.05 & 12.00 & 2.20 & 2.05 & 1.76 & 1.99 & 2.11 & 2.19 \\ 
   &  &  & 456.11 & 10.00 & 2.23 & 2.01 & 1.73 & 1.92 & 2.09 & 2.19 \\ 
   &  &  & 451.18 & 12.00 & 2.19 & 2.04 & 1.84 & 1.97 & 2.09 & 2.15 \\ 
   &  &  & 446.24 & 13.00 & 2.18 & 2.02 & 1.78 & 1.94 & 2.08 & 2.15 \\ 
   &  &  & 441.3 & 12.00 & 2.09 & 1.93 & 1.63 & 1.86 & 1.98 & 2.08 \\ 
   &  &  & 436.37 & 12.00 & 2.05 & 1.89 & 1.61 & 1.78 & 1.97 & 2.05 \\ 
   &  &  & 431.43 & 12.00 & 2.05 & 1.89 & 1.65 & 1.82 & 1.97 & 2.04 \\ 
   &  &  & 426.5 & 15.00 & 2.04 & 1.91 & 1.74 & 1.87 & 1.97 & 2.03 \\ 
   &  &  & 421.56 & 11.00 & 2.13 & 1.95 & 1.72 & 1.88 & 2.02 & 2.10 \\ 
   &  &  & 416.62 & 8.00 & 1.98 & 1.75 & 1.46 & 1.67 & 1.83 & 1.95 \\ 
   &  &  & 411.69 & 8.00 & 2.00 & 1.77 & 1.47 & 1.69 & 1.85 & 1.95 \\ 
   &  &  & 406.75 & 7.00 & 2.11 & 1.82 & 1.26 & 1.67 & 1.94 & 2.05 \\ 
   &  &  & 401.82 & 7.00 & 2.03 & 1.77 & 1.38 & 1.67 & 1.86 & 1.98 \\ 
   &  &  & 396.88 & 7.00 & 2.11 & 1.84 & 1.40 & 1.72 & 1.96 & 2.05 \\ 
   &  &  & 391.94 & 7.00 & 2.16 & 1.87 & 1.37 & 1.73 & 1.97 & 2.07 \\ 
   &  &  & 387.01 & 4.00 & 2.17 & 1.74 & 0.94 & 1.26 & 1.86 & 2.17 \\ 
   &  &  & 382.07 & 1.00 &  &  &  &  &  &  \\ 
   &  &  & 377.14 & 3.00 & 1.59 & 1.07 & 0.00 & 0.76 & 1.34 & 1.59 \\ 
   &  &  & 372.2 & 3.00 & 2.10 & 1.45 & 0.00 & 1.34 & 2.10 & 2.10 \\ 
   &  & proximity & 465.98 & 12.00 & 2.24 & 2.06 & 1.83 & 1.99 & 2.13 & 2.19 \\ 
   &  &  & 461.05 & 12.00 & 2.25 & 2.08 & 1.83 & 2.00 & 2.14 & 2.20 \\ 
   &  &  & 456.11 & 10.00 & 2.23 & 2.00 & 1.72 & 1.92 & 2.07 & 2.20 \\ 
   &  &  & 451.18 & 12.00 & 2.19 & 2.03 & 1.79 & 1.96 & 2.08 & 2.15 \\ 
   &  &  & 446.24 & 13.00 & 2.18 & 2.03 & 1.69 & 1.96 & 2.09 & 2.16 \\ 
   &  &  & 441.3 & 12.00 & 2.08 & 1.91 & 1.66 & 1.83 & 1.96 & 2.06 \\ 
   &  &  & 436.37 & 12.00 & 2.09 & 1.91 & 1.71 & 1.86 & 1.98 & 2.04 \\ 
   &  &  & 431.43 & 12.00 & 2.04 & 1.91 & 1.68 & 1.82 & 1.98 & 2.05 \\ 
   &  &  & 426.5 & 15.00 & 2.09 & 1.95 & 1.78 & 1.89 & 1.99 & 2.05 \\ 
   &  &  & 421.56 & 11.00 & 2.13 & 1.92 & 1.67 & 1.86 & 1.99 & 2.09 \\ 
   &  &  & 416.62 & 8.00 & 1.98 & 1.79 & 1.44 & 1.69 & 1.86 & 1.97 \\ 
   &  &  & 411.69 & 8.00 & 1.99 & 1.78 & 1.45 & 1.68 & 1.86 & 1.96 \\ 
   &  &  & 406.75 & 7.00 & 2.02 & 1.75 & 1.26 & 1.62 & 1.82 & 1.95 \\ 
   &  &  & 401.82 & 7.00 & 2.02 & 1.79 & 1.40 & 1.69 & 1.86 & 1.98 \\ 
   &  &  & 396.88 & 7.00 & 2.01 & 1.74 & 1.37 & 1.61 & 1.83 & 1.98 \\ 
   &  &  & 391.94 & 7.00 & 2.16 & 1.86 & 1.44 & 1.75 & 1.97 & 2.06 \\ 
   &  &  & 387.01 & 4.00 & 2.07 & 1.74 & 0.57 & 1.47 & 1.88 & 2.07 \\ 
   &  &  & 382.07 & 1.00 &  &  &  &  &  &  \\ 
   &  &  & 377.14 & 2.00 & 1.14 & 0.57 & 0.00 & 0.00 & 1.14 & 1.14 \\ 
   &  &  & 372.2 & 3.00 & 2.10 & 1.41 & 0.00 & 1.34 & 2.10 & 2.10 \\ 
   &  & equal.split & 465.98 & 14.00 & 2.09 & 2.02 & 1.79 & 1.92 & 2.06 & 2.14 \\ 
   &  &  & 461.05 & 14.00 & 2.22 & 2.06 & 1.87 & 2.01 & 2.11 & 2.16 \\ 
   &  &  & 456.11 & 13.00 & 2.15 & 2.04 & 1.75 & 1.96 & 2.08 & 2.15 \\ 
   &  &  & 451.18 & 14.00 & 2.14 & 2.02 & 1.73 & 1.96 & 2.06 & 2.16 \\ 
   &  &  & 446.24 & 14.00 & 2.17 & 2.01 & 1.82 & 1.95 & 2.06 & 2.13 \\ 
   &  &  & 441.3 & 15.00 & 2.02 & 1.89 & 1.73 & 1.84 & 1.95 & 2.01 \\ 
   &  &  & 436.37 & 13.00 & 2.07 & 1.90 & 1.70 & 1.84 & 1.96 & 2.04 \\ 
   &  &  & 431.43 & 12.00 & 2.09 & 1.90 & 1.67 & 1.81 & 1.96 & 2.03 \\ 
   &  &  & 426.5 & 17.00 & 2.05 & 1.96 & 1.80 & 1.91 & 2.00 & 2.09 \\ 
   &  &  & 421.56 & 12.00 & 2.15 & 1.98 & 1.74 & 1.94 & 2.04 & 2.09 \\ 
   &  &  & 416.62 & 8.00 & 2.03 & 1.79 & 1.39 & 1.70 & 1.87 & 1.95 \\ 
   &  &  & 411.69 & 8.00 & 2.00 & 1.76 & 1.40 & 1.67 & 1.84 & 1.95 \\ 
   &  &  & 406.75 & 7.00 & 2.03 & 1.80 & 1.45 & 1.68 & 1.92 & 2.09 \\ 
   &  &  & 401.82 & 7.00 & 2.11 & 1.79 & 1.43 & 1.64 & 1.90 & 2.03 \\ 
   &  &  & 396.88 & 7.00 & 2.11 & 1.83 & 1.14 & 1.73 & 1.94 & 2.04 \\ 
   &  &  & 391.94 & 8.00 & 2.09 & 1.82 & 1.38 & 1.65 & 1.94 & 2.04 \\ 
   &  &  & 387.01 & 4.00 & 2.07 & 1.77 & 0.36 & 1.48 & 1.87 & 2.17 \\ 
   &  &  & 382.07 & 1.00 &  &  &  &  &  &  \\ 
   &  &  & 377.14 & 3.00 & 1.59 & 1.07 & 0.00 & 0.76 & 1.41 & 2.10 \\ 
   &  &  & 372.2 & 3.00 & 2.10 & 1.41 & 0.00 & 1.34 & 1.45 & 2.10 \\ 
   &  & gradual.split & 465.98 & 14.00 & 2.15 & 2.03 & 1.83 & 1.94 & 2.08 & 2.15 \\ 
   &  &  & 461.05 & 14.00 & 2.22 & 2.07 & 1.87 & 1.99 & 2.12 & 2.19 \\ 
   &  &  & 456.11 & 13.00 & 2.15 & 2.00 & 1.71 & 1.93 & 2.08 & 2.15 \\ 
   &  &  & 451.18 & 14.00 & 2.12 & 2.02 & 1.75 & 1.96 & 2.07 & 2.13 \\ 
   &  &  & 446.24 & 14.00 & 2.17 & 1.98 & 1.75 & 1.91 & 2.04 & 2.11 \\ 
   &  &  & 441.3 & 15.00 & 2.02 & 1.90 & 1.71 & 1.83 & 1.95 & 2.02 \\ 
   &  &  & 436.37 & 13.00 & 2.07 & 1.93 & 1.70 & 1.83 & 1.98 & 2.05 \\ 
   &  &  & 431.43 & 12.00 & 2.04 & 1.90 & 1.67 & 1.80 & 1.96 & 2.03 \\ 
   &  &  & 426.5 & 17.00 & 2.14 & 1.98 & 1.78 & 1.94 & 2.02 & 2.10 \\ 
   &  &  & 421.56 & 12.00 & 2.15 & 1.96 & 1.77 & 1.90 & 2.01 & 2.09 \\ 
   &  &  & 416.62 & 8.00 & 1.98 & 1.78 & 1.53 & 1.70 & 1.86 & 1.97 \\ 
   &  &  & 411.69 & 8.00 & 1.98 & 1.78 & 1.41 & 1.65 & 1.85 & 1.95 \\ 
   &  &  & 406.75 & 7.00 & 2.03 & 1.78 & 1.32 & 1.64 & 1.86 & 1.99 \\ 
   &  &  & 401.82 & 7.00 & 2.02 & 1.75 & 1.33 & 1.65 & 1.87 & 1.97 \\ 
   &  &  & 396.88 & 7.00 & 2.01 & 1.77 & 1.30 & 1.68 & 1.92 & 2.07 \\ 
   &  &  & 391.94 & 8.00 & 2.09 & 1.83 & 1.43 & 1.66 & 1.92 & 2.05 \\ 
   &  &  & 387.01 & 4.00 & 2.07 & 1.55 & 0.57 & 1.17 & 1.87 & 2.07 \\ 
   &  &  & 382.07 & 1.00 &  &  &  &  &  &  \\ 
   &  &  & 377.14 & 3.00 & 1.47 & 0.76 & 0.00 & 0.00 & 0.76 & 0.76 \\ 
   &  &  & 372.2 & 3.00 & 2.10 & 1.41 & 0.00 & 1.34 & 1.45 & 2.10 \\ 
  Epoch & stratigraphy & equalbins & 470 - 458.4 & 19.00 & 2.09 & 1.99 & 1.82 & 1.94 & 2.04 & 2.10 \\ 
   &  &  & 458.4 - 443.8 & 35.00 & 2.12 & 2.08 & 1.97 & 2.04 & 2.11 & 2.15 \\ 
   &  &  & 443.8 - 433.4 & 16.00 & 2.02 & 1.89 & 1.71 & 1.85 & 1.94 & 2.02 \\ 
   &  &  & 433.4 - 427.4 & 15.00 & 2.04 & 1.92 & 1.74 & 1.86 & 1.98 & 2.05 \\ 
   &  &  & 427.4 - 423 & 17.00 & 2.09 & 1.98 & 1.81 & 1.92 & 2.02 & 2.08 \\ 
   &  &  & 423 - 419.2 & 19.00 & 2.16 & 2.05 & 1.95 & 2.01 & 2.08 & 2.13 \\ 
   &  &  & 419.2 - 393.3 & 20.00 & 2.08 & 1.98 & 1.87 & 1.94 & 2.02 & 2.08 \\ 
   &  &  & 393.3 - 382.7 & 9.00 & 2.12 & 1.93 & 1.53 & 1.86 & 2.00 & 2.10 \\ 
   &  & acctran & 464.2 & 12.00 & 2.25 & 2.07 & 1.83 & 2.01 & 2.12 & 2.19 \\ 
   &  &  & 451.1 & 12.00 & 2.20 & 2.02 & 1.84 & 1.96 & 2.08 & 2.16 \\ 
   &  &  & 438.6 & 12.00 & 2.05 & 1.90 & 1.65 & 1.80 & 1.97 & 2.04 \\ 
   &  &  & 430.4 & 12.00 & 2.09 & 1.93 & 1.60 & 1.87 & 2.00 & 2.06 \\ 
   &  &  & 425.2 & 15.00 & 2.14 & 2.01 & 1.84 & 1.97 & 2.06 & 2.10 \\ 
   &  &  & 421.1 & 11.00 & 2.13 & 1.96 & 1.70 & 1.89 & 2.01 & 2.09 \\ 
   &  &  & 406.25 & 7.00 & 2.11 & 1.86 & 1.51 & 1.73 & 1.96 & 2.09 \\ 
   &  &  & 388 & 7.00 & 2.16 & 1.91 & 1.46 & 1.77 & 2.00 & 2.07 \\ 
   &  & deltran & 464.2 & 12.00 & 2.18 & 2.01 & 1.67 & 1.93 & 2.09 & 2.18 \\ 
   &  &  & 451.1 & 12.00 & 2.19 & 2.03 & 1.74 & 1.96 & 2.08 & 2.16 \\ 
   &  &  & 438.6 & 12.00 & 2.08 & 1.91 & 1.77 & 1.86 & 1.98 & 2.07 \\ 
   &  &  & 430.4 & 12.00 & 2.05 & 1.90 & 1.63 & 1.79 & 1.95 & 2.03 \\ 
   &  &  & 425.2 & 15.00 & 2.04 & 1.90 & 1.69 & 1.85 & 1.97 & 2.05 \\ 
   &  &  & 421.1 & 11.00 & 2.13 & 1.95 & 1.71 & 1.86 & 2.01 & 2.10 \\ 
   &  &  & 406.25 & 7.00 & 2.03 & 1.78 & 1.34 & 1.70 & 1.87 & 1.98 \\ 
   &  &  & 388 & 7.00 & 2.17 & 1.88 & 1.40 & 1.77 & 1.96 & 2.12 \\ 
   &  & random & 464.2 & 12.00 & 2.19 & 2.03 & 1.76 & 1.97 & 2.08 & 2.18 \\ 
   &  &  & 451.1 & 12.00 & 2.20 & 2.04 & 1.84 & 1.97 & 2.09 & 2.15 \\ 
   &  &  & 438.6 & 12.00 & 2.08 & 1.93 & 1.68 & 1.87 & 1.99 & 2.07 \\ 
   &  &  & 430.4 & 12.00 & 2.04 & 1.88 & 1.65 & 1.80 & 1.93 & 2.02 \\ 
   &  &  & 425.2 & 15.00 & 2.04 & 1.89 & 1.74 & 1.83 & 1.97 & 2.05 \\ 
   &  &  & 421.1 & 11.00 & 2.12 & 1.94 & 1.73 & 1.86 & 2.00 & 2.09 \\ 
   &  &  & 406.25 & 7.00 & 2.03 & 1.76 & 1.35 & 1.61 & 1.86 & 1.98 \\ 
   &  &  & 388 & 7.00 & 2.16 & 1.85 & 1.46 & 1.75 & 1.96 & 2.08 \\ 
   &  & proximity & 464.2 & 12.00 & 2.24 & 2.05 & 1.80 & 2.00 & 2.11 & 2.19 \\ 
   &  &  & 451.1 & 12.00 & 2.19 & 2.05 & 1.81 & 2.00 & 2.10 & 2.16 \\ 
   &  &  & 438.6 & 12.00 & 2.09 & 1.90 & 1.70 & 1.85 & 1.97 & 2.04 \\ 
   &  &  & 430.4 & 12.00 & 2.04 & 1.88 & 1.72 & 1.81 & 1.95 & 2.02 \\ 
   &  &  & 425.2 & 15.00 & 2.14 & 2.00 & 1.84 & 1.95 & 2.03 & 2.09 \\ 
   &  &  & 421.1 & 11.00 & 2.13 & 1.95 & 1.75 & 1.89 & 2.01 & 2.08 \\ 
   &  &  & 406.25 & 7.00 & 2.02 & 1.75 & 1.35 & 1.66 & 1.85 & 1.94 \\ 
   &  &  & 388 & 7.00 & 2.16 & 1.89 & 1.55 & 1.78 & 1.97 & 2.09 \\ 
   &  & equal.split & 464.2 & 14.00 & 2.19 & 2.05 & 1.87 & 2.00 & 2.09 & 2.15 \\ 
   &  &  & 451.1 & 14.00 & 2.12 & 2.01 & 1.79 & 1.97 & 2.07 & 2.15 \\ 
   &  &  & 438.6 & 13.00 & 2.04 & 1.92 & 1.68 & 1.86 & 1.98 & 2.06 \\ 
   &  &  & 430.4 & 13.00 & 2.04 & 1.89 & 1.63 & 1.79 & 1.95 & 2.05 \\ 
   &  &  & 425.2 & 17.00 & 2.09 & 1.99 & 1.81 & 1.93 & 2.03 & 2.08 \\ 
   &  &  & 421.1 & 12.00 & 2.15 & 1.97 & 1.73 & 1.91 & 2.02 & 2.09 \\ 
   &  &  & 406.25 & 7.00 & 2.02 & 1.84 & 1.50 & 1.75 & 1.89 & 2.03 \\ 
   &  &  & 388 & 7.00 & 2.17 & 1.88 & 1.55 & 1.76 & 1.97 & 2.09 \\ 
   &  & gradual.split & 464.2 & 14.00 & 2.15 & 2.05 & 1.89 & 1.99 & 2.09 & 2.14 \\ 
   &  &  & 451.1 & 14.00 & 2.18 & 2.01 & 1.84 & 1.96 & 2.07 & 2.13 \\ 
   &  &  & 438.6 & 13.00 & 2.07 & 1.92 & 1.67 & 1.84 & 1.98 & 2.05 \\ 
   &  &  & 430.4 & 13.00 & 1.98 & 1.84 & 1.67 & 1.77 & 1.92 & 2.01 \\ 
   &  &  & 425.2 & 17.00 & 2.14 & 2.02 & 1.86 & 1.95 & 2.06 & 2.10 \\ 
   &  &  & 421.1 & 12.00 & 2.15 & 1.96 & 1.69 & 1.88 & 2.02 & 2.08 \\ 
   &  &  & 406.25 & 7.00 & 2.02 & 1.76 & 1.27 & 1.65 & 1.85 & 1.96 \\ 
   &  &  & 388 & 7.00 & 2.16 & 1.85 & 1.39 & 1.74 & 1.94 & 2.09 \\ 
   & duration & equalbins & 466.7 - 456.2 & 16.00 & 2.18 & 2.06 & 1.85 & 1.99 & 2.09 & 2.15 \\ 
   &  &  & 456.2 - 445.7 & 18.00 & 2.12 & 2.00 & 1.81 & 1.95 & 2.06 & 2.12 \\ 
   &  &  & 445.7 - 435.2 & 29.00 & 2.11 & 2.04 & 1.94 & 2.00 & 2.08 & 2.12 \\ 
   &  &  & 435.2 - 424.7 & 17.00 & 2.00 & 1.91 & 1.71 & 1.85 & 1.96 & 2.02 \\ 
   &  &  & 424.7 - 414.2 & 23.00 & 2.11 & 2.02 & 1.93 & 1.99 & 2.06 & 2.11 \\ 
   &  &  & 414.2 - 403.7 & 7.00 & 2.05 & 1.80 & 1.44 & 1.71 & 1.90 & 2.00 \\ 
   &  &  & 403.7 - 393.2 & 8.00 & 1.97 & 1.77 & 1.29 & 1.64 & 1.84 & 1.94 \\ 
   &  &  & 393.2 - 382.7 & 8.00 & 2.09 & 1.86 & 1.37 & 1.72 & 1.93 & 2.04 \\ 
   &  & acctran & 461.45 & 12.00 & 2.20 & 2.02 & 1.71 & 1.95 & 2.10 & 2.18 \\ 
   &  &  & 450.95 & 12.00 & 2.20 & 2.03 & 1.78 & 1.96 & 2.08 & 2.16 \\ 
   &  &  & 440.45 & 12.00 & 2.05 & 1.90 & 1.69 & 1.82 & 1.96 & 2.05 \\ 
   &  &  & 429.95 & 12.00 & 2.16 & 2.01 & 1.82 & 1.93 & 2.06 & 2.12 \\ 
   &  &  & 419.45 & 11.00 & 2.13 & 1.94 & 1.67 & 1.86 & 2.02 & 2.10 \\ 
   &  &  & 408.95 & 7.00 & 2.02 & 1.78 & 1.39 & 1.70 & 1.87 & 1.96 \\ 
   &  &  & 398.45 & 7.00 & 2.11 & 1.84 & 1.44 & 1.75 & 1.94 & 2.06 \\ 
   &  &  & 387.95 & 7.00 & 2.16 & 1.91 & 1.50 & 1.78 & 1.99 & 2.06 \\ 
   &  & deltran & 461.45 & 12.00 & 2.24 & 2.09 & 1.86 & 2.03 & 2.14 & 2.20 \\ 
   &  &  & 450.95 & 12.00 & 2.19 & 2.00 & 1.74 & 1.94 & 2.07 & 2.16 \\ 
   &  &  & 440.45 & 12.00 & 2.08 & 1.91 & 1.61 & 1.83 & 1.97 & 2.05 \\ 
   &  &  & 429.95 & 12.00 & 2.04 & 1.88 & 1.69 & 1.80 & 1.93 & 2.01 \\ 
   &  &  & 419.45 & 11.00 & 2.13 & 1.94 & 1.67 & 1.87 & 2.00 & 2.09 \\ 
   &  &  & 408.95 & 7.00 & 2.02 & 1.74 & 1.35 & 1.62 & 1.84 & 1.96 \\ 
   &  &  & 398.45 & 7.00 & 2.02 & 1.78 & 1.39 & 1.68 & 1.89 & 1.97 \\ 
   &  &  & 387.95 & 7.00 & 2.17 & 1.88 & 1.42 & 1.77 & 1.98 & 2.08 \\ 
   &  & random & 461.45 & 12.00 & 2.19 & 2.01 & 1.81 & 1.97 & 2.07 & 2.15 \\ 
   &  &  & 450.95 & 12.00 & 2.19 & 2.04 & 1.85 & 1.98 & 2.08 & 2.16 \\ 
   &  &  & 440.45 & 12.00 & 2.08 & 1.91 & 1.70 & 1.85 & 1.96 & 2.07 \\ 
   &  &  & 429.95 & 12.00 & 2.11 & 1.94 & 1.71 & 1.88 & 2.00 & 2.06 \\ 
   &  &  & 419.45 & 11.00 & 2.12 & 1.91 & 1.74 & 1.85 & 1.98 & 2.08 \\ 
   &  &  & 408.95 & 7.00 & 2.02 & 1.78 & 1.38 & 1.66 & 1.84 & 1.98 \\ 
   &  &  & 398.45 & 7.00 & 2.11 & 1.82 & 1.40 & 1.73 & 1.93 & 2.04 \\ 
   &  &  & 387.95 & 7.00 & 2.17 & 1.89 & 1.40 & 1.80 & 1.97 & 2.09 \\ 
   &  & proximity & 461.45 & 12.00 & 2.25 & 2.06 & 1.75 & 1.99 & 2.13 & 2.18 \\ 
   &  &  & 450.95 & 12.00 & 2.19 & 2.02 & 1.75 & 1.94 & 2.09 & 2.14 \\ 
   &  &  & 440.45 & 12.00 & 2.08 & 1.92 & 1.69 & 1.85 & 1.97 & 2.06 \\ 
   &  &  & 429.95 & 12.00 & 2.04 & 1.86 & 1.65 & 1.78 & 1.94 & 2.02 \\ 
   &  &  & 419.45 & 11.00 & 2.13 & 1.96 & 1.69 & 1.87 & 2.02 & 2.09 \\ 
   &  &  & 408.95 & 7.00 & 2.03 & 1.80 & 1.42 & 1.70 & 1.89 & 2.01 \\ 
   &  &  & 398.45 & 7.00 & 2.01 & 1.74 & 1.14 & 1.59 & 1.82 & 1.95 \\ 
   &  &  & 387.95 & 7.00 & 2.16 & 1.88 & 1.49 & 1.73 & 1.97 & 2.07 \\ 
   &  & equal.split & 461.45 & 14.00 & 2.19 & 2.05 & 1.87 & 2.00 & 2.10 & 2.16 \\ 
   &  &  & 450.95 & 14.00 & 2.18 & 2.03 & 1.84 & 1.98 & 2.08 & 2.13 \\ 
   &  &  & 440.45 & 13.00 & 2.03 & 1.91 & 1.73 & 1.85 & 1.96 & 2.04 \\ 
   &  &  & 429.95 & 14.00 & 2.06 & 1.96 & 1.69 & 1.86 & 2.00 & 2.07 \\ 
   &  &  & 419.45 & 11.00 & 2.13 & 1.97 & 1.76 & 1.90 & 2.02 & 2.11 \\ 
   &  &  & 408.95 & 7.00 & 2.02 & 1.75 & 1.41 & 1.65 & 1.86 & 2.00 \\ 
   &  &  & 398.45 & 7.00 & 2.02 & 1.77 & 1.20 & 1.67 & 1.87 & 2.07 \\ 
   &  &  & 387.95 & 7.00 & 2.17 & 1.89 & 1.47 & 1.80 & 1.96 & 2.10 \\ 
   &  & gradual.split & 461.45 & 14.00 & 2.22 & 2.07 & 1.87 & 2.02 & 2.11 & 2.18 \\ 
   &  &  & 450.95 & 14.00 & 2.12 & 2.01 & 1.84 & 1.97 & 2.07 & 2.13 \\ 
   &  &  & 440.45 & 13.00 & 2.07 & 1.92 & 1.70 & 1.87 & 1.98 & 2.02 \\ 
   &  &  & 429.95 & 14.00 & 2.01 & 1.90 & 1.68 & 1.83 & 1.95 & 2.02 \\ 
   &  &  & 419.45 & 11.00 & 2.13 & 1.97 & 1.68 & 1.91 & 2.01 & 2.08 \\ 
   &  &  & 408.95 & 7.00 & 2.03 & 1.79 & 1.37 & 1.65 & 1.88 & 1.99 \\ 
   &  &  & 398.45 & 7.00 & 2.01 & 1.84 & 1.36 & 1.69 & 1.94 & 2.09 \\ 
   &  &  & 387.95 & 7.00 & 2.16 & 1.92 & 1.49 & 1.83 & 1.98 & 2.09 \\ 
   & number & equalbins & 465.98 - 454.26 & 17.00 & 2.15 & 2.04 & 1.89 & 2.00 & 2.08 & 2.12 \\ 
   &  &  & 454.26 - 442.54 & 33.00 & 2.11 & 2.05 & 1.93 & 2.01 & 2.08 & 2.13 \\ 
   &  &  & 442.54 - 430.81 & 14.00 & 2.01 & 1.88 & 1.68 & 1.81 & 1.93 & 2.01 \\ 
   &  &  & 430.81 - 419.09 & 23.00 & 2.09 & 1.99 & 1.86 & 1.95 & 2.02 & 2.07 \\ 
   &  &  & 419.09 - 407.37 & 9.00 & 1.97 & 1.76 & 1.50 & 1.68 & 1.85 & 1.92 \\ 
   &  &  & 407.37 - 395.65 & 7.00 & 1.96 & 1.74 & 1.27 & 1.61 & 1.84 & 1.96 \\ 
   &  &  & 395.65 - 383.92 & 8.00 & 2.11 & 1.86 & 1.48 & 1.75 & 1.96 & 2.04 \\ 
   &  &  & 383.92 - 372.2 & 9.00 & 1.92 & 1.75 & 1.34 & 1.57 & 1.85 & 2.00 \\ 
   &  & acctran & 465.98 & 12.00 & 2.24 & 2.08 & 1.90 & 2.02 & 2.12 & 2.19 \\ 
   &  &  & 452.59 & 12.00 & 2.20 & 2.04 & 1.81 & 1.97 & 2.09 & 2.16 \\ 
   &  &  & 439.19 & 12.00 & 2.05 & 1.90 & 1.70 & 1.83 & 1.98 & 2.04 \\ 
   &  &  & 425.79 & 15.00 & 2.14 & 2.01 & 1.87 & 1.96 & 2.06 & 2.12 \\ 
   &  &  & 412.39 & 8.00 & 2.00 & 1.76 & 1.31 & 1.67 & 1.82 & 1.94 \\ 
   &  &  & 399 & 7.00 & 2.11 & 1.82 & 1.32 & 1.70 & 1.94 & 2.05 \\ 
   &  &  & 385.6 & 4.00 & 2.07 & 1.74 & 0.66 & 1.47 & 1.87 & 2.07 \\ 
   &  &  & 372.2 & 3.00 & 2.10 & 1.41 & 0.00 & 1.34 & 1.45 & 2.10 \\ 
   &  & deltran & 465.98 & 12.00 & 2.18 & 2.03 & 1.86 & 1.97 & 2.08 & 2.14 \\ 
   &  &  & 452.59 & 12.00 & 2.19 & 2.04 & 1.78 & 1.97 & 2.10 & 2.16 \\ 
   &  &  & 439.19 & 12.00 & 2.08 & 1.94 & 1.65 & 1.86 & 2.00 & 2.06 \\ 
   &  &  & 425.79 & 15.00 & 2.04 & 1.94 & 1.65 & 1.89 & 1.99 & 2.06 \\ 
   &  &  & 412.39 & 8.00 & 1.98 & 1.74 & 1.36 & 1.67 & 1.83 & 1.93 \\ 
   &  &  & 399 & 7.00 & 2.02 & 1.77 & 1.33 & 1.66 & 1.86 & 1.96 \\ 
   &  &  & 385.6 & 4.00 & 2.17 & 1.74 & 1.00 & 1.48 & 1.87 & 2.17 \\ 
   &  &  & 372.2 & 3.00 & 2.10 & 1.41 & 0.00 & 1.34 & 1.45 & 2.10 \\ 
   &  & random & 465.98 & 12.00 & 2.24 & 2.05 & 1.84 & 2.00 & 2.13 & 2.19 \\ 
   &  &  & 452.59 & 12.00 & 2.19 & 2.04 & 1.74 & 1.96 & 2.08 & 2.17 \\ 
   &  &  & 439.19 & 12.00 & 2.05 & 1.90 & 1.71 & 1.83 & 1.96 & 2.06 \\ 
   &  &  & 425.79 & 15.00 & 2.09 & 1.94 & 1.81 & 1.89 & 2.00 & 2.05 \\ 
   &  &  & 412.39 & 8.00 & 2.00 & 1.80 & 1.38 & 1.69 & 1.86 & 1.96 \\ 
   &  &  & 399 & 7.00 & 2.02 & 1.75 & 1.21 & 1.62 & 1.84 & 1.95 \\ 
   &  &  & 385.6 & 4.00 & 2.17 & 1.73 & 0.94 & 1.46 & 1.86 & 1.89 \\ 
   &  &  & 372.2 & 3.00 & 2.10 & 1.41 & 0.00 & 1.34 & 1.45 & 2.10 \\ 
   &  & proximity & 465.98 & 12.00 & 2.24 & 2.08 & 1.82 & 2.03 & 2.13 & 2.19 \\ 
   &  &  & 452.59 & 12.00 & 2.19 & 2.04 & 1.82 & 1.96 & 2.09 & 2.15 \\ 
   &  &  & 439.19 & 12.00 & 2.08 & 1.92 & 1.73 & 1.86 & 1.98 & 2.04 \\ 
   &  &  & 425.79 & 15.00 & 2.14 & 2.02 & 1.86 & 1.98 & 2.06 & 2.12 \\ 
   &  &  & 412.39 & 8.00 & 1.99 & 1.75 & 1.33 & 1.64 & 1.85 & 1.94 \\ 
   &  &  & 399 & 7.00 & 2.01 & 1.72 & 1.22 & 1.58 & 1.81 & 1.95 \\ 
   &  &  & 385.6 & 4.00 & 2.07 & 1.56 & 0.27 & 1.46 & 1.86 & 2.07 \\ 
   &  &  & 372.2 & 3.00 & 2.10 & 1.41 & 0.00 & 1.34 & 1.45 & 2.10 \\ 
   &  & equal.split & 465.98 & 14.00 & 2.15 & 1.98 & 1.69 & 1.93 & 2.05 & 2.16 \\ 
   &  &  & 452.59 & 16.00 & 2.14 & 2.00 & 1.73 & 1.94 & 2.06 & 2.13 \\ 
   &  &  & 439.19 & 13.00 & 2.04 & 1.92 & 1.69 & 1.85 & 1.97 & 2.05 \\ 
   &  &  & 425.79 & 17.00 & 2.10 & 1.99 & 1.79 & 1.94 & 2.04 & 2.10 \\ 
   &  &  & 412.39 & 8.00 & 1.99 & 1.76 & 1.34 & 1.66 & 1.86 & 1.93 \\ 
   &  &  & 399 & 7.00 & 2.02 & 1.81 & 1.36 & 1.69 & 1.93 & 2.09 \\ 
   &  &  & 385.6 & 4.00 & 2.07 & 1.79 & 0.85 & 1.48 & 1.87 & 2.17 \\ 
   &  &  & 372.2 & 3.00 & 2.10 & 1.41 & 0.00 & 1.34 & 2.10 & 2.10 \\ 
   &  & gradual.split & 465.98 & 14.00 & 2.12 & 2.00 & 1.80 & 1.96 & 2.07 & 2.17 \\ 
   &  &  & 452.59 & 16.00 & 2.10 & 1.99 & 1.74 & 1.94 & 2.04 & 2.10 \\ 
   &  &  & 439.19 & 13.00 & 2.04 & 1.90 & 1.63 & 1.83 & 1.97 & 2.05 \\ 
   &  &  & 425.79 & 17.00 & 2.14 & 1.99 & 1.80 & 1.93 & 2.03 & 2.09 \\ 
   &  &  & 412.39 & 8.00 & 1.99 & 1.78 & 1.49 & 1.66 & 1.86 & 1.96 \\ 
   &  &  & 399 & 7.00 & 2.11 & 1.79 & 1.42 & 1.67 & 1.88 & 2.03 \\ 
   &  &  & 385.6 & 4.00 & 2.07 & 1.54 & 0.00 & 1.17 & 1.85 & 2.07 \\ 
   &  &  & 372.2 & 3.00 & 2.10 & 1.41 & 0.00 & 1.34 & 1.45 & 2.10 \\ 
   \hline
\hline
\caption{Disparity measurement (sum of variance) for Wright2017 dataset} 
\end{longtable}

  
%\end{landscape}

\end{document}