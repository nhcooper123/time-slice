\documentclass[12pt,letterpaper]{article}

%Packages
\usepackage{pdflscape}
\usepackage{fixltx2e}
\usepackage{textcomp}
\usepackage{fullpage}
\usepackage{float}
\usepackage{latexsym}
\usepackage{url}
\usepackage{epsfig}
\usepackage{graphicx}
\usepackage{amssymb}
\usepackage{amsmath}
\usepackage{bm}
\usepackage{array}
\usepackage[version=3]{mhchem}
\usepackage{ifthen}
\usepackage{caption}
\usepackage{hyperref}
\usepackage{amsthm}
\usepackage{amstext}
\usepackage{enumerate}
\usepackage[osf]{mathpazo}
\usepackage{dcolumn}
\usepackage{lineno}
\usepackage{color}
\usepackage[usenames,dvipsnames]{xcolor}
\pagenumbering{arabic}

%Pagination style and stuff
%\linespread{2} 

\raggedright
\setlength{\parindent}{0.5in}
\setcounter{secnumdepth}{0} 
\renewcommand{\section}[1]{%
\bigskip
\begin{center}
\begin{Large}
\normalfont\scshape #1
\medskip
\end{Large}
\end{center}}
\renewcommand{\subsection}[1]{%
\bigskip
\begin{center}
\begin{large}
\normalfont\itshape #1
\end{large}
\end{center}}
\renewcommand{\subsubsection}[1]{%
\vspace{2ex}
\noindent
\textit{#1.}---}
\renewcommand{\tableofcontents}{}

\setlength\parindent{0pt}

\begin{document}

\textbf{RE: PALA-12-17-4130-OA}\\
\bigskip
Dear Dr Smith,\\
\bigskip

We are very grateful to the referees for their helpful and constructive comments. 
Below we provide their comments in blue text with our responses.

%TG: TODO: fix ages in table 1

% ~~~~~~~~~~~~~~~~~~~~~~~~
%
% REVIEWER 1
%
% ~~~~~~~~~~~~~~~~~~~~~~~~


\section{Reviewer 1 (Steve Brusatte)}

\begin{enumerate}

\item{\textcolor{blue}{In the section on adding in reconstructed ancestors to the dataset before ordination, I suppose you should mention that this is similar to the approach that we outlined in our Paleobiology paper (Brusatte et al. 2011) and mention any key differences.}}

We now explicitly refer to Brusatte et al. 2011 and mention the key difference:

\textit{This approach is similar to Brusatte et al. (2011) but uses model based estimations (rather than parsimony) allowing to control for ambiguous nodes (i.e. poorly estimated).} lines @@@

\item{\textcolor{blue}{I'm not sure about the use of the term ‘cladisto-spaces’. To me this implies that the space itself is ordinated using phylogenetic information (e.g., branch lengths). Now, you are indeed including some phylogenetic information in your morphospaces, in terms of reconstructed ancestors and inferred states for branches. So sure, I think what you have could be called a ‘cladisto-space’. Same with most of the morphospace options produced by Claddis. But the ones that most of us have done in the past, simply using discrete characters to ordinate a morphospace, haven't generally incorporated phylogeny. I’m talking about the references you cite in lines 141-143 here. These morphospaces been ordinated using discrete characters of the type that are used in phylogenetic analysis. But the characters are not phylogenetically transformed in any way. As far as the ordination analysis knows, they are simply discrete character statements. So I wonder if the term ‘cladisto-space’ is confusing, and is just yet another new term that may muddy the waters rather than clarify. If you want to keep using this term, that's fine, but a sentence or two of explanation (about how previous authors have used discrete characters in a non-phylogenetic way) would be warranted.}}

We've change the term cladisto-space to morphospace throughout the manuscript.
Although we prefer the ``cladisto-space'' term, we agree with this concern about adding a ``new term that may muddy the waters rather than clarify''.
We've emphasised the differences, however, with other uses of the word ``morphospace'' in an explanatory paragraph:

\textit{To explore disparity-through-time in our datasets, we used a morphospace approach (CITATIONS).
Morphospaces can be obtained from any multidimensional morphological data set but can differ in the data used (e.g. discrete or continuous), and whether they include phylogenetic data or not. % NC: do you mean cladistic data i.e. data used to build phylogenies?
Although empirical morphospaces from discrete or continuous data have been shown to have similar properties (CITATIONS), our morphospaces are based on discrete morphological data (originally collected for phylogenetic analysis;} c.f. \textit{geometric morphometric data) and includes some phylogenetic information (see above).
Mathematically, our morphospaces are $n$ dimensional objects that summarise the distances between discrete morphological characters of the taxa present and their ancestors.} lines @@@

%TG: what do you think? % NC: sounds good, made a few minor edits above

\item{\textcolor{blue}{Figure 3 would be easier to interpret if the dataset names were added to the graphic itself (like in Figure 2) rather than constrained to the figure caption.}}

We have done as suggested - we didn't do this the first time round because of issues with ggplot. We also made the same change in the appendix figures, and changed the colours in response to a comment by reviewer 2.

%TG: Also for these figures, I think we need to try to plot a line rather than two dots for the bins interval. Its actually not that straight forward to realise what's happening.

\item{\textcolor{blue}{I may be misinterpreting Figure 3, but it seems like peak disparity values for the Brusatte et al. dataset are pretty much the same regardless of method used (second plot from top). This doesn't jibe with the statement in the text (lines 435-437). Please clarify.}}
\label{fig3_feckup}

We apologise for mismatching the datasets and the patterns described in the text.
We've now corrected this error and double check this as follows:

\textit{In the Beck \& Lee (2014) and Bapst et al. (2016) datasets, disparity peaks occur much at much older ages when time-slicing rather than time binning approaches are used (Fig. 3; Supporting Information Appendix S2: Figs A3-A4).
This is also true for stratigraphic time bins in the Wright (2017) dataset, although when using equal time bins the peaks are later than the time-slicing methods, or very similar (Fig. 3; Supporting Information Appendix S2: Figs A3-A4).
Across the three time binning methods, the Brusatte et al. (2014) dataset has similar disparity peaks whichever method is used, the Wright (2017) dataset only had variation in peaks when using unequal time bins (stratigraphy), whereas in the Bapst et al. (2016) and Beck \& Lee (2014) datasets, stratigraphic (unequal) versus equally sized time bins make a large difference to where the disparity peak occurs (Fig. 3; Supporting Information Appendix S2: Figs A3-A4).} lines @@@

We've also added a note about discrepancies in the time-slicing models in Beck \& Lee (2014): 

\textit{Additionally, there seems to be small discrepancies within the time-slicing methods (gradual vs. proximity) except in the Beck \& Lee (2014) dataset where the gradual split model recovered disparity peaks at younger ages than the proximity model (Fig. 3; Supporting Information Appendix S2: Figs A3-A4).} lines @@@

\end{enumerate}

% ~~~~~~~~~~~~~~~~~~~~~~~~
%
% REVIEWER 2
%
% ~~~~~~~~~~~~~~~~~~~~~~~~

\section{Reviewer 2}

\begin{enumerate}

\item{\textcolor{blue}{First, if I understand lines 278-299 (and the R annotation) correctly, the gradual evolution models are not in fact fundamentally different from the punctuated models. They are both assigning scores along the branch that are the same as either the ancestor or descendant — the difference is only that the selection is based on a probability in the ``gradual'' models, rather than a rule, as in the ``punctuated'' models. The concept of gradual change would be more closely modelled by assigning values along a branch that were in-between those assigned to the ancestor and descendent but based on how far along the branch any given point is. For example, let's say the value of the ancestor along PCO1 is 0.04 and that of the descendent is 0.03. For a point that is halfway along the branch linking them, the value should be 0.035 (instead of assigning either 0.03 or 0.04 based on a probability). This, of course, makes assumptions about the course of evolution between the ancestor and descendant, but at least would be describing gradual change.}}

The reviewer does indeed understand the \texttt{equal.splits} and \texttt{gradual.splits} models correctly: they select the values in the matrix from the edge's descendant or ancestor with a probability value for each.
We believe however that the reviewer's proposition of making these models ``actually'' gradual (e.g. by selecting the mean value between two points) might lead to incorrect interpretation of the results.

The rational for proposing the \texttt{equal.splits} and \texttt{gradual.splits} models rather than a pure gradual model is to not create any extra data points in the ordinated space.
In fact, doing so pre-ordination can change the ordinated space geometry, while doing so post-ordination leads to the incorrect estimation of the space: (1) the number of elements changes hence changing parameters for future statistical tests (e.g. degrees of freedom) and (2) the variance per axis will change as well.
Our methods allow us to approach a pure gradual model without modifying the ordinated space's structure and properties.
In fact, when applying the \texttt{equal.split} model (or \texttt{gradual.splits}) on a decent number of bootstrap replicates (say 100), the ancestor's PCO1 value (0.04 in the reviewer's example) will be sampled 50\% of the time and the same for the descendant's value (0.03 - with the probability changing as a function of branch length for the \texttt{gradual.splits} model).
The resulting metric, based on this value would be $0.5\times0.04 + 0.5\times0.03 = 0.035$ thus identical to what the reviewer suggests, without modifying the multidimensional space.

We realise we had not made this clear, thus we have added the following explanation in the text:

\textit{For example, using the ``equal.splits'' model on an ancestor and a descendant with PCO1 values of respectively $0.04$ and $0.03$, after a sufficient number of bootstrap replicates (e.g. 100) the value along the branch will be close to $0.5\times0.04 + 0.5\times0.03 = 0.035$.
By estimating this value rather than generating it (i.e. creating a new element mid-way along the branch that would be the average of the descendant and ancestor - $0.035$) we obtain the same results without modifying the morphospace properties.} lines @@@

%TG: I can implement that pretty easily but I think that mathematically it's not cool (we can not add values post-ordination in the ordinated space). Two options: (1) implementing the suggestion but with heavy warnings; (2) writing down a clever response to this comment. In the dispRity package review, that was highlighted (I bet reviewer 2 is Graeme) and I responded something similar to option (2) so I'd rather go that way for consistency.

% NC: I like this response. I dunno if it is Graeme, they use US spellings...



\item{\textcolor{blue}{Second, I do not see how Figure 3 supports the text (lines 435-447). Overall, these results show the exact opposite of what is stated in the text (lines 435-447): in the figure the time-binned-approach (pink) results show earlier disparity peaks than the time-slicing approaches (green and blue). Are the y-axes reversed from what they should be? Also the Brusatte et al 2014 (second row) shows very similar values, not different values (compare with lines 435-436). And the difference between the ages of the peaks in the Bapst et 2016 study (third row) is at most 6 million years...but the study itself is over 150 million years divided very coarsely in Figure 2.  On that scale, these are basically the "same time". Finally, whether the six different time-slicing methods show ``similar'' patterns (lines 412-413) depends on the question... it is hard to distinguish them in the relative disparity plots, but the timing of peak disparity varies widely (Figure A3) while the two shown in Figure 3 happen to be similar.}}

We apologise for the confusion in the description of Figure 3 and have fixed it (see Reviewer 1, comment \ref{fig3_feckup}). We were rushing to get this submitted by the symposium deadline and this error slipped through the cracks.


\item{\textcolor{blue}{Finally, I have spent some time thinking about by the statement ``all time binning approaches (whether equally sized or not) assume that characters of taxa evolve following a punctuated equilibrium model.'' (lines 70-71). I see what the authors are getting at — they are worried that doing this requires an assumption of constant evolution across the time bin. But I would argue that it is not necessary to interpret it this way. Instead it could simply represent an average across the interval sampled without the need to make any assumption about how change occurred within that interval. Plus, even in scenarios where there was enough change that at a course scale, it appeared that disparity was static, some of that volatility might be revealed if the time-scale were altered.}}

%TG: tone done the statement? % NC: yeah, not sure what else we can do.

\item{\textcolor{blue}{But there is also a deeper issue here, in that the fossil record is at its heart a time-binned record: rates of sedimentation vary and sediment deposition is sporadic, such that local beds represent different amounts of time, or different ``packages'' of time, separated by gaps, and the fossils within them may be time-averaged to various degrees. [...] Therefore, even if you pick a ``point in time'' and count the number of branches/ranges that cross that point, some of those are being captured because the resolution of the end points only allows us to know that the taxon started and/or ended somewhere in the interval encompassing that point in time.}}

%TG: this is a fair point we should include in the discussion.
%TG: though that is just a "practical" point for the vertebrate fossil record. Foraminiferas, for example, do not behave this way (or at a way finer scale).



\item{\textcolor{blue}{As an aside, the Hunt et al 2015 citation does not clarify the matter (line 74-75). A case could be made that all time series are binned (if you consider that the samples are likely time-averaged and there are gaps between samples--although this is different from dividing up "everything" into adjacent boxes), and yet there are time series that are best characterized by gradual directional change (e.g., best fit by a biased random walk). In addition, Hunt et al found that directional change was actually more likely to be detected in series of long duration. Furthermore, homogenous rate patterns were more likely to be supported than heterogenous/punctuated change with time series of fewer samples, which includes cases where each ``step'' includes a longer interval of time over which shorter changes could be hidden.}}

%TG: not sure where the reviewer want us to go with that. Is that the same nuancing point as the comment before? In that case we can simply merge them and make a caveat paragraph. 
% NC: Yeah I'm not sure either, but maybe needs a paragraph in the discussion? Currently the discussion is short (and written by me in about an hour!) so would be good to add something there.


\end{enumerate}

\subsection{Reviewer 2 - Minor comments}

\begin{enumerate}

\item{\textcolor{blue}{It is not immediately clear what is meant by ``defined by the average duration of the stratigraphic period'' (line 337, 341) vs ``defined by the average number of stratigraphic periods'' (line 348, 353). After thinking about it for awhile and then checking the R code, this is what appears to be going on. Let's say we have a case study that spans 10 stratigraphic intervals over 40 million years. Thus for the stratigraphy approach, there are 10 sub-samples which vary in their duration. Let’s say the average duration of those intervals is 5 million years. For the duration approach, the 40 million year span is divided into 5-million-year bins, resulting in 8 sub-samples. For the number approach, the 40 million year span is divided into 10, each of which is 4 million years long. If this is the case, the use of the word ``average'' when describing the number approach is confusing. I am also having trouble reconciling this with the number of points in each panel in Figure 2, unless there is also a minimum sample size applied (for example in the Bapst 2016 study, there are only three disparity values for the time-binned series using the ``number'' approach, maybe because after dividing the time scale up this way, there were not enough taxa in the first and second time bins?). It may be helpful to include a graphic, even if in relegated to one of the appendices.}}

%TG: good idea, we (I) can make a figure designating the three different strategies.

\item{\textcolor{blue}{Line 158. Gower’s distance is not the Euclidean distance (= the square root of the squared distances) over the number of shared characters. For a set of unordered characters, it is the number of mismatched character states over the number of comparable/shared characters. For a set of ordered characters, it is sum of the differences in the values of the characters states for each pairwise comparison, over the number of comparable characters. This is an edit that needs to be made only to the text; because the authors used functions from the Claddis package (see github documentation), they would have calculated Gower’s distance correctly.}}


\item{\textcolor{blue}{Consider using ``stage'' instead of ``age'' when referring to the finer resolution time scale.}}

We have changed ``stratigraphic age'' to ``stratigraphic stage'' throughout the manuscript.

\item{\textcolor{blue}{Figure 2. It is very difficult to distinguish the green dots from the blue dots. For someone who is color-blind, it would also be difficult to distinguish the red dots from the green dots.}}

%TG: I'll let you change the colours (I still don't want to touch ggplot!) how about the darkgrey/orange/blue palette?
%Here are the hexadecimals:
%colours_blind <- c("orange" = "#F65205", "blue" = "#3E9CBA", "darkgrey" = "#1C1C1C")
%NC: The colours are colour blind friendly already, but I'll switch to using viridis.

Interestingly the palettes in ggplot2 are chosen so that they are distinguishable by people that are colour blind. But it's no problem to change the colours. Hopefully the new colour scheme is easier to discern. We have used the viridis colour palettes which are designed for precisely this purpose (https://cran.r-project.org/web/packages/viridis/vignettes/intro-to-viridis.html), and are colour-blind friendly too. We've done this for all figures in the text and in the appendices.

A bigger issue, and this may be what the reviewer was referring to, is that a lot of the points lie on top of each other. We have played around with this a lot (we tried jittering and dodging point positions, and various types of colour palettes and transparencies etc.) and there's no satisfactory solution. For now we have made the error bars more transparent so hopefully it is a bit easier to see which points lie where.


\end{enumerate}


% ~~~~~~~~~~~~~~~~~~~~~~~~
%
% REVIEWER 3
%
% ~~~~~~~~~~~~~~~~~~~~~~~~

%NC: It says in the instructions to respond to these too!

\section{Reviewer 3 (Sally Thomas)}
\subsection{Technical comments}

\begin{enumerate}

\item{\textcolor{blue}{Please submit your manuscript in Word format. Do not include the figures in the text file, but it is helpful if you do include a list of the figure captions at the end.}}

Done.
% NC: Need to put the figure numbers onto the file names before submission.

\item{\textcolor{blue}{Could I suggest adding ‘sub-sampling’ to your key words? We now recommend that as many key words as possible are also included in your title, as well as being well-used in the abstract.}}

Done.

\item{\textcolor{blue}{Please upload your supporting information files to Dryad as separate files (with the table in an editable format).}}

We have split the appendices into three files S1) additional details of the datasets; S2) additional figures; and S3) additional tables (now in .docx format rather than PDF).
% NC: To upload but splitting is done

\item{\textcolor{blue}{Please clarify what all the supplementary files are (there is a details field available for each file) and ensure that they are in a form that is commonly readable, or provide information on how to access them.}}

Done. 
% NC: To do when uploading

\item{\textcolor{blue}{Please add the Dryad review link (even though Dryad instructs you not to) to the Data Archiving Statement. This makes it easier for the Editor and Referees, and the link will be updated later as required.}}

Done.
% NC: To do when uploaded

\item{\textcolor{blue}{Where you refer to the data in the text please cite a reference in the form: CARPENTER, D. K., FALCON-LANG, H. J., BENTON, M. J. and GREY, M. 2015. Data
from: Early Pennsylvanian (Langsettian) fish assemblages from the Joggins
Formation, Canada, and their implications for palaeoecology and palaeogeography.
Dryad Digital Repository. https://doi.org/10.5061/dryad.b0551.}}

\end{enumerate}

\subsection{References}

\begin{enumerate}

\item{\textcolor{blue}{Please replace the GitHub URL in the Materials \& Methods section with a citation of the data repository reference (as it is quoted on GitHub; Cooper \& Guillerme 2018).}}

We have now archived the code on Zenodo, so provide the citation for this.
% NC: To do

\item{\textcolor{blue}{Please include all repository details and links in data references (e.g. Bapst et al. 2016a)(see sample Dryad reference above for general format).}}
% NC: To do

\item{\textcolor{blue}{Where you have two references with the same citation (e.g. Bapst et al. 2016), please list them in the same order as they are cited in the text, irrespective of the second author name (currently Bapst et al. 2016b and Brusatte et al. 2014b are both cited before their respective a references; in each case the first reference cited should be a and be listed first).}}

Sorry this is a bug in our reference manager software that we thought would be easier to fix at this stage. Now fixed!
% NC: This should be fixable at the converting to Word stage

\item{\textcolor{blue}{Guillerme (2016) reference requires some more information – is there a URL for the software that can be added? (Similar for any other software reference, e.g. Lloyd 2015.)}}

We have fixed this.
% NC: Fix at converting to Word stage, hopefully there will be a paper to cite by then...
% NC: Actually, I think we have cited the claddis package not the paper - should I change that?

\end{enumerate}

\subsection{Figures}
\textcolor{blue}{Please include the figure number in the file name.}

Done.

\subsection{Tables}
\textcolor{blue}{Journal style uses a single phrase for table captions, with any additional notes in footnotes. Discussion of the data should be included in the text.}

Done.

% NC: Made new word versions of these

We hope we have dealt with all these comments appropriately. 

Please let us know if you need any further information,

Thomas Guillerme and Natalie Cooper

\end{document}