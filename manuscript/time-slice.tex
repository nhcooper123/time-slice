% Preamble
\documentclass[12pt,a4paper]{article}
\usepackage{enumerate} 	
\usepackage{setspace}						
\usepackage{authblk}	
\usepackage{graphicx} 	
%\usepackage[nomarkers, nolists]{endfloat} 
\usepackage{pdflscape}	
\usepackage{mathtools}	
\usepackage[osf]{mathpazo} 
\usepackage{lineno} 
\usepackage{ms}    	
\usepackage{hyperref}
\usepackage[round]{natbib} 
\usepackage{setspace}

\setcounter{secnumdepth}{0} 
\raggedright 			
\pagenumbering{arabic}	
\linenumbers


% First order headings upper case bold
\usepackage{titlesec}
\titleformat*{\section}{\small\bfseries\uppercase}

% Second order headings normal case italics
\titleformat*{\subsection}{\small\itshape}

% Third order, italics, paragraph style
\titleformat*{\paragraph}{\small\itshape}


% lists - arabic in (1; (2); (3) .


% Title page information

\title{Time for a rethink: disparity-through-time}

\author{
	Thomas Guillerme$^{1,2*}$ and Natalie Cooper$^{2}$
}

\date{}

\affiliation{\noindent{\footnotesize
	$^1$ Imperial\\ 
	$^2$Department of Life Sciences, Natural History Museum, Cromwell Road, London, SW7 5BD, UK. natalie.cooper@nhm.ac.uk}\\
	$^*$Corresponding author\\}

\vfill

\runninghead{Time slicing}
\keywords{??} 
% up to 6

\begin{document}

\mstitlepage
\parindent = 1.5em
\addtolength{\parskip}{.3em}

\section{Abstract}
% 300 words max
	

\newpage
\raggedright
\doublespacing
\setlength{\parindent}{1cm}

\section{Introduction}
Disparity-through-time analyes are common in palaeontology.
Examples
They often provide us with exciting insights into mass extinctions, competitive replacements etc.
However, our methods for analysing diversity-through-time have not changed all that much in the XX years they have been popular. 
The way we perfrom these analyses may have profound effects on our conclusions.

One aspect that is not always carefully considered, is how to divide taxa into subsets through time. The nature of disparity, i.e. a diversity metric, means it cannot be calculated using just one individual, so some way of subsetting taxa is required.

Stratigraphy
Time bins
Time slicing (requires a phylogeny)

Changes in disparity through time are generally investigated by calculating the disparity of taxa that occupy the cladisto-space during specific time intervals \citep[e.g][]{cisneros2010,prentice2011,Hughes20082013,hopkinsdecoupling2013,bentonmodels2014,bensonfaunal2014}.
These time intervals are usually defined based on biostratigraphy \citep[e.g.][]{cisneros2010,prentice2011,Hughes20082013,bentonmodels2014} but can also be arbitrarily chosen time periods of equal duration \citep{Butler2012,hopkinsdecoupling2013,bensonfaunal2014}.
However, this approach suffers from two main biases. 
First, if biostratigraphy is used to determine the time intervals, disparity may be distorted towards higher differences between time intervals because biostratigraphical periods are geologically defined based on differences in the morphology of fossils found in the different strata.
Second, this approach assumes that all characters evolve following a punctuated equilibrium model, because disparity is only estimated once for each interval resulting in all changes in disparity occurring between intervals, rather than also allowing for gradual changes within intervals \citep{Hunt21042015}.

To address these issues, we used a ``time-slicing'' approach that considers subsets of taxa in the cladisto-space at specific equidistant points in time, as opposed to considering subsets of taxa between two points in time.
This results in even-sampling of the cladisto-space across time and permits us to define the underlying model of character evolution (punctuated or gradual).  
In practice, time-slicing considers the disparity of any element present in the phylogeny (branches, nodes and tips) at any point in time.

We present this using data from Beck \& Lee and Halliday???  
	
	
\section{Materials and Methods}
\subsection{Overview}
\label{overview-section}
To test the different time subsampling methods, we first need to prepare the data for the disparity 

\begin{enumerate}
  \item Collate data. Disparity-through-time analysis requires a matrix containing morphological characters for each taxon, and information on the time range over which each taxon existed. Time slicing methods also require a phylogeny.
  \item Estimate ancestral character states at the nodes of the phylogeny. This step is required for the time slicing methods.
  \item Build cladisto-spaces.  
    \begin{enumerate}
      \item Construct distance matrices.
      \item ordiantion
    \end{enumerate}
  \item time subsample
    \begin{enumerate}
      \item stratigraphy
      \item time bins
        \begin{enumerate}
          \item 5 million years
          \item 10 million years
          \item 15 million years
        \end{enumerate}  
      \item time slicing
        \begin{enumerate}
          \item punctuated
          \item gradual
          \item random
        \end{enumerate}  
    \end{enumerate}
  \item Estimate disparity for each time subsample
  \item Compare different methods
\end{enumerate}


\subsection{Example datasets}
\label{datasets}
To test the different time slicing methods we selected X datasets. 
Note that although disparity-through-time analyses are possible without a phylogeny, we only use examples with phylogenies here so that we can compare results of time slicing methods, which do require a phylogeny. 

\paragraph{\textit{Mammal data - Halliday.}}
We used the cladistic morphological matrices and the Total Evidence tip-dated trees from \citet[][103 taxa with 446 morphological characters;]{Slater2012MEE} and \citet[][102 taxa with 421 morphological characters]{beckancient2014}.
%We chose these two datasets because they have a similar number of taxa and morphological characters.
%\cite{Slater2012MEE} ranges from 310 million years ago (Ma; Late Carboniferous) to the present and focuses on the clade Mammaliaformes at the family-level and is called hereafter the Mammaliaformes dataset.
%\cite{beckancient2014} ranges from 170 Ma (Middle Jurassic) to the present and focuses on Eutheria at the genus-level and is called hereafter the Eutheria dataset.
%We used the first and last occurrences reported in \cite{Slater2012MEE} and \cite{beckancient2014} as the temporal range of each taxon in our analysis.
%Both phylogenies are illustrated in figures \ref{fig:SlaterTree} and \ref{fig:BeckTree}.
%Both trees contain few taxa compared to the overall species richness of living and fossil mammals \citep{bininda-emondsthe2007,archibald2011extinction}.
%This is because Total Evidence trees need a lot of data, particularly morphological data for living taxa that can be hard to locate \citep[Chapter \ref{chap:TEM_paper} and \ref{chap:missing_mammals};][]{GuillermeCooperMissing,GuillermeCooper}.
%Therefore, most Total Evidence studies to date contain one or two orders of magnitude fewer taxa than phylogenies based solely on molecular data (e.g. thousands of taxa in \citealt{bininda-emondsthe2007,meredithimpacts2011} \textit{vs.} hundreds in \citealt{ronquista2012,Slater2012MEE,beckancient2014}).


\subsection{Estimating ancestral character states}
\label{ace}
For each dataset we used the re-rooting method \citep{Yang01121995,Garland2000} to get Maximum Likelihood estimates of the ancestral states for each character at every node in the phylogeny, using the \texttt{rerootingMethod} function from the \texttt{R} package \texttt{phytools} version 0.4-45 \citep{phytools,R}. % Version still the same?
Where there were missing character data for a taxon we followed the method of \cite{Claddis} and treated missing data as any possible observed state for each character.
For example, if a character had two observed states ($0$ and $1$) across all taxa, we attributed the multi-state ``$0$\&$1$" value to the taxon with missing data, representing an equal probability of being either $0$ or $1$.
This allows the ancestral node of a taxon with missing data to be estimated with no assumptions other than that the taxon has one of the observed character states.
To prevent poor ancestral state reconstructions from biasing our results, especially when a lot of error is associated with the reconstruction, we only included ancestral state reconstructions with a scaled Likelihood $\geq$ $0.95$.
Ancestral state reconstructions with scaled Likelihoods below this threshold were recoded as missing data (``?'').

\subsection{Building cladisto-spaces} % Maybe reduce detail here!
To explore disparity-through-time in our datasets we used a cladisto-space approach \citep[e.g.][]{Foote01071994,Foote29111996,Wesley-Hunt2005,Brusatte12092008,friedmanexplosive2010,toljagictriassic-jurassic2013,Hughes20082013}.
This approach is similar to constructing a morphospace based on continuous morphological data \citep[e.g.][]{friedmanexplosive2010}, except a cladisto-space is an approximation of the morphospace based on cladistic data (i.e. the discrete morphological characters used to build a phylogenetic tree).
Mathematically, a cladisto-space is an $n$ dimensional object that summarises the cladistic distances between the taxa present in a cladistic matrix (see details below).
Although empirically inter-taxon distances are the same in a morphospace or a cladisto-space \citep{foth2012different,hetherington2015cladistic}, we prefer the term cladisto-space to make it clear that this space is estimated using cladistic data and not morphometric data. %and because both objects have slightly different properties.
%For example, because of its inherent combinatory properties, a cladisto-space is a finite theoretical object limited by the product of the number of character states, whereas a morphospace is an infinite theoretical object.
%Thus a cladisto-space will be overloaded if the number of taxa is higher than the product of the number of character states, although this is rarely an issue with empirical data (our cladisto-spaces have maximal capacities of $1.9$$\times$$10^{181}$ taxa for the Mammaliaformes dataset, i.e. 101 orders of magnitude more taxa than the number of particles in the universe; and $4.5$$\times$$10^{159}$ taxa for the Eutheria dataset).

\paragraph{Constructing distance matrices.}
To estimate the cladisto-spaces for each of our datasets we first constructed pairwise distance matrices of length $k$, where $k$ is the total number of tips and nodes in the dataset.
For each dataset separately, we calculated the $k$$\times$$k$ distances using the Gower distance \citep{Gower71}, i.e. the Euclidean distance between two taxa divided by the number of shared characters. 
This allows us to correct for distances between two taxa that share many characters and could be closer to each other than to taxa with fewer characters in common (i.e. because some pairs of taxa share more characters in common than others, they are more likely to be similar).
For cladistic matrices, using this corrected distance is preferable to the raw Euclidean distance because of its ability to deal with discrete or/and ordinated characters as well as with missing data \citep{anderson2012using}.
However, the Gower distance cannot calculate distances when taxa have no overlapping data.
Therefore, we used the \texttt{TrimMorphDistMatrix} function from the \texttt{Claddis} \texttt{R} package \citep{Claddis} to remove pairs of taxa with no cladistic characters in common.
%This led to us removing 11 taxa from the Mammaliaformes dataset but none from the Eutheria dataset.
% NC: Add info on species we have to remove.

\paragraph{Ordination.}
After constructing our distance matrices we transformed them using classical multidimensional scaling \citep[MDS;][]{torgerson1965multidimensional,GOWER01121966,cailliez1983analytical}.
This method (also referred to as PCO; e.g. \citealt{Brusatte2015}; or PCoA; e.g. \citealt{paradisape:2004}) is an eigen decomposition of the distance matrix.
Because we used Gower distances instead of raw Euclidean distances, negative eigenvalues can be calculated.
To avoid this problem, we first transformed the distance matrices by applying the Cailliez correction \citep{cailliez1983analytical} which adds a constant $c^*$ to the values in a distance matrix (apart from the diagonal) so that all the Gower distances become Euclidean ($d_{Gower}+c^*=d_{Euclidean}$; \citealt{cailliez1983analytical}). 
We were then able to extract $n$ eigenvectors for each matrix (representing the $n$ dimensions of the cladisto-space) where $n$ is equal to $k-2$, i.e. the number of taxa in the matrix ($k$) minus the last two eigenvectors that are always null after applying the Cailliez correction.
Contrary to previous studies \citep[e.g][]{brusatte50,cisneros2010,prentice2011,anderson2012using,Hughes20082013,bentonmodels2014}, we use all $n$ dimensions of our cladisto-spaces and not a subsample representing the majority of the variance in the distance matrix (e.g. selecting only $m$ dimensions that represent up to 90\% of the variance in the distance matrix; \citealt{Brusatte12092008,toljagictriassic-jurassic2013}). % NC: Hmmm should we do this the usual way???

Note that our cladisto-spaces represent an ordination of all possible morphologies coded in each study through time.
It is unlikely that all morphologies will co-occur at each time point, therefore, the disparity of the whole cladisto-space is expected to be greater than the disparity at any specific point in time.

\subsection{Calculating disparity}
\label{disparity_calc}
Disparity can be estimated in many different ways \citep[e.g.][]{Wills1994,Ciampaglio2004,thorneresetting2011,hopkinsdecoupling2013,huang2015origins}, however most studies estimate disparity using four metrics: the sum and products of ranges and variances, each of which gives a slightly different estimate of how the data fits within the cladisto-space \citep{Foote01071994,Wills1994,brusatte50,Brusatte12092008,cisneros2010,thorneresetting2011,prentice2011,brusattedinosaur2012,toljagictriassic-jurassic2013,ruta2013,bentonmodels2014,bensonfaunal2014}.
However, these metrics have limitations. 
First, the range metrics are affected by the uneven sampling of the fossil record \citep{Butler2012}.
Second, because we include all $n$ dimensions in the analysis (see above), the products of ranges and variances will tend towards zero since the scores of the last dimension are usually really close to zero themselves. 
We therefore only use the sum of variances metric.

\subsection{Estimating disparity through time} 
\label{time_slicing}

% Might make a table with all the subsets???


% Just tips or tips and nodes????
\paragraph{Stratigraphic time bins.}
We extracted the taxa found in each geological stage from each cladisto-space.
These consisted of the Middle Jurassic to the present (XX subsamples) for dataset 1, the Middle Jurassic to the present (XX subsamples) for dataset 2 and the Middle Jurassic to the present (XX subsamples) X for dataset 3.
For each subsample, we estimated its disparity.
To reduce the influence of outliers on our disparity estimates, we bootstrapped each disparity measurement by randomly resampling with replacement a new subsample of taxa from the observed taxa in the subsample 1000 times.
We then calculated the median disparity value for each subsample along with the 50\% and 95\% confidence intervals.
We also recorded the number of taxa in each subsample as a proxy for taxonomic diversity.


\paragraph{Equally sized time bins.}
We extracted the taxa found in each time bin for each cladisto-space.
We varied the size of the time bins from 5 million year, 10 million years and 20 million years, to determine if the length of the time bin affected our results.
There were XX subsamples for dataset 1, XX subsamples for dataset 2 and XX subsamples for dataset 3.
For each subsample, we estimated its disparity.
To reduce the influence of outliers on our disparity estimates, we bootstrapped each disparity measurement by randomly resampling with replacement a new subsample of taxa from the observed taxa in the subsample 1000 times.
We then calculated the median disparity value for each subsample along with the 50\% and 95\% confidence intervals.
We also recorded the number of taxa in each subsample as a proxy for taxonomic diversity.

\paragraph{Time slicing.}
The ``time-slicing'' approach considers subsets of taxa in the cladisto-space at specific equidistant points in time, as opposed to considering subsets of taxa between two points in time.
This results in even-sampling of the cladisto-space across time and permits us to define the underlying model of character evolution (punctuated or gradual).  
In practice, time-slicing considers the disparity of any element present in the phylogeny (branches, nodes and tips) at any point in time.
When the phylogenetic elements are nodes or tips, the eigenvector scores for the nodes (estimated using ancestral state reconstruction as described in section \ref{ace}) or tips are directly used for estimating disparity.
When the phylogenetic elements are branches we chose the eigenvector score for the branch using one of two evolutionary models:
\begin{enumerate}
    \item{\textbf{Punctuated evolution.}} 
    This model selects the eigenvector score from either the ancestral node or the descendant node/tip of the branch regardless of the position of the slice along the branch. 
    Similarly to the time interval approach, this reflects a model of punctuated evolution where changes in disparity occur either at the start or at the end of a branch over a relatively short time period and clades undergo long periods of stasis during their evolution \citep{Gould1977,Hunt20112007}.
    We applied this model in three ways: 
    \begin{enumerate}[(i)]
      \item selecting the eigenvector score of the ancestral node of the branch (ACCTRAN).
      \item selecting the eigenvector score of the descendant node/tip of the branch (DELTRAN).
      \item randomly selecting either the eigenvector score of the ancestral node or the descendant node/tip of the branch (random).
    \end{enumerate}
    Method (i) assumes that changes always occur early on the branch (accelerated transition, ACCTRAN) and (ii) assumes that changes always occur later (delayed transition, DELTRAN).
    We prefer not to make either assumption so we report the results from (iii), although the ACCTRAN and DELTRAN results are available in the appendix \ref{chap:Appendix_STD} (Fig. \ref{Supp_disparity_all_Mammaliaformes}, \ref{Supp_disparity_all_Eutheria}, \ref{Supp_disparity_all_Mammaliaformes_rarefied}, \ref{Supp_disparity_all_Eutheria_rarefied}).
    \item{\textbf{Gradual evolution.}}
    This model also selects the eigenvector score from either the ancestral node or the descendant node/tip of the branch, but the choice depends on the distance between the sampling time point and the end of the branch.
    If the sampling time point falls in the first half of the branch length the eigenvector score is taken from the ancestral node, conversely, if the sampling time point falls in the second half of the branch length the eigenvector score is taken from the descendant node/tip.
    This reflects a model of gradual evolution where changes in disparity are gradual and cumulative along the branch.
    Under this model, the gradual changes could be either directional or random, however, directional evolution has been empirically shown to be rare \citep[only 5\% of the time][]{Hunt20112007}.
    We therefore considered that changes from a character state A to B were only dependent on the branch length.
\end{enumerate}

We applied our time-slicing approach separately to each cladisto-space, time-slicing the phylogeny every five million years.
% Do we want to vary the time slice timings too?
This resulted in XX subsamples for dataset 1, XX for dataset 2 and XX for dataset 3.
For each subsample, we estimated its disparity assuming punctuated (ACCTRAN, DELTRAN and random) and gradual evolution as described above.
To reduce the influence of outliers on our disparity estimates, we bootstrapped each disparity measurement by randomly resampling with replacement a new subsample of taxa from the observed taxa in the subsample 1000 times.
We then calculated the median disparity value for each subsample along with the 50\% and 95\% confidence intervals.
We also recorded the number of phylogenetic elements (nodes and tips) in each subsample as a proxy for taxonomic diversity.

\subsection{Comparing the results}
%Testing the difference section


\subsection{Rarefaction} 

Finally, disparity may be higher in subsamples with more phylogenetic elements simply because there are more taxa represented.
To test whether this influenced our results, we repeated the analyses using rarefied disparities.
In dataset 1, the minimum number of taxa in each stratigraphy subsample was eight, time bin XX and time slice was XX.


	
\section{Results} 

%Fig. or Figs
%Fig width 80mm single column
%110mm - 2/3 width
%166mm full page


	
%PCA figure
	%\begin{figure}[!htbp]
	%\centering
	%\includegraphics[width=1\linewidth, height=1\textheight, keepaspectratio]{figures/PCA_threeplot.png}
	%\caption[Morphospace (principal components) plot of morphological diversity in tenrec and golden mole skulls.]
	%	{Principal components plots of the morphospaces occupied by tenrecs (triangles, n=31 species) and golden moles (circles, n=12 species) for skulls in dorsal (top left), ventral (top right) and lateral (bottom left) views. Each point represents the average skull shape of an individual species. Axes are principal component 1 (PC1) and principal component 2 (PC2) of the average scores from principal components analyses of mean Procrustes shape coordinates for each species.}
	%\label{fig:threePCA}
	%\end{figure}


% Results table
	%\begin{table}[!htbp]			
		%\caption[Comparing morphological diversity in tenrecs and golden moles.]
		%{Morphological diversity in tenrecs compared to golden moles (12 species). N is the number of tenrec species: 31 species or 17 species including just five representatives of the \textit{Microgale} Genus. Morphological diversity of the Family is the mean Euclidean distance from each species to the Family centroid. Significant differences between the two Families (p$<$0.05) from two-tailed t-tests are highlighted in bold.}
		%\input{tables/diversity.results} 
		%\label{tab:diversity}  
	%\end{table}
	

\section{Discussion}


\section{Conclusions}
	

\section{Acknowledgments}
	We thank all the PalAss folk
	
%\bibliographystyle{sysbio} 
%\bibliography{time-refs} 

\end{document}